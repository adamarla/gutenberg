% This is an empty shell file placed for you by the 'examiner' script.
% You can now fill in the TeX for your question here.

% Now, down to brasstacks. ** Writing good solutions is an Art **. 
% Eventually, you will find your own style. But here are some thoughts 
% to get you started: 
%
%   1. Write the solution as if you are writing it for your favorite
%      14-17 year old to help him/her understand. Could be your nephew, 
%      your niece, a cousin perhaps or probably even you when you 
%      were that age. Just write for them.
%
%   2. Use margin-notes to "talk" to students about the critical insights
%      in the question. The tone can be - in fact, should be - informal
%
%   3. Don't shy away from creating margin-figures you think will help
%      students understand. Yes, it is a little more work per question. 
%      But the question & solution will be written only once. Make that
%      attempt at writing a solution count.
%
%   4. At the same time, do not be too verbose. A long solution can
%      - at first sight - make the student think, "God, that is a lot to know".
%      Our aim is not to scare students. Rather, our aim should be to 
%      create many "Aha!" moments everyday in classrooms around the world
% 
%   5. Ensure that there are *no spelling mistakes anywhere*. We are an 
%      education company. Bad spellings suggest that we ourselves 
%      don't have any education. Also, use American spellings by default
% 
%   6. If a question has multiple parts, then first delete lines 40-41
%   7. If a question does not have parts, then first delete lines 43-69

\question[3] Find three numbers that form a geometric progression if their product is $64$ and their arithmetic mean is $\dfrac{14}{3}$.

\insertQR{QRC}

\ifprintanswers
  % stuff to be shown only in the answer key - like explanatory margin figures
\fi 

\begin{solution}[\halfpage]
  Let the three numbers be $\dfrac{a}{r}$, $a$ and $ar$, where $a$ is the middle term and $r$ is the common ratio. From the condition regarding the product of the numbers we can gather that,
  \begin{align}
    \dfrac{a}{r}\times a \times{ar} &= 64 \\
    a                               &= 4
  \end{align}
  Using this result in the arithmetic mean condition we get,
  \begin{align}
    \dfrac{\dfrac{4}{r}+4+4r}{3} &= \dfrac{14}{3} \\
    4r^2-10r+4                   &= 0 \\
    r                            &= 2,\dfrac{1}{2}
  \end{align}
  The geometric progression is ${2,4,8}$ or ${8,4,2}$ depending on which value of common ratio is picked.
\end{solution}

