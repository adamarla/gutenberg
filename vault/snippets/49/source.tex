\documentclass[14pt,fleqn]{extarticle}
\RequirePackage{prepwell}

\newcommand\dfx{ \dfrac{x-1}{x-2} }
\newcommand\ddfx{ -\frac{1}{\left(x-2 \right)^2}}
\previewoff 

\begin{document} 
\begin{snippet}
    \correct
    
    If $f'(x) = \dfx$, then $f(x)$ has a local maxima at $x = 1$ 
    
    
    \reason
    
    \begin{align}
    f'(x) &= 0 \text{ when } x = 1 \\
	f''(x) &= \ddx f'(x) = \ddx \left[\dfx \right] \\
	&= \underbrace{\dfrac{(x-2)\ddx (x-1) - (x-1)\ddx (x-2)}{\left(x-2 \right)^2}}_{\text{Quotient Rule}} \\
	&= \ddfx < 0 \text{ for all }x \neq 2 
\end{align}

Hence, we have an extrema at $x=1$. And $f''(1) < 0 \implies $maxima at $x=1$ 
    
\end{snippet} 
\end{document} 