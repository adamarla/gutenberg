
%\noprintanswers
%\setcounter{rolldice}{3}
%\printrubric

\ifnumequal{\value{rolldice}}{0}{
  % variables 
  \renewcommand{\vbone}{30}
  \renewcommand{\vbtwo}{40}
  \renewcommand{\vbthree}{3}
  \renewcommand{\vbfour}{50}
}{
  \ifnumequal{\value{rolldice}}{1}{
    % variables 
    \renewcommand{\vbone}{60}
    \renewcommand{\vbtwo}{80}
    \renewcommand{\vbthree}{5}
    \renewcommand{\vbfour}{100}
  }{
    \ifnumequal{\value{rolldice}}{2}{
      % variables 
      \renewcommand{\vbone}{80}
      \renewcommand{\vbtwo}{60}
      \renewcommand{\vbthree}{4}
      \renewcommand{\vbfour}{100}
    }{
      % variables 
      \renewcommand{\vbone}{40}
      \renewcommand{\vbtwo}{30}
      \renewcommand{\vbthree}{10}
      \renewcommand{\vbfour}{50}
    }
  }
}

\question $A$, $B$ and $C$ are three trains on three separate tracks $T_1, T_2$ and $T_3$
- \asif. $T_2\parallel T_3$ and $T_1\perp$ to both $T_2$ and $T_3$. Train $A$ has developed engine 
trouble and is therefore being towed towards $M$ by train $B$ using an iron chain welded to the 
the two trains. But when $B$ is $\vbone$ meters from $N$ (and $\vbtwo$ meters from $M$), it too
develops engine problems. So now, train $B$ is attached to train $C$ in much the same manner 
as $A$ is to $B$ and train $C$ now starts pulling $B$ towards $N$. The question for you then is
- if $C$ is moving at a speed of $\vbthree$ meters per second and the two chains ($AB$ and $BC$)
are $\vbfour$ meters long, then how fast is $A$ moving towards $M$?

\insertQR[4cm]{qrc}

\watchout[-5cm]

  \begin{marginfigure}[-4cm]
    \figinit{pt}
      \figpt 100:(0,80) % line 1
      \figpt 101:$A$(30,80)
      \figpt 102:$M$(70,80)
      \figpt 103:$T_2$(85,80)
      \figpt 104:$T_1$(70,95) % line 2
      \figpt 105:$B$(70,40)
      \figpt 106:$N$(70,10)
      \figpt 107:(70,0)
      \figpt 108:$T_3$(85,10) % line 3
      \figpt 109:$C$(30,10)
      \figpt 110:(0,10)
    \figdrawbegin{}
      \figdrawline [100,101,102,103]
      \figdrawline [104,105,106,107]
      \figdrawline [108,109,110]
      \figdrawline [101,105,109]
    \figdrawend
    \figvisu{\figBoxA}{}{%
      \figwriten 104:(3)
      \figwritee 103:(3)
      \figwritee 108:(3)
      \figset write(mark=$\bullet$)
      \figwrites 101:(3)
      \figwritese 102:(3)
      \figwritee 105:(3)
      \figwritene 106:(3)
      \figwrites 109:(3)
    }
    \centerline{\box\figBoxA}
  \end{marginfigure}


\ADD\vbone\vbtwo\p

\POWER\vbfour{2}\l % L^2
\POWER\vbone{2}\bns
\POWER\vbtwo{2}\bms

\SUBTRACT\l\bns\cn
\SUBTRACT\l\bms\am
\SQRT\cn\q
\SQRT\am\r

\FRACMULT\vbone\q\r\vbtwo\a\b
\FRACMULT\b\a\vbthree{1}\e\f

\begin{solution}
  The facts of the case are as follows
  \begin{align}
    BM +BN &= \p  \\
    \Rightarrow \dfrac{d}{dt}BM + \dfrac{d}{dt}BN &= 0 \nonumber\\
      \Rightarrow \dfrac{d}{dt}BM &= -\dfrac{d}{dt}BN \\
    AM^2 + BM^2 &= \vbfour^2 \\
    \Rightarrow 2AM\dfrac{d}{dt}AM + 2BM\dfrac{d}{dt}BM &= 0 \nonumber\\
    \Rightarrow \dfrac{d}{dt}AM &= -\WRITEFRAC\vbtwo\r\times\dfrac{d}{dt}BM \\
    CN^2 + BN^2 &= \vbfour^2 \\
    \Rightarrow 2CN\dfrac{d}{dt}CN + 2BN\dfrac{d}{dt}BN &= 0 \nonumber\\
    \Rightarrow \dfrac{d}{dt}CN &= -\WRITEFRAC\vbone\q\times\dfrac{d}{dt}BN \\
      &= \WRITEFRAC\vbone\q\times\dfrac{d}{dt}BM \to (2) \nonumber\\
      &= \WRITEFRAC\vbone\q\times\left(-\WRITEFRAC\r\vbtwo\dfrac{d}{dt}AM\right)\to (4) \nonumber 
  \end{align}

  Now, you would have recognized $\dfrac{d}{dt}CN$ as $C's$ speed and $\dfrac{d}{dt}AM$ as $A's$ speed. 

  \begin{align}
    \dfrac{d}{dt}AM &= -\WRITEFRAC\b\a\dfrac{d}{dt}CN = -\WRITEFRAC\e\f
  \end{align}
  Hence, $A$ is being pulled at a speed of $\WRITEFRAC\e\f$ meters per second. The negative 
  sign is because $A$ moves in opposite direction to $C$
\end{solution}

\ifprintrubric
  \begin{table}
  	\begin{tabular}{ p{5cm}p{5cm} }
  		\toprule % in brief (4-6 words), what should a grader be looking for for insights & formulations
  		  \sc{\textcolor{blue}{Insight}} & \sc{\textcolor{blue}{Formulation}} \\ 
  		\midrule % ***** Insights & formulations ******
        Applied Pythgoras on the two right-angled triangles & \\
        And from that, inferred relationship between train speeds & \\
  		\toprule % final numerical answers for the various versions
        \sc{\textcolor{blue}{If question has $\ldots$}} & \sc{\textcolor{blue}{Final answer}} \\
  		\midrule % ***** Numerical answers (below) **********
        C's speed = 3 & A's speed = $\dfrac{16}{3}$ \\
        C's speed = 4 & A's speed = $\dfrac{9}{4}$ \\
        C's speed = 5 & $\qquad$ A's speed = $\dfrac{80}{9}$ \\
        C's speed = 10 & $\qquad$ A's speed = $\dfrac{45}{8}$ \\
  		\bottomrule
  	\end{tabular}
  \end{table}
\fi
