\documentclass[14pt,fleqn]{extarticle}
\RequirePackage{prepwell}

\previewoff 

\begin{document} 

\begin{skill}
\textcolor{blue}{Limits (Trigonometric)}

Fundamental limits featuting $\sin x$ and $\cos x$ only 
\end{skill}

\newcard 

%text
Given below are the \underline{fundamental} limits from which one can 
derive all other limits involving trigonometric functions
%
\begin{center}
\begin{tabular}{MM} 
\midrule 
\text{Rule} & \text{Also} \\
\midrule 
\lim_{x\to 0}\sin x = 0 & \lim_{x\to 0}\sin Nx = 0  \\
\midrule 
\lim_{x\to 0}\cos x = 1 & \lim_{x\to 0}\cos Nx = 1 \\
\midrule 
\lim_{x\to 0}\dfrac{\sin x}{x} = 1 & \lim_{x\to 0}\dfrac{\sin Mx}{Nx} = \dfrac{M}{N} \\
\midrule
\end{tabular} 
\end{center} 

For example, 

\begin{align}
	\lim_{x\to 0}\frac{\tan x}{x} &= \lim_{x\to 0} \left(\frac{\sin x}{\cos x} \right) \cdot\frac{1}{x} \\
	&= \lim_{x\to 0} \left(\frac{\sin x}{x} \right)\cdot\frac{1}{\cos x} \\
	&= \lim_{x\to 0} \left(\frac{\sin x}{x} \right)\cdot \lim_{x\to 0}\frac{1}{\cos x} \\
	&= 1\times 1 = 1 
\end{align}

Hence, by knowing just the limits above, we figured out a new limit, namely 
\[ \qquad\qquad \lim_{x\to 0}\frac{\tan x}{x} = 1 \]

\end{document} 