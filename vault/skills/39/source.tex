\documentclass[14pt,fleqn]{extarticle}
\RequirePackage{prepwell-eng}

\previewoff

\begin{document} 

\begin{skill}
\begin{narrow}
\textcolor{blue}{Inverse Trig. Functions (Basics)}
Meaning of $\sin^{-1}x,\cos^{-1}x$ etc ;
domains and principal ranges 
\end{narrow}

\reason 

%text
A lot of people mistakenly assume 
that \[ \qquad \sin^{-1}x = \dfrac{1}{\sin x} \]

$\sin^{-1} x, \cos^{-1}x$ etc are angles!\newline 

By definition, $\sin\theta = x \implies \sin^{-1} x=\theta$
Same goes for $\cos^{-1}x$ or $\tan^{-1}x$ etc. \newline 

$\theta$ in the above equation belongs to what
is called the \underline{principal range}\newline 

It is the \underline{smallest range of angles} that will give 
all possible values for a trigonometric
function -- as summarized in the table below\newline 

%
\begin{center}
\begin{tabular}{MMM}
\midrule
\text{Function} & x \in & \text{Principal Range} \\
\midrule 
y = \sin^{-1}x & [-1,1] & \left[ -\frac\pi{2},\frac\pi{2} \right] \\
\midrule 
y = \cos^{-1} x & [-1,1] & [0,\pi] \\
\midrule 
y = \tan^{-1} x & \mathbb{R} & \left( -\frac\pi{2},\frac\pi{2} \right) \\ 
\midrule 
y = \csc^{-1}x  & \mathbb{R}-(-1,1) & \left[ -\frac\pi{2},\frac\pi{2} \right] - \lbrace 0\rbrace \\
\midrule 
y = \sec^{-1}x  & \mathbb{R}-(-1,1) & \left[ 0,\pi \right] - \lbrace \frac\pi{2}\rbrace \\
\midrule 
y = \cot^{-1} x & \mathbb{R} & (0,\pi) \\
\midrule
\end{tabular} 
\end{center} 
\end{skill}
\end{document}
