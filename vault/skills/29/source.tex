\documentclass[14pt,fleqn]{extarticle}
\RequirePackage{prepwell-eng}
\previewoff
\begin{document}

%text
\begin{skill}
\begin{narrow}
\textcolor{blue}{Sums of Trigonometric Ratios}
\[ \sin A \pm \sin B, \cos A \pm \cos B \ldots \]
\end{narrow}
%

\reason

%text
Useful identities (listed without proof) 

%
\begin{align}
  \sin A + \sin B &= 2\cdot\sin\frac{A+B}{2}\cdot\cos\frac{A-B}{2} \\
  \sin A - \sin B &= 2\cdot\cos\frac{A+B}{2}\cdot\sin\frac{A-B}{2}  \\
  \cos A + \cos B &= 2\cdot\cos\frac{A+B}{2}\cdot\cos\frac{A-B}{2} \\
  \cos A - \cos B &= 2\cdot\sin\frac{A+B}{2}\cdot\underbrace{\sin\frac{B-A}{2}}_{\text{Watch out!}}
\end{align}
%text

The above results can help turn a complicated expression into something simpler \newline 

For example 
\small\[y = \frac{\cos x + \cos 3x}{\sin x + \sin 3x} = \frac{2\cos 2x\cdot \cos x}{2\cdot \sin 3x\cdot \cos x} = \cot 2x\]\normalsize

Whether you are trying to evaluate $y$ for many values of $x$ or trying to differentiate (or integrate) $y$, you will find that working with $y=\cot 2x$ is going to be easier\newline 

In fact, if you can simplify an expression, it is usually a good idea to do so

\end{skill}
\end{document}
