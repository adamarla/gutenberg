\documentclass[14pt,fleqn]{extarticle}
\RequirePackage{prepwell-eng}

\previewoff 

\begin{document} 
\begin{skill}
    \begin{narrow}
         \textcolor{blue}{Explicit outcome counting}
         
         Counting outcomes by listing them first instead of using 
         $\permi{n}{r}$ or $\combi{n}{r}$ 
    \end{narrow}
    
    \reason 
    
    If the number of possible outcomes is small, then it just might be easier to 
    list them out first and count \newline 
    
    For example, the following is a good use-case for this approach\newline 
    
    \textit{In how many ways can one roll a total of $7$ using two dice?}\newline 
    
    But the following is perhaps not \newline 
    
    \textit{In how many ways is the following possible?}
    \[ \qquad a + b + c = 100\quad \left(a,b,c,\in\mathbb{N} \right)\]
    
    The number of potential outcomes is too large for one to count them all quickly and reliably by listing them out first
\end{skill}
\end{document} 