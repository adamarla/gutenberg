\documentclass[14pt,fleqn]{extarticle}
\RequirePackage{prepwell}

\previewoff

\begin{document}

\begin{skill}
\textcolor{blue}{Inverse Trigonometry (Negatives)}
	
\small\textbf{Example } $\sin^{-1} \left(-x \right) = \sin^{-1} x$ etc. 
\end{skill}

\newcard 

%text
We will prove one. And if you're 
so inclined, then you can try and 
prove the others

%
\begin{align}
\sin\theta &= x \implies \theta = \sin^{-1}x \\ 
\text{And }\sin (-\theta) &= -\sin\theta = -x  \\
\implies -\theta &= \sin^{-1}(-x)  \\
\text{But }\theta &= \sin^{-1} x \implies -\theta = -\sin^{-1} x \\
\therefore \sin^{-1} (-x) &= -\sin^{-1} x 
\end{align}
%text

Below are other such results\newline

%
\begin{tabular}{MM}
\midrule 
\textbf{Result} & \textbf{Condition} \\ 
\midrule 
\sin^{-1}(-x) = -\sin^{-1} x & x\in[-1,1] \\
\midrule 
\cos^{-1}(-x) = \pi -\cos^{-1}x & x\in[-1,1] \\
\midrule
\tan^{-1}(- x) = -\tan^{-1}x &x\in\mathbb{R} \\
\midrule 
\cot^{-1}(-x) = \pi - \cot^{-1}x & x\in\mathbb{R} \\
\midrule 
\sec^{-1}(-x) = \pi-\sec^{-1}x & x\in\mathbb{R} - (-1,1) \\
\midrule
\csc^{-1}(-x) = -\csc^{-1}x & x\in\mathbb{R} - (-1,1) \\
\midrule
\end{tabular} 
\end{document}
