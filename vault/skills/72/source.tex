\documentclass[14pt,fleqn]{extarticle}
\RequirePackage{prepwell-eng}

\previewoff 

\begin{document} 
\begin{skill}
\begin{narrow}
\textcolor{blue}{Probability Basics}

The concept and its mathematical formulation 
\end{narrow}

\reason 

When a dice is rolled, we can get 6 possible numbers. The likelihood of getting a specific number -- for example 1 -- is \underline{defined as} $p = \frac{1}{6}$ \newline 

Generally speaking, if an event $A$ can occur in $N$ ways out of possible $M$ ways, then the probability of $A$ occurring is \[\qquad P(A) = \dfrac{N}{M}, N\leq M\]

Note that probability is always a number between 0 and 1\newline 

Also understand that a probability is a probability - not a guarantee. In our dice example, the probability of getting a 1 was $\frac{1}{6}$ \newline 

This \underline{does not mean} that if we roll the dice 6 times, then we are 
guaranteed to get a 1 in one of the rolls. We might get no 1's -- or all 1's\newline 

What $p= \frac{1}{6}$ means is that if we roll a dice say $N =$ 10 times or 100 times or a million times, then \underline{in the long run}, we should \underline{expect to get} $\frac{N}{6}$ 1's


\end{skill} 
\end{document}
