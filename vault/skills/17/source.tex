\documentclass[14pt,fleqn]{extarticle}
\RequirePackage{prepwell-eng}
\previewoff
\begin{document}

\begin{skill}
    \begin{narrow}
\textcolor{blue}{Theorem of Total Probability}
    \end{narrow}
    
    \reason 
    
    %text
If the set of \underline{all possible events} $(S)$ be 
divided into smaller sets $E_1, E_2, \ldots$
such that 

%
\begin{center}
\begin{tabular}{Nc}
\midrule 
\text{Condition} & Meaning \\ 
\midrule 
E_i\cap E_j = \phi\,(i\neq j) & $E_i$ and $E_j$ are mutually \\
&exclusive $\forall \left( i, j\right) $ \\
\midrule 
E_i\cup E_j\cup\ldots E_n = S & $E_i\ldots E_n$ are exhaustive \\
\midrule 
E_i \neq \phi \,\forall\, $i$ & No $E_i$ is empty \\
\midrule 
\end{tabular}
\end{center} 
%text

then an event $A$ will intersect with \underline{some or all}
of $E_1, E_2,\ldots E_n$ \newline 

And hence, the \underline{probability of $A$} occurring will be the following 

\[ \qquad P(A) = \sum_{i=1}^n P\left( A\,\vert\, E_i\right)\cdot P(E_i)\]

%

\end{skill}
\end{document}