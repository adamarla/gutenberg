\documentclass[14pt,fleqn]{extarticle}
\RequirePackage{prepwell}
\previewoff
\begin{document}

%text
\begin{skill}
\textcolor{blue}{Conditional Probability Notation}

Understanding what $P\left( A\,\vert\, B\right),$

$P\left( B\,\vert\, A\right)$ and $P\left( A\cap B \right)$ mean

\end{skill}
%

\newcard

%text
The notation used to express conditional 
probability has a very special -- and a very
precise -- meaning\newline 

The table below should help you remember
\underline{when to use which notation}

\begin{center}
\begin{tabular}{Ncc}
\midrule
\text{Notation} & What you know & Probability to find \\
\midrule 
P\left( A\,\vert\, B\right) & B has happened & $A$ happened?\\
\midrule 
P\left( B\,\vert\, A\right) & A has happened & $B$ happened?\\
\midrule 
P\left(A\cap B\right) & Nothing & $A$ and $B$  \\
& & happening together \\
\midrule 
\end{tabular}
\end{center} 

\end{document}