\documentclass[14pt,fleqn]{extarticle}
\RequirePackage{prepwell}

\previewoff 

\begin{document} 
\begin{skill}
\textcolor{blue}{Set builder \& roster notations}

Understanding \& working with sets expressed in either of the two forms
\end{skill}

\newcard 

There are two ways of expressing sets \newline 

Either one can write out all the elements that are in the set. This is the \underline{roster form} -- from the English word 'roster'\newline 

For example, below is the set of positive even integers less than 10
\[ \quad E = \left\lbrace 2,4,6,8 \right\rbrace\]

Alternatively, one can write the \underline{recipe} for how the set is built. This is called the \underline{set-builder} form. For example, the above set can also be written as 
\[ \quad E = \left\lbrace x : x = 2k,\, 1\leq k < 5, k\in\mathbb{Z} \right\rbrace\]

It is the \underline{same set, written differently}
\end{document} 