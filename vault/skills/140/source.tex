\documentclass[14pt,fleqn]{extarticle}
\RequirePackage{prepwell}

\previewoff 

\begin{document} 
\begin{skill}
    \begin{narrow}
         \textcolor{blue}{The Inverse Element}
         
         What is it? 
    \end{narrow}
    \reason 
    
    For a binary operation $*$ defined as $a*b$ where \underline{both} $a,b\in A$, $b$ is \underline{inverse of $a$} if 
    \[ \qquad \qquad a * b = e = b * a \]
    where $e\in A$ is the \underline{identity element}\newline 
    
    Notice that every $a$ will have it's own inverse $b$ even though the $e$ is the same for all $a$ \newline 
    
    And it is possible that there is no inverse for some $a\in A$\newline 
    
    For example, for $A = \left\lbrace -1,1,-2,2,3 \right\rbrace$ and a binary operator $*$ defined as $a*b = a + b$ 
    \[ \qquad e = 0\text{ because } a * e = e * a = a \]
    
    Hence, one would think that the \underline{inverse} of $a=3$ would be 
    $b = a^{-1} = -3$ \newline 
    
    But $-3\notin A$. And therefore $a=3$ has no inverse for $a=3$ in $A$
\end{skill}
\end{document} 