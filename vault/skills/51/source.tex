\documentclass[14pt,fleqn]{extarticle}
\RequirePackage{prepwell}
\previewoff
\begin{document}
%text

\begin{skill}
\begin{narrow}
\textcolor{blue}{Differentiation (Chain Rule)}

$\dfrac{d}{dx}f(x)$ when $f(x) = \left(g\circ h \right)(x)$ 
\end{narrow}

%

\reason

%text
\textbf{Intuition}

Let $A$ run twice as fast as $B$ and 
$B$ run 3 times as fast as $C$. Then 
how fast does $A$ run compared to $C$?  \newline 

The answer is simply $6 (= 2\times 3)$\newline 

The idea behind Chain Rule is the same \newline 

If $f(x)$ be a composite function, that is, 
$\quad f(x) = g\left[ h(x) \right] = (g\circ h)(x)$, then 

\[ \qquad \frac{d}{dx} f(x) = \frac{d}{d h(x)}g(x)\times \frac{d}{dx} h(x) \]

For example, if $f(x) = \sin \left( x^2\right)$ , then 

\begin{align}
	\frac{d}{dx} f(x) &= \frac{d}{d x^2} \sin \left(x^2 \right)\times \frac{d}{dx} x^2  \\
	&= \cos \left(x^2 \right)\cdot 2x 
\end{align}
%

\end{skill}
\end{document}
