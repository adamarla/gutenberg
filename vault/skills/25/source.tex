\documentclass[14pt,fleqn]{extarticle}
\RequirePackage{prepwell}

\previewoff

\begin{document}

\begin{skill}
\begin{narrow}
\textcolor{blue}{Polynomial factorization}

Expressing $p(x)$ as the product or quotient of other polynomials
\[ \small p(x) = a(x)\cdot b(x)\text{ or } p(x) = \frac{a(x)}{b(x)}\]
\end{narrow}

\reason

You will be working a lot with polynomials in the  course of your studies. \newline 

And many a times, you will need to \underline{factorize} polynomials in order to simplify a complex expression \newline 

For example, you may need to do one or more of the following conversions
 
\begin{align}
	\left(x^2 - 3x + 2 \right) &\longleftrightarrow \left(x-1 \right)\cdot \left(x-2 \right) \\
	\left(x^2-x+1 \right)&\longleftrightarrow \frac{x^3-1}{x-1}
\end{align}


It would therefore help you to know the expansions of $(x+a)^2, (x^3-a^3), (x-a)^3$ etc. \newline 

Below are some cubic and bi-quadratic (with $x^4$) expansions for your convenience 

\begin{align}
	(x+a)^3 &= x^3 + 3ax^2 + 3a^2x + a^3 \\
	(x-a)^3 &= x^3 - 3ax^2 + 3a^2 x - a^3 \\
	x^3+a^3 &= (x+a)\cdot (x^2-ax+a^2) \\ 
	x^3 - a^3 &= (x-a)\cdot (x^2 + ax  -a^3) \\ 
	x^4 - a^4 &= \underbrace{(x-a)\cdot (x+a)}_{= x^2 - a^2}\cdot (x^2 + 1) 
\end{align}


\end{skill}

\end{document} 
