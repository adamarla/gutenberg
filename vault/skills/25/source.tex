\documentclass[14pt,fleqn]{extarticle}
\RequirePackage{prepwell}

\previewoff

\begin{document}

\begin{skill}
\textcolor{blue}{Polynomials \& Rational Functions}

Being able to do the following conversions as needed 
\[ p(x) \leftrightarrow \frac{a(x)}{b(x)}\text{ OR } \frac{p(x)}{q(x)} = \frac{a(x)}{b(x)} + \frac{c(x)}{d(x)} \]

\end{skill}

\newcard 

You will be working a lot with polynomials in the  course of your studies. \newline 

And many a times, you will need to change \underline{how the polynomial looks} without changing what it is \newline 

For example, you may need to do one or more of the following operations 
\begin{align}
	\left(x^2 - 3x + 2 \right) &\longleftrightarrow \left(x-1 \right)\cdot \left(x-2 \right) \\
	\left(x^2-x+1 \right)&\longleftrightarrow \frac{x^3-1}{x-1} \\
	\frac{x}{x^2-1} &\longleftrightarrow \frac{1}{2} \left[\frac{1}{x-1} + \frac{1}{x+1} \right]
\end{align}

The first two operations involve \underline{factorizing} a polynomial. \newline 

The last operation involves expressing a rational function (a function where both the numerator and denominator are polynomials) as a \underline{partial sum}
\end{document}
