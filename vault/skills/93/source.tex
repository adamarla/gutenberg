\documentclass[14pt,fleqn]{extarticle}
\RequirePackage{prepwell}

\previewoff

\begin{document} 
\begin{skill}
    \begin{narrow}
         \textcolor{blue}{Homogenous functions}
         
         Recognizing a function as a homogenous and finding it's degree 
    \end{narrow}
    
    \reason 
    
    A function $F(x,y)$ is homogenous if 
    \[ \qquad F \left(\lambda x, \lambda y \right) = \lambda^n F \left(x,y \right)\, (\lambda\neq 0)\]
    
    For example, if $x\to 2x$ and $y\to 2y$ then 
    \[ \qquad F(2x,2y) = 2^n\cdot F(x,y) \]
    
    Which means that $F(2x,2y)$ could be the same $(n=0)$, double $(n=1)$, quadruple $(n=2)$ etc \newline 
    
    It all depends on $n$ - which is called the \underline{degree} of the homogenous function \newline 
    
    For example, pressure $P = \dfrac{\text{Force}}{\text{Area}}$. If you double both the force $(F)$ applied and the area $(A)$ over which it is applied, the pressure $(P)$ will remain the same \newline 
    
    Hence $P=\frac{F}{A}$ is a homogenous function of degree $0$\newline 
    
    As regards \underline{differential equations}, you will be expected to 
    solve \underline{only} those where the homogenous function is of \underline{degree $0$}
\end{skill}
\end{document} 