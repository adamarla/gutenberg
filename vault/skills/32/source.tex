\documentclass[14pt,fleqn]{extarticle}
\RequirePackage{prepwell}

\previewoff

\begin{document}

\begin{skill}
    \begin{narrow}
\textcolor{blue}{Simplifying expressions}

Using techniques like rationalization, variable substitution etc. 
         
    \end{narrow}
    
    \reason 

In Math, you can change how something looks as long you don't change its value. For example     

\begin{align}
	\sqrt{\dfrac{1-\sin x}{1+\sin x}} &= \dfrac{\sqrt{1-\sin x}}{\sqrt{1+\sin x}}\times\dfrac{\sqrt{1+\sin x}}{\sqrt{1+\sin x}} \\
	&= \dfrac{\sqrt{1-\sin^2 x}}{\sqrt{\left(1+\sin x \right)^2}} = \dfrac{\cos x}{1+\sin x} 
\end{align}

And sometimes changing how an expression looks makes it easier to work with \newline 

Hence, if you see something that looks big \& complicated or looks like something you need (but not exactly), then you should look for ways of simplifying the expression\newline 

There are no rules. You will just have to be creative! 
    
\end{skill}



\end{document}
