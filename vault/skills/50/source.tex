\documentclass[14pt,fleqn]{extarticle}
\RequirePackage{prepwell}

\previewoff 

\begin{document}

\textcolor{blue}{Integration by Parts:} 

Evaluating $\int f(x)\cdot g(x)\cdot dx$ ; Applying $ILATE$ to ease integration

\newcard 

Differentiation has the \underline{product rule}
\[ \frac{d}{dx} f(x)\cdot g(x) = f'g + fg ' \]
And integration has \underline{integration by parts}
\[ \int f\cdot g\cdot dx = f\underbrace{\int g\cdot dx}_A - \underbrace{\int \overbrace{\left[f'\int g\cdot dx \right]}^B\cdot dx}_{C}\]
Note that there are \underline{3 separate integrals} to evaluate -- $A,B$ and $C$\newline 

\textbf{ILATE}

But what is $f(x)$ in, say, $\int x\cdot\sin x\cdot dx$?  

There are no right or wrong answers. Only better ones.\newline 

In the table below, let \underline{$f(x)$} be the function that \underline{matches first}

\begin{tabular}{c|c|c}
\hline
	 & Meaning & Example\\
\hline
	I & Inverse Trig. & $\sin^{-1} x, \tan^{-1} x$ \\
\hline
	L & Logarithmic & $\log x$\\
\hline
	A & Algebraic & $x^2, 4x^7$ \\
\hline
	T & Trigonometric & $\sin x, \cos 3x$ \\
\hline
	E & Exponential & $e^x, 7^x$\\
\hline
\end{tabular}\newline 

Hence, in $\int x\cdot\sin x\cdot dx$, let 
\[ \underbrace{f(x) = x}_{\text{A = Algebraic}}\text{ and } \underbrace{g(x) = \sin x}_{\text{T = Trigonometric}} \]
\end{document}
