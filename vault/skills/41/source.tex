\documentclass[14pt,fleqn]{extarticle}
\RequirePackage{prepwell}

\previewoff 

\begin{document} 

\begin{skill}
\textcolor{blue}{Inverse Trigonometry (Sums)}

\textit{Example } 
$\sin^{-1}x + \cos^{-1} x = ?$ 
\end{skill}

\newcard 

We will prove one result, namely, 
\[ \qquad \sin^{-1} x + \cos^{-1} x = \frac\pi{2} \]

And then you could perhaps try proving the others. The logic is similar. 


\begin{align}
\sin\theta &= \cos\left( \frac\pi{2} - \theta \right) = x \\
\implies \theta &= \sin^{-1} x \\
\text{and } \frac\pi{2} - \theta &= \cos^{-1} x \\
\text{ or } \theta &= \frac\pi{2} - \cos^{-1}x = \sin^{-1} x \\
\therefore \cos^{-1}x &+ \sin^{-1}x = \frac\pi{2}  
\end{align}
%text

Similarly, 

%
\begin{align}
\tan^{-1}x + \cot^{-1}x &= \frac\pi{2} \\ 
\sec^{-1} x + \csc^{-1} x &= \frac\pi{2} 
\end{align}
\end{document}