\documentclass[14pt,fleqn]{extarticle}
\RequirePackage{prepwell}
\previewoff
\begin{document}

%text
\begin{skill}
\textcolor{blue}{Random Variables} 

And their probability distribution
\end{skill}

%

\newcard

%text
If a random variable $(X)$ can have \underline{only}
values from the set $\lbrace x_1, x_2, \ldots x_n\rbrace$, then 
it's \underline{probability distribution is a table}
that lists the values and the probabilties
of getting those values - as shown below

%
\begin{center}
\begin{tabular}{NNNNN}
\midrule
X & x_1 & x_2 & \ldots & x_n \\
\midrule 
P(X=x_i) & p_1 &p_2& \ldots & p_n \\
\midrule 
\end{tabular}
\end{center} 

%text
As $X$ can only be one of $x_1, x_2 \ldots  x_n $, hence it should come as no surprise that \[ \qquad\qquad \sum_{k=1}^n p_k = 1 \] 
\end{document}