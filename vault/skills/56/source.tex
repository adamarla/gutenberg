\documentclass[14pt,fleqn]{extarticle}
\RequirePackage{prepwell}
\previewoff
\begin{document}

%text
\begin{skill}
\begin{narrow}
\textcolor{blue}{Parametric Curves}

What they are ; Finding a point, tangent or normal on the curve

\end{narrow}
%

\reason

%text
Ordinarily, we expect curves to
be of the form $y = f(x)$ \newline 

But if both $x$ and $y$ are expressed
in terms of a third variable/parameter,
then we have what is called a 
\underline{parametric curve}\newline 

And therefore, \underline{instead of one} $y = f(x)$ type equation,
\underline{we will have two equations} -- both in terms 
of the parameter $(t)$

\[ \quad x = f(t)  \text{ and }  y = g(t) \]

The curve is still plotted in $x-y$. It is only expressed in terms of $t$\newline 

 \textbf{Coordinates of a point $A$}\newline 
 
 A point $A$ on the curve will have 
 the coordinates 
 \[ \qquad A = (x,y) = \left(f(t), g(t) \right)\]

\textbf{Slope of the tangent at $A$}\newline
 
 The slope of the tangent will continue
 to be $\dfrac{dy}{dx}$. However, 

 \[ \qquad\qquad \dfrac{dy}{dx} = \frac{dy / dt}{dx / dt}\]
 
%


\end{skill}
\end{document}
