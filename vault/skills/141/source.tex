\documentclass[14pt,fleqn]{extarticle}
\RequirePackage{prepwell-eng}

\previewoff 

\begin{document} 
\begin{skill}
    \begin{narrow}
         \textcolor{blue}{Cartesian Product of Sets}
         
         What does $A\times B$ mean when $A$ and $B$ are both sets?
    \end{narrow}
    
    \reason 
    
    Given two \underline{non-empty} sets $A$ and $B$, we define 
    $A\times B$ as the \underline{set of all ordered pairs} $(x,y)$ 
    where $x\in A$ and $y\in B$ \newline 
    
    Symbolically, we write this as 
    \[ \quad A\times B = \left\lbrace (x,y)\,\vert\, x\in A, y\in B \right\rbrace\]
    Now, some examples  
    \begin{center}
  \begin{tabular}{NNN}
   \toprule
        A & B & A\times B \\
   \midrule 
   \left\lbrace 1,2\right\rbrace & \left\lbrace 3,4\right\rbrace & 
   \left\lbrace (1,3),(1,4),(2,3),(2,4) \right\rbrace\\
    \midrule 
    \left\lbrace a\right\rbrace & \left\lbrace b,c\right\rbrace & 
    \left\lbrace (a,b),(a,c)\right\rbrace \\
    \bottomrule
  \end{tabular}
\end{center} 

  Once again, $A\times B$ is also a set. But it is a set of elements like $(a,b)$ instead of just $a$ or $b$ \newline 
  
  Moreover, the idea can be extended to finding $A\times B\times C$\newline 
  
  The set $A\times B\times C$ will made up of elements like 
  $(a,b,c)$ 

\end{skill}
\end{document} 