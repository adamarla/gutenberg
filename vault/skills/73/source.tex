\documentclass[14pt,fleqn]{extarticle}
\RequirePackage{prepwell}

\previewoff

\begin{document} 
\begin{skill}
    \begin{narrow}
         \textcolor{blue}{When is $f(x) = \frac{p(x)}{q(x)} \geq 0\text{ or } \leq 0$?}
         
         Determining the values of $x$ for which a rational function is $\geq 0$ or $\leq 0$ 
    \end{narrow}
    
    \reason   
    
    \[ \qquad x + 2 \geq 0 \implies x \geq -2 \]
     That was easy. But what if $f(x)$ is a little more complicated, say a rational function, like 
    \[ \quad f(x) = \frac{(x-1)\cdot (x+3)}{x-2}, x\neq 2 \]
    When is $f(x) \geq 0$ and when is it $\leq 0$? \newline 
    
    One only needs to be methodical to solve such inequalities. First, determine the "roots" of $f(x)$ as shown below 
    
    \begin{center}
  \begin{tabular}{lN}
   \toprule
        & \text{Roots} \\
   \midrule 
   Numerator & x = 1, -3 \\
    \midrule 
    Denominator & x = 2 \\
    \bottomrule
  \end{tabular}
\end{center}

Then \underline{split the number line at the roots} and determine the 
sign of each individual term in $f(x)$ -- and therefore of $f(x)$  
\begin{center}
  \begin{tabular}{NNNNN}
   \toprule
        &  x < -3 & (-3,1) & (1,2) & > 2 \\
   \midrule 
   x-1 & - & - & + & + \\
    \midrule 
    x + 3  & - & + & + & + \\
    \midrule 
    x - 2 & - & - & - & + \\ 
    \midrule 
    \frac{(x-1)\cdot (x+3)}{x-2} & - & + & - & + \\
    \bottomrule
  \end{tabular}
\end{center}

\end{skill}
\end{document} 