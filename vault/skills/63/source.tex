\documentclass[14pt,fleqn]{extarticle}
\RequirePackage{prepwell-eng}

\previewoff

\begin{document} 

\begin{skill}
\begin{narrow}
\textcolor{blue}{Interpreting $f'(x)$ and $f''(x)$}

Using $f'(x)$ to classify functions as increasing/decreasing; Conditions for maxima \& minima at $x=a$
\end{narrow}

\reason 

A function $f(x)$ defined in $[a,b]$ is 

\begin{center}
  \begin{tabular}{cNN}
  \toprule
        & \text{If} & \text{Remember}  \\
   \midrule 
   Increasing & f'(x) \geq 0 & f'(x)\text{ can be } 0 \\ 
    \midrule 
    Strictly increasing & f'(x) > 0 & f'(x)\neq 0 \\ 
    \midrule
    Decreasing & f'(x) \leq 0 & f'(x)\text{ can be } 0 \\
    \midrule
    Strictly decreasing & f'(x) < 0 & f'(x)\neq 0 \\
    \bottomrule
  \end{tabular}
\end{center}

\textbf{Maxima \& Minima}

The point at which $f(x)$ stops rising and start falling is called a \underline{maxima}\newline 

Similarly, the point at which $f(x)$ stops falling and starts rising is called a \underline{minima}\newline 

Maxima \& minima are both \underline{extrema} points \newline 

The conditions for classifying an extrema as a maxima or a minima are given below 

\begin{center}
  \begin{tabular}{cNN}
  \toprule
        & f'(x) & f''(x) \\
   \midrule
   Maxima & 0 & < 0 \\ 
    \midrule 
    Minima & 0 & > 0 \\
    \bottomrule
  \end{tabular}
\end{center}

\textbf{Remember:} $f'(x) = 0$ at all extrema points. But one must look at $f''(x)$ to know whether the extrema is a \underline{maxima or a minima}
\end{skill} 
\end{document}
