\documentclass[14pt,fleqn]{extarticle}
\RequirePackage{prepwell-eng}

\previewoff 

\begin{document} 
\begin{skill}
    \begin{narrow}
         \textcolor{blue}{The Identity Element}
         
         What is it? 
    \end{narrow}
    
    \reason 
    
Let a binary operation $*$ on some \underline{non-empty} set $A$ be defined as 
$a*b$\newline 

The \underline{identity element $e$} is an element - or elements - in $A$ $(e\in A)$ such that for \underline{every} $a\in A$ 
\[ \qquad\quad a * e = a = e* a \text{ for all }a\in A \]

Once again, $e\in A$. And the same $e$ satisfies the above condition for all $a\in A$\newline 

\underline{As far as your syllabus is concerned}, the above definition is enough. But it gets a little more interesting\newline 

It is possible that $a*e = a$ but $e*a \neq a$. Or that $e*a = a$ but $a*e \neq a$. Or, that there is no $e$. Below are some examples 

\begin{center}
  \begin{tabular}{NNNN}
   \toprule
        a * b & e & a * e = a & e * b = b  \\
   \midrule 
   a + (b-1)\cdot(b-2) & 1,2 & \checkmark & \times \\
    \midrule 
    a\cdot (a^2-1) + b & 0,1,-1 & \times & \checkmark \\
    \midrule 
    ab + 1 & \left\lbrace \phi\right\rbrace & \times & \times \\
    \midrule 
    a + b & 0 & \checkmark & \checkmark \\
    \bottomrule
  \end{tabular}
\end{center}

\end{skill}

\end{document} 