\documentclass[14pt,fleqn]{extarticle}
\RequirePackage{prepwell-eng}
\previewoff


\begin{document}
\begin{skill}
    \begin{narrow}
         \textcolor{blue}{Relation Types}
         
         Meaning of reflexive, transitive, symmetric and 
         equivalent relations 
    \end{narrow}
    
    \reason 
    
    Best if we just took some examples before reading any definitions\newline 
    
    Given a set $A = \left\lbrace 1,2,3\right\rbrace$, let us define the following 
    relations 
    
    \begin{center}
  \begin{tabular}{NNN}
   \toprule
        & \text{Recipe} & \text{Will give} \\
   \midrule 
   R_1 & \left\lbrace (a,b): a \leq b\right\rbrace & \left\lbrace (1,1),(1,2),(1,3),(2,2), (2,3), (3,3)\right\rbrace \\
    \midrule 
    R_2 & \left\lbrace (a,b): a = b \right\rbrace & \left\lbrace (1,1),(2,2), (3,3)\right\rbrace \\
    \midrule 
    R_3 & \left\lbrace (a,b): a \geq b\right\rbrace & \left\lbrace (1,1), (2,1), (2,2), (3,1), (3,2), (3,3) \right\rbrace\\
    \bottomrule
  \end{tabular}
\end{center}

The three relations have also been classified below. As you read the table, remember the "recipes" we defined above 
\begin{center}
  \begin{tabular}{lMNNN}
   \toprule
       Type & \text{Meaning} & R_1 & R_2 & R_3 \\
   \midrule 
   Reflexive & (a,a)\in R\text{ for all } a & \checkmark & \checkmark & \checkmark \\ 
   \midrule 
   Symmetric & (a,b)\in R\implies (b,a)\in R & X & \checkmark & X \\ 
    \midrule 
    Transitive & (a,b), (b,c)\in R \implies (a,c)\in R & \checkmark & \checkmark & \checkmark \\
    \midrule 
    Equivalent & \text{Is all three of the above} & X & \checkmark & X \\
    \bottomrule
  \end{tabular}
\end{center}
Note that these are \underline{not either-or} conditions\newline 

A relation can be transitive but not reflexive or reflexive but not symmetric. 
It comes down to what the relation is 
\end{skill}
\end{document}