% This is an empty shell file placed for you by the 'examiner' script.
% You can now fill in the TeX for your question here.

% Now, down to brasstacks. ** Writing good solutions is an Art **. 
% Eventually, you will find your own style. But here are some thoughts 
% to get you started: 
%
%   1. Write to be understood - but be crisp. Your own solution should not take 
%      more space than you will give to the student. Hence, if you take more than 
%      a half-page to write a solution, then give the student a full-page and so on...
%
%   2. Use margin-notes to "talk" to students about the critical insights
%      in the question. The tone can be - in fact, should be - informal
%
%   3. Don't shy away from creating margin-figures you think will help
%      students understand. Yes, it is a little more work per question. 
%      But the question & solution will be written only once. Make that
%      attempt at writing a solution count.
%      
%      3b. Use bc_to_fig.tex. Its an easier way to generate plots & graphs 
% 
%   4. Ensure that there are *no spelling mistakes anywhere*. We are an 
%      education company. Bad spellings suggest that we ourselves 
%      don't have any education. Also, use American spellings by default
% 
%   5. If a question has multiple parts, then first delete lines 40-41
%   6. If a question does not have parts, then first delete lines 43-69
%   
%   7. Create versions of the question when possible. Use commands defined in 
%      tufte-tweaks.sty to do so. Its easier than you think

%\noprintanswers
%\setcounter{rolldice}{3}

\ifnumequal{\value{rolldice}}{0}{
  % variables 
  \renewcommand{\vbone}{2}
  \renewcommand{\vbtwo}{3}
  \renewcommand{\vbthree}{9}
  \renewcommand{\vbfive}{6} % reqd. n. Greatest = n + 1
  \renewcommand{\vbsix}{\dfrac{7\cdot 3^{13}}{2}} % reduced expression
}{
  \ifnumequal{\value{rolldice}}{1}{
    % variables 
    \renewcommand{\vbone}{4}
    \renewcommand{\vbtwo}{5}
    \renewcommand{\vbthree}{7}
    \renewcommand{\vbfive}{4}
    \renewcommand{\vbsix}{\dfrac{7\cdot 5^9}{4}} % reduced expression
  }{
    \ifnumequal{\value{rolldice}}{2}{
      % variables 
      \renewcommand{\vbone}{2}
      \renewcommand{\vbtwo}{5}
      \renewcommand{\vbthree}{8}
      \renewcommand{\vbfive}{7}
      \renewcommand{\vbsix}{\dfrac{5^{14}}{8}} % reduced expression
    }{
      % variables 
      \renewcommand{\vbone}{4}
      \renewcommand{\vbtwo}{3}
      \renewcommand{\vbthree}{7}
      \renewcommand{\vbfive}{2}
      \renewcommand{\vbsix}{7\cdot 4^3 \cdot 3^5}
    }
  }
}

\renewcommand{\vbfour}{\dfrac{\vbtwo}{\vbone}}
\gcalcexpr[0]\tp{\vbthree + 1}
\gcalcexpr[0]\tq{(\vbtwo * \vbtwo)}
\gcalcexpr[0]\tr{(\vbone * \vbone)}
\gcalcexpr[0]\ts{\tq + \tr}
\gcalcexpr[0]\tt{\tp * \tq}
\gcalcexpr[0]\tg{\vbfive + 1} % greatest term
\gcalcexpr[0]\tu{\tg - 1} % = \vbfive

\question[5] Find the largest term (numerically) in the expansion of $(\vbone + \vbtwo x)^{\vbthree}$ if 
$x = \frac{\vbtwo}{\vbone}$

\insertQR[-15pt]{QRC}

\watchout

\ifprintanswers
\fi 

\begin{solution}[\fullpage]
	If $T_n$ be the $n^{\text{th}}$ term and $T_{n+1}$ be the $(n+1)^{\text{th}}$ term - $n \in\aleph$ - then
	the terms keep on increasing as along as $\dfrac{T_{n+1}}{T_n} \geq 1$
	
	And if we re-write $(\vbone + \vbtwo x)^\vbthree$ as $\vbone^\vbthree\cdot\left(\vbfour x + 1\right)^\vbthree$, then
	\begin{align}
		\dfrac{T_{n+1}}{T_n} &= \dfrac
		{\vbone^\vbthree\cdot\encr\vbthree{n}\cdot\left( \vbfour\cdot x\right)^{n}}
		{\vbone^\vbthree\cdot\encr\vbthree{n-1}\cdot\left( \vbfour\cdot x\right)^{n-1}} \\
		&= \dfrac{\fncr\vbthree{n}}{\fncr\vbthree{(n-1)}}\cdot\left( \vbfour x \right) \\
		&= \dfrac{\tp - n}{n}\cdot\left( \vbfour x \right)
	\end{align}	 
	When $x = \vbfour$, then 
	\begin{align}
		\dfrac{T_{n+1}}{T_n} &= \dfrac{\tp -n}{n}\cdot\dfrac{\tq}{\tr} \\
		\text{And therefore, if } \dfrac{T_{n+1}}{T_n} \geq 1 &\Rightarrow n \leq \dfrac{\tt}{\ts}
	\end{align}
	But as $n \in \aleph,\, n = \vbfive$ is the only sensible answer and $T_{\tg}$ the greatest term. Moreover, 
	$T_{\tg} = \vbone^{\vbthree}\cdot\encr\vbthree\tu\cdot\left( \vbfour\cdot\dfrac{\vbtwo}{\vbone} \right)^{\tu}
  = \vbsix$
\end{solution}

