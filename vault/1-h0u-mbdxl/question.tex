% This is an empty shell file placed for you by the 'examiner' script.
% You can now fill in the TeX for your question here.

% Now, down to brasstacks. ** Writing good solutions is an Art **. 
% Eventually, you will find your own style. But here are some thoughts 
% to get you started: 
%
%   1. Write the solution as if you are writing it for your favorite
%      14-17 year old to help him/her understand. Could be your nephew, 
%      your niece, a cousin perhaps or probably even you when you 
%      were that age. Just write for them.
%
%   2. Use margin-notes to "talk" to students about the critical insights
%      in the question. The tone can be - in fact, should be - informal
%
%   3. Don't shy away from creating margin-figures you think will help
%      students understand. Yes, it is a little more work per question. 
%      But the question & solution will be written only once. Make that
%      attempt at writing a solution count.
%
%   4. At the same time, do not be too verbose. A long solution can
%      - at first sight - make the student think, "God, that is a lot to know".
%      Our aim is not to scare students. Rather, our aim should be to 
%      create many "Aha!" moments everyday in classrooms around the world
% 
%   5. Ensure that there are *no spelling mistakes anywhere*. We are an 
%      education company. Bad spellings suggest that we ourselves 
%      don't have any education. And, use American spellings

\question[2] In the adjoining figure, $O$ is the center of the circle and $\angle BOC = \ang{100}$.
What is $\angle BDC$ equal to ?

\ifprintanswers
  % stuff to be shown only in the answer key - like explanatory margin figures
\fi 
\begin{marginfigure}
	\figinit{pt}
    \figpt 1: (60,0)
    \figptcirc 2:$A$: 1;50(70)
    \figptcirc 3:$B$: 1;50(200)
    \figptcirc 4:$C$: 1;50(310)
    \figptcirc 5:$D$: 1;50(250)
	\figdrawbegin{}
	\figdrawcirc 1(50)
    \figdrawline [3,1,4,5,3]
    \ifprintanswers
      \figdrawline [3,2,4]
    \fi
	\figdrawend
  \figvisu{\figBoxA}{}{
    \figsetmark{$\bullet$}
    \figwrites 1:$O$(5)
    \figwriten 2:(5)
    \figwritew 3:(5)
    \figwritee 4:(5)
    \figwrites 5:(5)
  }
  \centerline{\box\figBoxA}
\end{marginfigure}
\begin{solution}[\halfpage]
	If we pick a point $A$ on the circle's circumfrence, then $ACDB$ is a cyclic 
	quadrilateral in which,
	\begin{align}
		\angle BAC &= \dfrac{1}{2}\cdot\angle BOC = \ang{50}
	\end{align}
	Moreover, because it is a cyclic quadrilateral, 
	\begin{align}
		\angle BAC + \angle BDC &= \ang{180} \\
		\Rightarrow \angle BDC &= \ang{180} - \angle BAC = \ang{130}
	\end{align}
\end{solution}
% This is an empty shell file placed for you by the 'examiner' script.
% You can now fill in the TeX for your question here.

% Now, down to brasstacks. ** Writing good solutions is an Art **. 
% Eventually, you will find your own style. But here are some thoughts 
% to get you started: 
%
%   1. Write the solution as if you are writing it for your favorite
%      14-17 year old to help him/her understand. Could be your nephew, 
%      your niece, a cousin perhaps or probably even you when you 
%      were that age. Just write for them.
%
%   2. Use margin-notes to "talk" to students about the critical insights
%      in the question. The tone can be - in fact, should be - informal
%
%   3. Don't shy away from creating margin-figures you think will help
%      students understand. Yes, it is a little more work per question. 
%      But the question & solution will be written only once. Make that
%      attempt at writing a solution count.
%
%   4. At the same time, do not be too verbose. A long solution can
%      - at first sight - make the student think, "God, that is a lot to know".
%      Our aim is not to scare students. Rather, our aim should be to 
%      create many "Aha!" moments everyday in classrooms around the world
% 
%   5. Ensure that there are *no spelling mistakes anywhere*. We are an 
%      education company. Bad spellings suggest that we ourselves 
%      don't have any education. And, use American spellings

\question[2] A necklace is appraised at \texteuro 6,300. If the value of 
the necklace has increased 7\% year-on-year, then how much was it worth 
10 years back?

\ifprintanswers
  % stuff to be shown only in the answer key - like explanatory margin figures
\fi 

\begin{solution}[\halfpage]
	\begin{align}
		P_{\texttt{now}} &= P_{\texttt{10 years back}}\cdot\left( 1 + \dfrac{7}{100}\right)^{10} \\
		\Rightarrow P_{\texttt{10 years back}} &= \dfrac{\text{\texteuro 6,300}}{(1.07)^{10}} \\
		&= \text{\texteuro} 3,202.60
	\end{align}
\end{solution}

\newpage
% This is an empty shell file placed for you by the 'examiner' script.
% You can now fill in the TeX for your question here.

% Now, down to brasstacks. ** Writing good solutions is an Art **. 
% Eventually, you will find your own style. But here are some thoughts 
% to get you started: 
%
%   1. Write the solution as if you are writing it for your favorite
%      14-17 year old to help him/her understand. Could be your nephew, 
%      your niece, a cousin perhaps or probably even you when you 
%      were that age. Just write for them.
%
%   2. Use margin-notes to "talk" to students about the critical insights
%      in the question. The tone can be - in fact, should be - informal
%
%   3. Don't shy away from creating margin-figures you think will help
%      students understand. Yes, it is a little more work per question. 
%      But the question & solution will be written only once. Make that
%      attempt at writing a solution count.
%
%   4. At the same time, do not be too verbose. A long solution can
%      - at first sight - make the student think, "God, that is a lot to know".
%      Our aim is not to scare students. Rather, our aim should be to 
%      create many "Aha!" moments everyday in classrooms around the world
% 
%   5. Ensure that there are *no spelling mistakes anywhere*. We are an 
%      education company. Bad spellings suggest that we ourselves 
%      don't have any education. And, use American spellings

\question[3] If one invests \texteuro 8,000 for a period of 3 years in a scheme
that offers 10\% per annum compounded annually, then how much interest would
one earn in the \emph{second} year?

\ifprintanswers
  % stuff to be shown only in the answer key - like explanatory margin figures
\fi 

\begin{solution}[\halfpage]
  If $A_0$ be the initial amount, $A_1$ the amount at the end of the first year
  and $A_2$ the amount at the end of the second year, then the interest earned 
  in the second year $(= I_2)$ would be
  
  \begin{align}
  	I_2 &= A_2 - A_1 \\ 
  	    &= A_0\cdot(1+R)^2 - A_0\cdot(1+R) \\
  	    &= \text{\texteuro 8,000}\cdot\left[ (1+0.1)^2-(1+0.1)\right] \\
  	    &= \text{\texteuro 880}
  \end{align}
\end{solution}
% This is an empty shell file placed for you by the 'examiner' script.
% You can now fill in the TeX for your question here.

% Now, down to brasstacks. ** Writing good solutions is an Art **. 
% Eventually, you will find your own style. But here are some thoughts 
% to get you started: 
%
%   1. Write the solution as if you are writing it for your favorite
%      14-17 year old to help him/her understand. Could be your nephew, 
%      your niece, a cousin perhaps or probably even you when you 
%      were that age. Just write for them.
%
%   2. Use margin-notes to "talk" to students about the critical insights
%      in the question. The tone can be - in fact, should be - informal
%
%   3. Don't shy away from creating margin-figures you think will help
%      students understand. Yes, it is a little more work per question. 
%      But the question & solution will be written only once. Make that
%      attempt at writing a solution count.
%
%   4. At the same time, do not be too verbose. A long solution can
%      - at first sight - make the student think, "God, that is a lot to know".
%      Our aim is not to scare students. Rather, our aim should be to 
%      create many "Aha!" moments everyday in classrooms around the world
% 
%   5. Ensure that there are *no spelling mistakes anywhere*. We are an 
%      education company. Bad spellings suggest that we ourselves 
%      don't have any education. And, use American spellings

\question[3] A bank is offering two investment schemes. The first scheme offers
7\% compound interest in the first year and - in each subsequent year - 25\% less 
than what was offered in the previous year. The second scheme offers a constant 5.75\%
throughout. If you are investing for a 2-year period, which scheme would you choose?

\ifprintanswers
  % stuff to be shown only in the answer key - like explanatory margin figures
\fi 

\begin{solution}[\halfpage]
	The amount in-hand at the end of 2 years with the first scheme is
	\begin{align}
		P_1 &= P_0\cdot\underbrace{\left( 1 + \dfrac{7}{100}\right)}_{\texttt{first year}}
		\cdot\underbrace{\left( 1 + \dfrac{5.25}{100}\right)}_{\texttt{second year}} \\
		&= 1.126\cdot P_0
	\end{align}
	
	Similarly, with the second scheme, you would get
	\begin{align}
		P_2 &= P_0\cdot\left( 1 + \dfrac{5.75}{100}\right)^{2} \\
		    &= 1.118\cdot P_0
	\end{align}
	
	Clearly, you would be better off going with the first scheme
\end{solution}

\newpage
% This is an empty shell file placed for you by the 'examiner' script.
% You can now fill in the TeX for your question here.

% Now, down to brasstacks. ** Writing good solutions is an Art **. 
% Eventually, you will find your own style. But here are some thoughts 
% to get you started: 
%
%   1. Write the solution as if you are writing it for your favorite
%      14-17 year old to help him/her understand. Could be your nephew, 
%      your niece, a cousin perhaps or probably even you when you 
%      were that age. Just write for them.
%
%   2. Use margin-notes to "talk" to students about the critical insights
%      in the question. The tone can be - in fact, should be - informal
%
%   3. Don't shy away from creating margin-figures you think will help
%      students understand. Yes, it is a little more work per question. 
%      But the question & solution will be written only once. Make that
%      attempt at writing a solution count.
%
%   4. At the same time, do not be too verbose. A long solution can
%      - at first sight - make the student think, "God, that is a lot to know".
%      Our aim is not to scare students. Rather, our aim should be to 
%      create many "Aha!" moments everyday in classrooms around the world
% 
%   5. Ensure that there are *no spelling mistakes anywhere*. We are an 
%      education company. Bad spellings suggest that we ourselves 
%      don't have any education. And, use American spellings

\question[3] In the adjoining figure, $O$ is the center of the circle. Find $\angle BAC$ 
given that $\angle BOA = \ang{80}$ and $\angle AOC = \ang{120}$

\ifprintanswers
  % stuff to be shown only in the answer key - like explanatory margin figures
\fi 
\begin{marginfigure}
	\figinit{pt}
    \figpt 1: (60,0)
    \figptcirc 2:$A$: 1;50(100)
    \figptcirc 3:$B$: 1;50(190)
    \figptcirc 4:$C$: 1;50(340)
	\figdrawbegin{}
	\figdrawcirc 1(50)
    \figdrawline [1,2]
    \figdrawline [1,3]
    \figdrawline [1,4]
    \figdrawarccircP 1;8 [2,3]
    \figdrawarccircP 1;10 [4,2]
    \ifprintanswers
      \figset (dash=4)
      \figdrawline [3,2,4]
    \fi
	\figdrawend
  \figvisu{\figBoxA}{}{
    \figsetmark{$\bullet$}
    \figwrites 1:$O$(5)
    \figwriten 2:(5)
    \figwritew 3:(5)
    \figwritee 4:(5)
  }
  \centerline{\box\figBoxA}
\end{marginfigure}
\begin{solution}[\halfpage]
	If we join $A$ with $B$ and $A$ with $C$ as shown, then we get two isoceles 
	triangles, $\triangle AOB$ and $\triangle AOC$
	
	In $\triangle AOB$,
	\begin{align}
		\angle OBA = \angle OAB &= \dfrac{1}{2}\cdot(\ang{180}-\angle AOB) \\
		                        &= \ang{50}
	\end{align}
	
	Similarly, in $\triangle AOC$
	\begin{align}
		\angle OAC = \angle OCA &= \dfrac{1}{2}\cdot(\ang{180}-\angle AOC) \\
		                        &= \ang{40}
	\end{align}
	
	And therefore,
	\begin{align}
		\angle BAC &= \angle OAB + \angle OAC \\
		           &= \ang{50} + \ang{40} = \ang{90}
	\end{align}
\end{solution}

\newpage
% This is an empty shell file placed for you by the 'examiner' script.
% You can now fill in the TeX for your question here.

% Now, down to brasstacks. ** Writing good solutions is an Art **. 
% Eventually, you will find your own style. But here are some thoughts 
% to get you started: 
%
%   1. Write the solution as if you are writing it for your favorite
%      14-17 year old to help him/her understand. Could be your nephew, 
%      your niece, a cousin perhaps or probably even you when you 
%      were that age. Just write for them.
%
%   2. Use margin-notes to "talk" to students about the critical insights
%      in the question. The tone can be - in fact, should be - informal
%
%   3. Don't shy away from creating margin-figures you think will help
%      students understand. Yes, it is a little more work per question. 
%      But the question & solution will be written only once. Make that
%      attempt at writing a solution count.
%
%   4. At the same time, do not be too verbose. A long solution can
%      - at first sight - make the student think, "God, that is a lot to know".
%      Our aim is not to scare students. Rather, our aim should be to 
%      create many "Aha!" moments everyday in classrooms around the world
% 
%   5. Ensure that there are *no spelling mistakes anywhere*. We are an 
%      education company. Bad spellings suggest that we ourselves 
%      don't have any education. And, use American spellings

\question[3] If the difference between the simple interest earned over one year
and the compound interest - payed semi-annually - over the same year is \texteuro 180,
then what is the initial investment amount. The annualized interest rate in both cases is 10\% 

\ifprintanswers
  % stuff to be shown only in the answer key - like explanatory margin figures
\fi 

\begin{solution}[\fullpage]
	If $I_1$ be the interest earned from \emph{compounding} in one year, $P_0$ the 
	initial investment amount and $P_1$ the amount at the end of the first year, then
	
	\begin{align}
		I_1 &= P_1 - P_0 \\ 
		    &= P_0\cdot\left[ 1+\dfrac{R}{N}\right]^N - P_0 \\
		    &= P_0\cdot\left[ (1+\dfrac{R}{N})^N - 1 \right]
	\end{align}
	where $N$ = number of compounding periods = 2
	
	Over the same period, the \emph{simple} interest earned - $I_2$ - is equal to
	\begin{align}
		I_2 &= P_0\cdot R\cdot T(=1)
    \end{align}
    
    And therefore, if $I_1-I_2 = \text{\texteuro 180}$, then 
    \begin{align}
    	\text{\texteuro 180} &= P_0\cdot\left[ (1+\dfrac{0.1}{2})^2 - 1 - 0.1\right] \\
    	                     &= 0.0025\cdot P_0 \\
    	\Rightarrow P_0 &= \text{\texteuro 72,000}
    \end{align}
\end{solution}

\newpage
% This is an empty shell file placed for you by the 'examiner' script.
% You can now fill in the TeX for your question here.

% Now, down to brasstacks. ** Writing good solutions is an Art **. 
% Eventually, you will find your own style. But here are some thoughts 
% to get you started: 
%
%   1. Write the solution as if you are writing it for your favorite
%      14-17 year old to help him/her understand. Could be your nephew, 
%      your niece, a cousin perhaps or probably even you when you 
%      were that age. Just write for them.
%
%   2. Use margin-notes to "talk" to students about the critical insights
%      in the question. The tone can be - in fact, should be - informal
%
%   3. Don't shy away from creating margin-figures you think will help
%      students understand. Yes, it is a little more work per question. 
%      But the question & solution will be written only once. Make that
%      attempt at writing a solution count.
%
%   4. At the same time, do not be too verbose. A long solution can
%      - at first sight - make the student think, "God, that is a lot to know".
%      Our aim is not to scare students. Rather, our aim should be to 
%      create many "Aha!" moments everyday in classrooms around the world
% 
%   5. Ensure that there are *no spelling mistakes anywhere*. We are an 
%      education company. Bad spellings suggest that we ourselves 
%      don't have any education. And, use American spellings

\question[4] Commander Spock - of Star-Trek fame - has some \textit{Vulcan} dollars that he
would like to invest for a period of one Vulcan year. He has a choice of 2 banks - an Earth-based
bank and a bank on his home planet Vulcan. The Earth bank offers compounding at the
rate of 3\% every 6 \textit{Earth} months. The Vulcan bank - on the other hand - offers 6\% compounding
every 6 \textit{Vulcan} months. Which bank should Commander Spock put his money in given that
1 Vulcan year = 2 Earth years

\ifprintanswers
  % stuff to be shown only in the answer key - like explanatory margin figures
  \marginnote[-0.5cm]{Note that the \textit{nominal} interest rate offered by banks 
  is the same - 12\% per Vulcan year}
  \marginnote[0.5cm] {But in order to do a comparison, one should consider only how 
  often the compounding happens and at what rate. In other words, one must consider
  only the \textit{effective} interest rate}
\fi 

\begin{solution}[\fullpage]
	The number of times the Earth-based bank would compound the amount in one \textit{Vulcan}
	year is 
	\begin{align}
		&= \SI{1}{vulcan-year}\times\SI{2}{earth-years\per vulcan-year}\times
		\dfrac{\SI{1}{time}}{\frac{1}{2}\text{earth-year}} \\
		&= 4\text{ times}
	\end{align}
	
	Similarly, the bank on Vulcan will compound the amount
	\begin{align}
		&= \SI{1}{vulcan-year}\times\dfrac{\SI{1}{time}}{\frac{1}{2}\text{vulcan-year}} \\
		&= 2\text{ times}
	\end{align}
	And therefore, if $P_0$ be the amount Commander Spock puts in initially, 
	$P_{\texttt{earth}}$ be the amount he realizes at the end of his investment period with the bank
	on Earth and $P_{\texttt{vulcan}}$ the amount he realizes on Vulcan, then 
	
	\begin{align}
		\dfrac{P_\texttt{earth}}{P_0} &= \left( 1 + \dfrac{3}{100}\right)^4 \\
		                       &= 1.1255 \\
		\dfrac{P_\texttt{vulcan}}{P_0} &= \left( 1 + \dfrac{6}{100}\right)^2 \\
		                        &= 1.1236
	\end{align}
	Clearly, Commander Spock is better off investing his
	money with the bank on Earth
\end{solution}

\newpage
% This is an empty shell file placed for you by the 'examiner' script.
% You can now fill in the TeX for your question here.

% Now, down to brasstacks. ** Writing good solutions is an Art **. 
% Eventually, you will find your own style. But here are some thoughts 
% to get you started: 
%
%   1. Write the solution as if you are writing it for your favorite
%      14-17 year old to help him/her understand. Could be your nephew, 
%      your niece, a cousin perhaps or probably even you when you 
%      were that age. Just write for them.
%
%   2. Use margin-notes to "talk" to students about the critical insights
%      in the question. The tone can be - in fact, should be - informal
%
%   3. Don't shy away from creating margin-figures you think will help
%      students understand. Yes, it is a little more work per question. 
%      But the question & solution will be written only once. Make that
%      attempt at writing a solution count.
%
%   4. At the same time, do not be too verbose. A long solution can
%      - at first sight - make the student think, "God, that is a lot to know".
%      Our aim is not to scare students. Rather, our aim should be to 
%      create many "Aha!" moments everyday in classrooms around the world
% 
%   5. Ensure that there are *no spelling mistakes anywhere*. We are an 
%      education company. Bad spellings suggest that we ourselves 
%      don't have any education. And, use American spellings

\question[6] Shown alongside is a circle centered at $O$ having radius = $4\sqrt2$.
The two chords - $AB$ and $CD$ - intersect at $X$. If $X$ divides $AB$ in the ratio $2:1$
and $CD$ in the ratio $3:1$, then what is $\angle OCD$ given that length of $AB = 3\sqrt3$ units?


\begin{marginfigure}
	\figinit{pt}
    \figpt 0: (60,0)
    \figptcirc 1:$A$: 0;50(40)
    \figptcirc 2:$B$: 0;50(140)
    \figptcirc 3:$C$: 0;50(200)
    \figptcirc 4:$D$: 0;50(100)
    \figvectP 10 [1,2]
    \figvectP 20 [3,4]
    \figptinterlines 5:$X$[1,10;3,20]
    \figptorthoprojline 6:$M$= 0/3,4/
	\figdrawbegin{}
		\figdrawcirc 0(50)
    \figdrawline [1,2]
    \figdrawline [3,4]
    \figdrawarccircP 3;10 [0,4]
    \figset (dash=7)
    \figdrawline [0,3]
    \ifprintanswers
      \figdrawline [0,6]
    \fi
	\figdrawend
  \figvisu{\figBoxA}{}{
    \figsetmark{$\bullet$}
    \figwrites 0:$O$(5)
    \figwriten 1:(5)
    \figwritew 2:(5)
    \figwrites 3:(5)
    \figwriten 4:(5)
    \ifprintanswers
      \figwritese 5:(5)
      \figwritew 6:(5)
    \fi
  }
  \centerline{\box\figBoxA}
\end{marginfigure}

\ifprintanswers
  % stuff to be shown only in the answer key - like explanatory margin figures
  \marginnote[0.5cm]{Whether $\frac{AX}{BX} = \frac{2}{1}$ or $\frac{BX}{AX} = \frac{2}{1}$ is 
  unimportant. We are only interested in the total length of the chord}
  \marginnote[0.5cm]{Same goes for chord $CD$ }
\fi 

\begin{solution}[\fullpage]
	Here is what we know about intersecting chords in a circle,
	\begin{align}
		AX\cdot BX &= CX\cdot DX
	\end{align}
	Assuming that $\dfrac{AX}{BX} = \dfrac{2}{1}$ and $\dfrac{CX}{DX} = \dfrac{3}{1}$, we get
	\begin{align}
		\dfrac{2}{3}L_{AB}\cdot\dfrac{1}{3}L_{AB} &= \dfrac{3}{4}L_{CD}\cdot\dfrac{1}{4}L_{CD} \\
		\Rightarrow \dfrac{2}{9}\cdot(3\sqrt{3})^2 &= \dfrac{3}{16}\cdot L_{CD}^2 \\
		\Rightarrow L_{CD} &= 4\sqrt{2}
	\end{align}
	
	Now, if we drop a perpendicular from $O$ on $CD$ at $M$, then we know that 
	$CM = MD = \dfrac{4\sqrt2}{2} = 2\sqrt{2}$
	
	And therefore, in $\triangle OCM$, 
	\begin{align}
		\cos\angle OCM &= \dfrac{CM}{OC} \\
		               &= \dfrac{2\sqrt{2}}{4\sqrt{2}} = \dfrac{1}{2} \\
		\Rightarrow \angle OCM &= \cos^{-1}\dfrac{1}{2} = \ang{60}
	\end{align}
\end{solution}

