% This is an empty shell file placed for you by the 'examiner' script.
% You can now fill in the TeX for your question here.

% Now, down to brasstacks. ** Writing good solutions is an Art **. 
% Eventually, you will find your own style. But here are some thoughts 
% to get you started: 
%
%   1. Write to be understood - but be crisp. Your own solution should not take 
%      more space than you will give to the student. Hence, if you take more than 
%      a half-page to write a solution, then give the student a full-page and so on...
%
%   2. Use margin-notes to "talk" to students about the critical insights
%      in the question. The tone can be - in fact, should be - informal
%
%   3. Don't shy away from creating margin-figures you think will help
%      students understand. Yes, it is a little more work per question. 
%      But the question & solution will be written only once. Make that
%      attempt at writing a solution count.
%      
%      3b. Use bc_to_fig.tex. Its an easier way to generate plots & graphs 
% 
%   4. Ensure that there are *no spelling mistakes anywhere*. We are an 
%      education company. Bad spellings suggest that we ourselves 
%      don't have any education. Also, use American spellings by default
% 
%   5. If a question has multiple parts, then first delete lines 40-41
%   6. If a question does not have parts, then first delete lines 43-69
%   
%   7. Create versions of the question when possible. Use commands defined in 
%      tufte-tweaks.sty to do so. Its easier than you think

% \noprintanswers
% \setcounter{rolldice}{0}

\ifnumequal{\value{rolldice}}{0}{
  % variables 
  \renewcommand{\vbone}{3}
  \renewcommand{\vbtwo}{3}
  \renewcommand{\vbthree}{120}
  \renewcommand{\vbfour}{21}
  \renewcommand{\vbfive}{99}
  \renewcommand{\vbsix}{7}
  \renewcommand{\vbseven}{}
  \renewcommand{\vbeight}{}
  \renewcommand{\vbnine}{}
  \renewcommand{\vbten}{}
}{
  \ifnumequal{\value{rolldice}}{1}{
    % variables 
    \renewcommand{\vbone}{4}
    \renewcommand{\vbtwo}{4}
    \renewcommand{\vbthree}{5040}
    \renewcommand{\vbfour}{138}
    \renewcommand{\vbfive}{4902}
    \renewcommand{\vbsix}{21}
    \renewcommand{\vbseven}{}
    \renewcommand{\vbeight}{}
    \renewcommand{\vbnine}{}
    \renewcommand{\vbten}{}
  }{
    \ifnumequal{\value{rolldice}}{2}{
      % variables 
      \renewcommand{\vbone}{4}
      \renewcommand{\vbtwo}{3}
      \renewcommand{\vbthree}{720}
      \renewcommand{\vbfour}{93}
      \renewcommand{\vbfive}{627}
      \renewcommand{\vbsix}{12}
      \renewcommand{\vbseven}{}
      \renewcommand{\vbeight}{}
      \renewcommand{\vbnine}{}
      \renewcommand{\vbten}{}
    }{
      % variables 
      \renewcommand{\vbone}{3}
      \renewcommand{\vbtwo}{4}
      \renewcommand{\vbthree}{720}
      \renewcommand{\vbfour}{26}
      \renewcommand{\vbfive}{694}
      \renewcommand{\vbsix}{14}
      \renewcommand{\vbseven}{}
      \renewcommand{\vbeight}{}
      \renewcommand{\vbnine}{}
      \renewcommand{\vbten}{}
    }
  }
}

\question Find the number of ways in which $\vbone$ rings can be worn on
$\vbtwo$ fingers given that -

\insertQR{}

\watchout

\ifprintanswers
  % stuff to be shown only in the answer key - like explanatory margin figures
  \begin{marginfigure}
    \figinit{pt}
      \figpt 100:(0,0)
      \figpt 101:(0,0)
    \figdrawbegin{}
      \figdrawline [100,101]
    \figdrawend
    \figvisu{\figBoxA}{}{%
    }
    \centerline{\box\figBoxA}
  \end{marginfigure}
\fi 

\begin{parts}
  \part All of the rings are identical

  \insertQR{}
  \begin{solution}
  Let us consider cases as shown here. The numbers in the columns
  indicate number of rings on a finger,\\
  \ifnumequal{\value{rolldice}}{0}{
    \begin{tabular}{cccc}
      \textit{Finger}&\textit{Finger}&\textit{Finger}
        &\textit{Permutations}\\
      3&0&0&$\encr{3}{1}$ \\
      2&1&0&$\encr{3}{2}$ \\
      1&1&1&$\encr{3}{3}$ \\
    \end{tabular}\\    
  }{
    \ifnumequal{\value{rolldice}}{1}{
      \begin{tabular}{ccccccccc}
	    \textit{Finger}&\textit{Finger}&\textit{Finger}
	      &\textit{Finger}&\textit{Permutations}\\
	    4&0&0&0&$\encr{4}{1}$ \\
	    3&1&0&0&$\encr{4}{2}$ \\
	    2&2&0&0&$\encr{4}{2}$ \\
	    2&1&1&0&$\encr{4}{3}$ \\
	    1&1&1&1&$\encr{4}{4}$ \\
	  \end{tabular}\\    
	}{
	  \ifnumequal{\value{rolldice}}{2}{
	    \begin{tabular}{cccc}
	      \textit{Finger}&\textit{Finger}&\textit{Finger}
	        &\textit{Permutations}\\
	      4&0&0&$\encr{3}{1}$ \\
	      3&1&0&$\encr{3}{2}$ \\
	      2&2&0&$\encr{3}{2}$ \\
	      2&1&1&$\encr{3}{1}$ \\
	    \end{tabular}\\
      }{
	    \begin{tabular}{ccccc}
	      \textit{Finger}&\textit{Finger}&\textit{Finger}
	        &\textit{Finger}&\textit{Permutations}\\
          3&0&0&0&$\encr{4}{1}$ \\
          2&1&0&0&$\encr{4}{2}$ \\
          1&1&1&0&$\encr{4}{3}$ \\
	    \end{tabular}\\
      }
    }    
  }
  The total number of combinations, which is the sum of combinations 
  for each of these individual cases equals $\vbsix$.
  \end{solution}
  
  \part Each ring is distinct and the order in which it is worn on a 
  finger matters.

  \insertQR{}
  \begin{solution}
  Since the order in which the rings go on the finger matters, the total
  number of permutations would be given by,
  \begin{align}
    (\vbone + \vbtwo -1)! = \vbthree \nonumber
  \end{align}
  \end{solution}

  \part Each ring is distinct but the order in which it goes on the 
  finger does \underline{not} matter.

  \insertQR{}
  \begin{solution}
  To obtain this, from the total permutations in part b), remove the 
  cases where the rings are only re-ordered on a particular finger. We 
  can borrow from the cases created in part a) in order to do this\\
  \ifnumequal{\value{rolldice}}{0}{
    \begin{tabular}{cccc}
      \textit{Finger}&\textit{Finger}&\textit{Finger}
                     &\textit{Permutations}\\
      3&0&0&$(3!-1)\times\encr{3}{1}$ \\
      2&1&0&$(2!-1)\times\encr{3}{2}$ \\
    \end{tabular}\\    
  }{
    \ifnumequal{\value{rolldice}}{1}{
      \begin{tabular}{ccccc}
	    \textit{Finger}&\textit{Finger}&\textit{Finger}
	                   &\textit{Finger}&\textit{Permutations}\\
	    4&0&0&0&$(4!-1)\times\encr{4}{1}$ \\
	    3&1&0&0&$(3!-1)\times\encr{4}{2}$ \\
	    2&2&0&0&$2\times(2!-1)\times\encr{4}{2}$ \\
	    2&1&1&0&$(2!-1)\times\encr{4}{3}$ \\
	  \end{tabular}\\
    }{
      \ifnumequal{\value{rolldice}}{2}{
        \begin{tabular}{cccc}
          \textit{Finger}&\textit{Finger}&\textit{Finger}
                         &\textit{Permutations}\\
	      4&0&0&$(4!-1)\times\encr{3}{1}$ \\
	      3&1&0&$(3!-1)\times\encr{3}{2}$ \\
	      2&2&0&$2\times(2!-1)\times\encr{3}{2}$ \\
	      2&1&1&$(2!-1)\times\encr{3}{1}$ \\
	    \end{tabular}\\  
      }{
        \begin{tabular}{ccccc}
          \textit{Finger}&\textit{Finger}&\textit{Finger}
	                     &\textit{Finger}&\textit{Permutations}\\
	      3&0&0&0&$(3!-1)\times\encr{4}{1}$ \\
	      2&1&0&0&$(2!-1)\times\encr{4}{2}$ \\
	    \end{tabular}\\  
      }
    }
  }
  Adding all these and subtracting from the result in part b) we get,
  \begin{align}  
    \vbthree - \vbfour = \vbfive
  \end{align}
  \end{solution}

\end{parts}
