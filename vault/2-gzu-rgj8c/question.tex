% This is an empty shell file placed for you by the 'examiner' script.
% You can now fill in the TeX for your question here.

% Now, down to brasstacks. ** Writing good solutions is an Art **. 
% Eventually, you will find your own style. But here are some thoughts 
% to get you started: 
%
%   1. Write the solution as if you are writing it for your favorite
%      14-17 year old to help him/her understand. Could be your nephew, 
%      your niece, a cousin perhaps or probably even you when you 
%      were that age. Just write for them.
%
%   2. Use margin-notes to "talk" to students about the critical insights
%      in the question. The tone can be - in fact, should be - informal
%
%   3. Don't shy away from creating margin-figures you think will help
%      students understand. Yes, it is a little more work per question. 
%      But the question & solution will be written only once. Make that
%      attempt at writing a solution count.
%
%   4. At the same time, do not be too verbose. A long solution can
%      - at first sight - make the student think, "God, that is a lot to know".
%      Our aim is not to scare students. Rather, our aim should be to 
%      create many "Aha!" moments everyday in classrooms around the world
% 
%   5. Ensure that there are *no spelling mistakes anywhere*. We are an 
%      education company. Bad spellings suggest that we ourselves 
%      don't have any education. And, use American spellings

\question[2]  Adya needs Rs.4,00,000 in in order to pay for travel expenses in order to play in tournaments overseas. She found a bank that pays 5\% interest compounded quarterly. How much does she need to raise in sponsorships so that she can participate in a tournament 2 years down the road.
\insertQR{QRC}

\ifprintanswers
  % stuff to be shown only in the answer key - like explanatory margin figures
  \marginnote[3cm] {Since interest is compounded quarterly there are 4 cycles per year and 8 overall}
\fi 

\begin{solution}[\halfpage]

	Say the amount Adya needs to raise is Rs.A,
	\begin{align}
			   400000 &= A\left(1+\dfrac{5}{100}\right)^8 \\
		4 \times 10^5 &= A\left(\dfrac{105}{100}\right)^8 \\
		   \log 4 + 5 &= \log A + 8\left(\log 21 -\log 20\right) \\
		       	 5.43 &= \log A \\
		    	 	A &= \text{Rs.}2,69,153
	\end{align}

\end{solution}
