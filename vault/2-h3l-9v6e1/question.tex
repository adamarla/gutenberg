% This is an empty shell file placed for you by the 'examiner' script.
% You can now fill in the TeX for your question here.

% Now, down to brasstacks. ** Writing good solutions is an Art **. 
% Eventually, you will find your own style. But here are some thoughts 
% to get you started: 
%
%   1. Write the solution as if you are writing it for your favorite
%      14-17 year old to help him/her understand. Could be your nephew, 
%      your niece, a cousin perhaps or probably even you when you 
%      were that age. Just write for them.
%
%   2. Use margin-notes to "talk" to students about the critical insights
%      in the question. The tone can be - in fact, should be - informal
%
%   3. Don't shy away from creating margin-figures you think will help
%      students understand. Yes, it is a little more work per question. 
%      But the question & solution will be written only once. Make that
%      attempt at writing a solution count.
%
%   4. At the same time, do not be too verbose. A long solution can
%      - at first sight - make the student think, "God, that is a lot to know".
%      Our aim is not to scare students. Rather, our aim should be to 
%      create many "Aha!" moments everyday in classrooms around the world
% 
%   5. Ensure that there are *no spelling mistakes anywhere*. We are an 
%      education company. Bad spellings suggest that we ourselves 
%      don't have any education. Also, use American spellings by default
% 
%   6. If a question has multiple parts, then first delete lines 40-41
%   7. If a question does not have parts, then first delete lines 43-69

\question How many ways can $4$ boys and $4$ girls sit alternating around a circular table (Boy-Girl-Boy...)?

\insertQR{}

\ifprintanswers
  % stuff to be shown only in the answer key - like explanatory margin figures
\fi 

\begin{solution}
  The only way to satisfy this condition is to seat the girls (or boys) in four alternating seats leaving a gap between each one, and then seating the others in those gaps. \\
  Number of ways we can seat the $4$ girls on four seats around the circular table is,
  \begin{align}
    N_{girls} = 3!
  \end{align}
  Now the table is no longer without reference as each spot on the table has a unique position. Therefore, number of ways we can seat the four boys in these four spots is,
  \begin{align}
    N_{boys} = 4!
  \end{align}

  Therefore total number of ways $N$, in which they can all be arranged is,
    \begin{align}
      N &= N_{girls} \times N_{boys} \nonumber \\
        &= 3! \times 4! \nonumber \\
        &= 144
    \end{align}

\end{solution}

