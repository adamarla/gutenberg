% This is an empty shell file placed for you by the 'examiner' script.
% You can now fill in the TeX for your question here.

% Now, down to brasstacks. ** Writing good solutions is an Art **. 
% Eventually, you will find your own style. But here are some thoughts 
% to get you started: 
%
%   1. Write the solution as if you are writing it for your favorite
%      14-17 year old to help him/her understand. Could be your nephew, 
%      your niece, a cousin perhaps or probably even you when you 
%      were that age. Just write for them.
%
%   2. Use margin-notes to "talk" to students about the critical insights
%      in the question. The tone can be - in fact, should be - informal
%
%   3. Don't shy away from creating margin-figures you think will help
%      students understand. Yes, it is a little more work per question. 
%      But the question & solution will be written only once. Make that
%      attempt at writing a solution count.
%
%   4. At the same time, do not be too verbose. A long solution can
%      - at first sight - make the student think, "God, that is a lot to know".
%      Our aim is not to scare students. Rather, our aim should be to 
%      create many "Aha!" moments everyday in classrooms around the world
% 
%   5. Ensure that there are *no spelling mistakes anywhere*. We are an 
%      education company. Bad spellings suggest that we ourselves 
%      don't have any education. Also, use American spellings by default
% 
%   6. If a question has multiple parts, then first delete lines 40-41
%   7. If a question does not have parts, then first delete lines 43-69


\question In the adjoining figure, $R$ is the region in the first quadrant bounded by the graph of  $y =4\ln (3-x)$, the horizontal line $y=6$, and the vertical line $x=2$. Answer the following questions based on this information. \textit{(hint $\ln 3$ = 1.0986)}

\begin{marginfigure}
\figinit{pt}
%Origin
\figpt 1:$O$ (0,0)
%Straight line
\figpt 2: (0,60)
\figpt 3: (20,0)
\figpt 4:(20,60)
%axes
\figpt 10: (100,0)
\figpt 11: (0,100)
%region
\figpt 14: (10,35)
%draw
\figdrawbegin{}
%\figdrawcurve [5,6,7,8]
\figdrawlineC(
0 43.99999,
.68965 43.06855,
1.37931 42.11492,
2.06896 41.13803,
2.75862 40.13671,
3.44827 39.10972,
4.13793 38.05570,
4.82758 36.97318,
5.51724 35.86060,
6.20689 34.71622,
6.89655 33.53818,
7.58620 32.32443,
8.27586 31.07275,
8.96551 29.78069,
9.65517 28.44555,
10.34482 27.06436,
11.03448 25.63383,
11.72413 24.15031,
12.41379 22.60971,
13.10344 21.00748,
13.79310 19.33846,
14.48275 17.59686,
15.17241 15.77607,
15.86206 13.86854,
16.55172 11.86560,
17.24137 9.75719,
17.93103 7.53159,
18.62068 5.17499,
19.31034 2.67102,
20.00000 0
)
\figdrawline [2,4]
\figdrawline [3,4]
\figdrawaxes 1(-20,100, -20,100)
\figdrawend
%write
\figvisu{\figBoxA}{Figure}{
\figwritesw 1:(3)
\figwritew 2:$[0, 6]$ (3)
\figwrites 3:$[2, 0]$ (3)
\figwritee 10:$x$(3)
\figwriten 11:$y$(3)
\figwritene 14:$R$(0)
}
\centerline{\box\figBoxA}
\end{marginfigure}


\begin{parts}
  \part[3] Find the area of $R$
  \insertQR{QRC}
\begin{solution}[\fullpage]

    Let area of region $R$ be $A$.

    \begin{align}
      \text{A}  &= \int_0^2 (6-4\ln(3-x)) \ud x \\
    \end{align}

    Using integration by parts technique,

    \begin{align}
      f(x)g(x) = \int f'(x)g(x) + \int f(x)g'(x) \nonumber
    \end{align}

    For $\ln(3-x)$ term we get, \\

    \begin{align}
      I &= \int \ln(3-x) \ud x \\
        &= \ln(3-x) x - \int \dfrac{-x}{(3-x)} \ud x \\
        &= (3-x)(1-\ln \left(3-x\right))
    \end{align}

    Combining the results we get,

    \begin{align}
      \text{A} &= \left[6x - 4\left(3-x\right)\left(1-\ln\left(3-x\right)\right)\right]_0^2 \\ 
               &= 12 - 4 (3\ln 3 -2) \\
	       &= 6.8167
    \end{align}

  \end{solution}

\newpage
  \part[3] Find the volume of the solid generated when $R$ is revolved about the horizontal line $y=8$.
  \insertQR{QRC}
\begin{solution}[\halfpage]
    To find the volume of the solid generated let us first consider the volume of a single disk of width $dx$ at a distance $x$ from the $y$ axis.
    \begin{align}
      \ud v &= \left(\pi (8 - 4\ln(3-x))^2 - \pi (8 - 6)^2\right) \ud x \\ 
    \end{align}
    Adding up the volume of all such disks from $x=0$ to $x=2$, we get,
    \begin{align}
      \text{V} &= \int_0^2 \left(\pi (8 - 4\ln(3-x)^2)^2 - \pi (8 - 6)^2\right) \ud x \\
      \text{V} &= \pi \int_0^2 \left((8 - 4\ln(3-x))^2 - (2)^2\right) \ud x 
               &= 168.179
    \end{align}
  \end{solution}

  \part[3] The region $R$ is the base of a solid. For this solid, each cross section perpendicular to the $x-axis$ is a square. Find the volume of the solid.
  \insertQR{QRC}
\begin{solution}[\halfpage]
    If we divide the solid into many little square discs perpendicular to the $x-axis$ of width $dx$, the volume of each little disk would be,
    \begin{align}
      \ud v &= \left(6 - 4\ln(3-x)\right)^2 \ud x
    \end{align}    
    Therefore, volume of the entire solid can be obtained by integrating each such disk from $x=0$ to $x=2$ would be, 
    \begin{align}
      \text{V} &= \int_0^2 \left(6 - 4\ln(3-x)\right)^2 \ud x \\
               &= 26.266
    \end{align}
  \end{solution}
\end{parts}
