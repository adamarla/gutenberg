% This is an empty shell file placed for you by the 'examiner' script.
% You can now fill in the TeX for your question here.

% Now, down to brasstacks. ** Writing good solutions is an Art **. 
% Eventually, you will find your own style. But here are some thoughts 
% to get you started: 
%
%   1. Write to be understood - but be crisp. Your own solution should not take 
%      more space than you will give to the student. Hence, if you take more than 
%      a half-page to write a solution, then give the student a full-page and so on...
%
%   2. Use margin-notes to "talk" to students about the critical insights
%      in the question. The tone can be - in fact, should be - informal
%
%   3. Don't shy away from creating margin-figures you think will help
%      students understand. Yes, it is a little more work per question. 
%      But the question & solution will be written only once. Make that
%      attempt at writing a solution count.
%      
%      3b. Use bc_to_fig.tex. Its an easier way to generate plots & graphs 
% 
%   4. Ensure that there are *no spelling mistakes anywhere*. We are an 
%      education company. Bad spellings suggest that we ourselves 
%      don't have any education. Also, use American spellings by default
% 
%   5. If a question has multiple parts, then first delete lines 40-41
%   6. If a question does not have parts, then first delete lines 43-69
%   
%   7. Create versions of the question when possible. Use commands defined in 
%      tufte-tweaks.sty to do so. Its easier than you think

% \noprintanswers
% \setcounter{rolldice}{3}

\ifnumequal{\value{rolldice}}{0}{
  % variables 
  \renewcommand{\vbone}{1} % x 
  \renewcommand{\vbtwo}{3} % y = 3x
  \renewcommand{\vbthree}{3} % p
  \renewcommand{\vbfour}{4} % q 
  \renewcommand{\vbfive}{5} % d 
  \renewcommand{\vbeight}{1}
}{
  \ifnumequal{\value{rolldice}}{1}{
    % variables 
    \renewcommand{\vbone}{2}
    \renewcommand{\vbtwo}{6}
    \renewcommand{\vbthree}{5}
    \renewcommand{\vbfour}{12}
    \renewcommand{\vbfive}{26}
    \renewcommand{\vbeight}{2}
    }{
    \ifnumequal{\value{rolldice}}{2}{
      % variables 
      \renewcommand{\vbone}{2}
      \renewcommand{\vbtwo}{6}
      \renewcommand{\vbthree}{4}
      \renewcommand{\vbfour}{3}
      \renewcommand{\vbfive}{10}
      \renewcommand{\vbeight}{2}
    }{
      % variables 
      \renewcommand{\vbone}{1}
      \renewcommand{\vbtwo}{3}
      \renewcommand{\vbthree}{12}
      \renewcommand{\vbfour}{5}
      \renewcommand{\vbfive}{13}
      \renewcommand{\vbeight}{1}
    }
  }
}

\renewcommand{\vbsix}{\dfrac{\vbthree}{\vbfour}}
\gcalcexpr[0]{\vbseven}{(3*\vbfour - \vbthree)*\vbone}
\gcalcexpr[0]{\vbnine}{(\vbfive / \vbeight) + \vbone}
\gcalcexpr[0]{\vbten}{\vbone - (\vbfive / \vbeight)}

\question[4] If a line that passes through a point $A = (\vbone, \vbtwo)$ has slope equal to $\vbsix$, then 
find the points on it that are $\vbfive$ units away from $A$

\insertQR{QRC}

\watchout

\ifprintanswers
\fi 

\begin{solution}[\halfpage]
	The equation of the given line would be
	\begin{align}
		\dfrac{y-\vbtwo}{x-\vbone} &= \vbsix \Rightarrow y = \vbsix\cdot x + \dfrac{\vbseven}{\vbfour}
	\end{align}
	Moreover, any point that is $\vbfive$ units away from $A$ would satisy 
	the following condition 
	\begin{align}
		(x - \vbone)^2 + (y-\vbtwo)^2 &= (\vbfive)^2 
	\end{align}
	And if this point also happens to be on the line, then 
	\begin{align}
		(x-\vbone)^2 + \left[ \left( \vbsix\cdot x + \dfrac{\vbseven}{\vbfour}\right) - \vbtwo \right]^2 &= \vbfive^2 \\
		\Rightarrow (x - \vbone)^2\cdot\left[ 1 + \left( \vbsix \right)^2\right] &= \vbfive^2 \\
		\Rightarrow x - \vbone &= \pm\dfrac{\vbfive}{\vbeight} \\
		\text{ or } x &= \vbnine, \vbten
	\end{align}
	
	\gcalcexpr{\vbfive}{\vbthree / \vbfour} % slope 
	\gcalcexpr[2]{\vbtwo}{(\vbfive * \vbnine) + (3-\vbfive)*\vbone} % y_1
    \gcalcexpr[2]{\vbthree}{(\vbfive * \vbten) + (3-\vbfive)*\vbone} % y_2
    
    Plug these values of $x$ into the equation for the line - $(1)$ - and you will 
    get the required coordinates -  $(\vbnine, \vbtwo)$ and $(\vbten, \vbthree)$
\end{solution}
