% This is an empty shell file placed for you by the 'examiner' script.
% You can now fill in the TeX for your question here.

% Now, down to brasstacks. ** Writing good solutions is an Art **. 
% Eventually, you will find your own style. But here are some thoughts 
% to get you started: 
%
%   1. Write to be understood - but be crisp. Your own solution should not take 
%      more space than you will give to the student. Hence, if you take more than 
%      a half-page to write a solution, then give the student a full-page and so on...
%
%   2. Use margin-notes to "talk" to students about the critical insights
%      in the question. The tone can be - in fact, should be - informal
%
%   3. Don't shy away from creating margin-figures you think will help
%      students understand. Yes, it is a little more work per question. 
%      But the question & solution will be written only once. Make that
%      attempt at writing a solution count.
%      
%      3b. Use bc_to_fig.tex. Its an easier way to generate plots & graphs 
% 
%   4. Ensure that there are *no spelling mistakes anywhere*. We are an 
%      education company. Bad spellings suggest that we ourselves 
%      don't have any education. Also, use American spellings by default
% 
%   5. If a question has multiple parts, then first delete lines 40-41
%   6. If a question does not have parts, then first delete lines 43-69
%   
%   7. Create versions of the question when possible. Use commands defined in 
%      tufte-tweaks.sty to do so. Its easier than you think

%\noprintanswers
%\setcounter{rolldice}{1}
%\printrubric

\ifnumequal{\value{rolldice}}{0}{
  % variables 
  \renewcommand{\vbone}{4}
  \renewcommand{\vbtwo}{5}
  \renewcommand{\vbthree}{3}
  \renewcommand{\vbfour}{-3} % 1st soln
  \renewcommand{\vbfive}{5} % 2nd soln
}{
  \ifnumequal{\value{rolldice}}{1}{
    % variables 
    \renewcommand{\vbone}{3}
    \renewcommand{\vbtwo}{6}
    \renewcommand{\vbthree}{2}
    \renewcommand{\vbfour}{-4}
    \renewcommand{\vbfive}{3}
  }{
    \ifnumequal{\value{rolldice}}{2}{
      % variables 
      \renewcommand{\vbone}{5}
      \renewcommand{\vbtwo}{2}
      \renewcommand{\vbthree}{4}
      \renewcommand{\vbfour}{-1}
      \renewcommand{\vbfive}{8}
    }{
      % variables 
      \renewcommand{\vbone}{6}
      \renewcommand{\vbtwo}{4}
      \renewcommand{\vbthree}{4}
      \renewcommand{\vbfour}{-2}
      \renewcommand{\vbfive}{8}
    }
  }
}

\question Find the equation of the line - or lines - that pass through $A = (\vbone, -\vbtwo)$ 
and for which $ON - OM = \vbthree$ - \asif

\insertQR{}

\watchout[-10pt]

\begin{marginfigure}
  % 1. Definition of characteristic points
\figinit{pt}
\def\Xmin{-16.00000}
\def\Ymin{-31.99999}
\def\Xmax{64.00000}
\def\Ymax{48.00000}
\def\Xori{16.00000}
\def\Yori{31.99999}
\figpt 100: $M$(48,31)
\figpt 101: $N$(15,64)
\figpt 102 :$A$(61,19)
\figpt0:(\Xori,\Yori)
% 2. Creation of the graphical file
\figdrawbegin{}
\def\Xmaxx{\Xmax} % To customize the position
\def\Ymaxx{\Ymax} % of the arrow-heads of the axes.
\figset arrowhead(length=4, fillmode=yes) % styling the arrowheads
\figdrawaxes 0(\Xmin, \Xmaxx, \Ymin, \Ymaxx)
\figdrawlineC(
0 79.99999, % y = 4.50
2.75862 77.24137, % y = 4.24
5.51724 74.48275, % y = 3.98
8.27586 71.72413, % y = 3.72
11.03448 68.96551, % y = 3.46
13.79310 66.20689, % y = 3.20
16.55172 63.44827, % y = 2.94
19.31034 60.68965, % y = 2.68
22.06896 57.93103, % y = 2.43
24.82758 55.17241, % y = 2.17
27.58620 52.41379, % y = 1.91
30.34482 49.65517, % y = 1.65
33.10344 46.89655, % y = 1.39
35.86206 44.13793, % y = 1.13
38.62068 41.37931, % y = .87
41.37931 38.62068, % y = .62
44.13793 35.86206, % y = .36
46.89655 33.10344, % y = .10
49.65517 30.34482, % y = -.15
52.41379 27.58620, % y = -.41
55.17241 24.82758, % y = -.67
57.93103 22.06896, % y = -.93
60.68965 19.31034, % y = -1.18
63.44827 16.55172, % y = -1.44
66.20689 13.79310, % y = -1.70
68.96551 11.03448, % y = -1.96
71.72413 8.27586, % y = -2.22
74.48275 5.51724, % y = -2.48
77.24137 2.75862, % y = -2.74
79.99999 0
)
\figdrawend
% 3. Writing text on the figure
\figvisu{\figBoxA}{}{%
\figptsaxes 1:0(\Xmin, \Xmaxx, \Ymin, \Ymaxx)
% Points 1 and 2 are the end points of the arrows
\figwritee 1:(5pt)     \figwriten 2:(5pt)
\figptsaxes 1:0(\Xmin, \Xmax, \Ymin, \Ymax)
\figset write(mark = $\bullet$)
\figwritesw 100:(1)
\figwritene 101:(2)
\figwritene 102:(2)
\figwritesw 0:$O$(2)
}
\centerline{\box\figBoxA}

\end{marginfigure} 

\gcalcexpr[0]\tp{\vbone - \vbtwo + \vbthree} 
\gcalcexpr[0]\tq{\vbtwo * \vbthree}
\gcalcexpr[0]\tr{\vbfour - \vbthree} % L1 x-intercept
\gcalcexpr[0]\ts{\vbfive - \vbthree} % L2 x-intercept

\begin{solution}
	If $ON = n$, then $OM = n - \vbthree$, and 
	\begin{align}
		\dfrac{x}{n-\vbthree} + \dfrac{y}{n} &= 1
	\end{align}
	And as the line passes through $A = (\vbone, -\vbtwo)$, 
	\begin{align}
		\dfrac{\vbone}{n-\vbthree}-\dfrac{\vbtwo}{n} &= 1 \\
		\Rightarrow n^2 - \tp n - \tq &= 0 \\
		\Rightarrow n &= \vbfour,\, \vbfive
	\end{align}
	There would be two lines and their equations would be 
	\begin{align}
		\dfrac{x}{\tr} + \dfrac{y}{\vbfour} &= 1 \\
		\dfrac{x}{\ts} + \dfrac{y}{\vbfive} &= 1
	\end{align}
\end{solution}

\ifprintrubric
  \begin{table}	
  	\begin{tabular}{ p{5cm}p{5cm} }
  		\toprule % in brief (4-6 words), what should a grader be looking for for insights & formulations
  		  \sc{\textcolor{blue}{Insight}} & \sc{\textcolor{blue}{Formulation}} \\ 
  		\midrule % ***** Insights & formulations ******
        There are two valid lines & Used slope-intercept form \\ 
  		\toprule % final numerical answers for the various versions
        \sc{\textcolor{blue}{If question has $\ldots$}} & \sc{\textcolor{blue}{Final answer}} \\
  		\midrule % ***** Numerical answers (below) **********
  			$A=(4,-5)$ & $ x + 2y + 6 = 0$ and $5x + 2y = 10$ \\
  			$A=(5,-2)$ & $x + 5y + 5 = 0$ and $2x + y = 8$ \\
  			$A=(6,-4)$ & $x + 3y + 6 = 0$ and $2x + y = 8$ \\
  			$A=(3,-6)$ & $2x + 3y + 12 = 0$ and $3x + y = 3$ \\
  		\bottomrule
  	\end{tabular}
  \end{table}
\fi
