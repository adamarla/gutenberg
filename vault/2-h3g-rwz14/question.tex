% This is an empty shell file placed for you by the 'examiner' script.
% You can now fill in the TeX for your question here.

% Now, down to brasstacks. ** Writing good solutions is an Art **. 
% Eventually, you will find your own style. But here are some thoughts 
% to get you started: 
%
%   1. Write the solution as if you are writing it for your favorite
%      14-17 year old to help him/her understand. Could be your nephew, 
%      your niece, a cousin perhaps or probably even you when you 
%      were that age. Just write for them.
%
%   2. Use margin-notes to "talk" to students about the critical insights
%      in the question. The tone can be - in fact, should be - informal
%
%   3. Don't shy away from creating margin-figures you think will help
%      students understand. Yes, it is a little more work per question. 
%      But the question & solution will be written only once. Make that
%      attempt at writing a solution count.
%
%   4. At the same time, do not be too verbose. A long solution can
%      - at first sight - make the student think, "God, that is a lot to know".
%      Our aim is not to scare students. Rather, our aim should be to 
%      create many "Aha!" moments everyday in classrooms around the world
% 
%   5. Ensure that there are *no spelling mistakes anywhere*. We are an 
%      education company. Bad spellings suggest that we ourselves 
%      don't have any education. Also, use American spellings by default
% 
%   6. If a question has multiple parts, then first delete lines 40-41
%   7. If a question does not have parts, then first delete lines 43-69

\question A tank of capacity $2400m^3$ is full of water. The rate at which water discharges from the tank is $10m^3/min$ higher than the rate at which it fills. As a result it takes $8min$ more to fill the tank than it takes to empty it. Find the rate at which the tank fills?

\insertQR{}

\ifprintanswers
  % stuff to be shown only in the answer key - like explanatory margin figures
\fi 

\begin{solution}

  Let the rate at which the tank fulls be $r_f(m^3/min)$ and the rate at which it empties be $r_e(m^3/min)$. As per the question,
  \begin{align}
    r_f(m^3/min) = r_e(m^3/min) - 10(m^3/min)
  \end{align}
  
  Time it takes to empty $t_{empty}$ and time to fill $t_{fill}$ will be,
  \begin{align}
    t_{fill(min)}  &= \dfrac{2400(m^3)}{r_f(m^3/min)} \\
    t_{empty(min)} &= \dfrac{2400(m^3)}{r_e(m^3/min)} \\    
  \end{align}
  
  Since it takes $8min$ less to empty than it takes to fill,
  \begin{align}
    t_{fill}(min)                   &= t_{empty}(min)+8(min) \\
    \dfrac{2400(m^3)}{r_f(m^3/min)} &= 
    	\dfrac{2400(m^3)}{r_e(m^3/min)}+8(min) \\
    \dfrac{2400}{r_f}               &= \dfrac{2400}{r_e}+8 \\
    \dfrac{2400}{r_f}               &= \dfrac{2400}{r_f+10}+8 \\
    r_f^2 + 10r_f - 3000            &= 0 \\
    r_f                             &= 50, -60
  \end{align}
  Negative value is rejected, rate at which the tank fills is $50(m^3/min)$.
\end{solution}

