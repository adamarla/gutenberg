
 %\setcounter{rolldice}{1}

\renewcommand{\vbthree}{(\lambda + 2)^2 + 40}
\ifnumequal{\value{rolldice}}{0}{
  % variables 
  \renewcommand{\vbone}{1}
  \renewcommand{\vbtwo}{1}
}{
  \ifnumequal{\value{rolldice}}{1}{
    % variables 
    \renewcommand{\vbone}{2}
    \renewcommand{\vbtwo}{3}
  }{
    \ifnumequal{\value{rolldice}}{2}{
      % variables 
      \renewcommand{\vbone}{5}
      \renewcommand{\vbtwo}{4}
    }{
      % variables 
      \renewcommand{\vbone}{3}
      \renewcommand{\vbtwo}{8}
    }
  }
}

\MULTIPLY\vbone{6}\a
\MULTIPLY\vbtwo{2}\b
\SUBTRACT\a\b\c
\ADD\c{2}\d
\MULTIPLY\d{2}\e
\SQUARE\d\f
\SUBTRACT{44}\f\n
\SUBTRACT\e{4}\d

\question[2] The scalar product of the vector $\vec{V} = \hat{i} + \vbone\hat{j} + \vbtwo\hat{k}$ with the 
\textit{unit} vector along the sum of $\vec{A} = \WRITEVEC{2}{4}{-5}$ and $\vec{B} =\WRITEVECGENERAL{\lambda}{2}{3}$ is $1$. 
Find the value of $\lambda$


\watchout

\ifprintanswers
\fi 

\begin{solution}[\mcq]
	\begin{align}
		\vec{A} + \vec{B} &= (\lambda + 2)\hat{i} + 6\hat{j} - 2\hat{k}
	\end{align}
	And therefore, the unit vector along $\vec{A} + \vec{B}$ would be 
	\begin{align}
		= \dfrac{\lambda + 2}{\sqrt{\vbthree}}\hat{i} + \dfrac{6}{\sqrt{\vbthree}}\hat{j} 
		- \dfrac{2}{\sqrt{\vbthree}} \hat{k}
	\end{align}
	Now, if the dot product of $\vec{V}$ and the unit vector is $1$, then 
	\begin{align}
		\dfrac{\lambda + 2}{\sqrt{\vbthree}}\times{1} &+ \dfrac{6}{\sqrt{\vbthree}}\times\vbone
		- \dfrac{2}{\sqrt{\vbthree}}\times\vbtwo = 1 \\
		\Rightarrow (\lambda + 2) + \c &= \sqrt{\vbthree} \\
		\Rightarrow \lambda^{2} + \e\lambda + \f &= \lambda^{2} + 4\lambda + 44 \\
		\Rightarrow \lambda &= \WRITEFRAC\n\d
	\end{align}
\end{solution}

