


\ifnumequal{\value{rolldice}}{0}{
  % variables 
  \renewcommand{\vbone}{9}
  \renewcommand{\vbtwo}{6}
  \renewcommand{\vbthree}{84}
}{
  \ifnumequal{\value{rolldice}}{1}{
    % variables 
    \renewcommand{\vbone}{8}
    \renewcommand{\vbtwo}{4}
    \renewcommand{\vbthree}{70}
  }{
    \ifnumequal{\value{rolldice}}{2}{
      % variables 
      \renewcommand{\vbone}{10}
      \renewcommand{\vbtwo}{7}
      \renewcommand{\vbthree}{120}
    }{
      % variables 
      \renewcommand{\vbone}{7}
      \renewcommand{\vbtwo}{3}
      \renewcommand{\vbthree}{35}
    }
  }
}

\SUBTRACT\vbone\vbtwo\p

\question[3] What is the coefficient of $x^{\vbtwo} y^{\p}$ in the expansion of $(x + y)^{\vbone}$?


\watchout

\ifprintanswers
\fi 

\begin{solution}[\halfpage]
  The general form of a term in the expansion of $(x+y)^{\vbone}$ is 
  \begin{align}
    a_m &= \encr\vbone{m}\cdot x^{m}\cdot y^{\vbone - m}, \text{where } m \in [0,\vbone]
  \end{align}

  What we need is $m=\vbtwo$ so that $x^m = x^{\vbtwo}$. Which means, that 
  \begin{align}
    a_{\vbtwo} &= \encr\vbone\vbtwo\cdot x^{\vbtwo}\cdot y^{\p}
  \end{align}

  And therefore, the required coefficient is simply
  \begin{align}
     = \encr\vbone\vbtwo  &= \fncr\vbone\vbtwo = \vbthree
  \end{align}
\end{solution}

