


\ifnumequal{\value{rolldice}}{0}{
  % variables 
  \renewcommand{\vbone}{5} % required
  \renewcommand{\vbtwo}{9} % total 
  \renewcommand{\vbsix}{91}
}{
  \ifnumequal{\value{rolldice}}{1}{
    % variables 
    \renewcommand{\vbone}{6}
    \renewcommand{\vbtwo}{8}
    \renewcommand{\vbsix}{22}
  }{
    \ifnumequal{\value{rolldice}}{2}{
      % variables 
      \renewcommand{\vbone}{7}
      \renewcommand{\vbtwo}{11}
      \renewcommand{\vbsix}{246}
    }{
      % variables 
      \renewcommand{\vbone}{4}
      \renewcommand{\vbtwo}{7}
      \renewcommand{\vbsix}{25}
    }
  }
}

\gcalcexpr[0]{\vbthree}{\vbtwo - 2}
\gcalcexpr[0]{\vbfour}{\vbtwo - 1}
\gcalcexpr[0]{\vbfive}{\vbone - 2}

\question[2] A committee of $\vbone$ persons needs to be formed from amongst $\vbtwo$ people. However, 
if Mr. Laurel is picked for the committee, then Mr. Hardy must be too. In how many 
ways can the committee be formed?


\watchout[-30pt]

\ifprintanswers
\fi 

\begin{solution}[\mcq]
	If Mr. Laurel is picked - and therefore also Mr. Hardy - then we need to pick $\vbfive$ more individuals 
	from amongst $\vbthree$ persons
	
	And if Mr. Laurel is \textit{not} picked, then we need to pick $\vbone$ individuals from amongst $\vbfour$ individuals. Hence,
	\begin{align}
		N_{\texttt{total}} &= \encr\vbthree\vbfive + \encr\vbfour\vbone \\
		&= \fncr\vbthree\vbfive + \fncr\vbfour\vbone \\
		&= \vbsix
	\end{align}
\end{solution}
