
\ifnumequal{\value{rolldice}}{0}{
  % variables 
  \renewcommand{\va}{13}
  \renewcommand{\vb}{24}
}{
  \ifnumequal{\value{rolldice}}{1}{
    % variables 
    \renewcommand{\va}{19}
    \renewcommand{\vb}{36}
  }{
    \ifnumequal{\value{rolldice}}{2}{
      % variables 
      \renewcommand{\va}{25}
      \renewcommand{\vb}{48}
    }{
      % variables 
      \renewcommand{\va}{31}
      \renewcommand{\vb}{60}
    }
  }
}

\DIVIDE\vb{12}\vc
\MULTIPLY{7}\vc\vd

\question Prove - or disprove - the following proposition using mathematical induction 
\[ S(n) = \sum_{k=1}^n\dfrac{1}{n + k} > \dfrac\va\vb,\,n > 1,\, n\in\mathbb{N} \]

\watchout
\insertQR{}

\begin{solution}
  First, let us see whether the proposition is true for $n=2$.
  \begin{align}
    S(n) &= \dfrac{1}{2+1} + \dfrac{1}{2+2} = \dfrac{7}{12} = \dfrac\vd\vb > \dfrac\va\vb
  \end{align}

  It is and therefore \textbf{let us assume} that it also true for some $n\implies S(n) > \frac\va\vb$. 
  Now, 
  \begin{align}
    S(n+1) &= \sum_{k=1}^{n+1}\dfrac{1}{(n+1) + k} \\ 
           &= \underbrace{\dfrac{1}{n+2} + \dfrac{1}{n+3} + \ldots + \dfrac{1}{2n}}_{= S(n) - \frac{1}{n+1}} 
           + \dfrac{1}{2n+1} + \dfrac{1}{2n + 2 } \\
           &= S(n) + \dfrac{1}{2n+1} + \dfrac{1}{2\cdot (n+1)} - \dfrac{1}{n+1} \\
           &= S(n) + \underbrace{\dfrac{1}{2\cdot(n+1)\cdot(2n+1)}}_{ > 0\text{ for } n\in\mathbb{N}}
  \end{align}

  Hence, if $S(n) > \frac\va\vb$, then so is $S(n+1)$. And as $S(2) > \frac\va\vb$, then so 
  must $S(3), S(4),\ldots$. Hence proved.
\end{solution}

