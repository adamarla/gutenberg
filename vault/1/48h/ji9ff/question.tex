% This is an empty shell file placed for you by the 'examiner' script.
% You can now fill in the TeX for your question here.

% Now, down to brasstacks. ** Writing good solutions is an Art **. 
% Eventually, you will find your own style. But here are some thoughts 
% to get you started: 
%
%   1. Write to be understood - but be crisp. Your own solution should not take 
%      more space than you will give to the student. Hence, if you take more than 
%      a half-page to write a solution, then give the student a full-page and so on...
%
%   2. Use margin-notes to "talk" to students about the critical insights
%      in the question. The tone can be - in fact, should be - informal
%
%   3. Don't shy away from creating margin-figures you think will help
%      students understand. Yes, it is a little more work per question. 
%      But the question & solution will be written only once. Make that
%      attempt at writing a solution count.
%      
%      3b. Use bc_to_fig.tex. Its an easier way to generate plots & graphs 
% 
%   4. Ensure that there are *no spelling mistakes anywhere*. We are an 
%      education company. Bad spellings suggest that we ourselves 
%      don't have any education. Also, use American spellings by default
% 
%   5. If a question has multiple parts, then first delete lines 40-41
%   6. If a question does not have parts, then first delete lines 43-69
%   
%   7. Create versions of the question when possible. Use commands defined in 
%      tufte-tweaks.sty to do so. Its easier than you think

%\printrubric
%\noprintanswers
%\setcounter{rolldice}{2}

\ifnumequal{\value{rolldice}}{0}{
  % variables 
  \renewcommand{\vbone}{2}
  \renewcommand{\vbtwo}{1}
  \renewcommand{\vbthree}{5}
  \renewcommand{\vbfour}{7}
  \renewcommand{\vbseven}{-\sqrt{3}}
  \renewcommand{\vbeight}{120}
}{
  \ifnumequal{\value{rolldice}}{1}{
    % variables 
    \renewcommand{\vbone}{2\sqrt{11}}
    \renewcommand{\vbtwo}{5}
    \renewcommand{\vbthree}{5}
    \renewcommand{\vbfour}{11}
    \renewcommand{\vbseven}{\dfrac{1}{\sqrt{3}}}
    \renewcommand{\vbeight}{30}
  }{
    \ifnumequal{\value{rolldice}}{2}{
      % variables 
      \renewcommand{\vbone}{\sqrt{23}}
      \renewcommand{\vbtwo}{4}
      \renewcommand{\vbthree}{2}
      \renewcommand{\vbfour}{3}
      \renewcommand{\vbseven}{\dfrac{1}{\sqrt{3}}}
      \renewcommand{\vbeight}{30}
    }{
      % variables 
      \renewcommand{\vbone}{\sqrt{7}}
      \renewcommand{\vbtwo}{2}
      \renewcommand{\vbthree}{2}
      \renewcommand{\vbfour}{4}
      \renewcommand{\vbseven}{\sqrt{3}}
      \renewcommand{\vbeight}{60}
      }
  }
}

\renewcommand{\vbfive}{(\vbone - \vbtwo\cdot\sqrt{3})}
\renewcommand{\vbsix}{(\vbone + \vbtwo\cdot\sqrt{3})}
\gcalcexpr[0]{\vbnine}{180-\vbeight}

\question[3] Find the angle at which the two lines $y=\vbfive\cdot(x + \vbthree)$ and 
$y=\vbsix\cdot(x - \vbfour)$ intersect

\insertQR[-15pt]{QRC}

\watchout

\ifprintanswers

\fi 

\begin{solution}[\halfpage]
	The equations of the two lines can be re-written as 
	\begin{align}
		y &= \overbrace{\vbfive\cdot x}^{m_1} + \overbrace{\vbfive\cdot \vbthree}^{\text{not important}} \\
		y &= \underbrace{\vbsix\cdot x}_{m_2} - \vbsix\cdot \vbfour \\
		\Rightarrow & \fAngleOfIntersection{1}{2} = \dfrac{\vbfive - \vbsix}{1 + \vbfive\cdot\vbsix} \\
		&= \vbseven \\ 
		\Rightarrow \theta_1 &= \ang{\vbeight} \text{ or } \ang{\vbnine}
	\end{align}
\end{solution}

\ifprintrubric
  \begin{table}
  	\begin{tabular}{ p{5cm}p{5cm} }
  		\toprule % in brief (4-6 words), what should a grader be looking for for insights & formulations
  		  \sc{\textcolor{blue}{Insight}} & \sc{\textcolor{blue}{Formulation}} \\ 
  		\midrule % ***** Insights & formulations ******
        Rewrote equations of the lines to be of the form $y = mx + C$ & 
        Applied formula for angle between two lines \\
        Recognized that there are two possible angles - $\theta$ and $\pi - \theta$ & \\
  		\toprule % final numerical answers for the various versions
        \sc{\textcolor{blue}{If question has $\ldots$}} & \sc{\textcolor{blue}{Final answer}} \\
  		\midrule % ***** Numerical answers (below) **********
        $(2-\sqrt{3})$ & $\ang{60}$ and $\ang{120}$ \\
        $(2\sqrt{11}-5\sqrt{3})$ & $\ang{30}$ and $\ang{150}$ \\
        $(\sqrt{23}-4\sqrt{3})$ & $\ang{30}$ and $\ang{150}$ \\
        $(\sqrt{7}-2\sqrt{3})$ & $\ang{60}$ and $\ang{120}$ \\
  		\bottomrule
  	\end{tabular}
  \end{table}
\fi
