

\question[5] Three numbers form an arithmetic progression. If we add 8 to the first number,
 we get a geometric progression whose terms add upto 26. Find the numbers in the original 
 arithmetic progression


\ifprintanswers
\fi 

\begin{solution}[\fullpage]
	If the original arithmetic progression is $A = \lbrace a, a + d, a+2d\rbrace$, then the
	resulting geometric progression is $G = \lbrace a + 8, a + d, a + 2d \rbrace$
	\begin{align}
		\Rightarrow a +d &= (a+8)\cdot r \\
		a + 2d &= (a+d)\cdot r = (a + 8)\cdot r^2 
	\end{align}
	Moreover, 
	\begin{align}
		(a+8) + (a+d)+(a+2d) &= 3\cdot(a+d) + 8 = 26 \\
		\Rightarrow (a+d) = 6
	\end{align}
	So, our geometric sequence - $G$ - now is $\frac{6}{r}, 6, 6r$
	\begin{align}
		\Rightarrow 6\cdot\left( \dfrac{1}{r} + 1 + r\right) &= 26 \\
		\text{ or } r + \dfrac{1}{r} &= \dfrac{10}{3} \Rightarrow 3r^2-10r+3 = 0 \\
		\Rightarrow r = 3, \frac{1}{3}
	\end{align}
	
	Which means that the \textit{geometric} progression is either $(18,6,2)$ or $(2,6,18)$.
	And that in turn means that the original \textit{arithmetic} progression is either
	$(18-8,6,2) = (10,6,2)$ or $(2-8,6,18) = (-6,6,18)$
\end{solution}
