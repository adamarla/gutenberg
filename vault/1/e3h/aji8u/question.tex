
\ifnumequal{\value{rolldice}}{0}{
  % variables 
  \renewcommand{\va}{37} % initial angle 
  \renewcommand{\vb}{53} % final angle  
  \renewcommand{\vh}{50} % Height 
}{
  \ifnumequal{\value{rolldice}}{1}{
    % variables 
    \renewcommand{\va}{27}
    \renewcommand{\vb}{43}
    \renewcommand{\vh}{60} % Height 
  }{
    \ifnumequal{\value{rolldice}}{2}{
      % variables 
      \renewcommand{\va}{15}
      \renewcommand{\vb}{35}
      \renewcommand{\vh}{95} % Height 
    }{
      % variables 
      \renewcommand{\va}{22}
      \renewcommand{\vb}{51}
      \renewcommand{\vh}{80} % Height 
    }
  }
}

\DEGREESCOT\va\vc
\DEGREESCOT\vb\vd
\DEGREESTAN\va\vx
\DEGREESTAN\vb\vy

\MULTIPLY\vc\vy\vm %cot A tan B
\MULTIPLY\vd\vx\vn% tan A cot B 

\ROUND[2]\vm\vj
\ROUND[2]\vn\vk
\MULTIPLY\vh\vk\vl
\ROUND[2]\vl\vz

\question[2] A tower and a building stand opposite each other. If the angle of elevation
from the bottom of the tower to the top of the building is $\ang\va$ and the angle
of elevation from the bottom of the building to the top of the tower is $\ang\vb$, then
what is the height of the building given that the tower is $\vh$ meters tall.

\watchout

\begin{calcaid}
  \begin{tabular}{c c}
    $\cot\ang\va\cdot\tan\ang\vb=\vj$ & $\tan\ang\va\cdot\cot\ang\vb=\vk$
  \end{tabular}
\end{calcaid}

\ifprintanswers
  % stuff to be shown only in the answer key - like explanatory margin figures
  \vspace{0.7cm}
  \figinit{pt}
    \figpt 100: $A$(0,0)
    \figpt 101: $B$(100,0)
    \figpt 102: $N$(100,70)
    \figpt 103: $M$(0,45)
    \figpt 105: $H_{\text{tower}} = \vh\text{m}$(100,25)
    \figpt 106: $\ang\va$(77,7)
    \figpt 107: $\ang\vb$(11,7)
  \figdrawbegin{}
    \figdrawline [100,101,102,100]
    \figdrawline [101,103,100]
    \figdrawarccircP 100 ; 12 [101,102] 
    \figdrawarccircP 101 ; 12 [103,100] 
  \figdrawend
  \figvisu{\figBoxA}{}{%
    \fontfamily{cmss}\selectfont\Large
    \figwrites 100:(2)
    \figwrites 101:(2)
    \figwritee 102:(2)
    \figwritew 103:(2)
    \figwritee 105:(4)
    \figwritew 106:(2)
    \figwritee 107:(2)
  }
  \centerline{\box\figBoxA}
\fi 

\begin{solution}[\halfpage]
	The situation is \asif. And one can see that,
	
	\begin{align}
		AB = \dfrac{BN}{\tan\ang\vb} &= \dfrac{AM}{\tan\ang\va} \\
    \implies AM &= BN\cdot\dfrac{\tan\ang\va}{\tan\ang\vb} \\
		\text{or } AM = BN\times\vk &= \vh\times\vk = \vz\text{ meters}
	\end{align}
\end{solution}

\ifprintanswers\begin{codex}$\vz$ meters\end{codex}\fi
