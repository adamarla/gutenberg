



\ifprintanswers
  % stuff to be shown only in the answer key - like explanatory margin figures
  	\begin{tabular}{ccc}
  	   \toprule
  	   Combination & First child & Second child \\
  	   \midrule
  	   1 & Girl & Girl \\
  	   2 & Girl & Boy \\
  	   3 & Boy & Girl \\
  	   4 & Boy & Boy \\
  	   \bottomrule
  	\end{tabular}
\fi 

\begin{parts}
  \part[1] Atleast one child is a girl

\begin{solution}[\mcq]
     Which means that combinations 1,2 and 3 are the only possible outcomes. And of these, 
     only the first combination is a favorable outcome. Therefore, the required probability
     is $\dfrac{1}{3}$
  \end{solution}

  \part[1] The older child is a girl 

\begin{solution}[\mcq]
     If the first child was a girl, then our set of possible outcomes is down to just
     the first two. And therefore, the required probability is $\dfrac{1}{2}$
  \end{solution}

\end{parts}
