% This is an empty shell file placed for you by the 'examiner' script.
% You can now fill in the TeX for your question here.

% Now, down to brasstacks. ** Writing good solutions is an Art **. 
% Eventually, you will find your own style. But here are some thoughts 
% to get you started: 
%
%   1. Write the solution as if you are writing it for your favorite
%      14-17 year old to help him/her understand. Could be your nephew, 
%      your niece, a cousin perhaps or probably even you when you 
%      were that age. Just write for them.
%
%   2. Use margin-notes to "talk" to students about the critical insights
%      in the question. The tone can be - in fact, should be - informal
%
%   3. Don't shy away from creating margin-figures you think will help
%      students understand. Yes, it is a little more work per question. 
%      But the question & solution will be written only once. Make that
%      attempt at writing a solution count.
%
%   4. At the same time, do not be too verbose. A long solution can
%      - at first sight - make the student think, "God, that is a lot to know".
%      Our aim is not to scare students. Rather, our aim should be to 
%      create many "Aha!" moments everyday in classrooms around the world
% 
%   5. Ensure that there are *no spelling mistakes anywhere*. We are an 
%      education company. Bad spellings suggest that we ourselves 
%      don't have any education. Also, use American spellings by default
% 
%   6. If a question has multiple parts, then first delete lines 40-41
%   7. If a question does not have parts, then first delete lines 43-69
\question[3] An 1.5m tall girl is standing some distance from a 30m tall building.
The angle of elevation from her eyes to the top of the tower changes from $\ang{30}$
to $\ang{60}$ as she walks towards the building. Find the distance she walked towards the building

\insertQR{QRC}

\ifprintanswers
  % stuff to be shown only in the answer key - like explanatory margin figures
  \begin{marginfigure}
    \figinit{pt}
      \figpt 100: $A$(0,0)
      \figpt 101: $B$(35,0)
      \figpt 102: $C$(90,0)
      \figpt 103: $G$(90,45)
      \figpt 104: $D$(90,70)
      \figpt 105: $F$(35,45)
      \figpt 106: $E$(0,45)
    \figdrawbegin{}
      \figdrawline [100,101,102,103,105,106,100]
      \figdrawline [101,105,104,106]
      \figdrawline [103,104]
      \figdrawarccircP 106 ; 12 [105,104] 
      \figdrawarccircP 105 ; 12 [103,104] 
    \figdrawend
    \figvisu{\figBoxA}{}{%
      \figwrites 100:(2)
      \figwrites 101:(2)
      \figwrites 102:(2)
      \figwritee 103:(2)
      \figwritee 104:(2)
      \figwritese 105:(4)
      \figwritese 106:(2)
    }
    \centerline{\box\figBoxA}
  \end{marginfigure}
\fi 

\begin{solution}[\halfpage]
	The situation is \asif. The distance covered by the girl is $EF = EG - FG$, where
	\begin{align}
		\tan\angle{DEG} = \tan\ang{30} &= \dfrac{DG}{EG} \\
		\Rightarrow EG &= \dfrac{DG}{\tan\ang{30}} \\
		\text{And, }\tan\angle{DFG} = \tan\ang{60} &= \dfrac{DG}{FG} \\
		\Rightarrow FG &= \dfrac{DG}{\tan\ang{60}}
	\end{align}
	
	Now, $DG =$ (height of the building - girl's height) $= 30 - 1.5 = 28.5m$. And therefore, 
	
	\begin{align}
		EF = EG - EF &= 28.5\cdot\left(\dfrac{1}{\tan\ang{30}} - \dfrac{1}{\tan\ang{60}}\right) \\
		&= 32.91 \text{ meters}
	\end{align}
	
\end{solution}

