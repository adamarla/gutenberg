


\ifnumequal{\value{rolldice}}{0}{
  \renewcommand{\va}{5}
  \renewcommand{\vb}{3}
  \renewcommand{\vc}{1}
  \renewcommand{\vd}{2}
}{
  \ifnumequal{\value{rolldice}}{1}{
    \renewcommand{\va}{7}
    \renewcommand{\vb}{2}
    \renewcommand{\vc}{2}
    \renewcommand{\vd}{7}
  }{
    \ifnumequal{\value{rolldice}}{2}{
      \renewcommand{\va}{3}
      \renewcommand{\vb}{11}
      \renewcommand{\vc}{4}
      \renewcommand{\vd}{3}
    }{
      \renewcommand{\va}{9}
      \renewcommand{\vb}{2}
      \renewcommand{\vc}{5}
      \renewcommand{\vd}{1}
    }
  }
}

\ADD\va\vd\p
\SUBTRACT\vc\vb\q
\def\logtwo{0.301}
\EXPR[3]\a{ ((\va - (\va + \vd)*\logtwo) / (\vb + (\vc - \vb)*\logtwo)) }

\question[2] Using logarithms, solve the following equation for $x$.
\texttt{Hint:} $\log_{10}5 = \log_{10}\frac{10}{2} = 1 - \log_{10} 2$. ($\log_{10}2 = 0.301$)
\[ 5^{\va - \vb x} = 2^{\vc x + \vd}\]

\watchout[-40pt]

\begin{solution}[\mcq]
  \begin{align}
    5^{\va -\vb x} &= 2^{\vc x + \vd} \implies 
    (\va - \vb x)\cdot\log_{10} 5 = (\vc x + \vd)\cdot\log_{10} 2 \\
    &\implies (\va - \vb x)\cdot(1-\log_{10} 2) = (\vc x + \vd)\cdot\log_{10} 2 \\
    \text{ or } x &= \dfrac{\va - (\va + \vd)\log_{10} 2}
    {\vb + (\vc - \vb)\log_{10} 2} = \a
  \end{align}
\end{solution}

\ifprintanswers\begin{codex}\end{codex}\fi
