

\question[2] Two circles intersect at points $A$ and $B$ - \asif. If $AP$ and $AQ$ be their 
respective diameters, then prove that $PBQ$ is a line


  % stuff to be shown only in the answer key - like explanatory margin figures
  \begin{marginfigure}
    \figinit{pt}
    	\figpt 100: $C_1$(40,40)
    	\figpt 101: $C_2$(85,40)
      \figpt 102: (70,40) % right extreme of C_1
      \figpt 103: (40,70) % top extreme of C_1
      \figpt 104: (50,40) % left C_2
      \figpt 105: (85,75)
      \figptsintercirc 200 [100,30;101,35] % 201 is the other point
      \figptsinterlinellP 300 [100,102,103 ; 201,100] % 301 is the other point
      \figptsinterlinellP 310 [101,104,105 ; 201,101] % 311 is the other point
    \figdrawbegin{}
      \figdrawcirc 100(30)
      \figdrawcirc 101(35)
      \figdrawline [201,301]
      \figdrawline [201,311]
      \figdrawline [301,311]
      \ifprintanswers
        \figset (dash=5)
        \figdrawline [201,200]
      \fi
    \figdrawend
    \figvisu{\figBoxA}{}{%
      \figset write(mark=$\bullet$)
      \figwritew 100:(2)
      \figwritee 101:(1)
      \figwrites 200:$B$(3) 
      \figwriten 201:$A$(3) 
      \figwrites 301:$P$(3)
      \figwrites 311:$Q$(3)
    }
    \centerline{\box\figBoxA}
  \end{marginfigure} 

\begin{solution}[\halfpage]
	If $AP$ and $AQ$ are diameters, then 
  \[ \underbrace{\angle{PBA}=\angle{QBA}=\ang{90}}_{\text{angles within a semi-circle}} \]
	
	And therefore, 
	\begin{align}
		\angle PBA + \angle QBA &= \ang{90} + \ang{90} = \ang{180} \\
		     &\implies \text{ points P,B and Q are collinear }
	\end{align}
\end{solution}

