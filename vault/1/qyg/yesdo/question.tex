
\ifnumequal{\value{rolldice}}{0}{
	\renewcommand\va{3}
	\renewcommand\vb{4}
	\renewcommand\vc{4}
	\renewcommand\vd{6}
	\renewcommand\ve{5}
	\renewcommand\vf{7}
  \renewcommand\vx{(4,5)}
  \renewcommand\vy{(2,3)}
  \renewcommand\vz{(6,9)}
}{
	\ifnumequal{\value{rolldice}}{1}{
		\renewcommand\va{2}
		\renewcommand\vb{5}
		\renewcommand\vc{6}
		\renewcommand\vd{7}
		\renewcommand\ve{5}
		\renewcommand\vf{7}
    \renewcommand\vx{(1,5)}
    \renewcommand\vy{(3,5)}
    \renewcommand\vz{(9,9)}
	}{
		\ifnumequal{\value{rolldice}}{2}{
			\renewcommand\va{4}
			\renewcommand\vb{7}
			\renewcommand\vc{8}
			\renewcommand\vd{2}
			\renewcommand\ve{7}
			\renewcommand\vf{5}
      \renewcommand\vx{(3,10)}
      \renewcommand\vy{(5,4)}
      \renewcommand\vz{(11,0)}
		}{
			\renewcommand\va{5}
			\renewcommand\vb{4}
			\renewcommand\vc{7}
			\renewcommand\vd{9}
			\renewcommand\ve{6}
			\renewcommand\vf{4}
      \renewcommand\vx{(4,-1)}
      \renewcommand\vy{(6,9)}
      \renewcommand\vz{(8,9)}
		}
	}
}

\question[5]   The mid-points of the sides of a triangle are 
\[ A = (\va,\vb),\, B=(\vc,\vd)\text{ and }C=(\ve,\vf) \] 
What are the coordinates of the vertices - $X,Y$ and $Z$?

\watchout

\vspace{0.7cm}
\figinit{pt}
\figpt 1:(0,0)
\figpt 2:(90,0)
\figpt 3:(30,60)
\figvectP 10 [1,2]
\figvectP 20 [2,3]
\figvectP 30 [3,1]
\figpttra 11:$A(\va,\vb)$= 1/0.5,10/
\figpttra 21:$B(\vc,\vd)$= 2/0.5,20/
\figpttra 31:$C(\ve,\vf)$= 3/0.5,30/
\figdrawbegin{}
	\figdrawline[1,2,3,1]
	\ifprintanswers\figdrawline[11,21,31,11]\fi
\figdrawend
\figvisu{\figBoxA}{}{
	\figsetmark{$\bullet$}
	\figwritew 1:$X$(5)
	\figwritee 2:$Y$(5)
	\figwriten 3:$Z$(5)
	\figwrites 11:(5)
	\figwritee 21:(5)
	\figwritew 31:(5)
}
\centerline{\box\figBoxA}

% AB 
\LINESLOPE\va\vb\vc\vd\a\b
\LINEINTERCEPT\va\vb\a\b\c\d

%BC
\LINESLOPE\vc\vd\ve\vf\m\n
\LINEINTERCEPT\ve\vf\m\n\e\f

%CA
\LINESLOPE\va\vb\ve\vf\o\p
\LINEINTERCEPT\va\vb\o\p\g\h

%XZ
\LINEINTERCEPT\ve\vf\a\b\aa\bb
%XY
\LINEINTERCEPT\va\vb\m\n\mm\nn
%YZ
\LINEINTERCEPT\vc\vd\o\p\oo\pp

\begin{solution}[\fullpage]
	\textbf{Insight \#1: Line joining two mid-points $\parallel$ third side}

	\begin{align}
		AB &: \dfrac{y-\vb}{x-\va} = \dfrac{\vd-\vb}{\vc-\va}  = \WRITEFRAC\a\b \implies y = \WRITELINE\a\b\c\d  \\
		CA &: \dfrac{y-\vf}{x-\ve} = \dfrac{\vf-\vb}{\ve-\va}  = \WRITEFRAC[false]\o\p\implies y = \WRITELINE\o\p\g\h \\
    BC &: \dfrac{y-\vf}{x-\ve} = \dfrac{\vf-\vd}{\ve-\vc} = \WRITEFRAC\m\n\implies y = \WRITELINE\m\n\e\f \\
	\end{align}
	\textbf{Insight \#2: Parallel lines have the same slope} 

  And therefore, the \textbf{slopes} of the \textbf{sides of the triangle} will be 
  \begin{align}
    m_{XZ} &= m_{AB} = \WRITEFRAC[false]\a\b \\
    m_{XY} &= m_{BC} = \WRITEFRAC[false]\m\n \\
    m_{YZ} &= m_{AC} = \WRITEFRAC[false]\o\p
  \end{align}

  \textbf{Insight \#3: Each mid-point lies on some side}

  And therefore, given the slope of a side and the mid-point through which it passes, 
  we can easily find the \textbf{equation of the sides}.
	
	\begin{align}
		XZ &: \dfrac{y-\vf}{x-\ve} = \WRITEFRAC[false]\a\b \implies y = \WRITELINE\a\b\aa\bb \\
		XY &: \dfrac{y-\vb}{x-\va} = \WRITEFRAC[false]\m\n \implies y = \WRITELINE\m\n\mm\nn \\
		YZ &: \dfrac{y-\vd}{x-\vc} = \WRITEFRAC[false]\o\p \implies y = \WRITELINE\o\p\oo\pp
	\end{align}
	
	\textbf{Insight \#4: Vertices are where the sides intersect}

  \begin{tabular}{c c c c}
    \toprule
      Vertex & Eqn \#1 & Eqn \#2 & Coordinates \\
    \midrule
      X & $y=\WRITELINE\a\b\aa\bb$ & $y=\WRITELINE\m\n\mm\nn$ & $\vx$\\
      Y & $y=\WRITELINE\o\p\oo\pp$ & $y=\WRITELINE\m\n\mm\nn$ & $\vy$\\
      Z & $y=\WRITELINE\a\b\aa\bb$ & $y=\WRITELINE\o\p\oo\pp$ & $\vz$\\
    \bottomrule
  \end{tabular}
\end{solution}

\ifprintanswers\begin{codex}
  $X=\vx\qquad Y=\vy\qquad Z = \vz$
\end{codex}\fi
