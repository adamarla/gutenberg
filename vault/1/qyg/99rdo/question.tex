

\question[3] Compute the area of the figure contained between the curves
\[ y = \dfrac{1}{1+x^2}\text{ and } y=\dfrac{x^2}{2} \]

\ifprintanswers
\vspace{0.7cm}
	\figinit{pt}
	\def\Xmin{-39.99999}
	\def\Ymin{-20}
	\def\Xmax{39.99999}
	\def\Ymax{55}
	\def\Xori{39.99999}
	\def\Yori{31.99999}
	\figpt0:(\Xori,\Yori)
	\figpt 100: (42,56)
	\figpt 101: $y=\dfrac{x^2}{2}$(80,80)
	\figpt 102: $y=\dfrac{1}{1+x^2}$(80,45)
	\figdrawbegin{}
	\def\Xmaxx{\Xmax} % To customize the position
	\def\Ymaxx{\Ymax} % of the arrow-heads of the axes.
	\figset arrowhead(length=4, fillmode=yes) % styling the arrowheads
	\figdrawaxes 0(\Xmin, \Xmaxx, \Ymin, \Ymaxx)
	\figdrawlineC(
	0 45.12820,
	2.75862 46.46152,
	5.51724 47.96736,
	8.27586 49.66531,
	11.03448 51.57325,
	13.79310 53.70431,
	16.55172 56.06214,
	19.31034 58.63400,
	22.06896 61.38191,
	24.82758 64.23235,
	27.58620 67.06735,
	30.34482 69.72159,
	33.10344 71.99182,
	35.86206 73.66347,
	38.62068 74.55282,
	41.37931 74.55282,
	44.13793 73.66347,
	46.89655 71.99182,
	49.65517 69.72159,
	52.41379 67.06735,
	55.17241 64.23235,
	57.93103 61.38191,
	60.68965 58.63400,
	63.44827 56.06214,
	66.20689 53.70431,
	68.96551 51.57325,
	71.72413 49.66531,
	74.48275 47.96736,
	77.24137 46.46152,
	79.99999 45.12820
	)
	\figdrawlineC(
	0 79.99999,
	2.75862 73.60760,
	5.51724 67.67181,
	8.27586 62.19262,
	11.03448 57.17003,
	13.79310 52.60404,
	16.55172 48.49464,
	19.31034 44.84185,
	22.06896 41.64565,
	24.82758 38.90606,
	27.58620 36.62306,
	30.34482 34.79667,
	33.10344 33.42687,
	35.86206 32.51367,
	38.62068 32.05707,
	41.37931 32.05707,
	44.13793 32.51367,
	46.89655 33.42687,
	49.65517 34.79667,
	52.41379 36.62306,
	55.17241 38.90606,
	57.93103 41.64565,
	60.68965 44.84185,
	63.44827 48.49464,
	66.20689 52.60404,
	68.96551 57.17003,
	71.72413 62.19262,
	74.48275 67.67181,
	77.24137 73.60760,
	79.99999 79.99999
	)
	\figdrawend
	\figvisu{\figBoxA}{}{%
	\fontfamily{cmss}\selectfont\large
	\figptsaxes 1:0(\Xmin, \Xmaxx, \Ymin, \Ymaxx)
	\figwritee 1:(5pt)     \figwriten 2:(5pt)
	\figptsaxes 1:0(\Xmin, \Xmax, \Ymin, \Ymax)
	\figwritee 100:$A$(2)
	\figwritee 101:(2)
	\figwritee 102:(2)
	}
\centerline{\box\figBoxA}
\fi

\begin{solution}[\fullpage]
	The two curves are as shown in the figure above. And they intersect when
  \begin{align}
     \dfrac{1}{1+x^2} &= \dfrac{x^2}{2} \\
     \implies x^4+x^2-2 &= 0 \\
     \text{Setting } z = x^2, \text{ we get }
     z^2+z-2 &= 0 \\
     \implies z = 1, -2
  \end{align}
  As $z = x^2$, it has to be > 0. And therefore, 
	\[ z = 1 \implies x = \pm 1 \]
	The required area $=A$ therefore is 
  \begin{align}
     A &= 2\cdot\left[ \int_{-1}^1 \dfrac{1}{1+x^2}\ud x - \int_{-1}^1\dfrac{x^2}{2}\ud x\right] \\
       &= \underbrace{2\cdot\left[ \int_0^1 \dfrac{1}{1+x^2}\ud x - \int_0^1\dfrac{x^2}{2}\ud x\right]}_{\text{due to symmetry}} \\
     &= 2\cdot\left[ \left( \tan^{-1} x\right)_0^1 - \left( \dfrac{x^3}{6}\right)_0^1\right] \\
     &= \dfrac{\pi}{2} - \dfrac{1}{3} 
  \end{align}
\end{solution}
\ifprintanswers\begin{codex}$\dfrac\pi{2}-\dfrac{1}{3}$\end{codex}\fi
