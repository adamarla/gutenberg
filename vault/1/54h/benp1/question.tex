

\question[2] Find the inverse of the one-to-one function
$f:\Re-\{-5\}\rightarrow \Re-\{2\}$ defined by $f(x)=\dfrac{2x-1}{x+5}$


\ifprintanswers
	\marginnote{$\Re-\{-5\}$ means all real numbers \textit{except} -5. Similarly, $\Re-\{-5,7\}$ means
	all real numbers \textit{except} -5 and 7. Getting the picture?}
	\marginnote[1cm]{If $f(x) = y$, then $f^{-1}(y) = x$. This is the definition of an inverse function }
\fi 

\begin{solution}[\mcq]
	\begin{align}
		\text{If } y &= f(x) = \dfrac{2x-1}{x+5} \\
		\text{then } x &= \dfrac{5y+1}{2-y} = f^{-1}(y)
	\end{align}
	In $y = f(x)$, $y$ is the dependent variable and $x$ the independent variable. 
	And the convention is to write the \textit{independent} variable as $x$ and the 
	dependent variable as $y$
	
	So, we could also write (2) as $f^{-1}(x) = \dfrac{5x+1}{2-x}$. But remember that 
	the $x$ here \textit{is not} the same as the $x$ in (1). We are just following a naming
	convention here
	
	Alternately, we could use the fact that $f \circ f^{-1}(y) = y$ (or $f \circ f^{-1}(x) = x$, if 
	we follow the naming convention)
	
	\begin{align}
		\Rightarrow \dfrac{2f^{-1}(x)-1}{f^{-1}(x)+5} &= x \\
		\Rightarrow f^{-1}(x) &= \dfrac{5x+1}{2-x}
	\end{align}
\end{solution}
