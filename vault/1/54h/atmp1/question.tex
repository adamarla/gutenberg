
\ifnumequal{\value{rolldice}}{0}{
  \renewcommand\va{2}
  \renewcommand\vc{1}
  \renewcommand\vd{3}
  \renewcommand\vk{2}
}{
  \ifnumequal{\value{rolldice}}{1}{
    \renewcommand\va{3}
    \renewcommand\vc{5}
    \renewcommand\vd{3}
    \renewcommand\vk{3}
  }{
    \ifnumequal{\value{rolldice}}{2}{
      \renewcommand\va{3}
      \renewcommand\vc{2}
      \renewcommand\vd{3}
      \renewcommand\vk{2}
    }{
      \renewcommand\va{5}
      \renewcommand\vc{4}
      \renewcommand\vd{3}
      \renewcommand\vk{2}
    }
  }
}

\POWER\vk{2}\vm
\POWER\vd{2}\vn
\SUBTRACT\vn\vm\vo
\MULTIPLY\va{4}\vp
\FRACADD{1}{1}\vo\vp\vx\vy % b = vx / vy
\MULTIPLY\vd{-1}\ve
\ADD\ve\vk\vf
\SUBTRACT\ve\vk\vg
\MULTIPLY\va{2}\vh % roots = vf/vh, vg/vh

\FRACMULT\vd{1}\vf\vh\p\q
\FRACMINUS\vc{1}\p\q\a\b
\FRACMULT\vd{1}\vg\vh\r\s
\FRACMINUS\vc{1}\r\s\c\d

\question Given two functions 
\[
  f:\mathbb{R}\rightarrow\mathbb{R} = \va x^2 -\frac\vx\vy \text{ and }
  g:\mathbb{R}\rightarrow\mathbb{R} = \vc-\vd x 
\]

\begin{parts}
  \part Find the domain for which the two functions are equal
  \begin{solution}
    The two functions would be equal when 
    \begin{align}
      \va x^2 -\frac\vx\vy &= \vc-\vd x \\
      \implies \va x^2 +\vd x -\left( \frac\vx\vy - \vc\right) &= 0 \\
      \implies x &= \WRITEFRAC\vf\vh, \WRITEFRAC\vg\vh
    \end{align}

    Hence, the domain is $\left\{ \WRITEFRAC\vf\vh, \WRITEFRAC\vg\vh \right\}$
  \end{solution}
  
  \part What would this same domain of equality be if $f(x) = f:\mathbb{R}\rightarrow\mathbb{N}$ (instead of 
  $f:\mathbb{R}\rightarrow\mathbb{R}$) and $g(x)=g:\mathbb{R}\rightarrow\mathbb{N}$ (instead of 
  $g:\mathbb{R}\rightarrow\mathbb{R}$)? Provide justification for credit.
  \begin{solution}
    In part (a), when we said that $f(x) = f:\mathbb{R}\rightarrow\mathbb{R}$, what we were 
    saying was that $x\in\mathbb{R}$ and $f(x)\in\mathbb{R}$. Same for $g(x)$. 

    Now, we are saying that $x\in\mathbb{R}$ \textbf{but} $f(x)$ and $g(x)\in\mathbb{N}$.

    Which means, we must find $f(x)$ ($ = g(x)$) for the domain we found in part (a) and see if 
    the either of the values $\in\mathbb{N}$

    \begin{align}
      f\left(\WRITEFRAC\vf\vh\right) &= g\left(\WRITEFRAC\vf\vh\right) = \WRITEFRAC\a\b \\
      f\left(\WRITEFRAC\vg\vh\right) &= g\left(\WRITEFRAC\vg\vh\right) = \WRITEFRAC\c\d
    \end{align}
    As $f(x)$ and $g(x)\notin\mathbb{N}$ for either of the $x$ we found in part (a), all that we 
    can say is that $x=\WRITEFRAC\vf\vh,\WRITEFRAC\vg\vh$ \textbf{cannot even be} in the domain of 
    $f:\mathbb{R}\rightarrow\mathbb{N}$ or $g:\mathbb{R}\rightarrow\mathbb{N}$. 

    And as $f(x)=g(x)$ \textbf{for only those two values} of $x$, the domain of equality 
    for $f:\mathbb{R}\rightarrow\mathbb{N}$ and $g:\mathbb{R}\rightarrow\mathbb{N}$ is empty = $\phi$ 
  \end{solution}

\end{parts}

