
\ifnumequal{\value{rolldice}}{0}{
  \renewcommand\va{3}
}{
  \ifnumequal{\value{rolldice}}{1}{
    \renewcommand\va{3}
  }{
    \ifnumequal{\value{rolldice}}{2}{
      \renewcommand\va{4}
    }{
      \renewcommand\va{5}
    }
  }
}

\SUBTRACT\va{2}\vb
\SQUARE\va\vc
\SUBTRACT\va{1}\vd
\ADD\vc\vd\ve

\question[4] The figure below shows the plot of the curve \[y=(x^2+\va x)\cdot e^{-x}\]
Find the area contained between it and the $x-$axis.

\watchout

\figinit{pt}
\def\Xmin{-66.66666}
\def\Ymin{-14.57502}
\def\Xmax{13.33333}
\def\Ymax{65.42497}
\def\Xori{66.66666}
\def\Yori{14.57502}
\figpt0:(\Xori,\Yori)
\figpt 100: (32,14)
\figdrawbegin{}
\def\Xmaxx{\Xmax} % To customize the position
\def\Ymaxx{\Ymax} % of the arrow-heads of the axes.
\figset arrowhead(length=4, fillmode=yes) % styling the arrowheads
\figdrawaxes 0(\Xmin, \Xmaxx, \Ymin, \Ymaxx)
\figdrawlineC(
0 79.99999,
2.75862 59.42819,
5.51724 43.17884,
8.27586 30.51152,
11.03448 20.79838,
13.79310 13.50832,
16.55172 8.19326,
19.31034 4.47623,
22.06896 2.04110,
24.82758 .62368,
27.58620 .00406,
30.34482 0,
33.10344 .46119,
35.86206 1.26443,
38.62068 2.30933,
41.37931 3.51477,
44.13793 4.81577,
46.89655 6.16088,
49.65517 7.50989,
52.41379 8.83194,
55.17241 10.10387,
57.93103 11.30887,
60.68965 12.43526,
63.44827 13.47555,
66.20689 14.42559,
68.96551 15.28387,
71.72413 16.05097,
74.48275 16.72903,
77.24137 17.32142,
79.99999 17.83234
)
\figdrawend
\figvisu{\figBoxA}{}{%
\figptsaxes 1:0(\Xmin, \Xmaxx, \Ymin, \Ymaxx)
\figwritee 1:(5pt)     \figwriten 2:(5pt)
\figptsaxes 1:0(\Xmin, \Xmax, \Ymin, \Ymax)
\ifprintanswers
  \large
  \figwrites 100:$A$(2)
\fi
}

\vspace{0.7cm}
\centerline{\box\figBoxA}

\begin{solution}[\fullpage]
 \begin{align}
    (x^2+\va x)\cdot e^{-x} &= x(x+\va)\cdot e^{-x}
 \end{align}
 $\implies$ the curve cuts the $x-axis$ at $x=0$ and $x=-\va$.
 
 The required area $A$ of the region is therefore
 \begin{align}
    A &= \int_{-\va}^0(x^2+\va x)e^{-x}\ud x \\
      &= \int_{-\va}^0 (x^2\cdot e^{-x})\ud x + \int_{-\va}^0 (\va x\cdot e^{-x})\ud x
  \end{align}
  
  Setting $-x = z\implies \ud x = -\ud z$, we get 
  \begin{align}
    A &= -\int_{\va}^0(z^2\cdot e^z)\ud z - \int_{\va}^0 (\va z\cdot e^z)\ud z \\
      &= \int_0^{\va}(z^2\cdot e^z)\ud z + \int_0^{\va} (\va z\cdot e^z)\ud z
  \end{align}
  Looks like we will soon have to evaluate the following integrals 
  \[ \int z^2\cdot e^z\ud z\text{ and } \int z\cdot e^z\ud z \]
  And here is how you should think about them. If, 
  \begin{align}
    y = z^n\cdot e^z\text{ then }\dfrac{\ud y}{\ud z} &= z^n\cdot e^z + n\cdot z^{n-1}\cdot e^z \\
    \implies \int\ud y &= \underbrace{\int z^n\cdot e^z\ud z}_{\text{ what we need}} + 
    n\int z^{n-1}\cdot e^z\ud z \\
    \implies \int z^n\cdot e^z\ud z &= \underbrace{z^n\cdot e^z}_{=y} - n\cdot\int z^{n-1}\cdot e^z\ud z
  \end{align}
  \textbf{Using this general result,} we can now say that 
  \begin{align}
    \int z^2\cdot e^z\ud z &= z^2\cdot e^z -2\int z\cdot e^z\ud z \text{ and } \\
    \int z\cdot e^z\ud z &= z\cdot e^z - \int z^0\cdot e^z\ud z = ze^z - e^z + C \\
    &= (z-1)\cdot e^z + C 
  \end{align}
  And therefore, 
  \begin{align}
    A &= \int_0^{\va}(z^2\cdot e^z)\ud z + \int_0^{\va} (\va z\cdot e^z)\ud z \nonumber\\
      &= \left[ z^2\cdot e^z - 2\int z\cdot e^z\ud z \right]_0^{\va} 
      + \left[ \va\int z\cdot e^z\ud z \right]_0^{\va} \\ 
      &= \left[ z^2\cdot e^z + \vb\cdot (z-1)\cdot e^z \right]_0^{\va} = \ve\cdot e^{\va} - 1
  \end{align}
\end{solution}

\ifprintanswers\begin{codex}$\ve\cdot e^{\va} - 1$ sq. units\end{codex}\fi
