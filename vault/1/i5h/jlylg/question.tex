
\ifnumequal{\value{rolldice}}{0}{
  % variables 
  \renewcommand\va{8}
  \renewcommand\vb{5}
  \renewcommand\vc{12}
  \renewcommand\vd{9}
}{
  \ifnumequal{\value{rolldice}}{1}{
    % variables 
    \renewcommand\va{7}
    \renewcommand\vb{2}
    \renewcommand\vc{11}
    \renewcommand\vd{6}
  }{
    \ifnumequal{\value{rolldice}}{2}{
      % variables 
      \renewcommand\va{13}
      \renewcommand\vb{9}
      \renewcommand\vc{7}
      \renewcommand\vd{3}
    }{
      % variables 
      \renewcommand\va{15}
      \renewcommand\vb{6}
      \renewcommand\vc{21}
      \renewcommand\vd{12}
    }
  }
}

\ADD\va\vb\ve
\ADD\vc\vd\vf
\SUBTRACT\vf\ve\vg
\ABSVALUE\vg\vh
\DIVIDE\vh{2}\vi

\question[4] For what value of $N$ would the following hold 
  \[\dfrac{\cos{\va A}\cos{\vb A}-\cos{\vc A}\cos{\vd A}}{\sin{\va A}\cos{\vb A}+\cos{\vc A}\sin{\vd A}} = \tan{\textbf{N}A}\]
\watchout
\begin{solution}[\halfpage]
	Given that,
	\begin{align}
		\fProdOfCos{x}{y} \\
		\fProdSinCos{x}{y}
     \end{align}
     , one can \textit{re-write} the above expression as 
   \[\dfrac{\frac{1}{2}\cdot\left[\eProdOfCos{\va A}{\vb A}\right]-\frac{1}{2}\cdot\left[\eProdOfCos{\vc A}{\vd A}\right]}
     	{\frac{1}{2}\cdot\left[\eProdSinCos{\va A}{\vb A}\right] + \frac{1}{2}\cdot\left[\eProdSinCos{\vd A}{\vc A}\right]}\]

     \begin{align}
     	&= \dfrac{\cos \ve A - \cos \vf A}{\sin \ve A + \sin \vf A} \\
     	&= \dfrac{\eDiffOfCos{\ve A}{\vf A}}{\eSumOfSin{\ve A}{\vf A}} \\
     	&= \dfrac{\sin \vi A}{\cos \vi A} = \tan \vi A
     \end{align}
     Hence, $N = \vi$.
\end{solution}
