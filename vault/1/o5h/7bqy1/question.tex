

\question[3] For what value(s) of $x$ is the following true? 
\[ 2\cos^2 x + 3\sin x = 0 \]

\begin{solution}[\halfpage]
	\begin{align}
		2\cos^2 x + 3\sin x = 0 \implies 2\cdot(1-\sin^2 x) &+ 3 \sin x = 0 \\
		\implies 2\sin^2 x - 3\sin x - 2 &= 0 \\
    \implies 2\sin^2 x - 4\sin x + \sin x - 2 &= 0 \\
    \implies (2\sin x + 1)\cdot (\sin x - 2) &= 0\\
    \implies \sin x &= -\dfrac{1}{2}\text{ or } 2
	\end{align}

  But $-1\leq \sin x \leq 1\implies \sin x = -\frac{1}{2}$. 

  \textbf{The first }$x\in\underbrace{\left[-\frac\pi{2},\frac\pi{2} \right]}_{\text{domain}}$ 
  for which $\sin x = -\frac{1}{2}$ is $x=-\frac\pi{6}$.
	
  The other angles (and there are infinite of them) are given by 
  \[ x = n\pi + (-1)^n\left( -\dfrac\pi{6}\right) = n\pi + (-1)^{n+1}\dfrac\pi{6} \] 
\end{solution}

\ifprintanswers\begin{codex}$n\pi + (-1)^{n+1}\dfrac\pi{6}$\end{codex}\fi
