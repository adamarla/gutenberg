
\ifnumequal{\value{rolldice}}{0}{
  % variables 
  \renewcommand{\va}{13}
  \renewcommand{\vb}{11}
  \renewcommand{\vc}{4}
  \renewcommand{\vd}{3}
  \renewcommand{\ve}{eight}
}{
  \ifnumequal{\value{rolldice}}{1}{
    % variables 
    \renewcommand{\va}{12}
    \renewcommand{\vb}{9}
    \renewcommand{\vc}{2}
    \renewcommand{\vd}{3}
    \renewcommand{\ve}{second}
  }{
    \ifnumequal{\value{rolldice}}{2}{
      % variables 
      \renewcommand{\va}{11}
      \renewcommand{\vb}{13}
      \renewcommand{\vc}{4}
      \renewcommand{\vd}{3}
      \renewcommand{\ve}{seventh}
    }{
      % variables 
      \renewcommand{\va}{10}
      \renewcommand{\vb}{12}
      \renewcommand{\vc}{5}
      \renewcommand{\vd}{3}
      \renewcommand{\ve}{third}
    }
  }
}

\EXPR[0]\p{(\vb - (\vc + \vd))}
\SUBTRACT\va{2}\q

\question[3] In a $\va-$storey building, $\vb$ people enter an elevator. 
It is known that they will leave the elevator in groups of 
$\vc$, $\vd$ and $\p$ at different storeys. In how many ways can they do 
so if the elevator does not stop at the $\ve$ storey?

\watchout[-1cm]

\begin{solution}[\mcq]
  Right upfront, we know that two floors are off-limits from consideration - the $\ve$ floor
  and the floor at which everyone got on the elevator. 

  Which means, the 3-groups have a choice of $\q$ floors. And therefore they could get off 
  in $\binom\q{3}\times 3! = \enpr\q{3}$ ways
\end{solution}

\ifprintanswers\begin{codex}$\enpr\q{3}$ ways\end{codex}\fi
