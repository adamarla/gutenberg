


\ifnumequal{\value{rolldice}}{0}{
  % variables 
  \renewcommand{\vbone}{1,5,6,7}
  \renewcommand{\vbtwo}{5671}
  \renewcommand{\vbthree}{1} % units
  \renewcommand{\vbfour}{7} % tens
  \renewcommand{\vbfive}{6} % hundreds
  \renewcommand{\vbsix}{5} % thousands
  \renewcommand{\vbseven}{6,7} % guaranteed > (thousands) 
  \renewcommand{\vbeight}{7} % hundreds
  \renewcommand{\vbnine}{7} % tens
  \renewcommand{\vbten}{5,6,7} % units
  \gcalcexpr[0]\nth{2}
  \gcalcexpr[0]\nh{1}
  \gcalcexpr[0]\nten{1}
  \gcalcexpr[0]\nu{3}
}{
  \ifnumequal{\value{rolldice}}{1}{
    % variables 
    \renewcommand{\vbone}{2,4,7,9}
    \renewcommand{\vbtwo}{4792}
    \renewcommand{\vbthree}{2}
    \renewcommand{\vbfour}{9}
    \renewcommand{\vbfive}{7}
    \renewcommand{\vbsix}{4}
    \renewcommand{\vbseven}{7,9}
    \renewcommand{\vbeight}{9}
    \renewcommand{\vbnine}{9}
    \renewcommand{\vbten}{4,7,9}
    \gcalcexpr[0]\nth{2}
    \gcalcexpr[0]\nh{1}
    \gcalcexpr[0]\nten{1}
    \gcalcexpr[0]\nu{3}
  }{
    \ifnumequal{\value{rolldice}}{2}{
      % variables 
      \renewcommand{\vbone}{3,4,6,7}
      \renewcommand{\vbtwo}{6473}
      \renewcommand{\vbthree}{3}
      \renewcommand{\vbfour}{7}
      \renewcommand{\vbfive}{4}
      \renewcommand{\vbsix}{6}
      \renewcommand{\vbseven}{7}
      \renewcommand{\vbeight}{6,7}
      \renewcommand{\vbnine}{7}
      \renewcommand{\vbten}{4,6,7}
      \gcalcexpr[0]\nth{1}
      \gcalcexpr[0]\nh{2}
      \gcalcexpr[0]\nten{1}
      \gcalcexpr[0]\nu{3}
    }{
      % variables 
      \renewcommand{\vbone}{2,5,6,9}
      \renewcommand{\vbtwo}{6592}
      \renewcommand{\vbthree}{2}
      \renewcommand{\vbfour}{9}
      \renewcommand{\vbfive}{5}
      \renewcommand{\vbsix}{6}
      \renewcommand{\vbseven}{9}
      \renewcommand{\vbeight}{6,9}
      \renewcommand{\vbnine}{9}
      \renewcommand{\vbten}{5,6,9}
      \gcalcexpr[0]\nth{1}
  	  \gcalcexpr[0]\nh{2}
  	  \gcalcexpr[0]\nten{1}
  	  \gcalcexpr[0]\nu{3}
    }
  }
}

\gcalcexpr[0]\tp{\nth * 4 * 4 * 4}
\gcalcexpr[0]\tq{\nh * 4 * 4}
\gcalcexpr[0]\tr{\nten * \nu}
\gcalcexpr[0]\ngreater{\tp + \tq + \tr}
\gcalcexpr[0]\nlessequal{256 - \ngreater}
\gcalcexpr[0]\nreqd{84 + \nlessequal}

\question[5] How many numbers $\leq \vbtwo$ can be formed using $\vbone$ if the digits are allowed to repeat?


\watchout

\ifprintanswers
\fi 

\begin{solution}[\halfpage]
	Any one-, two- or three-digit number made using digits from $A=\lbrace \vbone \rbrace$ 
	is guaranteed to be $\leq \vbtwo$. They are also easy to count
	\begin{align}
		N_{\texttt{< 4-digits}} &= N_1 + N_2 + N_3 \\ 
		 &= 4 + 4^2 + 4^3 = 84
	\end{align}
	
	It is when counting the 4-digit numbers that one has to be careful. But here is what we know about what 
	would make a 4-digit number $ > \vbtwo$ (strictly greater)
	
	
	\begin{tabular}{ccccc}
		\toprule
		Thousand & Hundreds & Tens & Units & \# possibilities \\
		\midrule
		$\lbrace\vbseven\rbrace$ & 4 & 4 & 4 & \tp \\ % thousands
		$\lbrace\vbsix\rbrace$ & $\lbrace\vbeight\rbrace$ & 4 & 4 & \tq \\ % hundreds
		$\lbrace\vbsix\rbrace$ & $\lbrace\vbfive\rbrace$ & $\lbrace\vbnine\rbrace$ & $\lbrace\vbten\rbrace$ & \tr \\ % tens 
	    \bottomrule
	\end{tabular}
	
	This means that there are $\ngreater (= \tp + \tq + \tr)$ 4-digit numbers $ > \vbtwo$ that can be formed from 
	$A = \lbrace\vbone\rbrace$. And hence, there are $\nlessequal( = 4^4 - \ngreater)$ numbers that are $ \leq \vbtwo$
	
	So, in total, there are $\nreqd (= \nlessequal + 84)$ numbers that meet our initial criterion
	
\end{solution}
