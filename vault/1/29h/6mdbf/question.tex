


\ifnumequal{\value{rolldice}}{0}{
  % variables 
  \renewcommand{\vbone}{6}
  \renewcommand{\vbtwo}{30,240}
}{
  \ifnumequal{\value{rolldice}}{1}{
    % variables 
    \renewcommand{\vbone}{7}
    \renewcommand{\vbtwo}{2,540,160}
  }{
    \ifnumequal{\value{rolldice}}{2}{
      % variables 
      \renewcommand{\vbone}{5}
      \renewcommand{\vbtwo}{25,200}
    }{
      % variables 
      \renewcommand{\vbone}{8}
      \renewcommand{\vbtwo}{29,030,400}
    }
  }
}

\gcalcexpr[0]\tp{\vbone + 2}
\gcalcexpr[0]\tq{\tp - 1}
\gcalcexpr[0]\tr{\tp - 2}

\question[3] At a buffet, guests get one type of soup and one type of dessert (or sweet-dish) alongwith $\vbone$ other 
main-course dishes. In how many ways can the buffet table be laid if the soup and the dessert should \textit{never} be
placed together? 


\watchout

\ifprintanswers
\fi 

\begin{solution}[\mcq]
  Including the soup and the dessert, there are a total of $\tp = (\vbone + 2)$ dishes. The number of ways, therefore, 
  of laying the table in which the dessert and the soup are not placed together is 
  \begin{align}
  	N &= N_{\texttt{all possible ways}} - N_{\texttt{soup \& dessert together}} \\
  	  &= \tp ! - 2!\cdot\left( \tp - 2 + 1\right) ! = \tp ! - 2!\cdot\tq ! = \tq !\cdot(\tp - 2) \\
  	  &= \tp!\cdot\tr = \vbtwo
  \end{align}
\end{solution}
