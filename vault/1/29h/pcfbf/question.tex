

\ifnumequal{\value{rolldice}}{0}{
  % variables 
  \renewcommand{\va}{16}
  \renewcommand{\vb}{37}
}{
  \ifnumequal{\value{rolldice}}{1}{
    % variables 
    \renewcommand{\va}{12}
    \renewcommand{\vb}{29}
  }{
    \ifnumequal{\value{rolldice}}{2}{
      % variables 
      \renewcommand{\va}{17}
      \renewcommand{\vb}{44}
    }{
      % variables 
      \renewcommand{\va}{19}
      \renewcommand{\vb}{25}
    }
  }
}

\SUBTRACT\va{1}\vc
\SUBTRACT\vb{1}\vd
\SUBTRACT\vb\va\ve
\FRACTIONSIMPLIFY\ve\vc\vx\vy

\question[3] The first, $\va^\text{th}$ and $\vb^\text{th}$ terms of an arithmetic progression are also 
\textit{successive} terms of a geometric progression. Find the common ratio of the geometric progression.

\watchout

\begin{solution}[\halfpage]
	Here is what we know 
	\begin{align}
		a &= b \\
		a + \vc\cdot d &= b\cdot r \\
		a + \vd\cdot d &= b\cdot r^2 
	\end{align}
	
	From this, we can infer 
	\begin{align}
		(a + \vc\cdot d) - a &= b\cdot r - b = b\cdot(r-1) \\
		(a + \vd\cdot d) - a &= b\cdot r^2 - b = b\cdot(r^2-1) \\ 
		&= \underbrace{b\cdot(r-1)}_{\vc\cdot d}\cdot(r+1) \\
		\implies \vd\cdot d &= \vc\cdot d \cdot (r + 1) \\ 
		\implies r &= \frac\vx\vy 
	\end{align}
\end{solution}

\ifprintanswers
  \begin{codex}
    $\dfrac\vx\vy$
  \end{codex}
\fi
