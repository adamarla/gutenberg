
\ifnumequal{\value{rolldice}}{0}{
  % variables 
  \renewcommand\va{2}
  \renewcommand\vb{7}
}{
  \ifnumequal{\value{rolldice}}{1}{
    % variables 
    \renewcommand\va{3}
    \renewcommand\vb{8}
  }{
    \ifnumequal{\value{rolldice}}{2}{
      % variables 
      \renewcommand\va{4}
      \renewcommand\vb{9}
    }{
      % variables 
      \renewcommand\va{5}
      \renewcommand\vb{11}
    }
  }
}

\ADD\va\vb\vc
\MULTIPLY\va\vb\vd
\ADD\va{1}\vx
\ADD\vb{1}\vy

\question If $\lfloor x\rfloor$ is the greatest integer function, then for what values of $x$ would 
the following hold 
  \[
    \lfloor x\rfloor^2-\vc\lfloor x\rfloor + \vd = 0
  \]
  Do not assume that the above is true for one or two values of $x$. The above is true for infinite values of $x$.


\watchout

\begin{solution}
  Let $\lfloor x\rfloor = z$. Which means,
  \begin{align}
    z^2 -\vc z + \vd &= 0 \implies (z-\va)\cdot (z-\vb) = 0 \\
    \implies z = \lfloor x\rfloor &= \va, \vb \\
    \text{ Now, } \lfloor x\rfloor &= \va \implies \va\leq x <\vx \implies x\in [\va,\vx) \\
    \text{ and, } \lfloor x\rfloor &= \vb \implies \vb\leq x <\vy \implies x\in [\vb,\vy)
  \end{align}

  And therefore, $x\in [\va,\vx)\cup [\vb,\vy)$

\end{solution}

