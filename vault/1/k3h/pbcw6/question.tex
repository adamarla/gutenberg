

\question Two men - $M$ and $N$ - are playing a rather dangerous game. They are taking
turns firing at each other from a revolver that has 2 bullets with space for six (6). 
The revolver's bullet chamber is circular and the bullets are in \textit{adjacent} chambers.
$M$ took the first shot and it was blank. Now its $N's$ turn. Should $N$ fire the gun as-is
or should he rotate the chamber first? 


\ifprintanswers
  % stuff to be shown only in the answer key - like explanatory margin figures
  \begin{marginfigure}
    \figinit{pt}
      \figpt 100: (50,50)
      \figptcirc 200 :$A$: 100; 50(0)
      \figptcirc 201 :$B$: 100; 50(60)
      \figptcirc 202 :$C$: 100; 50(120)
      \figptcirc 203 :$D$: 100; 50(180)
      \figptcirc 204 :$E$: 100; 50(240)
      \figptcirc 205 :$F$: 100; 50(300)
    \figdrawbegin{}
      \figdrawcirc 200 (5)
      \figdrawcirc 201 (5)
      \figdrawcirc 202 (5)
      \figdrawcirc 203 (5)
      \figset (fillmode=yes)
      \figdrawcirc 204 (5)
      \figdrawcirc 205 (5)
    \figdrawend
    \figvisu{\figBoxA}{}{%
      \figwritee 200:(7)
      \figwritee 201:(7)
      \figwritee 202:(7)
      \figwritee 203:(7)
      \figwritee 204:(7)
      \figwritee 205:(7)
    }
    \centerline{\box\figBoxA}
  \end{marginfigure}
\fi 

\begin{solution}
	The two men obviously don't like each other. Otherwise they wouldn't be playing such 
	a dangerous game. Be that as it may, it is obvious that $N$ would want to fire a bullet - and not a 
	blank - at $M$ when his turn comes
	
	The question for $N$ then is - is the probability that there is a bullet in the current
	chamber higher if he rotates the chamber or if he sticks with the chamber he got when $M$
	passed him the gun. The situation is \asif where chambers $E$ and $F$ have the bullets whilst 
	others are blank. Lets assume that the chamber rotates \textit{counter clockwise} after each shot
	
	And so, if $X$ be the event that the first shot is blank and $Y$ the event the second
	is blank, then we are seeking $\bayesp{\textoverline{Y}}{X}$
	\begin{align}
		\bayesp{\textoverline{Y}}{X} &= \dfrac{\bayesf{X}{\textoverline{Y}}}{P(X)} \\
		\text{where }\bayesp{X}{\textoverline{Y}} = P(\text{first shot = A}) &= \dfrac{1}{6} \\
		P(\overline{Y}) = P(\text{first shot = A or F}) &= \dfrac{2}{6} \\
		\text{and } P(X) &= \dfrac{4}{6} \\
		\Rightarrow \bayesp{\textoverline{Y}}{X} &= \dfrac{\frac{1}{6}\times\frac{2}{6}}{\frac{2}{3}} = \dfrac{1}{12}
	\end{align}
	
	If, however, $N$ were to rotate the chamber, then his probability of firing a \textit{non-blank} 
	chamber would be $\dfrac{1}{3} > \dfrac{1}{12}$. And therefore, $N$ should rotate the chamber before firing
	
\end{solution}
