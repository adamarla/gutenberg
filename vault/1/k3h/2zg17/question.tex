
\ifnumequal{\value{rolldice}}{0}{
  % variables 
  \renewcommand{\va}{14}
  \renewcommand{\vb}{2}
}{
  \ifnumequal{\value{rolldice}}{1}{
    % variables 
    \renewcommand{\va}{11}
    \renewcommand{\vb}{5}
  }{
    \ifnumequal{\value{rolldice}}{2}{
      % variables 
      \renewcommand{\va}{12}
      \renewcommand{\vb}{6}
    }{
      % variables 
      \renewcommand{\va}{17}
      \renewcommand{\vb}{5}
    }
  }
}

\MULTIPLY\vb{2}\vc
\SUBTRACT\vc{1}\vd
\SQUARE\vb\ve
\SUBTRACT\ve\va\vf

\SQUARE\vd\vw
\EXPR\vx{\vw - 4*\vf}
\SQUAREROOT\vx\vy

\ADD{-\vd}\vy\vm
\SUBTRACT{-\vd}\vy\vn
\DIVIDE\vm{2}\vp
\DIVIDE\vn{2}\vq
\MULTIPLY\vp{-1}\vr
\MULTIPLY\vq{-1}\vs

\question[3] For what values of $x$ would the following hold?
\[ \sqrt{x+\va} < x+\vb \]

\watchout

\begin{solution}[\halfpage]
  \textbf{Insigh \#1: $x+\va\nless 0$}

  Because if it is, then $\sqrt{x+\va}\notin\mathbb{R}$. And therefore, 
  \[ x + \va\geq 0\implies x\geq -\va \] 
  \textbf{Insigh \#2: $x+\vb > 0$}

  The \textbf{square root function} is defined to mean the positive square root. Which means, if
  \[ x + \vb > \sqrt{x + \va}\text{ and } \sqrt{x + \va} > 0\text{ then } x + \vb > 0 \implies x > -\vb \]
  Moreover, for the inequality to hold 
	\begin{align}
    x + \va < (x+\vb)^2 &\implies x + \va < x^2 + \vc x + \ve \\
    \implies \startpoly\WRITEPOLYTERM{1}{2}\WRITEPOLYTERM{\vd}{1}\WRITEPOLYTERM{\vf}{0} &> 0 \text{ or } 
    \startpoly\left(\WRITEPOLYTERM{1}{1}\WRITEPOLYTERM\vr{0}\right)\cdot
    \startpoly\left(\WRITEPOLYTERM{1}{1}\WRITEPOLYTERM\vs{0}\right) > 0 \\
    \implies x &> \vp\text{ or } x < \vq
	\end{align}

  But $x\nless\vq$ because $x > -\vb$. Hence, the \textbf{only acceptable answer} is $x > \vp$.
\end{solution}
\ifprintanswers\begin{codex}$x>\vp$\end{codex}\fi
