% This is an empty shell file placed for you by the 'examiner' script.
% You can now fill in the TeX for your question here.

% Now, down to brasstacks. ** Writing good solutions is an Art **. 
% Eventually, you will find your own style. But here are some thoughts 
% to get you started: 
%
%   1. Write to be understood - but be crisp. Your own solution should not take 
%      more space than you will give to the student. Hence, if you take more than 
%      a half-page to write a solution, then give the student a full-page and so on...
%
%   2. Use margin-notes to "talk" to students about the critical insights
%      in the question. The tone can be - in fact, should be - informal
%
%   3. Don't shy away from creating margin-figures you think will help
%      students understand. Yes, it is a little more work per question. 
%      But the question & solution will be written only once. Make that
%      attempt at writing a solution count.
%      
%      3b. Use bc_to_fig.tex. Its an easier way to generate plots & graphs 
% 
%   4. Ensure that there are *no spelling mistakes anywhere*. We are an 
%      education company. Bad spellings suggest that we ourselves 
%      don't have any education. Also, use American spellings by default
% 
%   5. If a question has multiple parts, then first delete lines 40-41
%   6. If a question does not have parts, then first delete lines 43-69
%   
%   7. Create versions of the question when possible. Use commands defined in 
%      tufte-tweaks.sty to do so. Its easier than you think

% \noprintanswers
% \setcounter{rolldice}{0}
% \printrubric

\ifnumequal{\value{rolldice}}{0}{
  % variables 
  \renewcommand{\vbone}{}
  \renewcommand{\vbtwo}{}
  \renewcommand{\vbthree}{}
  \renewcommand{\vbfour}{}
  \renewcommand{\vbfive}{}
  \renewcommand{\vbsix}{}
  \renewcommand{\vbseven}{}
  \renewcommand{\vbeight}{}
  \renewcommand{\vbnine}{}
  \renewcommand{\vbten}{}
}{
  \ifnumequal{\value{rolldice}}{1}{
    % variables 
    \renewcommand{\vbone}{}
    \renewcommand{\vbtwo}{}
    \renewcommand{\vbthree}{}
    \renewcommand{\vbfour}{}
    \renewcommand{\vbfive}{}
    \renewcommand{\vbsix}{}
    \renewcommand{\vbseven}{}
    \renewcommand{\vbeight}{}
    \renewcommand{\vbnine}{}
    \renewcommand{\vbten}{}
  }{
    \ifnumequal{\value{rolldice}}{2}{
      % variables 
      \renewcommand{\vbone}{}
      \renewcommand{\vbtwo}{}
      \renewcommand{\vbthree}{}
      \renewcommand{\vbfour}{}
      \renewcommand{\vbfive}{}
      \renewcommand{\vbsix}{}
      \renewcommand{\vbseven}{}
      \renewcommand{\vbeight}{}
      \renewcommand{\vbnine}{}
      \renewcommand{\vbten}{}
    }{
      % variables 
      \renewcommand{\vone}{}
      \renewcommand{\vbtwo}{}
      \renewcommand{\vbthree}{}
      \renewcommand{\vbfour}{}
      \renewcommand{\vbfive}{}
      \renewcommand{\vbsix}{}
      \renewcommand{\vbseven}{}
      \renewcommand{\vbeight}{}
      \renewcommand{\vbnine}{}
      \renewcommand{\vbten}{}
    }
  }
}

\question The function $g$ is defined and differentiable on the closed interval $[-7,5]$ and satisfied
$g(0) = 5$. The graph of $y=g'(x)$, the derivative of $g$, consists of a semicircle and three line
segments, \asif.

\insertQR{}

%\watchout

%\ifprintanswers
  \begin{marginfigure}
    \figinit{pt}
      \figpt 10:(0,0)
      \figpt 20:(-70,-10)
      \figpt 30:(-20,0)
      \figpt 40:(20,0)
      \figpt 50:(30,30)
      \figpt 60:(50,-10)
      % extremeties
      \def\Xmax{60}
      \def\Ymax{40}
      \def\Xmin{-80}
      \def\Ymin{-20}
      % pts for numbering the axes (5, 10, 15...)
      \figpt 200:$ $(-80,0)
      \figpt 201:$ $(-70,0)
      \figpt 202:$ $(-60,0)
      \figpt 203:$ $(-50,0)
      \figpt 204:$ $(-40,0)
      \figpt 205:$ $(-30,0)
      \figpt 206:$ $(-20,0)
      \figpt 207:$ $(-10,0)
      \figpt 208:$\tiny\text{O}$(0,0)
      \figpt 209:$1$(10,0)
      \figpt 210:$ $(20,0)
      \figpt 211:$ $(30,0)
      \figpt 212:$ $(40,0)
      \figpt 213:$ $(50,0)
      \figpt 214:$ $(0,-20)
      \figpt 215:$ $(0,-10)
      \figpt 216:$1$(0,10)
      \figpt 217:$ $(0,20)
      \figpt 218:$ $(0,30)
      % label graph
      \figpt 100:(\Xmax,0)
      \figpt 101:(0,\Ymax)
      \figpt 102:(0,\Ymin)
    \figdrawbegin{}
      \figset arrowhead(length=4, fillmode=yes)
      \figdrawaxes 10(\Xmin, \Xmax, \Ymin, \Ymax)
      \figdrawline [20,30]
      \figdrawarccirc 10 ; 20 (0,180)
      \figdrawline [40,50]
      \figdrawline [50,60]
    \figdrawend
    \figvisu{\figBoxA}{}{%
      \figwritee 100: $x$(2 pt)
      \figsetmark{$\tiny\text{|}$}
      \figwrites 200,201,202,203,205,206,207,209,210,211,212,213 :(2 pt)
      \figwritesw 204: (2 pt)
      \figsetmark{$\tiny\text{-}$}
      \figwritew 214,215,216,217,218 :(2 pt)
      \figsetmark{$\bullet$}
      \figwrites 20:$\text{(-7,-1)}$(2 pt)
      \figwriten 50:$\text{(3,3)}$(2 pt)
      \figwrites 60:$\text{(5,-1)}$(2 pt)
      \figwrites 102: $\text{Graph of g'}$(10 pt)
    }
    \centerline{\box\figBoxA}
  \end{marginfigure}
%\fi

\begin{parts}
  \part Find $g(3)$ and $g(-2)$.
  \begin{solution}
    $g(x)$ can be found by calculating area under the curve $g'(x)$.
    \begin{align}
      g(3) &= g(0) + \int_0^3 g'(x)\ud x \\ 
           &= 5 + \dfrac{\pi\times 2^2}{4}+\dfrac{3}{2} \\
           &= \dfrac{13}{2}+\pi
    \end{align}
    \begin{align}
      g(-2) &= g(0) + \int_{0}^{-2} g'(x)\ud x \\ 
           &= 5 - \dfrac{\pi\times 2^2}{4} \\
           &= 5-\pi
    \end{align}
  \end{solution}

  \part Find the $x$-coordinate of each point of inflection of the graph of $y=g(x)$ on the interval
  $-7 < x < 5$. Explain your reasoning.
  \begin{solution}
    Points of inflection occur when $g'(x)$ goes from increasing to decreasing or decreasing to
    increasing. Examining the graph of $g'(x)$, the inflection points are $x=0$ (increasing to 
    decreasing), $x=2$ (decreasing to increasing) and $x=3$ (increasing to decreasing).
  \end{solution}

  \part The function $h$ is defined by $h(x)=g(x)-\frac{1}{2}x^2$. Find the $x$-coordinate of each 
  critical point of $h$, where $-7 < x < 5$, and classify each critical point as the location of a
  relative minimum, relative maximum, or neither a minimum, nor a maximum. Explain your reasoning.
  \begin{solution}
    To find the critical points, we first need to find $f'(x)$.
    \begin{align}
      f'(x) &= g'(x) - \dfrac{\ud}{\ud x}1/2 x^2 \\
            &= g'(x) - x 
    \end{align}
    At critical points, $f'(x)=0$, therefore $g'(x)=x$.
    In the interval $-2 \leq x \leq 2$,
    \begin{align}
      g'(x)=\sqrt{4-x^2}=x \Rightarrow x=\sqrt{2}
    \end{align}
    From an inspection of the Graph of $g'(x)$, we can see that the only other time $g'(x)=x$ in
    the interval $-7 < x < 5$ is when $x=3$. Since
    \begin{align}
      h'(x) = g'(x) - x > 0 \quad\text{for}\quad 0 \leq x < \sqrt{2} \\
      h'(x) = g'(x) - x \leq 0 \quad\text{for}\quad \sqrt{2} \leq x < 5
    \end{align}
    Therefore $h$ has a relative maximum at $x=\sqrt{2}$ and neither a maximum nor a minimum at 
    $x=3$.
  \end{solution}
\end{parts}

\ifprintrubric
  \begin{table}
  	\begin{tabular}{ p{5cm}p{5cm} }
  		\toprule % in brief (4-6 words), what should a grader be looking for for insights & formulations
  		  \sc{\textcolor{blue}{Insight}} & \sc{\textcolor{blue}{Formulation}} \\ 
  		\midrule % ***** Insights & formulations ******
  		\toprule % final numerical answers for the various versions
        \sc{\textcolor{blue}{If question has $\ldots$}} & \sc{\textcolor{blue}{Final answer}} \\
  		\midrule % ***** Numerical answers (below) **********
  		\bottomrule
  	\end{tabular}
  \end{table}
\fi
