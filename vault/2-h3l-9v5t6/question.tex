% This is an empty shell file placed for you by the 'examiner' script.
% You can now fill in the TeX for your question here.

% Now, down to brasstacks. ** Writing good solutions is an Art **. 
% Eventually, you will find your own style. But here are some thoughts 
% to get you started: 
%
%   1. Write the solution as if you are writing it for your favorite
%      14-17 year old to help him/her understand. Could be your nephew, 
%      your niece, a cousin perhaps or probably even you when you 
%      were that age. Just write for them.
%
%   2. Use margin-notes to "talk" to students about the critical insights
%      in the question. The tone can be - in fact, should be - informal
%
%   3. Don't shy away from creating margin-figures you think will help
%      students understand. Yes, it is a little more work per question. 
%      But the question & solution will be written only once. Make that
%      attempt at writing a solution count.
%
%   4. At the same time, do not be too verbose. A long solution can
%      - at first sight - make the student think, "God, that is a lot to know".
%      Our aim is not to scare students. Rather, our aim should be to 
%      create many "Aha!" moments everyday in classrooms around the world
% 
%   5. Ensure that there are *no spelling mistakes anywhere*. We are an 
%      education company. Bad spellings suggest that we ourselves 
%      don't have any education. Also, use American spellings by default
% 
%   6. If a question has multiple parts, then first delete lines 40-41
%   7. If a question does not have parts, then first delete lines 43-69

%\noprintanswers

\question[5] In how many ways can the six faces of a cube be coloured - using six different colours - 
so that no two cubes that have been coloured differently look the same no matter how you hold them?
\texttt{Hint:} Unroll the cube and lay its faces flat out on a table

\insertQR[15pt]{QRC}

\ifprintanswers
  % stuff to be shown only in the answer key - like explanatory margin figures
  \begin{marginfigure}
    \figinit{pt}
      \figpt 100:(20,0)
      \figpt 101:(40,0)
      \figpt 102:(40,20)
      \figpt 103:(40,40)
      \figpt 104:(60,40)
      \figpt 105:(60,60)
      \figpt 106:(40,60)
      \figpt 107:(40,80)
      \figpt 108:(20,80) 
      \figpt 109:(20,60)
      \figpt 110:(0,60)
      \figpt 111:(0,40)
      \figpt 112:(20,40)
      \figpt 113:(20,20)
      \figpt 200:$1$(30,70)
      \figpt 201:$2$(50,50)
      \figpt 202:$3$(30,30)
      \figpt 203:$4$(10,50)
    \figdrawbegin{}
      \figdrawline[100,101,102,103,104,105,106,107,108,109,110,111,112,113,100]
      \figset (dash=8)
      \figdrawline[103,106,109,112,103]
      \figset (fillmode=yes, color=0.7)
      \figdrawline[100,101,102,113,100]
    \figdrawend
    \figvisu{\figBoxA}{}{%
      \figwriten 200:(0)
      \figwritee 201:(0)
      \figwrites 202:(0)
      \figwritew 203:(0)
    }
    \centerline{\box\figBoxA}
  \end{marginfigure}
  
  \marginnote[10pt]{These $30$ cubes are called MacMohan cubes after the British mathematician Percy Alexander MacMohan}
\fi 

\begin{solution}[\halfpage]
	Unroll the cube - \asif. Faces $1,2,3$ and $4$ are the vertical faces. The coloured face is the top face
	
	Its easiest to first think of the \textit{distinct} permutations in which the vertical faces can be 
	coloured. And remember, 
	\begin{itemize}\itemsep0pt
		\item Because the cube can be held in the hand and rotated, we should be looking 
		for the number of distinct \textit{circular} permutations possible for faces $1,2,3$ and $4$ \\
		\item Each face can be the top-, bottom- or some other face - depending on how the cube held. In other words, 
		each face can be called six different things - even though nothing has changed
	\end{itemize}		
	
	
	Given the above, we get 
	\begin{align}
		N_{\texttt{total}} &= \dfrac{N_{\texttt{sides}} \times N_{\texttt{top-face}}}{6} \\
		&= \dfrac{1}{6}\cdot\left[ \encr{6}{4}\cdot (4-1)! \times 2 \text{ (remaining colours)}\right] = 30
	\end{align}
\end{solution}

