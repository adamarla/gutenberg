
\ifnumequal{\value{rolldice}}{0}{
  \renewcommand{\va}{7}
  \renewcommand{\vb}{1}
  \renewcommand{\vc}{5}
  \renewcommand{\vd}{3}
  \renewcommand{\ve}{16}
  \renewcommand{\vf}{0.845}
  \renewcommand{\vg}{0.699}
}{
  \ifnumequal{\value{rolldice}}{1}{
    \renewcommand{\va}{3}
    \renewcommand{\vb}{3}
    \renewcommand{\vc}{7}
    \renewcommand{\vd}{2}
    \renewcommand{\ve}{18}
    \renewcommand{\vf}{0.477}
    \renewcommand{\vg}{0.845}
  }{
    \ifnumequal{\value{rolldice}}{2}{
      \renewcommand{\va}{2}
      \renewcommand{\vb}{7}
      \renewcommand{\vc}{9}
      \renewcommand{\vd}{2}
      \renewcommand{\ve}{17}
      \renewcommand{\vf}{0.301}
      \renewcommand{\vg}{0.954}
    }{
      \renewcommand{\va}{5}
      \renewcommand{\vb}{4}
      \renewcommand{\vc}{3}
      \renewcommand{\vd}{2}
      \renewcommand{\ve}{21}
      \renewcommand{\vf}{0.699}
      \renewcommand{\vg}{0.477}
    }
  }
}

\POWER\va\vb\m
\POWER\vc\vd\n
\MULTIPLY\m\n\p
\EXPR[2]\q{(\ve * (\vb * \vf + \vd * \vg ) )}
\FLOOR\q\a
\ADD\a{1}\b

\question Given a number $N = \p^{\ve}$

\watchout

\begin{parts}
  \part[2] Find $\log_{10} N$ ($\log_{10}\va=\vf,\,\log_{10}\vc=\vg$)

\begin{solution}[\mcq]
    \begin{align}
      \log_{10} N &= \log_{10} \p^{\ve} = \ve\cdot\log_{10}\p \\
      &= \ve\cdot\log_{10}(\va^\vb\times\vc^\vd) \\
      &= \ve\cdot (\vb\cdot\log_{10}\va + \vd\cdot\log_{10}\vc) = \q
    \end{align}
  \end{solution}

  \part[1] How many digits would there be in an \textbf{integer} M that lies between $10^{k}$ and $10^{k+1},\,k\in\mathbb{N}$? Your answer should be in terms of $k$

\begin{solution}[\mcq]
    There are \textbf{k+1} digits in an \textbf{integer} that lies between $10^k$ and $10^{k+1}$.
    
    Think about integers that lie between $10^0 = 1$ and $10^1 = 10$. Or, for that matter
    between $10^2 = 100$ and $10^3 = 1000$.  
  \end{solution}

  \part[2] Using your answers for part (a) and (b), how many digits do you think there are in  $\p^{\ve}$? 

\begin{solution}[\mcq]
    In part (a), we found $\log_{10}\p^{\ve} = \q\implies 10^{\q} = \p^{\ve}$.
    
    It also means that $\p^{\ve} \in [10^{\a}, 10^{\b}]$. And from 
    part (b), we know that $N$ would therefore have $\b$ digits
  \end{solution}

\end{parts}

\ifprintanswers\begin{codex}$(c)\,\b\text{ digits}$\end{codex}\fi
