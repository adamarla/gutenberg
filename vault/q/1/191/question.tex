
\ifnumequal{\value{rolldice}}{0}{
  % variables 
  \renewcommand{\va}{15}
}{
  \ifnumequal{\value{rolldice}}{1}{
    % variables 
    \renewcommand{\va}{30}
  }{
    \ifnumequal{\value{rolldice}}{2}{
      % variables 
      \renewcommand{\va}{45}
    }{
      % variables 
      \renewcommand{\va}{60}
    }
  }
}

\FRACTIONSIMPLIFY\va{180}\vp\vq
\MULTIPLY\vp{2}\vr
\SUBTRACT\vq\vp\vs
\FRACTIONSIMPLIFY\vr\vs\vx\vy
\FRACADD\vy\vx{1}{1}\vm\vn
\FRACMULT\vn\vm{60}{1}\vi\vj

\FRACMULT\vy\vx{2}{1}\vt\vu
\FRACADD\vt\vu{1}{1}\vv\vw
\FRACMULT\vv\vw\vi\vj\p\q

\question[3] The angles of a triangle are in arithmetic progression and the ratio of 
the number of \textit{degrees} of the \textit{smallest} angle to the number of 
\textit{radians} of the \textit{largest} angle is $\frac\va\pi$. Find the angles 
of the triangle - in degrees 

\watchout

\begin{solution}[\halfpage]
  \textbf{Insight \#1}
  
  If $\theta_D$ is the measure of an angle \textbf{in degrees} and $\theta_R$ the measure 
  of the same angle \textbf{in radians}, then 
  \[ \theta_R = \pi\cdot\dfrac{\theta_D}{180}\text{ and }\theta_D = \dfrac{180}\pi\cdot\theta_R \]

  Hence, if the three angles be $A,B$ and $C,\,(A < B < C)$, then 
  \[ A = a\qquad B = a+d\qquad C = a + 2d \] 
  Moreover
  \begin{align}
    \dfrac{A_D}{C_R} &= \dfrac{A_D}{\frac\pi{180}\cdot C_D} = \dfrac\va\pi\implies\dfrac{A_D}{C_D}=\dfrac\vp\vq \\
    \implies \dfrac{a}{a+2d} &= \dfrac\vp\vq\implies a = \WRITEFRAC\vx\vy\cdot d\implies d = \WRITEFRAC\vy\vx\cdot a 
  \end{align} 

  And as 
  \begin{align}
    a + (a+d) + (a+2d) &=\ang{180}\implies (a+d) = B = \ang{60} \\
    \text{also } a+d &=\ang{60}\implies a\cdot\left(1 + \WRITEFRAC\vy\vx\right) = \ang{60} \\
    &\implies a = \WRITEFRAC\vi\vj^\circ = A \\
    \therefore C = a + 2d &= \WRITEFRAC\vi\vj\cdot\left( 1+2\times\WRITEFRAC\vy\vx \right) = \WRITEFRAC\p\q^\circ
  \end{align}
\end{solution}

\ifprintanswers
  \begin{codex}
    $\WRITEFRAC\vi\vj^\circ,\,\ang{60}\text{ and }\WRITEFRAC\p\q^\circ$
  \end{codex}
\fi
