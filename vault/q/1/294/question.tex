

 %\printrubric

\ifnumequal{\value{rolldice}}{0}{
  % variables 
  \renewcommand{\va}{SUCCESS }
  \renewcommand{\vb}{S } % do not begin with
  \renewcommand{\vc}{C } % end with
  \renewcommand{\vd}{3} % no-begin count 
  \renewcommand{\ve}{2} % end count
  \renewcommand{\vf}{7} % word-length
  \renewcommand{\vg}{60}
}{
  \ifnumequal{\value{rolldice}}{1}{
    % variables 
    \renewcommand{\va}{ALBATROSS }
    \renewcommand{\vb}{A }
    \renewcommand{\vc}{S }
    \renewcommand{\vd}{2}
    \renewcommand{\ve}{2}
    \renewcommand{\vf}{9}
    \renewcommand{\vg}{15,120}
  }{
    \ifnumequal{\value{rolldice}}{2}{
      % variables 
      \renewcommand{\va}{DAMASCUS }
      \renewcommand{\vb}{S }
      \renewcommand{\vc}{A }
      \renewcommand{\vd}{2}
      \renewcommand{\ve}{2}
      \renewcommand{\vf}{8}
      \renewcommand{\vg}{1800}
    }{
      % variables 
      \renewcommand{\va}{BENEFICIAL }
      \renewcommand{\vb}{I }
      \renewcommand{\vc}{E }
      \renewcommand{\vd}{2}
      \renewcommand{\ve}{2}
      \renewcommand{\vf}{10}
      \renewcommand{\vg}{141,120}
    }
  }
}

\SUBTRACT\vf{1}\vx
\SUBTRACT\vx\vd\vy
\SUBTRACT\vf{2}\vz

\question[4] How many \textit{distinct} words can be made by jumbling \va so that none of the words
begins with a \vb but all end with a \vc

\watchout[-10pt]

\ifprintanswers
  \marginnote[20pt]{We start out by treating all characters as different - even if they are actually the same. Later on we 
  will adjust the answer to account for the "same-ness" of characters}
\fi 

\begin{solution}[\halfpage]
	The word is $\vf$ characters long - with $\vd$ \vb 's and $\ve$ \vc 's. We therefore have $\ve$
	 options for the \textit{last} character
	 
	 Once we fix the last character, we would ordinarily have $\vx$ options for the first character. 
	 But we cannot use the $\vd$ \vb 's. And so, we are down to $\vy$ options for the first character
	 
	 For the middle characters, however, there are no constraints. The remaining $\vz = (\vf - 2)$ characters 
	 can some in any permutation
	 
	 Moreover, given that we have more than one occurrence of the \vb and \vc, we must discount 
	 permutations that are essentially the same word. And so, the number of distinct words is 
	 
	 \begin{align}
	 	N &= \dfrac{\vy \times \vz ! \times \ve}{\vd ! \times \ve !} = \vg
	 \end{align}
\end{solution}

\ifprintanswers
  \begin{codex}
    $\vg$
  \end{codex}
\fi
