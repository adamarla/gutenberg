\documentclass[14pt,fleqn]{extarticle}
\RequirePackage{prepwell-eng}

\newcommand\yx{\left(\frac{y}{x} \right)}
\newcommand\ymx{\left(y-x \right)}

\previewoff

\begin{document} 
\begin{problem}
	\statement 
    
     Find the equation of the curve passing through origin if the slope of the 
     tangent to the curve at any point $(x,y)$ is equal to the square 
     of the difference of the abscissa and ordinate of the point 
     
     \begin{step}
  \begin{options} 
     \correct 
      
     The required curve satisfies the following condition
     \[ \qquad\quad \dydx = (y-x)^2 \]
     
     \incorrect
     
     The required curve satisfies the following condition
     \[ \qquad\quad \dydx = y^2 - x^2 \]
        
    \end{options} 
     \reason 
     
     Abscissa and ordinate are \underline{fancy words} for $x$ and $y$ respectively 
     (as shown below) 
     \[ \qquad\qquad (\overbrace{x}^{\text{Abscissa}}, \overbrace{y}^{\text{Ordinate}} )\]
     
     And hence, as per the problem, the required curve satisfies the following the condition 
     \[ \qquad \underbrace{\dydx = (y-x)^2 \text{ or } \dydx = (x-y)^2}_{\text{Either is fine}}\]
       
\end{step} 

\begin{step}
  \begin{options} 
     \correct 
       
     Of the following three strategies, \underline{strategy $III$} is the best
     
     \begin{center}
  \begin{tabular}{NNl}
   \toprule
         & \text{Re-write as} & Solve using \\
   \midrule 
   I & \dydx + 2yx = y^2 + x^2 & Integrating factor \\
    \midrule 
    II & \dydx = x^2 \left[\yx^2 - 2\yx + 1 \right] & $ y = vx$ \\
    \midrule 
    III & \dydx = (y-x)^2 & Let $z = y-x$ \\
    \bottomrule
  \end{tabular}
\end{center}
     \incorrect
     
     Of the following three strategies, \underline{strategy $II$} is the best
     
     \begin{center}
  \begin{tabular}{NNl}
   \toprule
         & \text{Re-write as} & Solve using \\
   \midrule 
   I & \dydx + 2yx = y^2 + x^2 & Integrating factor \\
    \midrule 
    II & \dydx = x^2 \left[\yx^2 - 2\yx + 1 \right] & $ y = vx$ \\
    \midrule 
    III & \dydx = (y-x)^2 & Let $z = y-x$ \\
    \bottomrule
  \end{tabular}
\end{center}
        
    \end{options} 
     \reason

The table below summarizes why the given differential equation \underline{is not} a linear or a homogenous differential equation. Hence, it \underline{cannot} be solved as those 
\begin{center}
  \begin{tabular}{NlN}
   \toprule
        \text{If written as} & Looks like & \text{Is not because}  \\
   \midrule 
   \dydx + 2xy = x^2 + y^2 & Linear & \dydx + 2xy \neq F(x) \\
    \midrule 
    \dydx = x^2 \left[\yx^2 - 2\yx + 1 \right] & Homogenous & \dydx \neq F\yx \\
    \bottomrule
  \end{tabular}
\end{center}
    
    Which leaves \underline{strategy $III$} as the best option. So, let us 
    try and work with that 

\end{step}

\begin{step}
  \begin{options} 
     \correct 
       
     If $z = y-x$ then 
     \begin{align}
	\frac{dz}{z^2-1} = dx &\implies \int \frac{dz}{z^2-1} = \int dx \\
	\text{or } x &= \frac{1}{2}\log\left\vert \frac{y-x-1}{y-x+1}\right\vert + C
\end{align}

But as $y = 0$ when $x=0$, therefore $C = 0$. Hence, 
\[ \quad \underbrace{y = e^{2x}\cdot \left(y-x+1 \right) + (x+1)}_{\text{Answer}}\]
     \incorrect
     
     If $z = y-x$ then 
     \begin{align}
	\frac{dz}{z^2-1} &= dx \implies \int \frac{dz}{z^2-1} = \int dx \\
	\text{or } x &= \log\left\vert \ymx + \sqrt{\ymx^2 + 1}\right\vert + C
\end{align}

But as $y = 0$ when $x=0$, therefore $C = 0$. Hence, 
\[ \quad \underbrace{e^x = \ymx + \sqrt{\ymx^2 + 1}}_{\text{Answer}}\]   
    \end{options} 
     \reason 
     
     \begin{align}
     z = y -x &\implies \frac{dz}{dx} = \dydx - 1\\
     \therefore \frac{dz}{dx} + 1 &= \ymx^2 = z^2  \\
     \implies \frac{dz}{z^2 - 1} &= dx \text{ or } \int \frac{dz}{z^2 - 1} = \int dx \\
     \implies x &= \frac{1}{2} \log \left\vert \dfrac{z-1}{z + 1}\right\vert + C \\
     &= \frac{1}{2}\log \left\vert \dfrac{\ymx - 1}{\ymx+ 1} \right\vert + C 
\end{align}

But the curve \underline{passes through the origin}. Hence,
\begin{align}
0 &= \underbrace{\frac{1}{2}\log \left\vert\dfrac{(0-0) - 1}{(0-0) + 1} \right\vert + C \implies C = 0}_{\log 1 = 0}
\end{align}

And so, 
\begin{align}
x &= \frac{1}{2}\log\left\vert \dfrac{\ymx - 1}{\ymx + 1}\right\vert  \\
\implies e^{2x} &= \underbrace{\dfrac{\ymx - 1}{\ymx + 1}}_{\log a = b \implies a = e^b} \\
\text{or } y &= e^{2x} \left[\ymx+1 \right] + \left(x+1 \right)
\end{align}
       
\end{step}
\end{problem} 
\end{document} 