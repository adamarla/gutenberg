


\ifnumequal{\value{rolldice}}{0}{
  % variables 
  \renewcommand{\va}{1}
  \renewcommand{\vb}{2}
  \renewcommand{\vc}{3}
  \renewcommand{\vd}{5}
}{
  \ifnumequal{\value{rolldice}}{1}{
    % variables 
    \renewcommand{\va}{3}
    \renewcommand{\vb}{6}
    \renewcommand{\vc}{7}
    \renewcommand{\vd}{9}
  }{
    \ifnumequal{\value{rolldice}}{2}{
      % variables 
      \renewcommand{\va}{2}
      \renewcommand{\vb}{7}
      \renewcommand{\vc}{5}
      \renewcommand{\vd}{9}
    }{
      % variables 
      \renewcommand{\va}{3}
      \renewcommand{\vb}{8}
      \renewcommand{\vc}{5}
      \renewcommand{\vd}{13}
     }
  }
}

\SUBTRACT\vd\vb\ve
\SUBTRACT\vc\va\vf
\EXPR[2]\vm{\ve / \vf}

\DEGREESTAN{15}\p
\ROUND[2]\p\q

\ADD\vm\q\r
\MULTIPLY\vm\q\s
\SUBTRACT{1}\s\t
\EXPR[2]\vz{\r / \t}

\MULTIPLY\vz{-\va}\vp
\EXPR[2]\vq{\vp + \vb}

\question[3] If the line joining the points $A = (\va, \vb)$ and $B = (\vc, \vd)$ 
\ is rotated \textit{counter-clockwise} about $A$ by $\ang{15}$, then what is the equation 
of the resulting new line ? 

\begin{calcaid}
  $\tan\ang{15}=\q$
\end{calcaid}

\watchout

\begin{solution}[\halfpage]
    First, the slope of $AB$ is given by 
    \begin{align}
        m &= \tan\theta = \dfrac{\vd - \vb}{\vc - \va} = \vm
    \end{align}

    The slope of the \textit{new} line would therefore be 
    \[ m_2 = \tan(\theta + \ang{15}) = \dfrac{\tan\theta + \tan\ang{15}}{1-\tan\theta\cdot\tan\ang{15}} = \vz \] 
    
    Also, as the \textbf{new} line came about because of rotation around $A$, the new 
    line also passes through $A$. 
    
    And hence, its equation would be
    \begin{align}
      \dfrac{y-\vb}{x-\va} &= \vz  \\
      \implies y &= \vz x + \vq
    \end{align}
\end{solution}

\ifprintanswers
  \begin{codex}
    $y = \vz x + \vq$
  \end{codex}
\fi
