\documentclass[14pt,fleqn]{extarticle}
\RequirePackage{prepwell}
\previewoff
\begin{document}


\newcommand\ec{e^{\cos x}}
\newcommand\emc{e^{-\cos x}}
\newcommand\dnm{\ec + \emc} 

%text
Evaluate the following \[ \qquad \int_0^\pi \frac{\ec}{\dnm}\cdot dx \]
%

\newcard

\begin{align}
	A &= \int_0^\pi \frac{\ec}{\dnm}\cdot dx = \int_0^\pi \frac{\emc}{\dnm}\cdot dx 
\end{align}

\newcard 

When evaluating definite integrals, we have access to some additional results that don't apply to indefinite integrals \newline 

One of these is \[\qquad \int_0^a f(x)\cdot dx = \int_0^a f \left(a-x \right)\cdot dx\]

When applied to the given integral, we get 
\begin{align}
	A &= \int_0^\pi \frac{e^{\cos \left(\pi -x \right)}}{e^{\cos \left(\pi -x \right)} + e^{\cos \left(\pi - x \right)}}\cdot dx  \\
	&= \int_0^\pi \frac{\emc}{\emc + \ec}\cdot dx 
\end{align}

\newcard 

\[ \qquad A +A = \pi \implies A = \frac\pi{2} \]

\newcard 

\begin{align}
A &= \int_0^\pi \frac{\ec}{\dnm}\cdot dx \\ 
&= \int_0^\pi \frac{\emc}{\dnm}\cdot dx \\
\therefore A + A &= \int_0^\pi \left[\frac{\ec + \emc}{\dnm} \right]\cdot dx \\
\implies 2A &= \int_0^\pi dx = \pi \text{ or } A = \frac\pi{2} 
\end{align}

\end{document}