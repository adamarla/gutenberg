\documentclass[14pt,fleqn]{extarticle}
\RequirePackage{prepwell-eng}

\newcommand\hpi{\frac\pi{2}}
\newcommand\fxa{\dfrac{1-\sin^3 x}{3\cos^2x}}
\newcommand\fxb{\dfrac{q \left(1-\sin x \right)}{\left(\pi-2x \right)^2}}
\newcommand\lhl{\lim_{x\to\hpi^-}}
\newcommand\rhl{\lim_{x\to\hpi^+}}
\newcommand\hz{\frac{z}{2}}

\previewoff

\begin{document} 
\begin{problem}
	\statement 
    
     Find the value of $p$ and $q$ for which the following function is 
     continuous at $x=\dfrac\pi{2}$ 
     
     \[ \quad f(x) = \begin{cases}
	\fxa, &\text{ if }x<\hpi \\
	p, &\text{ if } x = \hpi \\
	\fxb,&\text{ if } x > \hpi 
	
\end{cases}\] 

\begin{step}
  \begin{options} 
     \correct 
     
     For $f(x)$ to be continuous at $x=\hpi$
     \[ \lhl \fxa = p = \rhl\fxb \]  
       
     \incorrect
     
     For $f(x)$ to be continuous at $x=\hpi$
     \[ \quad \lhl \fxa = \rhl\fxb \]  
        
    \end{options} 
     \reason 
     
     $f(x)$ will be continuous at $x=\hpi$ when 
     \[ \qquad \underbrace{\lhl f(x)}_{\text{Left-hand limit}} = \underbrace{f \left(\hpi \right)}_{p} = \underbrace{\rhl f(x)}_{\text{Right-hand limit}} \]
     
     This is the standard condition for $f(x)$ to be continuous at $x = a$ 
       
\end{step}

\begin{step}
  \begin{options} 
     \correct 
     
     Given below are two strategies for finding 
     \[ \qquad \lhl\fxa \]
     \begin{center}
  \begin{tabular}{NN}
   \toprule
        \text{Strategy} &  \text{Rewrite }\lhl f(x)\text{ as }\\
   \midrule 
   A & \lim_{z\to 0^-} \left[ \dfrac{1-\cos^3 z}{3\sin^2 z}\right] \,\text{ where } z = \hpi-x\\
    \midrule 
   B & \lhl \left[\dfrac{\sin x + 1 -\sin^3x -\sin x}{3\cos^2x} \right]\\
    \bottomrule
  \end{tabular}
\end{center}  

\underline{Strategy $B$} is better and it gives 
\[ \qquad \lhl\fxa = \frac{1}{2} \]
       
     \incorrect
     
     Given below are two strategies for finding 
     \[ \qquad \lhl\fxa \]
     \begin{center}
  \begin{tabular}{NN}
   \toprule
        \text{Strategy} &  \text{Rewrite }\lhl f(x)\text{ as }\\
   \midrule 
   A & \lim_{z\to 0^-} \left[ \dfrac{1-\cos^3 z}{3\sin^2 z}\right] \,\text{ where } z = \hpi-x\\
    \midrule 
   B & \lhl \left[\dfrac{\sin x + 1 -\sin^3x -\sin x}{3\cos^2x} \right]\\
    \bottomrule
  \end{tabular}
\end{center}  

\underline{Strategy $A$} is better and it gives 
\[ \qquad \lhl\fxa = \frac{1}{3} \]
       
        
    \end{options} 
     \reason 
     
     You can try Strategy $A \left(z = \hpi -x \right)$. But the resulting expression 
     \[  \lim_{z\to 0^-} \dfrac{1-\sin^3 \left(\hpi - z \right)}{3\sin^2 \left(\hpi - z \right)} = \lim_{z\to 0^-} \dfrac{1-\cos^2 z}{3\sin^2 z} \]
       is not that much more elegant or useful \newline 
       
       But if we \underline{add and subtract} $\sin x$ in the numerator \underline{only}, then 
       \begin{align}
       \lhl f(x) &= \lhl \left[\dfrac{1 + \sin x -\sin x -\sin^3 x}{3\cos^2 x} \right] \\[-10pt]
       &= \lhl \left[\dfrac{\left(1-\sin x \right) + \sin x \left(1-\sin^2  x \right)}{3\cos^2 x} \right] \\
       &= \frac{1}{3}\lhl \left[ \frac{1-\sin x}{\cos^2 x} + \frac{\sin x\cdot\cos^2}{\cos^2 x}\right] \\
       &= \frac{1}{3} \lhl \left[\frac{1-\sin x}{1-\sin^2 x} + \sin x \right] \\
       &= \frac{1}{3}\lhl \left[\frac{1}{1+\sin x} + \sin x \right] \\
       &= \frac{1}{3} \left[\frac{1}{1+1} + 1 \right] = \frac{1}{2}
\end{align}
\end{step}

\begin{step}
  \begin{options} 
     \correct 
       
       Given below are two strategies for finding 
       \[ \qquad \rhl \fxb \]
       \begin{center}
  \begin{tabular}{NN}
   \toprule
        \text{Strategy} & \text{Substitution} \\
   \midrule 
   I & \text{Let } z = x - \hpi \implies 2z = 2x - \pi \\
    \midrule 
    II & \text{Let }x = 2\pi - z \\
    \bottomrule
  \end{tabular}
\end{center}



Of these, \underline{Strategy $I$} is better and will yield 
\begin{align}
\rhl\fxb &= \lim_{z\to0^+} \left[\frac{2\sin^2 \hz}{16 \left(\hz \right)^2}\right] = \frac{q}{8}
\end{align}

And therefore, for $f(x)$ to be continuous 
\[ \quad \frac{1}{2} = p = \frac{q}{8} \implies p = \frac{1}{2}, q = 4 \]
        
    \end{options} 
     \reason 
     
     We know that $\lim_{x\to 0}\dfrac{\sin x}{x} = 1$. There is nothing 
     in the expression that looks exactly like this. But note that 
     \[ \qquad \pi - 2x \to 0^+ \text{ as } x\to \hpi^+ \]
       
     Hence, let us try $I$ 
     \begin{align}
     z = x-\hpi &\implies 2x-\pi = 2z \\
     \therefore \rhl f(x) &= \underbrace{\lim_{z\to 0^+}\left[\frac{q \left(1-\sin \left(z+\hpi \right) \right)}{\left(-2z \right)^2} \right]}_{\text{Note that }z\to 0^+\text{ as } x\to\hpi^+} \\[-10pt]
     &= \lim_{z\to 0^+} \left[\dfrac{q \left(1-\cos z \right)}{4z^2} \right] \\
     &= \underbrace{\lim_{z\to 0^+} \left[\dfrac{q\cdot 2\sin^2\hz}{4 \left(\hz \right)^2\cdot 4}\right]}_{1-\cos2\theta=2\sin^2\theta} \\[5pt]
     &= \underbrace{\frac{q}{8}\lim_{z\to 0^+} \left[\dfrac{\sin \hz}{\hz}\cdot\dfrac{\sin\hz}{\hz} \right] = \frac{q}{8} }_{\lim_{\theta\to 0}\frac{\sin\theta}{\theta} = 1}
\end{align}

Hence, for $f(x)$ to be continuous at $x=\hpi$
\begin{align}
\lhl f(x) &= p = \rhl f(x) \\
\implies \frac{1}{2} &= p = \frac{q}{8} \text{ or }p = \frac{1}{2}\text{ and } q = 4 
\end{align}
\end{step}
\end{problem} 
\end{document} 