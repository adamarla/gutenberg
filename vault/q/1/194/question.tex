

\ifnumequal{\value{rolldice}}{0}{
  % variables 
  \renewcommand{\va}{1}
  \renewcommand{\vb}{7}
  \renewcommand{\vc}{3}
}{
  \ifnumequal{\value{rolldice}}{1}{
    % variables 
    \renewcommand{\va}{3}
    \renewcommand{\vb}{7}
    \renewcommand{\vc}{5}
  }{
    \ifnumequal{\value{rolldice}}{2}{
      % variables 
      \renewcommand{\va}{9}
      \renewcommand{\vb}{5}
      \renewcommand{\vc}{7}
    }{
      % variables 
      \renewcommand{\va}{11}
      \renewcommand{\vb}{5}
      \renewcommand{\vc}{3}
    }
  }
}

\ADD\va\vb\a
\ADD\a\vc\vd

\FRACADD\va{2}\vb{2}\p\d
\FRACMINUS\va{2}\vb{2}\b\d
\ABSVALUE\b\q

\FRACADD\vc{2}\vd{2}\r\d
\FRACMINUS\vc{2}\vd{2}\c\d
\ABSVALUE\c\s

\FRACADD\r{2}\q{2}\vz\d
\FRACMINUS\r{2}\q{2}\k\d
\ABSVALUE\k\m

\question[3] For what value of $N$ would the following hold? 
\[ \sin\dfrac{\va\theta}{2}\sin\dfrac{\vb\theta}{2} + 
\sin\dfrac{\vc\theta}{2}\sin\dfrac{\vd\theta}{2} = \sin N\theta\cdot\sin\m\theta \]

\watchout

\ifprintanswers
\fi 

\begin{solution}[\halfpage]
  \begin{align}
      &\sin\dfrac{\va\theta}{2}\sin\dfrac{\vb\theta}{2} + \sin\dfrac{\vc\theta}{2}\sin\dfrac{\vd\theta}{2} \nonumber\\
      &= \dfrac{1}{2}\cdot\left[ 2\cdot\sin\dfrac{\va\theta}{2}\sin\dfrac{\vb\theta}{2} + 
         2\cdot\sin\dfrac{\vc\theta}{2}\sin\dfrac{\vd\theta}{2} \right] \\
       &= \dfrac{1}{2}\overbrace{\left[ \cos\left( \dfrac{\va\theta}{2} + \dfrac{\vb\theta}{2}\right)
           - \cos\left( \dfrac{\vb\theta}{2} - \dfrac{\va\theta}{2}\right)\right]}
           ^{2\sin A\sin B=\cos\dfrac{A+B}{2}-\cos\dfrac{B-A}{2}}\nonumber\\
       &+ \dfrac{1}{2}\left[ \cos\left( \dfrac{\vc\theta}{2} + \dfrac{\vd\theta}{2}\right) 
           - \cos\left( \dfrac{\vc\theta}{2} - \dfrac{\vd\theta}{2}\right)\right] \\
       &= \dfrac{1}{2}\left[ (\cos \p\theta - \cos\q\theta) + ( \cos\r\theta - \cos\s\theta)\right] \\ 
       &= \dfrac{1}{2}\left( \cos\r\theta - \cos\q\theta \right) \\
       &= \dfrac{1}{2}\cdot 2\cdot\sin\left( \dfrac{\r\theta + \q\theta}{2}\right)\sin\left( \dfrac{\r\theta - \q\theta}{2}\right) \\
       &= \sin\vz\theta\cdot\sin\m\theta \implies N = \vz
  \end{align}
\end{solution}
\ifprintanswers
  \begin{codex}
    $\vz$
  \end{codex}
\fi
