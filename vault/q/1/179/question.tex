
\ifnumequal{\value{rolldice}}{0}{
  \renewcommand{\va}{3}
  \renewcommand{\vb}{2}
  \renewcommand{\vc}{7}
  \renewcommand{\vd}{4}
  \renewcommand{\ve}{6}
  \renewcommand{\vf}{0,1,2}
  \renewcommand{\vg}{3,4,5,6,7}
  \renewcommand{\vh}{4,6}
  \renewcommand{\vi}{3,5,7}
}{
  \ifnumequal{\value{rolldice}}{1}{
    \renewcommand{\va}{2}
    \renewcommand{\vb}{1}
    \renewcommand{\vc}{5}
    \renewcommand{\vd}{3}
    \renewcommand{\ve}{5}
    \renewcommand{\vf}{0,1} % = A
    \renewcommand{\vg}{2,3,4,5} % = B
    \renewcommand{\vh}{3,5} % B \cap C
    \renewcommand{\vi}{2,4} % B - C 
  }{
    \ifnumequal{\value{rolldice}}{2}{
      \renewcommand{\va}{3}
      \renewcommand{\vb}{3}
      \renewcommand{\vc}{6}
      \renewcommand{\vd}{5}
      \renewcommand{\ve}{7}
      \renewcommand{\vf}{0,1,2}
      \renewcommand{\vg}{4,5,6}
      \renewcommand{\vh}{5}
      \renewcommand{\vi}{4,6}
    }{
      \renewcommand{\va}{2}
      \renewcommand{\vb}{2}
      \renewcommand{\vc}{6}
      \renewcommand{\vd}{4}
      \renewcommand{\ve}{5}
      \renewcommand{\vf}{0,1}
      \renewcommand{\vg}{3,4,5,6}
      \renewcommand{\vh}{4,5}
      \renewcommand{\vi}{3,6}
    }
  }
}

\providecommand\qcnuwrt[1] { 
  (\vz,#1),
} 

\providecommand\qcnucaploop[1]{
  \renewcommand\vz{#1}
  \expandafter\forcsvlist\expandafter\qcnuwrt\expandafter{\vh}
}

\providecommand\qcnuminuslp[1]{
  \renewcommand\vz{#1}
  \expandafter\forcsvlist\expandafter\qcnuwrt\expandafter{\vi}
}

\question If $A = \{ x:x\in\mathbb{W}, x < \va\}, B = \{ x:x\in\mathbb{N}, \vb<x\leq\vc\}$ and 
$C=\{\vd,\ve\}$, then find the following given that $\mathbb{W}$ is the set of whole numbers

\watchout

\begin{parts}
  \part[2] $A\times (B\cap C)$

  \begin{explanation}
    As per the definitions of $A,B$ and $C$, 
    \begin{align}
      A &= \{ \vf \} \\
      B &= \{ \vg \} \text{ and } \\
      C &= \{ \vd,\ve \} \\
      \implies (B\cap C) &= \{ \vh \} \\
      \text{ and }\therefore A\times (B\cap C) &= 
      \{\expandafter\forcsvlist\expandafter\qcnucaploop\expandafter{\vf} \}
    \end{align}
  \end{explanation}

  \part[2] $A\times(B - C)$

  \begin{explanation}
    Using the same sets $A,B$ and $C$, we get 
    \begin{align}
      B - C &= \{ \vi \} \\
      \text{ and }\therefore A\times (B\cup C) &=
      \{ \expandafter\forcsvlist\expandafter\qcnuminuslp\expandafter{\vf} \}
    \end{align}
  \end{explanation}
\end{parts}

\ifprintanswers
  \begin{codex}
    \begin{tabular}{l}
    $(a)\,\{\expandafter\forcsvlist\expandafter\qcnucaploop\expandafter{\vf} \}$ \\
    $(b)\,\{ \expandafter\forcsvlist\expandafter\qcnuminuslp\expandafter{\vf} \}$
    \end{tabular}
  \end{codex}
\fi

