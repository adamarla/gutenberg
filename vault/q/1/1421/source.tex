\documentclass[14pt,fleqn]{extarticle}
\RequirePackage{prepwell-eng}

\newcommand\nextrd{\frac{55}{72}}
\newcommand\bwins{\frac{5}{72}}
\newcommand\expa{ \left(\nextrd \right)}

\previewoff 

\begin{document} 
\begin{problem}
	\statement 
    
    $A$ and $B$ throw a pair of dice alternately. 
    $A$ wins the game if he gets a total of $7$ and $B$ wins the game if 
    he gets a total of $10$. If $A$ starts the game, then find the 
    probability that $B$ wins

    \begin{step}
  \begin{options} 
     \correct 
      
      Favourable outcomes for $A$ and $B$ are as follows where 
      $\left(a,b \right)$ are the 
      values on the first and second die respectively 
      
      \begin{center}
  \begin{tabular}{NN}
   \toprule
        A & B \\
   \midrule 
   \left(1,6 \right) & \left(4,6 \right) \\
    \midrule 
    \left(2,5 \right) & \left(5,5 \right) \\
    \midrule 
    \left(3,4 \right) & \left(6,4 \right) \\
    \midrule
    \left(6,1 \right) & \\
    \midrule
    \left(5,2 \right) & \\
    \midrule
    \left(4,3 \right) & \\
    \bottomrule
  \end{tabular}
\end{center}

     \incorrect
     
     Favourable outcomes for $A$ and $B$ are as follows where 
      $\left(a,b \right)$ are the 
      values on the first and second die respectively 
      
      \begin{center}
  \begin{tabular}{NN}
   \toprule
        A & B \\
   \midrule 
   \left(1,6 \right) & \left(4,6 \right) \\
    \midrule 
    \left(2,5 \right) & \left(5,5 \right) \\
    \midrule 
    \left(3,4 \right) & \\
    \bottomrule
  \end{tabular}
\end{center}
        
    \end{options} 
     \reason 
     
     There are two ways of rolling a sum of $a+b$ with two dice -- $\left(a,b \right)$ or $\left(b,a \right)$\newline 
     
     Given this, below are the ways in which one can roll a total of $7$ ($A$ wins) or $10$ ($B$ wins) 
     
     \begin{center}
  \begin{tabular}{NN}
   \toprule
        A & B \\
   \midrule 
   \left(1,6 \right) & \left(4,6 \right) \\
    \midrule 
    \left(2,5 \right) & \left(5,5 \right) \\
    \midrule 
    \left(3,4 \right) & \left(6,4 \right) \\
    \midrule
    \left(6,1 \right) & \\
    \midrule
    \left(5,2 \right) & \\
    \midrule
    \left(4,3 \right) & \\
    \bottomrule
  \end{tabular}
\end{center}
\end{step}  

\begin{step}
  \begin{options} 
     \correct 
       
      Let a round be defined as $A$ rolling the pair of dice first followed by $B$ doing the same\newline 
      
     Given this
     \begin{center}
  \begin{tabular}{cN}
   \toprule
       Event & \text{Probability} \\
   \midrule 
   No one wins in a round & \nextrd \\
    \midrule 
    $B$ wins in the $N-$th round & \left(\nextrd \right)^{N-1}\cdot\bwins \\
    \bottomrule
  \end{tabular}
\end{center}

     \incorrect
     
     Let a round be defined as $A$ rolling the pair of dice first followed by $B$ doing the same\newline 
      
     Given this
     \begin{center}
  \begin{tabular}{cN}
   \toprule
       Event & \text{Probability} \\
   \midrule 
   No one wins in a round & \nextrd \\
    \midrule 
    $B$ wins in the $N-$th round & \left(\nextrd\right)^{N-1}\cdot\frac{1}{12}\\
    \bottomrule
  \end{tabular}
\end{center}
        
    \end{options} 
     \reason 
     
     For $B$ to win, $A$ should \underline{not} already have won the game by rolling a 
     total of $7$ \newline 
     
     Hence, for $B$ to win \underline{in the $N-$th round}, no one should
     have won in the \underline{preceding} $N-1$ rounds\newline 
    
    \begin{center}
  \begin{tabular}{cN}
   \toprule
        Event & \text{Probability} \\
   \midrule 
   $A$ rolls $7$ & \frac{6}{36} = \frac{1}{6} \\
    \midrule 
    $B$ rolls $10$ & \frac{3}{36} = \frac{1}{12} \\
    \midrule 
    No one wins in a round & \underbrace{\frac{5}{6}\cdot\frac{11}{12} = \nextrd}_{\left(1-\frac{1}{6} \right)\cdot \left(1-\frac{1}{12} \right)}\\
    \midrule
    $B$ wins in a round & \underbrace{\frac{5}{6}\cdot\frac{1}{12} = \bwins}_{\text{$A$ doesn't win but $B$ does}} \\ 
    \midrule
    $B$ wins in the $N-$th round & \left(\nextrd \right)^{N-1}\cdot\bwins \\
    \bottomrule
  \end{tabular}
\end{center} 
       
\end{step}

\begin{step}
  \begin{options} 
     \correct 
      
     The probability that $B$ wins is therefore 
     \[ \quad\prob{B} = \sum_{N=1}^\infty \left(\nextrd \right)^{N-1}\cdot\bwins = \frac{5}{17}\]
        
    \end{options} 
     \reason 
     
     $B$ could win in the $1-$st round or the $100-$th round or the $1000-$th round. 
     Hence, the \underline{total probability} of $B$ winning is 
     
     \begin{align}
     P &= \expa^0\bwins + \expa^1\bwins + \cdots \infty \\[10pt]
     &= \frac{5}{72}\underbrace{\sum_{N=1}^\infty\expa^{N-1}}_{\text{Infinite geometric series}} \\[5pt]
     &= \bwins \left[\dfrac{1}{1-\nextrd} \right] = \bwins\cdot\frac{72}{17} = \frac{5}{17} 
     \end{align}

\end{step}
\end{problem} 
\end{document} 