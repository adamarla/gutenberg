\documentclass[14pt,fleqn]{extarticle}
\RequirePackage{prepwell-eng}

\newcommand\fx{x\cdot e^x} 
\newcommand\dfx{\left(x+1 \right)\cdot e^x } 
\newcommand\ddfx{ \left(x+2 \right)\cdot e^x }

\previewoff 

\begin{document} 
\begin{problem}
	\statement 
    
     Find the minimum (or maximum) value of 
     \[ \qquad \qquad f(x) = x\cdot e^x \] 
     
     \begin{step}
  \begin{options} 
     \correct 
     
     $f'(x) = \dfx$. And therefore, $f(x)$ has an extrema at $x = -1$
       
     \incorrect
     
     $f'(x) = \dfx$. And therefore, $f(x)$ has an extrema at $x = 0,-1$   
        
    \end{options} 
     \reason 
     
     \begin{align}
     f'(x) &= \underbrace{x\ddx e^x + e^x \ddx x }_{\text{Product Rule}} \\
     &= x\cdot e^x + e^x = \dfx 
\end{align}

Now, $e^x \neq 0$ for all values of $x$. Hence,  
\[ \qquad\qquad f'(x) = 0 \implies x = -1 \]
       
\end{step}

\begin{step}
  \begin{options} 
     \correct 
      
      \begin{center}
  \begin{tabular}{NcN}
   \toprule
        & Type & \text{Value} \\
   \midrule 
   x = -1 & Minima & -\frac{1}{e} \\
    \bottomrule
  \end{tabular}
\end{center}

     \incorrect


      \begin{center}
  \begin{tabular}{NcN}
   \toprule
        & Type & \text{Value} \\
   \midrule 
   x = -1 & Minima & -\frac{1}{2e} \\
    \bottomrule
  \end{tabular}
\end{center}
        
    \end{options} 
     \reason 
       
     We know that $f'(x) = 0$ at $x=-1$. To know, however, whether it is a maxima or a minima, one must look at $f''(x)$ 
     
     \begin{align}
	f''(x) &= \ddx f'(x) = \ddx \dfx \\
	&= (x+1)\ddx e^x + e^x\ddx (x+1) \\
	&= (x+1) \cdot e^x + e^x = \ddfx \\
	f''(-1) &= e^{-1} = \frac{1}{e} > 0 
\end{align}  

As $f''(x) > 0$  at $x=-1$, therefore we have a minima at $x=-1$. And the minimum value is 
\[ \qquad f_{min} = f(-1) = -1\cdot e^{-1} = -\frac{1}{e}\]

\end{step}
\end{problem} 
\end{document} 