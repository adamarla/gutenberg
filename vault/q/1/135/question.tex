
\ifnumequal{\value{rolldice}}{0}{
  % variables 
  \renewcommand\vx{7}
  \renewcommand\vp{80}
}{
  \ifnumequal{\value{rolldice}}{1}{
    % variables 
    \renewcommand\vx{6}
    \renewcommand\vp{88}
  }{
    \ifnumequal{\value{rolldice}}{2}{
      % variables 
      \renewcommand\vx{5}
      \renewcommand\vp{96}
    }{
      % variables 
      \renewcommand\vx{4}
      \renewcommand\vp{85}
    }
  }
}

\SUBTRACT{100}\vp\vl
\MULTIPLY\vx{6}\vs
\MULTIPLY\vx{4}\vw
\FRACTIONSIMPLIFY\vp{100}\va\vb
\FRACMULT\va\vb{16}{24}\vc\vd
\FRACMULT\vc\vd{27}{2}\ve\vf
\FRACMULT\ve\vf\vx{1}\vg\vh

\question[2] A metal sphere of diameter $\vs$ cm is melted and drawn into a wire of diameter $\vw$ cm.
What would be the length of the drawn wire if $\vl$\% of the metal was lost during the melting process?

\watchout

\begin{solution}[\halfpage]
  \begin{align}
    \text{Metal available after melting} &= \text{Volume of drawn wire} \\
    \implies \left( 1-\dfrac\vl{100} \right)&\times\dfrac{4}{3}\pi\left(\dfrac{D_{\text{sphere}}}{2}\right)^3 = 
    	\pi\left(\dfrac{D_{\text{wire}}}{2}\right)^2\cdot L_{\text{wire}} \\
    \implies L_{\text{wire}} &= \dfrac\vp{100}\times\dfrac{16}{24}\times\dfrac{(\vs\text{ cm})^3}{(\vw\text{ cm})^2} \\
    &=\WRITEFRAC\vg\vh\text{ cm}
  \end{align}
\end{solution}

\ifprintanswers\begin{codex}$\dfrac\vg\vh\text{ cm}$\end{codex}\fi
