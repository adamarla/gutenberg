\documentclass[14pt,fleqn]{extarticle}
\RequirePackage{prepwell}

\previewoff 

\begin{document} 

%text
Evaluate 
\[ \qquad \sin \left[ \frac\pi{3} - \sin^{-1}\left(-\frac{1}{2} \right) \right] \] 
%

\newcard 

\begin{align}
\sin\left( \pi + \frac\pi{6}\right) &= \sin\left( \frac{7\pi}{6} \right)  = -\frac{1}{2} \\
\therefore \sin^{-1}\left( -\frac{1}{2} \right) &= \frac{7\pi}{6} 
\end{align} 

\newcard 

%text
There are infinite angles for which 
\[ \qquad\qquad \sin\theta = -\frac{1}{2} \]

But when we say \[\qquad \qquad\sin^{-1}\left(-\frac{1}{2}\right) = \theta\]
the $\theta$ is an angle in what is called the

\underline{principal range}\newline 


For $\sin^{-1} x$, that principal range is $ \left[-\frac\pi{2},\frac\pi{2}\right]$\newline 


And as $\sin\left( -\frac\pi{6}\right) = -\frac{1}{2}$ and $-\frac\pi{6}\in\left[ -\frac\pi{2},\frac\pi{2}\right]$,
\[ \qquad \therefore \sin^{-1}\left( -\frac{1}{2}\right) = -\frac\pi{6} \]
%

\newcard 

\[ \sin \left[ \frac\pi{3} - \sin^{-1}\left(-\frac{1}{2} \right) \right] = 1 \]

\newcard 

\[ \sin \left[ \frac\pi{3} - \sin^{-1}\left(-\frac{1}{2} \right) \right] = \frac{1}{2} \]

\newcard 

\begin{align}
\sin^{-1}\left(-\frac{1}{2}  \right) &= -\frac\pi{6} \\
\therefore \frac\pi{3} - \sin^{-1}\left(-\frac{1}{2}  \right)  &= \frac\pi{3} - \left( -\frac\pi{6} \right)\\
&= \frac\pi{2} \\
\implies \sin \left[ \frac\pi{3} - \sin^{-1}\left(-\frac{1}{2} \right) \right] &= \sin\frac\pi{2}= 1
\end{align}
\end{document} 