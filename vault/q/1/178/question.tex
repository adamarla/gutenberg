
\ifnumequal{\value{rolldice}}{0}{
  \renewcommand\va{2}
  \renewcommand\vb{5}
}{
  \ifnumequal{\value{rolldice}}{1}{
    \renewcommand\va{4}
    \renewcommand\vb{7}
  }{
    \ifnumequal{\value{rolldice}}{2}{
      \renewcommand\va{3}
      \renewcommand\vb{2}
    }{
      \renewcommand\va{5}
      \renewcommand\vb{9}
    }
  }
}


\question[2] Find the inverse of the one-to-one function
\[ f:\mathbb{R}-\{-\vb\}\to \mathbb{R}-\{\va\} =\dfrac{\va x-1}{x+\vb }\]
\[\mathbb{R}-\lbrace X\rbrace\text{ means all real numbers except } X \]

\watchout

\begin{solution}[\mcq]
  \textbf{Method \#1}

	\begin{align}
		\text{If } y &= f(x) = \dfrac{\va x-1}{x+\vb} \\
		\text{then } yx + \vb y &= \va x - 1 \implies x = \dfrac{\vb y + 1}{\va - y} 
	\end{align}

	In iany function $y = f(x)$, $y$ is the dependent variable and $x$ the independent variable. 
	
  The \textbf{convention} is to write the \textit{independent} variable as $x$ and the 
	dependent variable as $y$.
	
	So, we could also write (2) as $f^{-1}(x) = \dfrac{\vb x+1}{\va-x}$. But remember, this $x$ 
	\textit{is not} the same as the $x$ in (1). We are just following a naming
	convention here.
	
  \textbf{Method \#2}

	Alternately, we could use the fact that 
  \[ f \circ f^{-1}(x) = x \]
	\begin{align}
		\implies \dfrac{\va f^{-1}(x)-1}{f^{-1}(x)+\vb} &= x \\
		\implies f^{-1}(x) &= \dfrac{\vb x + 1}{\va - x }
	\end{align}
\end{solution}

\ifprintanswers
  \begin{codex}
    $\dfrac{\vb x + 1}{\va -x}$
  \end{codex}
\fi 

