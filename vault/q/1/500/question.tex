
\question[3] Consider the R code snippet below. Explain the functions and arguments used as fully as you can

  \texttt{model $\leftarrow$ glm($y\sim x1+x2+x3$, family = binomial(link=``logit''))}\\
  \texttt{scale.factor = mean(dlogis(predict(model, type=``link'')))}

\begin{solution}[\halfpage]

  \underline{\texttt{model $\leftarrow$ glm($y\sim x1+x2+x3$, family = binomial(link=``logit''))}}
  \begin{itemize}
  \item{\texttt{glm} is the estimation call for Generalized Linear Models}
  \item{\texttt{family=binomial} instructs R that the estimation is for a binary dependent variable}
  \item{\texttt{link=``logit'' } instructs R to use the logistic function as the Generalized Linear Model}
  \end{itemize}

  \underline{\texttt{scale.factor = mean(dlogis(predict(model, type=``link'')))}}
  \begin{itemize}
  \item{The innermost \texttt{predict} determines fitted values for each observation}
  \item{\texttt{dlogis} then finds the partial effect for each observation}
  \item{Finally, \texttt{mean} calculates the average partial effect, which can then be used as the scale factor}
  \end{itemize}
\end{solution}

