
\question[2] Explain the statement as briefly as you can - 
\textit{Generalized Linear Models (GLM) allow for partial effects that are not ``constant''}

\begin{solution}[\halfpage]
  In a GLM we model the conditional expectation of $y$ as a non-linear function of the explanatory 
  variables in the form $E(y\vert\,X) = G(X\beta)$. While the function $G()$ is non-linear, the structure 
  within is still linear in the $\beta$'s. The partial effects for any variable are obtained through 
  calculus (for continuous variables) as 
    \[\frac{\partial E(y\vert\,X)}{\partial x_{j}} = \frac{\partial G(X\beta)}{\partial x_{j}} \beta_{j}\]

  Therefore, the partial effects are not constant. Note, $G()$ does not have to be a CDF. In general, 
  we should avoid automatically replacing $\dfrac{\partial G}{\partial x}$ with $g()$. In case you do so, 
  specify that by $g()$, you mean the derivative of $G()$
\end{solution}

