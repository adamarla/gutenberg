


\ifnumequal{\value{rolldice}}{0}{
  % variables 
  \renewcommand{\va}{\frac{1}{\sqrt{3}}} % tan \vd 
  \renewcommand{\vb}{\sqrt{3}} % 1/ \va 
  \renewcommand{\vc}{8\sqrt{3}}
  \renewcommand{\vd}{30} % angle of \perp with x-axis
  \renewcommand{\ve}{4} %m 
  \renewcommand{\vf}{4\sqrt{3}} %n 
}{
  \ifnumequal{\value{rolldice}}{1}{
    % variables 
    \renewcommand{\va}{\sqrt{3}}
    \renewcommand{\vb}{\dfrac{1}{\sqrt{3}}}
    \renewcommand{\vc}{\dfrac{8}{\sqrt{3}}}
    \renewcommand{\vd}{60}
    \renewcommand{\ve}{4}
    \renewcommand{\vf}{\frac{4}{\sqrt{3}}}
  }{
    \ifnumequal{\value{rolldice}}{2}{
      % variables 
      \renewcommand{\va}{\frac{1}{\sqrt{3}}}
      \renewcommand{\vb}{\sqrt{3}}
      \renewcommand{\vc}{32\sqrt{3}}
      \renewcommand{\vd}{30} 
      \renewcommand{\ve}{8}
      \renewcommand{\vf}{8\sqrt{3}}
    }{
      % variables 
      \renewcommand{\va}{\sqrt{3}}
      \renewcommand{\vb}{\dfrac{1}{\sqrt{3}}}
      \renewcommand{\vc}{\dfrac{32}{\sqrt{3}}}
      \renewcommand{\vd}{60}
      \renewcommand{\ve}{8}
      \renewcommand{\vf}{\frac{8}{\sqrt{3}}}
    }
  }
}

\question[4] If the perpendicular from the origin to a line $L_1$ makes an angle of $\ang{\vd}$ 
with the $x$-axis and the triangle formed by $L_1$ and the axes has an area of $\vc$ units,
then find the equation of $L_1$.

\watchout

\ifprintanswers
  \figinit{pt}
  \def\Xmin{-13.33333}
  \def\Ymin{-13.33333}
  \def\Xmax{66.66666}
  \def\Ymax{66.66666}
  \def\Xori{13.33333}
  \def\Yori{13.33333}
  \figpt0:(\Xori,\Yori)
  \figpt 100: $M$(66.2,13.8)
  \figpt 101: $N$(13.5,66)
  \figpt 102: $L_1$(80,0)
  \figptorthoprojline 200 :$K$= 0 /100, 101/
  \figdrawbegin{}
  \def\Xmaxx{\Xmax} % To customize the position
  \def\Ymaxx{\Ymax} % of the arrow-heads of the axes.
  \figdrawline [0,200]
  \figset arrowhead(length=4, fillmode=yes) % styling the arrowheads
  \figdrawaxes 0(\Xmin, \Xmaxx, \Ymin, \Ymaxx)
  \figdrawlineC(
  0 79.99999, % y = 3.75
  2.75862 77.24137, % y = 3.59
  5.51724 74.48275, % y = 3.43
  8.27586 71.72413, % y = 3.28
  11.03448 68.96551, % y = 3.12
  13.79310 66.20689, % y = 2.97
  16.55172 63.44827, % y = 2.81
  19.31034 60.68965, % y = 2.66
  22.06896 57.93103, % y = 2.50
  24.82758 55.17241, % y = 2.35
  27.58620 52.41379, % y = 2.19
  30.34482 49.65517, % y = 2.04
  33.10344 46.89655, % y = 1.88
  35.86206 44.13793, % y = 1.73
  38.62068 41.37931, % y = 1.57
  41.37931 38.62068, % y = 1.42
  44.13793 35.86206, % y = 1.26
  46.89655 33.10344, % y = 1.11
  49.65517 30.34482, % y = .95
  52.41379 27.58620, % y = .80
  55.17241 24.82758, % y = .64
  57.93103 22.06896, % y = .49
  60.68965 19.31034, % y = .33
  63.44827 16.55172, % y = .18
  66.20689 13.79310, % y = .02
  68.96551 11.03448, % y = -.12
  71.72413 8.27586, % y = -.28
  74.48275 5.51724, % y = -.43
  77.24137 2.75862, % y = -.59
  79.99999 0
  )
  \figdrawend
  \figvisu{\figBoxA}{}{%
  \figptsaxes 1:0(\Xmin, \Xmaxx, \Ymin, \Ymaxx)
  \figwritee 1:(5pt)     \figwriten 2:(5pt)
  \figptsaxes 1:0(\Xmin, \Xmax, \Ymin, \Ymax)
  \figwrites 102:(1)
  \figset write(mark=$\bullet$)
  \figwritesw 100:(2)
  \figwritene 101:(2)
  \figwritesw 0:$O$(2)
  \figwritene 200:(2)
  }
  \vspace{0.7cm}
  \centerline{\box\figBoxA}
\fi 

\begin{solution}[\halfpage]
	The situation is \asif. 
  \[ OK \perp L_1 \implies\text{ slope of }L_1 = -\dfrac{1}{\tan\ang{\vd}} = -\vb \]
	Now, if $M = (m,0)$ and $N = (0,n)$, then looking at $L_1$
	\begin{align}
		\dfrac{n-0}{0-m} &= -\vb \implies n = \vb m
	\end{align}
	Moreover, the area of $\triangle OMN$ is 
	\begin{align}
		\dfrac{1}{2}m\cdot n &= \dfrac{1}{2}\vb m\cdot m = \vc \\
		\implies m &= \pm\ve \text{ and } \therefore n = \pm\vf
	\end{align}
	There would be \textit{two} lines that meet our criterion. And their equations would be 
	\begin{align}
		\dfrac{x}{\ve} + \dfrac{y}{\vf} &= 1 \\
		\text{ and } -\dfrac{x}{\ve} - \dfrac{y}{\vf} &= 1
  \end{align}
  That we got two lines means there are two such triangles. 

  One of them is in the first quadrant (shown). The other will be formed by the points 
  \[ M'=(-\ve,0), N'=(0,-\vf)\text{ and } O = (0,0) \] 
  in the third quadrant (not shown). 

  You can say that one could have similarly sized triangles in the second and fourth 
  quadrants also ($\implies$ same area).

  But for those triangles, the perpendicular from the origin will not be at $\ang\vd$.
\end{solution}

\ifprintanswers\begin{codex}
  $\dfrac{x}{\ve} + \dfrac{y}{\vf} = 1\text{ and }
  -\dfrac{x}{\ve} - \dfrac{y}{\vf} = 1$
\end{codex}\fi
