\documentclass[14pt,fleqn]{extarticle}
\RequirePackage{prepwell-eng}
\previewoff

\newcommand\xtb{\frac{99}{100}}
\newcommand\xntb{\frac{9}{1000}}
\newcommand\xray{$X-$ray }

\begin{document}
\begin{problem}
\statement
	
%text
It is standard practice to take a chest
\xray of a person suspected of having 
tuberculosis $(TB)$. The probability of 
detecting $TB$ in a person actually suffering
from it is $\frac{99}{100}$. However, the probability 
of a healthy person being diagnosed with 
$TB$ is $\xntb$\newline 

In a certain city, 1 in 600 people suffers from $TB$. A person is selected 
at random and is diagnosed with $TB$. What is the probability that the person 
actually has the disease?
%

\begin{step}
  \begin{options} 
     \correct 
      
      If we define the following two events $\ldots$ 
      \begin{center}
  \begin{tabular}{Nl}
   \toprule
        \text{Event} & Meaning  \\
   \midrule 
      X & \xray says person has $TB$ \\
    \midrule 
    TB & Person actually has $TB$ \\
    \bottomrule
  \end{tabular}
\end{center} 

$\ldots$ then 

\begin{center}
  \begin{tabular}{NN}
   \toprule
        \text{What we know} & \text{What we have to find} \\
   \midrule 
   \prob{TB} = \frac{1}{600} & \condp{TB}{X} = ? \\
    \midrule 
    \condp{X}{TB} = \xtb & \\
    \midrule
    \condp{X}{\overline{TB}} = \xntb & \\
    \bottomrule
  \end{tabular}
\end{center}
       
     \incorrect
     
      If we define the following two events $\ldots$ 
      \begin{center}
  \begin{tabular}{Nl}
   \toprule
        \text{Event} & Meaning  \\
   \midrule 
      X & \xray says person has $TB$ \\
    \midrule 
    TB & Person actually has $TB$ \\
    \bottomrule
  \end{tabular}
\end{center} 

$\ldots$ then 

\begin{center}
  \begin{tabular}{NN}
   \toprule
        \text{What we know} & \text{What we have to find} \\
   \midrule 
    \condp{TB}{X} = \xtb & \prob{TB} = ? \\
    \midrule
    \condp{\overline{TB}}{X} = \xntb & \\
    \bottomrule
  \end{tabular}
\end{center}
        
    \end{options} 
     \reason 
     
     The \xray is a \underline{seemingly reliable} way to detect $TB$. But it is obviously not perfect     
     \[ \qquad \underbrace{\condp{X}{TB} = \xtb}_{\text{\xray says $TB$ and the person actually has $TB$}}\]
     
     As we have been told, it is entirely possible that the \xray says a person has $TB$ when he actually does not     
     \[ \qquad \underbrace{\condp{X}{\overline{TB}}= \xntb}_{\text{\xray says $TB$ for a healthy person}} \]
     Moreover, in this town, 1 in every 600 people has $TB$
     \[ \qquad \therefore \prob{TB} = \frac{1}{600} \]
     And what we would like to know is that \underline{if the \xray says $TB$} then does the person \underline{actually have it}
     \[ \qquad \condp{TB}{X} = ? \]
     
       
\end{step}

\begin{step}
  \begin{options} 
     \correct 
       
       \[ \condp{TB}{X} = \fcondp{TB}{X} = \frac{110}{709} \]
     \incorrect
     
            \[ \condp{TB}{X} = \fcondp{TB}{X} = \frac{410}{837} \]
        
    \end{options} 
     \reason 
     
     From Bayes' Theorem, we know that 
     \begin{align}
	\condp{TB}{X} &= \fcondp{TB}{X} \\[-20pt]
	\text{where } \prob{X} &= \condp{X}{TB}\prob{TB} \\
	&+ \condp{X}{\overline{TB}}\prob{\overline{TB}} \\
	&= \underbrace{\xtb\cdot\frac{1}{600} + \xntb\cdot\frac{599}{600}}_{\text{Rule of Total Probability}} \\[-10pt]
	&= \left[ \frac{6381}{600\times 1000}\right] \\[10pt]
	\therefore \condp{TB}{X} &= \dfrac{\xtb\cdot\frac{1}{600}}{\frac{6381}{600\cdot 1000}} = \frac{990}{6381} = \frac{110}{709} 
\end{align}

This is a surprisingly low number. If a person is known to have $TB$, then there is a $99\%$ chance that his \xray will confirm the disease \newline 

But, the converse is not true. If a person's \xray shows $TB$, then there is only about a $15.6\% = \frac{110}{709}$ chance that he actually does have $TB$
\end{step}
\end{problem}
\end{document}
