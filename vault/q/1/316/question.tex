

\ifnumequal{\value{rolldice}}{0}{
  % variables 
  \renewcommand\va{4}
  \renewcommand\vb{5}
  \renewcommand\vc{3}
  \renewcommand\vd{-3} % 1st soln
  \renewcommand\ve{5} % 2nd soln
}{
  \ifnumequal{\value{rolldice}}{1}{
    % variables 
    \renewcommand\va{3}
    \renewcommand\vb{6}
    \renewcommand\vc{2}
    \renewcommand\vd{-4}
    \renewcommand\ve{3}
  }{
    \ifnumequal{\value{rolldice}}{2}{
      % variables 
      \renewcommand\va{5}
      \renewcommand\vb{2}
      \renewcommand\vc{4}
      \renewcommand\vd{-1}
      \renewcommand\ve{8}
    }{
      % variables 
      \renewcommand\va{6}
      \renewcommand\vb{4}
      \renewcommand\vc{4}
      \renewcommand\vd{-2}
      \renewcommand\ve{8}
    }
  }
}

\question[3] Find the equation of the lines that pass through $A = (\va, -\vb)$ 
and for which $ON - OM = \vc$.

\watchout[-10pt]

\figinit{pt}
\def\Xmin{-16.00000}
\def\Ymin{-31.99999}
\def\Xmax{64.00000}
\def\Ymax{48.00000}
\def\Xori{16.00000}
\def\Yori{31.99999}
\figpt 100: $M$(48,31)
\figpt 101: $N$(15,64)
\figpt 102 :$A$(61,19)
\figpt0:(\Xori,\Yori)
\figdrawbegin{}
\def\Xmaxx{\Xmax} % To customize the position
\def\Ymaxx{\Ymax} % of the arrow-heads of the axes.
\figset arrowhead(length=4, fillmode=yes) % styling the arrowheads
\figdrawaxes 0(\Xmin, \Xmaxx, \Ymin, \Ymaxx)
\figdrawlineC(
0 79.99999, % y = 4.50
2.75862 77.24137, % y = 4.24
5.51724 74.48275, % y = 3.98
8.27586 71.72413, % y = 3.72
11.03448 68.96551, % y = 3.46
13.79310 66.20689, % y = 3.20
16.55172 63.44827, % y = 2.94
19.31034 60.68965, % y = 2.68
22.06896 57.93103, % y = 2.43
24.82758 55.17241, % y = 2.17
27.58620 52.41379, % y = 1.91
30.34482 49.65517, % y = 1.65
33.10344 46.89655, % y = 1.39
35.86206 44.13793, % y = 1.13
38.62068 41.37931, % y = .87
41.37931 38.62068, % y = .62
44.13793 35.86206, % y = .36
46.89655 33.10344, % y = .10
49.65517 30.34482, % y = -.15
52.41379 27.58620, % y = -.41
55.17241 24.82758, % y = -.67
57.93103 22.06896, % y = -.93
60.68965 19.31034, % y = -1.18
63.44827 16.55172, % y = -1.44
66.20689 13.79310, % y = -1.70
68.96551 11.03448, % y = -1.96
71.72413 8.27586, % y = -2.22
74.48275 5.51724, % y = -2.48
77.24137 2.75862, % y = -2.74
79.99999 0
)
\figdrawend
\figvisu{\figBoxA}{}{%
\figptsaxes 1:0(\Xmin, \Xmaxx, \Ymin, \Ymaxx)
\figwritee 1:(5pt)     \figwriten 2:(5pt)
\figptsaxes 1:0(\Xmin, \Xmax, \Ymin, \Ymax)
\figset write(mark = $\bullet$)
\figwritesw 100:(1)
\figwritene 101:(2)
\figwritene 102:(2)
\figwritesw 0:$O$(2)
}

\vspace{0.7cm}
\centerline{\box\figBoxA}

\ADD\vb\vc\vm
\SUBTRACT\vm\va\vn
\MULTIPLY\vb\vc\vo
\SUBTRACT\vd\vc\vp
\SUBTRACT\ve\vc\vq

\LCM\vp\vd\vf
\DIVIDE\vf\vp\vg
\DIVIDE\vf\vd\vh
\LCM\vq\ve\vi
\DIVIDE\vi\vq\vj
\DIVIDE\vi\ve\vk

\begin{solution}[\halfpage]
  If $ON = y-$intercept $=n$ and $OM = x-$intercept =$n - \vc$,
  then the equation of a line with these intercepts would be 

  \begin{align}
    \dfrac{x}{n-\vc} + \dfrac{y}{n} &= 1
  \end{align}

  Moreover, as the line passes through $A = (\va, -\vb)$,
  the following would also be true
  \begin{align}
    \dfrac{\va}{n-\vc}-\dfrac{\vb}{n} &= 1 \\
    \implies n^2 + \vn n - \vo &= 0 \\
    \implies n &= \vd,\, \ve
  \end{align}

  As there are two distinct values of $n = y-$intercept, 
  there would be two distinct lines given by 
  \begin{align}
    \frac{x}{\vp} + \frac{y}{\vd} &= 1 \implies \vg x + \vh y = \vf \\
    \frac{x}{\vq} + \frac{y}{\ve} &= 1 \implies \vj x + \vk y = \vi
  \end{align}
\end{solution}

\ifprintanswers
  \begin{codex}
    \begin{tabular}{l l}
      $L_1: \vg x + \vh y = \vf$ & 
      $L_2: \vj x + \vk y = \vi$ \\
    \end{tabular}
  \end{codex}
\fi
