
\question

\begin{parts}
  \part[3] What do you understand by  ``regression diagnostics?'' 

\begin{solution}[\halfpage]
    These are procedures or methods to test the assumptions made for estimation and/or inference.  
    The methods can be based on visual inspection of some standard plots involving the residuals, or 
    formal statistical tests on them, \emph{e.g.} the White test for heteroskedasticity. 
    In addition to checking the validity of statistical assumptions, one could also study the impact of potential outliers.
  \end{solution}

  \part[3] What is a quantile? Explain the idea behind a normal Q-Q plot.

\begin{solution}[\halfpage]
    A quantile is a set of points obtained by inverting a given CDF at regularly spaced points from 0 to 1

    A normal Q-Q plot compares the quantiles of the \emph{standardized residuals} with  quantiles from the 
    standard normal distribution. If the residuals are close to being normally distributed, then the points 
    will all line up with the reference line. So this is a test of the normality assumption. 
  \end{solution}

  \part[3] Briefly explain how we might test for presence of heteroskedasticity?

\begin{solution}[\halfpage]
    The assumption we wish to test is that of constant conditional variance, that is 
    \[ var(u\vert\, X) = \sigma^2 \]
    
    Since $var(u\vert\,X) = E(u^{2}\vert\,X) - (E(u\vert\,X))^{2} = E(u^{2}\vert\,X)$. Thus, one approach 
    to test for the presence of heteroskedasticity is to test the null hypothesis 
    \[ H_{0}: E(\hat{u}^{2}\vert\,X) = k\text{ (a constant)} \]
    
    A specific way of doing this to regress the square of the residuals with the explanatory variables 
    \textbf{(Bruesh Pagan Test)} or with the set of explanatory variables along with their squares and 
    pair-wise interaction terms \textbf{(the White Test)}
  \end{solution}

\end{parts}

