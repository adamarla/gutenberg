

\ifnumequal{\value{rolldice}}{0}{
  % variables 
  \renewcommand{\va}{3}
  \renewcommand{\vb}{6}
  \renewcommand{\vc}{-2}
  \renewcommand{\vd}{-3}
  \renewcommand{\ve}{4}
}{
  \ifnumequal{\value{rolldice}}{1}{
    % variables 
    \renewcommand{\va}{4}
    \renewcommand{\vb}{3}
    \renewcommand{\vc}{5}
    \renewcommand{\vd}{7}
    \renewcommand{\ve}{8}
  }{
    \ifnumequal{\value{rolldice}}{2}{
      % variables 
      \renewcommand{\va}{2}
      \renewcommand{\vb}{5}
      \renewcommand{\vc}{3}
      \renewcommand{\vd}{7}
      \renewcommand{\ve}{9}
    }{
      % variables 
      \renewcommand{\va}{7}
      \renewcommand{\vb}{3}
      \renewcommand{\vc}{4}
      \renewcommand{\vd}{7}
      \renewcommand{\ve}{-3}
    }
  }
}

\SUBTRACT\vc\va\a
\SUBTRACT\vb\vd\b
\SUBTRACT\vc\ve\c
\MULTIPLY\a\b\m
\FRACTIONSIMPLIFY\m\c\p\q
\FRACMINUS\vb{1}\p\q\j\k

\question[2]  Find the value of $k$ for which the points 
\[ A=(k, \va),\, B=(\vb, \vc)\text{ and }C=(\vd, \ve)\] 
are collinear.

\watchout

\begin{solution}[\mcq]
	The slope of a line does \emph{not} change depending on which two points were
	used to calculate the slope. After all, its the same line no matter where you 
	are on it 
	
	
	Given this, the following would hold true
	\begin{align}
	  \dfrac{\vc - (\va)}{\vb - k} &= 
	  \dfrac{\vc - \ve}{\vb - \vd}\\
	  \implies \vb - k &= \dfrac{(\vc - \va)\cdot(\vb - \vd)}{\vc - \ve} = \WRITEFRAC\p\q \\
	  \text{ or, } k &= \vb - \dfrac{\p}{\q} = \WRITEFRAC[false]{\j}{\k}
	\end{align}
\end{solution}

\ifprintanswers\begin{codex}$\WRITEFRAC[false]\j\k$\end{codex}\fi
