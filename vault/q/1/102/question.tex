

\ifnumequal{\value{rolldice}}{0}{
  \renewcommand{\va}{18}
}{
  \ifnumequal{\value{rolldice}}{1}{
    \renewcommand{\va}{21}
  }{
    \ifnumequal{\value{rolldice}}{2}{
      \renewcommand{\va}{24}
    }{
      \renewcommand{\va}{30}
    }
  }
}

\DIVIDE\va{3}\vb

\question[3] A right conical funnel with a slant height of $\va$ cm must be made. What should 
be its height for it to have the greatest possible volume?

\watchout

\begin{solution}[\halfpage]
   If $L$ be the slant height of the funnel and $V$ its volume, then 
   \begin{align}
       L^2 &= h^2 + R^2 \implies R^2 = L^2 - h^2 \\ 
       V &= \dfrac{\pi}{3}\cdot R^2\cdot h = \dfrac{\pi}{3}\cdot(L^2 - h^2)\cdot h
   \end{align}
   where $R$ is the radius of the cone's base and $h$ the cone's height
   
   And so, 
   \begin{align}
      \dfrac{\ud V}{\ud h} &= \dfrac{\pi}{3}\cdot\dfrac{\ud}{\ud h}(L^2-h^2)\cdot h 
      = \dfrac{\pi}{3}(L^2 - 3h^2) \\
      \dfrac{\ud V}{\ud h} &= 0\implies 3h^2 = L^2 \implies h = \dfrac{L}{\sqrt{3}}
      = \dfrac\va{\sqrt{3}} = \vb\sqrt{3} \\
      \text{Also }\dfrac{\ud^2 V}{\ud h^2} &= -2\pi\cdot h < 0 \implies \text{ maxima at }
      h = \vb\sqrt{3}
   \end{align}
   The funnel would therefore have the greatest volume when 
   \[ h = \vb\sqrt{3}\text{ cm} \] 
   
\end{solution}
\ifprintanswers\begin{codex}$\vb\sqrt{3}$ cm\end{codex}\fi
