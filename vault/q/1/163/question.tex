
\ifnumequal{\value{rolldice}}{0}{
  % variables 
  \renewcommand{\va}{1}
  \renewcommand{\vb}{2}
  \renewcommand{\vc}{1}
  \renewcommand{\vd}{3}
}{
  \ifnumequal{\value{rolldice}}{1}{
    % variables 
    \renewcommand{\va}{3}
    \renewcommand{\vb}{5}
    \renewcommand{\vc}{1}
    \renewcommand{\vd}{4}
  }{
    \ifnumequal{\value{rolldice}}{2}{
      % variables 
      \renewcommand{\va}{2}
      \renewcommand{\vb}{5}
      \renewcommand{\vc}{3}
      \renewcommand{\vd}{4}
    }{
      % variables 
      \renewcommand{\va}{2}
      \renewcommand{\vb}{7}
      \renewcommand{\vc}{3}
      \renewcommand{\vd}{8}
    }
  }
}

\FRACPOWER\va\vb{2}\ve\vf
\FRACMINUS{1}{1}\ve\vf\vx\vy
\FRACMULT\va\vb{2}{1}\vg\vh
\FRACDIV\vg\vh\vx\vy\vp\vq

\FRACADD\vp\vq\vc\vd\a\b
\FRACMULT\vp\vq\vc\vd\c\d
\FRACMINUS{1}{1}\c\d\e\f
\FRACDIV\a\b\e\f\vj\vk

\question[2] Find the value of $\tan (2A+B)$ if 
 \[ \tan A = \frac\va\vb\text{ and }\tan B = \frac\vc\vd \]

\watchout

\begin{solution}[\halfpage]
	\begin{align}
		\tan(2A+B) &= \dfrac{\tan 2A + \tan B}{1-\tan 2A\tan B} \\
		\text{where } \tan 2A &= \dfrac{2\tan A}{1-\tan^2 A} = 
    \dfrac{2\times \frac\va\vb}{1-\left(\frac\va\vb\right)^2} = \dfrac\vp\vq \\
		 \implies \tan(2A+B) &= \dfrac{\frac\vp\vq + \frac\vc\vd}{1-\frac\vp\vq\cdot\frac\vc\vd} = \WRITEFRAC\vj\vk
	\end{align}
\end{solution}
\ifprintanswers\begin{codex}$\WRITEFRAC\vj\vk$\end{codex}\fi
