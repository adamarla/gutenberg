\ifnumequal{\value{rolldice}}{0}{
  % variables 
  \renewcommand{\va}{2}
  \renewcommand{\vb}{-1}
  \renewcommand{\vc}{1}
  \renewcommand{\vd}{1}
  \renewcommand{\ve}{-3}
  \renewcommand{\vf}{-5}
  \renewcommand{\vg}{3}
  \renewcommand{\vh}{-4}
  \renewcommand{\vi}{-4}
}{
  \ifnumequal{\value{rolldice}}{1}{
    % variables 
    \renewcommand{\va}{8}
    \renewcommand{\vb}{10}
    \renewcommand{\vc}{-6}
    \renewcommand{\vd}{1}
    \renewcommand{\ve}{-3}
    \renewcommand{\vf}{-5}
    \renewcommand{\vg}{3}
    \renewcommand{\vh}{-4}
    \renewcommand{\vi}{-4}
  }{
    \ifnumequal{\value{rolldice}}{2}{
      % variables 
      \renewcommand{\va}{2}
      \renewcommand{\vb}{-1}
      \renewcommand{\vc}{1}
      \renewcommand{\vd}{11}
      \renewcommand{\ve}{2}
      \renewcommand{\vf}{1}
      \renewcommand{\vg}{3}
      \renewcommand{\vh}{-4}
      \renewcommand{\vi}{-4}
    }{
      % variables 
      \renewcommand{\va}{2}
      \renewcommand{\vb}{-1}
      \renewcommand{\vc}{1}
      \renewcommand{\vd}{1}
      \renewcommand{\ve}{-3}
      \renewcommand{\vf}{-5}
      \renewcommand{\vg}{3}
      \renewcommand{\vh}{-4}
      \renewcommand{\vi}{-4}
    }
  }
}

\VECTORSUB(\vd,\ve,\vf)(\va,\vb,\vc)(\a,\b,\c)
\VECTORSUB(\vg,\vh,\vi)(\vd,\ve,\vf)(\d,\e,\f)
\VECTORSUB(\va,\vb,\vc)(\vg,\vh,\vi)(\g,\h,\k)

\SCALARPRODUCT(\a,\b,\c)(\d,\e,\f)\vx
\SCALARPRODUCT(\d,\e,\f)(\g,\h,\k)\vy
\SCALARPRODUCT(\a,\b,\c)(\g,\h,\k)\vz
% \VECTORSUB\vd\ve\vf\va\vb\vc\a\b\c
% \VECTORSUB\vd\ve\vf\va\vb\vc\a\b\c

\question[3] The \textbf{position vectors} of three points in space 
are given by 
\[ \vec{A} = \WRITEVECTOR\va\vb\vc,\,\vec{B} = \WRITEVECTOR\vd\ve\vf 
\text{ and }\vec{C} = \WRITEVECTOR\vg\vh\vi \] 
What is the \textbf{largest} angle (in degrees) in the triangle formed by joining 
these three points?

\watchout

\ifprintanswers
  % stuff to be shown only in the answer key - like explanatory margin figures
  \begin{marginfigure}
    \figinit{pt}
      \figpt 100:(0,0)
      \figpt 101:(0,0)
    \figdrawbegin{}
      \figdrawline [100,101]
    \figdrawend
    \figvisu{\figBoxA}{}{%
    }
    \centerline{\box\figBoxA}
  \end{marginfigure}
\fi 

\begin{solution}[\fullpage]
\textbf{Step \#1: Find the sides of the triangle}

To know the angles between sides of the triangle, we must first know what the 
sides are. 
\begin{align}
  \vec{AB} &= \vec{B}-\vec{A} = \WRITEVECTOR\a\b\c \\
  \vec{BC} &= \vec{C}-\vec{B} = \WRITEVECTOR\d\e\f \\
  \vec{CA} &= \vec{A}-\vec{C} = \WRITEVECTOR\g\h\k 
\end{align}

\textbf{Step \#2: Calculate the dot-products}
\begin{align}
  \vec{AB}\cdot\vec{BC} &= (\WRITEVECTOR\a\b\c)\cdot(\WRITEVECTOR\d\e\f) = \vx \\
  \vec{BC}\cdot\vec{CA} &= (\WRITEVECTOR\d\e\f)\cdot(\WRITEVECTOR\g\h\k) = \vy \\
  \vec{CA}\cdot\vec{AB} &= (\WRITEVECTOR\g\h\k)\cdot(\WRITEVECTOR\a\b\c) = \vz 
\end{align}

\textbf{Step \#3: Compare the dot-product values} 

Did you notice that one of the dot-products $=0\implies$ those two vectors (and the sides they represent) 
are at $\ang{90}$ to each other. 

Moreover, if a triangle has one angle $=\ang{90}$, then the other two angles have to be $<\ang{90}$. 

Hence, the largest angle is $\ang{90}$.
\end{solution}

\ifprintanswers\begin{codex}$\ang{90}$\end{codex}\fi
