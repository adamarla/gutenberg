
\ifnumequal{\value{rolldice}}{0}{
  \renewcommand\va{2}
}{
  \ifnumequal{\value{rolldice}}{1}{
    \renewcommand\va{3}
  }{
    \ifnumequal{\value{rolldice}}{2}{
      \renewcommand\va{4}
    }{
      \renewcommand\va{6}
    }
  }
}

\SQUARE\va\vb
\MULTIPLY\vb\va\vc % va^3
\MULTIPLY\vc{10}\vd
\FRACTIONSIMPLIFY{3}\vd\vp\vq

\question[5] A figure bounded by arcs of the parabolas $y=\va x^2$ and $\va y^2=x$ rotates
about the $x-axis$. Compute the volume of the solid thus generated.

\watchout

\ifprintanswers
  % stuff to be shown only in the answer key - like explanatory margin figures
\figinit{pt}
\def\Xmin{0}
\def\Ymin{0}
\def\Xmax{60.00000}
\def\Ymax{79.99999}
\def\Xori{0}
\def\Yori{0}
\figpt0:(\Xori,\Yori)
\figpt 100:$\va y^2=x$(60,42)
\figpt 101:$y=\va x^2$(60,80)
\figpt 102:$x_2$(8.3,16)
\figpt 103:$x_1$(27,16)
\figdrawbegin{}
\def\Xmaxx{\Xmax} % To customize the position
\def\Ymaxx{\Ymax} % of the arrow-heads of the axes.
\figset arrowhead(length=4, fillmode=yes) % styling the arrowheads
\figdrawaxes 0(\Xmin, \Xmaxx, \Ymin, \Ymaxx)
\figdrawline [102,103]
\figdrawlineC( % y^2 =x
0 0,
2.06896 8.08637,
4.13793 11.43586,
6.20689 14.00601,
8.27586 16.17275,
10.34482 18.08169,
12.41379 19.80750,
14.48275 21.39454,
16.55172 22.87173,
18.62068 24.25913,
20.68965 25.57137,
22.75862 26.81948,
24.82758 28.01203,
26.89655 29.15585,
28.96551 30.25646,
31.03448 31.31841,
33.10344 32.34551,
35.17241 33.34099,
37.24137 34.30760,
39.31034 35.24770,
41.37931 36.16338,
43.44827 37.05644,
45.51724 37.92848,
47.58620 38.78091,
49.65517 39.61500,
51.72413 40.43189,
53.79310 41.23260,
55.86206 42.01805,
57.93103 42.78909,
59.99999 43.54648
)
\figdrawlineC( % y = x^2
0 0,
2.06896 .09512,
4.13793 .38049,
6.20689 .85612,
8.27586 1.52199,
10.34482 2.37812,
12.41379 3.42449,
14.48275 4.66111,
16.55172 6.08799,
18.62068 7.70511,
20.68965 9.51248,
22.75862 11.51010,
24.82758 13.69797,
26.89655 16.07609,
28.96551 18.64447,
31.03448 21.40309,
33.10344 24.35196,
35.17241 27.49108,
37.24137 30.82045,
39.31034 34.34007,
41.37931 38.04994,
43.44827 41.95005,
45.51724 46.04042,
47.58620 50.32104,
49.65517 54.79191,
51.72413 59.45303,
53.79310 64.30439,
55.86206 69.34601,
57.93103 74.57788,
59.99999 79.99999
)
\figdrawend
\figvisu{\figBoxA}{}{%
\figptsaxes 1:0(\Xmin, \Xmaxx, \Ymin, \Ymaxx)
\figwritee 1:(5pt)     \figwriten 2:(5pt)
\figptsaxes 1:0(\Xmin, \Xmax, \Ymin, \Ymax)
\figwritene 100:(2)
\figwriten 101:(2)
\figwritew 102:(2)
\figwritee 103:(2)
}
\vspace{0.7cm}
\centerline{\box\figBoxA}
\fi 

\begin{solution}[\fullpage]
  What we have are two parabolas (see figure) that intersect in the first quadrant. 
  It is the region between the two curves that is rotated about the $x-$axis to get the solid.
  
  When the two curves intersect, then 
  \begin{align}
    y &= \va x^2 \implies x^2 = \dfrac{y}\va \\
    \va y^2 &= x \implies x^2 = \vb y^4 \\
    \implies \vb y^4 &= \dfrac{y}\va \text{ or } y\cdot (\vc y^3-1) = 0 \implies y = 0, \dfrac{1}\va
  \end{align}

  At any given $y,\, 0\leq y\leq\frac{1}\va$, imagine a thin strip like $x_1\to x_2$. When
  this strip is rotated about the $x-$axis, it will form a \textbf{hollow cylinder}. 

  The volume of that hollow cyclinder would be 
	\begin{align}
		\ud V &= \underbrace{(x_1-x_2)\ud y}_{\text{cross-section}}\times
		         \underbrace{2\pi y}_{\text{perimeter}} \\
		\text{where } x_1 &= \sqrt{\frac{y}\va} \text{ and } x_2 = \va y^2
	\end{align}

	And therefore, the total volume would be
	\begin{align}
		V &= \int_0^{\frac{1}\va} (\sqrt{\frac{y}\va} - \va y^2)\cdot 2\pi y\ud y \\
		  &= 2\pi\int_0^{\frac{1}\va}\left(\dfrac{y^{\frac{3}{2}}}{\sqrt\va} - \va y^3 \right) dy \\ 
		  &= 2\pi\left[ \dfrac{2y^{\frac{5}{2}}}{5\sqrt\va} - \dfrac\va{4} y^4\right]_0^{\frac{1}\va} = \dfrac{\vp\pi}\vq
	\end{align}
\end{solution}

\ifprintanswers\begin{codex}$\dfrac{\vp\pi}\vq$\end{codex}\fi
