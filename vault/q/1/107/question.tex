
\ifnumequal{\value{rolldice}}{0}{
  \renewcommand\vh{4}
}{
  \ifnumequal{\value{rolldice}}{1}{
    \renewcommand\vh{5}
  }{
    \ifnumequal{\value{rolldice}}{2}{
      \renewcommand\vh{3}
    }{
      \renewcommand\vh{6}
    }
  }
}

\question[2] Sand is pouring from a pipe at the rate of 12 cm$^3$/s.
The falling sand forms a cone on the ground in such a way that the height of the
cone is always one-sixth the radius of the base. How fast is the height of the
cone increasing when its height is $\vh$ cm? 

\watchout

\SQUARE\vh\vi
\MULTIPLY\vi{3}\vj

\begin{solution}[\halfpage]
  If $V_t$ be the volume of the cone at \textbf{any given point in time}, then 
  \[ V_t = \dfrac\pi{3}\cdot R_t^2\cdot H_t \]
  where $R_t$ and $H_t$ are the radius and height of the cone \textbf{at that point in time}.

  Moreover, 
  \[ R_t = 6\cdot H_t\text{ at all points in time }\]
  Hence, 
  \begin{align}
    V_t &= \dfrac\pi{3}\cdot R_t^2 \cdot H_t = \dfrac\pi{3}\cdot (36 H_t^2)\cdot H_t = 12\pi\cdot H_t^3 \\
    \implies\dfrac{\ud V_t}{\ud t} &= 12\text{ cm}^3\text{/s} = 36\pi\cdot H_t^2\dfrac{\ud H_t}{\ud t} \\
    \implies \left[\dfrac{\ud H_t}{\ud t}\right]_{\vh\text{ cm}} &= \dfrac{12\text{ cm}^3\text{/s}}{36\pi\cdot (\vh\text{ cm})^2} 
    = \dfrac{1}{\vj\pi}\text{ cm/s}
  \end{align}
\end{solution}
\ifprintanswers
  \begin{codex}
    $\dfrac{1}{\vj\pi}\text{ cm/s}$
  \end{codex}
\fi
