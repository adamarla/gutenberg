
\ifnumequal{\value{rolldice}}{0}{
  % variables 
  \renewcommand{\va}{2}
  \renewcommand\vb{8}
  \renewcommand\vc{2}
  \renewcommand\vd{5}
  \renewcommand\vm{38}
}{
  \ifnumequal{\value{rolldice}}{1}{
    % variables 
    \renewcommand{\va}{2}
    \renewcommand\vb{7}
    \renewcommand\vc{3}
    \renewcommand\vd{7}
    \renewcommand\vm{42}
  }{
    \ifnumequal{\value{rolldice}}{2}{
      % variables 
      \renewcommand{\va}{3}
      \renewcommand\vb{6}
      \renewcommand\vc{4}
      \renewcommand\vd{3}
      \renewcommand\vm{45}
    }{
      % variables 
      \renewcommand{\va}{4}
      \renewcommand\vb{9}
      \renewcommand\vc{2}
      \renewcommand\vd{9}
      \renewcommand\vm{72}
    }
  }
}

\MULTIPLY\vb{2}\ve
\MULTIPLY\va\vc\vf
\MULTIPLY\va\vd\vg
\SQUARE\vb\vh

\SUBTRACT\vf{1}\vi
\ADD\vg\ve\vj
\MULTIPLY\vi{2}\vn

\ADD{-\vj}\vm\vx
\SUBTRACT{-\vj}\vm\vy

\question[2] For what values of $x$ would $\va x$, $x-\vb$ and $\vc x+ \vd$ be successive terms 
of a geometric progression?

\watchout

\begin{calcaid}
  \begin{tabular}{c c c c}
    $38^2=1444$ & $42^2=1764$ & $45^2=2025$ & $72^2=5184$ 
  \end{tabular}
\end{calcaid}

\begin{solution}[\halfpage]
	If $\va x$, $x-\vb$ and $2x+\vd$ are to be successive terms of a geometric progression, then
	\begin{align}
		(x - \vb)^2 &= (\vc x+\vd)\cdot \va x \\
		\implies x^2-\ve x + \vh &= \vf x^2 + \vg x \text{ or } \vi x^2+\vj x- \vh = 0 \\
		\implies x &= \dfrac{-\vj \pm \sqrt{\vj^2-4\cdot \vi\cdot(-\vh)}}{2\times \vi} \\
		&=  \dfrac{-\vj \pm \vm}\vn\implies x = 
    \WRITEFRAC[false]\vx\vn, \WRITEFRAC[false]\vy\vn 
	\end{align}
\end{solution}

\ifprintanswers
  \begin{codex}
    $\WRITEFRAC[false]\vx\vn\text{ and }\WRITEFRAC[false]\vy\vn$
  \end{codex}
\fi
