

\ifnumequal{\value{rolldice}}{0}{
  % variables 
  \renewcommand{\va}{4}
  \renewcommand{\vb}{7}
  \renewcommand{\vc}{10}
}{
  \ifnumequal{\value{rolldice}}{1}{
    % variables 
    \renewcommand{\va}{6}
    \renewcommand{\vb}{10}
    \renewcommand{\vc}{30}
  }{
    \ifnumequal{\value{rolldice}}{2}{
      % variables 
      \renewcommand{\va}{6}
      \renewcommand{\vb}{8}
      \renewcommand\vc{20}
    }{
      % variables 
      \renewcommand{\va}{8}
      \renewcommand{\vb}{12}
      \renewcommand{\vc}{25}
    }
  }
}

\FRACTIONSIMPLIFY\va{100}\p\q
\FRACTIONSIMPLIFY\vb{100}\r\s
\FRACTIONSIMPLIFY\vc{100}\m\n
\FRACMINUS{1}{1}\m\n\a\b

\FRACMULT\p\q\m\n\k\j
\FRACMULT\r\s\a\b\y\z
\FRACDIV\y\z\k\j\tp\tq
\ADD\tp\tq\tr

\question[3] If $\va$\% of the people with blood group $O$ are left-handed and $\vb$\% 
of those with other blood groups are left-handed, then what is the probability that a left-handed
person selected at random has blood group $O$ given that overall $\vc$\% of people have 
blood group $O$?

\watchout[-40pt]

\begin{solution}[\halfpage]
  If $L =$ event that a person is left-handed and $O=$ event that a person has blood group $O$,
  then we have been told the following
  \begin{align}
    P(L\vert\,O) &= \WRITEFRAC\p\q \\
    P(L\vert\,O') &= \WRITEFRAC\r\s \\
    P(O) &= \WRITEFRAC\m\n \implies P(O') = \WRITEFRAC\a\b
  \end{align}
  What we need is $P(O\vert\,L)$ - which can be found using Baye's theorem
  \begin{align}
    P(O\vert\,L) &= \dfrac{P(L\vert\,O)\cdot P(O)}{P(L\vert\,O)\cdot P(O) + P(L\vert\,O')\cdot P(O')} \\
       &= \dfrac{\frac\p\q\cdot\frac\m\n}{\frac\p\q\cdot\frac\m\n + \frac\r\s\cdot\frac\a\b} = \WRITEFRAC\tq\tr 
  \end{align}
\end{solution}


\ifprintanswers\begin{codex}$\WRITEFRAC\tq\tr$\end{codex}\fi
