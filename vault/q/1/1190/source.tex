\documentclass[14pt,fleqn]{extarticle}
\RequirePackage{prepwell}
\previewoff
\begin{document}

\newcommand\ea{1+2x+3x^2} 
\newcommand\intga{ \frac{5}{6}\log \left(\ea \right)}
\newcommand\intgb{ \frac{11}{3\sqrt{2}}\tan^{-1} \left( \frac{3x+1}{\sqrt{2}}\right) }

\newcommand\trma{ \left(x + \frac{1}{3} \right)}
\newcommand\trmk{ \left(\frac{\sqrt{2}}{3} \right) }

%text
Evaluate the following integral 
\[ \quad I = \int \frac{5x-2}{\ea}\cdot dx \]

%

\newcard

\begin{align}
\text{If } f(x) &= \ea\text{ then } \\ 
	\frac{5x-2}{\ea} &= \frac{A\cdot\ddx f(x) + B}{\ea} 
\end{align}

because $f(x)$ cannot be factorized further 

\newcard 

\begin{align}
	\text{For } f(x) &= \ea \\ 
	D &= b^2-4ac = 2^2 - 4\cdot 3\cdot 1 < 0 
\end{align}

Therefore $f(x)$ has no real roots. Which means it cannot be factorized further \newline 

Moreover, the degree of $5x-2 (=1)$ is \underline{one less} than the degree of $f(x) (=2)$\newline 

Hence, it is best if we do the following 
\begin{align}
	\frac{5x-2}{\ea} &= \frac{A\cdot \ddx f(x) + B}{\ea}
\end{align}

\newcard 

\[ I = \underbrace{\frac{5}{6}\int \frac{6x+2}{\ea}\cdot dx }_P - \underbrace{\frac{11}{3}\int \frac{dx}{\ea}}_Q \]

\newcard 

\[ I = \underbrace{\frac{5}{6}\int \frac{6x+2}{\ea}\cdot dx }_P + \underbrace{\frac{11}{3}\int \frac{dx}{\ea}}_Q \]

\newcard 

\begin{align}
\frac{5x-2}{\ea} &= \frac{A\cdot \ddx f(x)+B}{\ea} \\
\text{where }\ddx f(x) &= \ddx \left(\ea \right) = 6x + 2 \\
\therefore \frac{5x-2}{\ea} &= \frac{A\cdot \left(6x+2 \right) + B}{\ea} \\
\text{or } 5x - 2 &= 6A x + \left(2A + B \right) \\
\therefore 6A &= 5 \implies A = \frac{5}{6} \\
\text{and } 2A + B &= -2 \implies B = -\frac{11}{3}
\end{align}

Which is why 
\[ I = \underbrace{\frac{5}{6}\int \frac{6x+2}{\ea}\cdot dx }_P - \underbrace{\frac{11}{3}\int \frac{dx}{\ea}}_Q \]

\newcard 

\[ P = \intga + C_1 \]

\newcard 

Recall that  $5x-2 = \frac{5}{6}\cdot\ddx f(x) -\frac{11}{3}$ 
\begin{align}
	\therefore P &= \frac{5}{6}\int \frac{f'(x)}{f(x)}\cdot dx = \underbrace{\frac{5}{6}\int \frac{dz}{z}}_{\text{Let } z = f(x)} \\
	&= \frac{5}{6}\log z + C_1 = \frac{5}{6}\log f(x) + C_1 \\
	&= \intga 
\end{align}

\newcard 

\begin{align}
	\ea &= 3 \left[ \trma^2  + \trmk^2 \right] \\
	\therefore Q &= \intgb + C_2 \\
	\text{and } I &= \intga \\ 
	&- \intgb 
\end{align}

\newcard 
\begin{align}
	\ea &= 3 \left[ \left(x-\frac{1}{3} \right)^2  + \left(\frac{1}{\sqrt{3}} \right)^2 \right] \\
	\therefore Q &= \frac{11}{\sqrt{3}}\tan^{-1} \left(\frac{3x-1}{\sqrt{3}} \right) + C_2 \\
	\text{and } I &= \intga \\
	&- \frac{11}{\sqrt{3}}\tan^{-1} \left(\frac{3x-1}{\sqrt{3}} \right) + C  
\end{align}

\newcard 

For integrals of the form $\int \frac{dx}{f(x)}$ where $f(x)$ \underline{cannot} be 
factorized further, it is best to express $f(x) = (x+a)^2 + b^2$ \newline 

This is called \underline{completing the squares}

\begin{align}
	\ea &= 3 \left[x^2 + \frac{2}{3}x + \frac{1}{3} \right] \\
	&= 3 \left[\trma^2 + \frac{1}{3} - \frac{1}{9} \right] \\
	&= 3 \left[\trma^2 + \trmk^2 \right]
\end{align}

As a result 
\begin{align}
Q &= \frac{11}{3}\cdot \int \frac{dx}{3 \left[\trma^2 + \trmk^2 \right]} \\
&= \frac{11}{9}\cdot \underbrace{\frac{3}{\sqrt{2}}\tan^{-1} \frac{\trma}{\trmk}}_{\int \frac{dx}{x^2 + a^2} = \frac{1}{a}\tan^{-1} \frac{x}{a}} \\
&= \intgb \\
\text{and } I &= \intga  \\
&- \intgb + C 
\end{align}

\end{document}