\documentclass[14pt,fleqn]{extarticle}
\RequirePackage{prepwell}

\previewoff

\begin{document}

%text
Using properties of definite integrals, 
evaluate the following 
\[ \qquad \qquad \int_0^\pi\dfrac{x\cdot\tan x}{\sec x + \tan x} \cdot dx \]
%

\newcard 

\begin{align}
A &= \int_0^\pi \dfrac{x\cdot\tan x}{\tan x+\sec x}dx \\
&= \int_0^\pi \dfrac{(\pi-x)\cdot\tan x}{\tan x+\sec x}dx \\
\implies A + A &= 2A = \pi\int_0^\pi\dfrac{\tan x}{\tan + \sec x}dx 
\end{align}

\newcard 

%text
There is a nice little rule that says 
\[ \qquad \int_0^a f(x)\cdot dx = \int_0^a f(a-x)\cdot dx \] 

When applied to the given integral, we get 
%
\begin{align}
\int_0^\pi \dfrac{x\cdot\tan x}{\tan x + \sec x} dx
&= \int_0^\pi\dfrac{(\pi -x)\cdot\tan (\pi-x)\cdot dx}{\tan(\pi-x) + \sec(\pi -x)} \\
&= \underbrace{\int_0^\pi\dfrac{(\pi - x)\cdot \tan x}{\tan x + \sec x}dx}
_{\tan(\pi-x) = -\tan x, \sec(\pi -x) = -\sec x}
\end{align}

%text

And hence, 
%
\begin{align}
  A &= \int_0^\pi\dfrac{x\cdot\tan x}{\tan x + \sec x}\cdot dx \\
  &= \int_0^\pi\dfrac{(\pi - x)\cdot \tan x}{\tan x + \sec x}\cdot dx \\
  \implies 2A &= A + A = \int_0^\pi\dfrac{\tan x\cdot (x + \pi -x)}{\tan x+\sec x}\cdot dx \\
  &= \pi\int_0^\pi\underbrace{\dfrac{\tan x}{\tan x + \sec x}\cdot dx}_{\text{Got rid of the x}}
\end{align}

\newcard 

\begin{align}
2A &= \pi\int_0^\pi\dfrac{\tan x}{\tan x + \sec x}dx \\ 
&= \pi\int_0^\pi\left[1-\dfrac{1}{1+\sin x}\right]\cdot dx \\
&= \pi\left[\pi - \int_0^\pi\dfrac{1}{1+\cos \left(\frac{\pi}{2}-x\right)}\cdot dx\right] \\
&= \pi\left[\pi - \frac{1}{2}\underbrace{\int_0^\pi\sec^2\left(\frac{\pi}{4}-\frac{x}{2}\right)\cdot dx}_{A_2}\right]
\end{align}

\newcard 

\begin{align}
2A &= \pi\int_0^\pi\dfrac{\tan x}{\tan x + \sec x}dx \\ 
&= \pi\int_0^\pi\left[1+\dfrac{1}{1+\sin x}\right]\cdot dx \\
&= \pi\left[\pi + \int_0^\pi\dfrac{1}{1+\cos \left(\frac{\pi}{2}-x\right)}\cdot dx\right] \\
&= \pi\left[\pi + \frac{1}{2}\underbrace{\int_0^\pi\csc^2\left(\frac{\pi}{4}-\frac{x}{2}\right)\cdot dx}_{A_2}\right]
\end{align}

\newcard 

\begin{align}
2A &= \pi\int_0^\pi\dfrac{1+\sin x-1}{1+\sin x}\cdot dx \\
&=\pi\int_0^\pi\left[1-\dfrac{1}{1+\sin x}\right]\cdot dx \\
&= \pi\left[\int_0^\pi dx -\int_0^\pi\dfrac{1}{1+\sin x}\cdot dx\right] \\
&= \pi\left[\pi - \int_0^\pi\dfrac{1}{\underbrace{1+\cos \left(\frac{\pi}{2}-x\right)}_{\sin x = \cos \left(\frac{\pi}{2}-x\right)}}\cdot dx\right] \\
&= \pi\left[\pi - \underbrace{\frac{1}{2}\int_0^\pi\sec^2\left(\frac{\pi}{4}-\frac{x}{2}\right)\cdot dx}_
{\frac{1}{1+\cos\theta} = \frac{1}{2\cos^2\frac{\theta}{2}} = \frac{\sec^2\theta}{2}}\right]
\end{align}

\newcard 

%text
\underline{Evaluating $A_2$ from the previous step}

%
\begin{align}
A_2 &= \int_0^\pi\sec^2 \left(\frac\pi{4}-\frac{x}{2} \right)\cdot dx \\
&= -2\int_{\frac\pi{4}}^{-\frac\pi{4}}\sec^2 z\cdot dz = 2\int_{-\frac\pi{4}}^{\frac\pi{4}}\sec^2 z\cdot dz
\end{align}
%text

where \[ \qquad z = \frac\pi{4}-\frac{x}{2} \] 
%

\newcard 

\begin{align}
A_2 &= \int_0^\pi\sec^2\left(\frac{\pi}{4}-\frac{x}{2}\right)\cdot dx \\
\text{Let } z &= \frac{\pi}{4}-\frac{x}{2}\implies dx = -2\cdot dz \\
\text{Also } x=0\implies z&=\frac\pi{4}\text{ and }x =\frac\pi{4}\implies z = -\frac\pi{4} \\
\therefore A_2 &= -2\int_{\frac{\pi}{4}}^{-\frac{\pi}{4}}\sec^2 z\cdot dz \\
&= 2\int_{-\frac{\pi}{4}}^{\frac{\pi}{4}}\sec^2 z\cdot dz 
\end{align}

\newcard 

\begin{align}
A_2 &= 2\int_{-\frac\pi{4}}^{\frac\pi{4}}\sec^2 z\cdot dz = 4\\
\therefore 2A &= \pi\left[ \pi-\frac{1}{2}A_2 \right] \implies 
A = \frac\pi{2}\cdot(\pi-2) 
\end{align}

\newcard 

\begin{align}
A_2 &= 2\int_{-\frac\pi{4}}^{\frac\pi{4}}\sec^2 z\cdot dz = -2\\
\therefore 2A &= \pi\left[ \pi-\frac{1}{2}A_2 \right] \implies 
A = \frac\pi{2}\cdot(\pi+ 1) 
\end{align}

\newcard 

\begin{align}
A_2 &= 2\int_{-\frac\pi{4}}^{\frac\pi{4}}\sec^2 z\cdot dz = 2\left[\tan z \right]_{-\frac\pi{4}}^{\frac\pi{4}}\\
&= 2\cdot (1 + 1) = 4 \\
\therefore 2A &= \pi\left[ \pi-\frac{1}{2}A_2 \right] = \pi\cdot(\pi - 2)\\
\implies A &= \frac\pi{2}\cdot(\pi-2) 
\end{align}

\end{document}
