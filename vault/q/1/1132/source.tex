\documentclass[14pt,fleqn]{extarticle}
\RequirePackage{prepwell-eng}
\previewoff
\begin{document}

\newcommand\intg{\int_{\frac{\pi}{6}}^{\frac{\pi}{3}}}
\newcommand\ea{ 1 + \sqrt{\cot x}}
\newcommand\eb{ \frac{\sqrt{\tan x}}{1+\sqrt{\tan x}} }
\begin{problem}
	\statement 
    
     Evaluate the following
     \[ \qquad A = \intg \frac{dx}{\ea} \]
     
     \begin{step}
  \begin{options} 
     \correct 
      
      Given that $\frac\pi{3} + \frac\pi{6} = \frac\pi{2}$ 
      \[ \qquad A = \intg \frac{dx}{1+\sqrt{\tan x}} \]
        
    \end{options} 
     \reason 
     
     There is a nice little rule that says 
     \[ \quad \int_a^b f(x)\cdot dx = \int_a^b f \left(a+b-x \right)\cdot dx \]
     When applied to the given integral 
     \begin{align}
	   \intg \frac{dx}{\ea} &= \intg \frac{dx}{1+\sqrt{\cot \left(\frac\pi{6}+\frac\pi{3} -x \right)}} \\
	   &= \intg \frac{dx}{1 + \sqrt{\cot \left(\frac\pi{2} - x \right)}} \\
	   &= \intg \frac{dx}{1+\sqrt{\tan x}}
\end{align}
       
\end{step}

\begin{step}
  \begin{options} 
     \correct 
      
      \begin{align}
	      A + A &= 2A = \intg dx \implies A = \frac{\pi}{12}
\end{align}
     \incorrect
      
      \begin{align}
	A + A &= 2A = \intg \tan x \cdot dx \\
	&= \left[\log \vert \sec x \vert \right]_{\frac\pi{6}}^{\frac\pi{3}} \\
	&= \log 2 - \log \frac{2}{\sqrt{3}} = \frac{\log 3}{2} 
\end{align}  
    \end{options} 
     \reason 
     
     \begin{align}
	    A &= \intg \frac{dx}{\ea} = \intg \frac{dx}{1 + \sqrt{\frac{1}{\tan x}}} \\
	     &= \intg \eb\cdot dx \\
	     A &= \intg \frac{dx}{1 + \sqrt{\tan x}}\text{ (see previous step) } \\
	     \therefore A + A &= \intg \eb\cdot dx + \intg \frac{dx}{1+\sqrt{\tan x}} \\
	     &= \intg \left[\eb + \frac{1}{1+\sqrt{\tan x}} \right]\cdot dx \\
	     &= \intg dx = \left(\frac\pi{3} - \frac\pi{6} \right) = \frac\pi{6} \\
	     \therefore A &= \frac\pi{12} 
\end{align}  
       
\end{step}
\end{problem} 
\end{document}