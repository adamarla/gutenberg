\documentclass[14pt,fleqn]{extarticle}
\RequirePackage{prepwell}
\previewoff
\begin{document}

%text
A pair of dice is thrown 4 times. If getting a 
doublet is considered a success, then find
the probability distribution of successes
%

\newcard

If the two die -- $I$ and $II$ -- are thrown, then we would get a doublet if 
either of the following combinations occur 

\begin{center}
  \begin{tabular}{N|NNNNNN}
  \toprule
   \text{Die} & & & & & & \\
   \midrule
   I & 1 & 2 & 3 & 4 & 5 & 6 \\      
  \midrule 
   II & 1 & 2 & 3 & 4 & 5 & 6 \\
   \bottomrule
  \end{tabular}
\end{center}

\newcard

If the two die -- $I$ and $II$ -- are thrown, then we would get a doublet if 
either of the following combinations occur 

\begin{center}
  \begin{tabular}{N|NNNNNN}
  \toprule
   \text{Die} & & & & & & \\
   \midrule
   I & 1 & 2 & 3 & 2 & 4 & 6 \\      
  \midrule 
   II & 2 & 4 & 6 & 1 & 2 & 3 \\
   \bottomrule
  \end{tabular}
\end{center}

\newcard 

A doublet is when the \underline{same number} occurs on both the die\newline 

Hence, we will get a doublet when 

\begin{center}
  \begin{tabular}{N|NNNNNN}
  \toprule
   \text{Die} & & & & & & \\
   \midrule
   I & 1 & 2 & 3 & 4 & 5 & 6 \\      
  \midrule 
   II & 1 & 2 & 3 & 4 & 5 & 6 \\
   \bottomrule
  \end{tabular}
\end{center}

\newcard 

If $p$ be the probability of getting a doublet in \underline{a given throw}, then $p= \frac{1}{6}$ 

\newcard 

If you roll a 1 (or a 2, or a 3 $\ldots$) on die $I$, then to get a doublet, die $II$ should also roll to the same number \newline 

Hence, the probability of getting a doublet \underline{on a given throw} is simply 
\[ \qquad p = P \left(II = I\vert I \right) = \frac{1}{6} \]

\newcard 

The probability of getting \underline{$N$ doublets} in four throws is 
\[ \qquad P(N) = {^4}C_N \cdot p^N\cdot q^{4-N},\, \left(N \leq 4\right) \]

where $q = 1-p = \frac{5}{6}$ 

\newcard 

The probability of getting \underline{$N$ doublets} in four throws is 
\[ \qquad P(N) = p^N\cdot q^{4-N},\, \left(N \leq 4\right) \]

where $q = 1-p = \frac{5}{6}$ 

\newcard 

You could get a doublet in first \& second throws or the first \& third throws or in any of the \underline{$^4C_N$ possible combinations} \newline 

Each of these $^4C_N$ combinations, however, represents $N$ successes and $4-N$ failures\newline 

Hence, each of these combinations will occur with a probability of $p^N\cdot q^{4-N}$\newline 

And therefore, the \underline{total probability} of getting $N$ doublets is 
\[ \qquad P(N) = ^4C_N\cdot p^N\cdot q^{4-N}\, \left(N\leq 4 \right)\]

\newcard

The probability distribution will therefore be as follows 

\begin{center}
\begin{tabular}{NN}
\toprule
\text{\# of doublets} = N & \text{Probability} \\
\midrule 
0 & \frac{625}{1296} \\
\midrule 
1 & \frac{125}{324} \\
\midrule 
2 & \frac{25}{216} \\
\midrule 
3 & \frac{5}{324} \\
\midrule 
4 & \frac{1}{1296} \\
\bottomrule
\end{tabular}
\end{center} 

\newcard

%text
We know that the probability of getting $N$ doublets in four throws is 
\[ \qquad P(N) = ^4C_N \cdot p^N\cdot q^{4-N}\, \left(N\leq 4 \right)\]

The probability distribution table is simply the value of $P(N)$ for $N=0,1,2,3,4$
%

\begin{center}
\begin{tabular}{NNN}
\toprule 
\text{\# of doublets} & \text{Expression} & \text{Probability} \\
\midrule 
0 & ^4C_0\cdot p^0\cdot q^4 & \frac{625}{1296} \\
\midrule 
1 & ^4C_1 \cdot p^1\cdot q^3 & \frac{125}{324} \\
\midrule 
2 & ^4C_2 \cdot p^2 \cdot q^2 & \frac{25}{216} \\
\midrule 
3 & ^4C_3\cdot p^3\cdot q^1 & \frac{5}{324} \\
\midrule 
4 & ^4C_4 \cdot p^4\cdot q^0 & \frac{1}{1296} \\
\bottomrule
\end{tabular}
\end{center}

\end{document}