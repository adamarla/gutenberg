\documentclass[14pt,fleqn]{extarticle}
\RequirePackage{prepwell}

\previewoff

\begin{document}

%text
Evaluate the following \[ \qquad\quad \int_0^1 \cot^{-1}(1-x+x^2)\cdot dx \]
% 

\newcard 

\begin{align}
\cot^{-1}\left( 1-x+x^2\right) = \tan^{-1}&\left( \dfrac{1}{1-x+x^2}\right) 
\end{align}

\newcard 

\begin{align}
\cot^{-1}\left( 1-x+x^2\right) &= \dfrac{1}{\tan^{-1} \left( 1-x+x^2\right)}
\end{align}

\newcard 

%text
Generally speaking, dealing with $\tan\phi$ or 
$\tan^{-1} x$ is easier than dealing with $\cot\phi$ 
or $\cot^{-1}x$\newline 

So, let us try to do a  
$\cot^{-1} x\rightarrow\tan^{-1}x$ conversion first

\begin{align}
	\cot^{-1} z &= \tan^{-1} \left(\dfrac{1}{z} \right)  \neq \dfrac{1}{\tan^{-1} z} \\
	\therefore \cot^{-1}\left( 1-x+x^2\right) &= \tan^{-1}\left( \dfrac{1}{1-x+x^2}\right)
\end{align}
 
\newcard 

\[ \tan^{-1}\dfrac{1}{1-x+x^2} = \tan^{-1} x + \tan^{-1} \left(1-x\right) \]

\newcard 

\[ \tan^{-1}\dfrac{1}{1-x+x^2} = \tan^{-1} x + \tan^{-1} \left(1+x\right) \]

\newcard 

%text
Recall that \[\qquad\tan^{-1} a + \tan^{-1} b = \tan^{-1}\left(\frac{a + b}{1-ab}\right)\]

And hence, 
%
\begin{align}
\tan^{-1}x + \tan^{-1}(1-x) &= \tan^{-1}\left[ \dfrac{x + (1-x)}{1-\lbrace x\cdot(1-x)\rbrace}\right] \\
&= \tan^{-1}\left[ \dfrac{1}{1-\lbrace x + x^2\rbrace}\right] \\
\end{align}

\newcard 

\begin{align}
	&\cot^{-1} \left(1-x+x^2 \right) = \tan^{-1} x + \tan^{-1} \left(1-x \right) \\
	&\therefore \int_0^1 \cot^{-1} \left(1-x+x^2 \right)\cdot dx = 2\int_0^1 \tan^{-1} x\cdot dx 
\end{align}

\newcard 

\begin{align}
A &= \int_0^1 \cot^{-1}(1-x+x^2)\cdot dx \\
&= \int_0^1\tan^{-1}x\cdot dx +\int_0^1 \tan^{-1}\left( 1-x \right)\cdot dx
\end{align}

But 
\begin{align}
	\int_0^1\tan^{-1} \left(1-x \right)\cdot dx &= \int_0^1\tan^{-1} \left[1-(1-x) \right]\cdot dx \\
	&= \int_0^1 \tan^{-1} x\cdot dx
\end{align}

And hence, 
\begin{align}
A = \int_0^1 \cot^{-1}(1-x+x^2)\cdot dx &= 2\int_0^1\tan^{-1} x\cdot dx 
\end{align}


\newcard 

\begin{align}
A &= 2\int_0^1 \tan^{-1} x\cdot dx  \\
\therefore \text{if }x &= \tan \phi\text{ then }  \\
A &= 2\int_0^{\frac\pi{4}}\phi\cdot\sec^2\phi\cdot d\phi
\end{align}

\newcard 

\begin{align}
A &= 2\int_0^1 \tan^{-1} x\cdot dx  \\
\therefore \text{if }x &= \tan \phi\text{ then }  \\
A &= 2\int_0^{\frac\pi{4}}\phi\cdot d\phi
\end{align}

\newcard 

\begin{align}
A &= 2\int_0^1\tan^{-1}x\cdot dx \\
\text{Let }x &= \tan\phi \implies dx = \sec^2\phi\cdot d\phi \\
\therefore A &= 2\underbrace{\int_0^{\frac\pi{4}}\phi\cdot\sec^2\phi\cdot d\phi}
_{x = 0\rightarrow 1\implies \phi = 0\rightarrow\frac\pi{4} }
\end{align}

\newcard 

\begin{align}
A &= 2\int_0^{\frac\pi{4}}\phi\cdot\sec^2\phi\cdot d\phi \\
&= 2\left[ \phi\tan\phi + \log \vert \cos \phi \vert \right]_0^{\frac\pi{4}} \\
&= \frac\pi{2} - \log 2 
\end{align}

\newcard 

\begin{align}
A &= 2\underbrace{\left[ \phi\tan\phi-\int\left[\frac{d\phi}{d\phi}\int\sec^2\phi\cdot d\phi\right]d\phi\right]_0^{\frac\pi{4}}}_
{\text{Integration by parts}} \\
&= 2\left[ \phi\tan\phi - \int\tan\phi\cdot d\phi \right]_0^{\frac\pi{4}} \\
&= 2\left[ \phi\tan\phi + \log\vert\cos\phi\vert \right]_0^{\frac\pi{4}}  \\
&= 2\left[ \left( \frac\pi{4} + \log\frac{1}{\sqrt{2}}\right) - 0 \right] \\
&= 2\left( \underbrace{\frac\pi{4} -\frac{1}{2}\log 2}_{\log\frac{1}{\sqrt{2}} = \log 2^{-\frac{1}{2}} = -\frac{1}{2}\log 2}\right)\\
&= \frac\pi{2} - \log 2 
\end{align}

\end{document}


