
 % \printrubric

\ifnumequal{\value{rolldice}}{0}{
  % variables 
  \renewcommand{\va}{5}
  \renewcommand{\vb}{2}
  \renewcommand{\vc}{6}
}{
  \ifnumequal{\value{rolldice}}{1}{
    % variables 
    \renewcommand{\va}{7}
    \renewcommand{\vb}{4}
    \renewcommand{\vc}{3}
  }{
    \ifnumequal{\value{rolldice}}{2}{
      % variables 
      \renewcommand{\va}{3}
      \renewcommand{\vb}{2}
      \renewcommand{\vc}{2}
    }{
      % variables 
      \renewcommand{\va}{9}
      \renewcommand{\vb}{4}
      \renewcommand{\vc}{5}
    }
  }
}

\renewcommand{\vd}{\va + \vb\cdot\sqrt{\vc}}
\renewcommand{\ve}{\va - \vb\cdot\sqrt{\vc}}
\POWER\va{2}\a
\EXPR[0]{\b}{(\vb * \vb * \vc)}
\ADD\va\va\c
\SQUARE\c\d
\SUBTRACT\d{2}\ans

\question[3] What is the value of $a^2+b^2$ if, 
  \[ a = \va + \vb\cdot\sqrt{\vc}\text{ and } b = \frac{1}{a} \]

\watchout

\begin{solution}[\halfpage]
	\begin{align}
		b = \dfrac{1}{a} &= \dfrac{1}{\vd} = \dfrac{1}{\vd}\times\dfrac{\ve}{\ve} \\
		&= \dfrac{\ve}{\a-\b} = \ve
	\end{align}
	Moreover, \begin{align}
		a^2 + b^2 &= (a+b)^2 - 2ab \\
		\text{where } (a+b) &= (\vd) + (\ve) = \c \\
		\text{and } a\cdot b &= (\vd)\cdot (\ve) = 1 \\
		\implies a^2 + b^2 &= \c^{2} - 2 = \ans
	\end{align}
\end{solution}

\ifprintanswers\begin{codex}$\ans$\end{codex}\fi
