\documentclass[14pt,fleqn]{extarticle}
\RequirePackage{prepwell}
\previewoff
\begin{document}

%text
A family has 2 children. Find the probability
that both are boys, if it is known that 

\begin{itemize}
\item {at least one of the children is a boy}
\item {the elder child is a boy}
\end{itemize}

%

\newcard

If $N_B = $ number of boys in the family and 
\begin{center}
  \begin{tabular}{Nc}
  \toprule
        \text{Event} & Meaning \\
   \midrule
   C_1 & First child is a boy \\ 
    \midrule 
    C_2 & Second child is a boy \\
    \bottomrule
  \end{tabular}
\end{center}

then we have to find 
\[ P\left( N_B = 2\,\vert\, N_B\geq 1\right)\text{ and } P\left( N_B=2\,\vert\, C_1\right)\]

\newcard

If $N_B = $ number of boys in the family and 
\begin{center}
  \begin{tabular}{Nc}
  \toprule
        \text{Event} & Meaning \\
   \midrule
   C_1 & First child is a boy \\ 
    \midrule 
    C_2 & Second child is a boy \\
    \bottomrule
  \end{tabular}
\end{center}

then we have to find 
\[ P\left( N_B = 2\cap N_B\geq 1\right)\text{ and } P\left( N_B=2\cap C_1\right)\]

\newcard 

In each of the two cases, we have to find the probability that $N_B=2$\newline 

But in each case, we have been told something. Either that the family has atleast one 
boy $N_B\geq 1$ or that the first child is a boy $C_1$\newline 

Hence, the probabilities we need to find are \underline{conditional}. And the 
expressions for these probabilities are simply 
\[ P\left( N_B=2\,\vert\, N_B\geq 1\right)\text{ and } P\left( N_B=2\,\vert\, C_1\right)\]

\newcard 

\begin{align}
	\condp{N_B=2}{N_B\geq 1} &= \fcondp{N_B=2}{N_B\geq 1}\\
	\text{where } P \left(N_B\geq 1 \right) &= \frac{3}{4} \text{ and } P \left(N_B = 2 \right) = \frac{1}{4} \\
	\therefore \condp{N_B=2}{N_B\geq 1} &= \frac{1}{3} 
\end{align}

\newcard 

From Baye's Theorem, we know that 
\small\[ \condp{N_B=2}{N_B\geq 1} = \fcondp{N_B=2}{N_B\geq 1}\]\normalsize 

But in the above expression 
\begin{align}
	\condp{N_B\geq 1}{N_B = 2} &= 1 \left( \because N_B = 2\implies N_B \geq 1\right) \\
	P \left(N_B =2 \right) & = P \left(C_1 \right)\cdot P \left(C_2 \right) \\
	&= \frac{1}{2}\cdot\frac{1}{2} = \frac{1}{4} \\ 
	P \left(N_B\geq 1 \right) &= 1- P \left(N_B = 0 \right) \\
	&= 1- \left[\frac{1}{2}\cdot\frac{1}{2} \right] = \frac{3}{4} 
\end{align}

And therefore 
\[ \condp{N_B=2}{N_B \geq 1 } = \frac{1\times\frac{1}{4}}{\frac{3}{4}} = \frac{1}{3} \] 

\newcard 

Similarly $\condp{N_B=2}{C_1}$ will be as follows 

\begin{align}
\condp{N_B=2}{C_1} &= \fcondp{N_B=2}{C_1}  \\ 
&= \frac{1}{2} 
\end{align}

\newcard 

Again, Bayes' Theorem tells us that 

\begin{align}
	\condp{N_B=2}{C_1} &= \fcondp{N_B=2}{C_1}  \\
	\text{But } \condp{C_1}{N_B=2} &= 1 \left(\because N_B=2 \implies C_1 \right) \\
	P \left(N_B=2 \right) &= \frac{1}{2}\cdot\frac{1}{2} = \frac{1}{4} \\
	P \left(C_1 \right) &= \frac{1}{2}
\end{align}

And therefore 
\[ \condp{N_B=2}{C_1} = \dfrac{1\times\frac{1}{4}}{\frac{1}{2}} = \frac{1}{2} \]



\end{document}