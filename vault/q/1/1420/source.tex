\documentclass[14pt,fleqn]{extarticle}
\RequirePackage{prepwell}

\newcommand\pa{\frac{1}{7}}
\newcommand\pb{\frac{2}{7}}
\newcommand\pc{\frac{4}{7}}

\previewoff

\begin{document} 
\begin{question}
	\statement 
	
    Three persons $A,B$ and $C$ apply for the job of a Manager in a Private Company. Chances of their selection are in the ratio $1:2:4$. The probabilities that $A,B$ and $C$ can introduce changes to improve the profits of the company are 
    $0.8,0.5$ and $0.3$ respectively\newline 
    
    If the change does not take place, then find the probability that it 
    is due to the appointment of $C$ 
      
   \begin{step}
  \begin{options} 
     \correct 
      
      If we define the following events $\ldots$ 
      \begin{center}
  \begin{tabular}{NlNl}
   \toprule
   \text{Event} & Meaning & \text{Event} & Meaning \\
   \midrule
        A & $A$ is hired & B & $B$ is hired \\
   \midrule
   C & $C$ is hired & I & Profits improve \\ 
    \bottomrule
  \end{tabular}
\end{center} 

$\ldots$ then we know the following 
\begin{center}
  \begin{tabular}{NNN}
   \toprule
        \prob{A} = \frac{1}{7} & \prob{B} = \frac{2}{7} & \prob{C} = \frac{4}{7} \\
   \midrule 
        \condp{I}{A} = 0.8 & \condp{I}{B} = 0.5 & \condp{I}{C} = 0.3 \\
    \bottomrule
  \end{tabular}
\end{center}
     \incorrect
     
     If we define the following events $\ldots$ 
      \begin{center}
  \begin{tabular}{NlNl}
   \toprule
   \text{Event} & Meaning & \text{Event} & Meaning \\
   \midrule
        A & $A$ is hired & B & $B$ is hired \\
   \midrule
   C & $C$ is hired & I & Profits improve \\ 
    \bottomrule
  \end{tabular}
\end{center} 

$\ldots$ then we know the following 
\begin{center}
  \begin{tabular}{NNN}
   \toprule
        \prob{A} = \frac{1}{7} & \prob{B} = \frac{2}{7} & \prob{C} = \frac{4}{7} \\
   \midrule 
        \prob{I\cap A} = 0.8 & \prob{I\cap B} = 0.5 & \prob{I\cap C} = 0.3 \\
    \bottomrule
  \end{tabular}
\end{center}
        
    \end{options} 
     \reason 
     
     It is not said explicitly in the question, but is understood, that one of $A,B$ or $C$ is \underline{definitely} hired. Therefore,
     \begin{align}
     \text{If }\prob{A} =x &\text{ then } \prob{B}= 2x\\
     &\text{ and }\prob{C} = 4x \\
     \implies x + 2x + 4x &= 1 \text{ or } x = \frac{1}{7} 
\end{align}\\[-25pt]
\[ \text{Hence, }\prob{A} = \pa, \prob{B} = \pb, \prob{C} = \pc \]
Moreover, we have been told the probabilities of profits improving \underline{given that} a certain person is hired\newline 

Hence $\condp{I}{A}$ --- and not $\prob{I\cap A}$ \newline 

These, we have been told, are 
\begin{center}
  \begin{tabular}{NlN}
   \toprule
        \text{Expression} & Meaning & \text{Value}  \\
   \midrule 
   \condp{I}{A} & Improvement if $A$ hired & 0.8 \\
    \midrule 
    \condp{I}{B} & Improvement if $B$ hired & 0.5 \\
    \midrule
    \condp{I}{C} & Improvement if $C$ hired & 0.3 \\
    \bottomrule
  \end{tabular}
\end{center}
       
\end{step}

\begin{step}
  \begin{options} 
     \correct 
      
      And we have to find $\condp{C}{I'}$ 
     \incorrect
     
     And we have to find $\condp{I'}{C}$ 
        
    \end{options} 
     \reason 
       
     Someone was hired --- and the profits \underline{didn't} improve. We want to know 
     the probability that $C$ was hired \newline 
     
     In other words, we want to know $\condp{C}{I'}$    
\end{step}

\begin{step}
  \begin{options} 
     \correct 
      
      \begin{align}
	\condp{C}{I'} &= \fcondp{C}{I'} \\[5pt]
	&= \dfrac{\condp{I'}{C}\cdot\prob{C}}{\sum_{X = A}^C \condp{I'}{X}\cdot\prob{X}} \\
	\text{where } \condp{I'}{A} &= 0.2, \condp{I'}{B} = 0.5 \\[-10pt]
	\text{and }\condp{I'}{C} &= 0.7  \\[10pt]
	\therefore \condp{C}{I'} &= \frac{7}{10} 
\end{align} 

Hence, if profits didn't improve, then there is a $70\%$ chance that $C$ was hired 
     
    \end{options} 
     \reason 
     
     Bayes' Theorem tells us that 
     \begin{align}
     \condp{C}{I'} &= \fcondp{C}{I'} \\[5pt]
     &= \dfrac{\condp{I'}{C}\cdot\prob{C}}{\underbrace{\sum_{X=A}^C\condp{I'}{X}\cdot\prob{X}}_{=\condp{I'}{A}\prob{A} + \ldots + \condp{I'}{C}\prob{C}}}
\end{align}

But we will need $\condp{I'}{A}$ etc. to find the above value. And given that 
\[ \qquad \condp{I}{X} + \condp{I'}{X} = 1 \] 
Therefore
\begin{align}
\condp{I'}{A} &= 1-\condp{I}{A} = 0.2 \\
\condp{I'}{B} &= 1-\condp{I}{B} = 0.5 \\
\condp{I'}{C} &= 1-\condp{I}{C} = 0.7 
\end{align}

And hence, 
\begin{align}
\condp{C}{I'} &= \dfrac{\condp{I'}{C}\cdot\prob{C}}{\sum_{X=A}^C\condp{I'}{X}\cdot\prob{X}} \\[10pt]
&= \dfrac{\frac{7}{10}\cdot\pc}{\left[\frac{2}{10}\cdot\pa + \frac{1}{2}\cdot\pb + \frac{7}{10}\cdot\pc \right]} \\[10pt]
&= \frac{7}{10} 
\end{align}
       
\end{step}
\end{question} 
\end{document} 