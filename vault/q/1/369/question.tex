
\ifnumequal{\value{rolldice}}{0}{
  % variables 
  \renewcommand{\va}{3}
  \renewcommand{\vb}{2}
  \renewcommand\vj{485}
  \renewcommand\vk{198}
}{
  \ifnumequal{\value{rolldice}}{1}{
    % variables 
    \renewcommand{\va}{5}
    \renewcommand{\vb}{2}
    \renewcommand\vj{1183}
    \renewcommand\vk{374}
  }{
    \ifnumequal{\value{rolldice}}{2}{
      % variables 
      \renewcommand{\va}{4}
      \renewcommand{\vb}{3}
      \renewcommand\vj{1351}
      \renewcommand\vk{390}
    }{
      % variables 
      \renewcommand{\va}{6}
      \renewcommand{\vb}{2}
      \renewcommand\vj{1664}
      \renewcommand\vk{480}
    }
  }
}

\renewcommand\vx{\sqrt\va}
\renewcommand\vy{\sqrt\vb}
\renewcommand\vz{\sqrt{\va\cdot\vb}}
\MULTIPLY\va\vb\vc % a*b 
\MULTIPLY\vj{2}\vl

\question If $S=\overbrace{(\sqrt\va + \sqrt\vb)^6}^{A} + \overbrace{(\sqrt\va - \sqrt\vb)^6}^{B}$, then

\watchout

\begin{parts}
  \part[1] Find the value of $A$. 

\begin{solution}[\mcq]
  	\begin{align}
  		A &= \sum_{k=0}^{6}\binom{6}{k}(\vx)^{6-k}\cdot(\vy)^{k} \\
  		&= \binom{6}{0}(\vx)^{6} + \binom{6}{1}(\vx)^{5}(\vy) + \binom{6}{2}(\vx)^4(\vy)^2 \nonumber\\
  		&+ \binom{6}{3}(\vx)^3(\vy)^3 + \binom{6}{4}(\vx)^2(\vy)^4 \nonumber\\
  		&+ \binom{6}{5}(\vx)(\vy)^5 + \binom{6}{6}(\vy)^6
  	\end{align}
    Calculate the individual terms in (2). And if you do the calculations correctly, 
    then you should get 
    \begin{align}
    	A &= \vj + \vk\sqrt\vc
    \end{align}
  \end{solution}

  \part[2] Find the value of $B$. 

\begin{solution}[\mcq]
  	If you have done (a), then this shouldn't be too much trouble either. However, do note that 
  	\begin{align}
  		B &= \sum_{k=0}^{6}\binom{6}{k}(\vx)^{6-k}\cdot\overbrace{(-\vy)^k} \nonumber\\
  		&= \sum_{k=0}^{6}\binom{6}{k}(\vx)^{6-k}\cdot(\vy)^k\cdot(-1)^k \\
       \implies B &= \binom{6}{0}(\vx)^{6} - \binom{6}{1}(\vx)^{5}(\vy) + 
       \binom{6}{2}(\vx)^4(\vy)^2 \nonumber\\
  		&- \binom{6}{3}(\vx)^3(\vy)^3 + \binom{6}{4}(\vx)^2(\vy)^4 \nonumber\\
  		&- \binom{6}{5}(\vx)(\vy)^5 + \binom{6}{6}(\vy)^6 \\
  		&= \vj-\vk\sqrt\vc
  	\end{align}
  \end{solution}

  \part[1] Find the value of $S$ using the values found in parts (a) and (b). 

\begin{solution}[\mcq]
  	\begin{align}
  		S &= (\vj + \vk\sqrt\vc) + (\vj-\vk\sqrt\vc) = \vl
  	\end{align}
  \end{solution}

\end{parts}

\ifprintanswers
  \begin{codex}
    $(a)\,\vj+\vk\sqrt\vc\quad (b)\,\vj-\vk\sqrt\vc\quad (c)\,\vl$
  \end{codex}
\fi 

