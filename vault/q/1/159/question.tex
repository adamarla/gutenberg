
\ifnumequal{\value{rolldice}}{0}{
  % variables 
  \renewcommand{\va}{17}
}{
  \ifnumequal{\value{rolldice}}{1}{
    % variables 
    \renewcommand{\va}{34}
  }{
    \ifnumequal{\value{rolldice}}{2}{
      % variables 
      \renewcommand{\va}{51}
    }{
      % variables 
      \renewcommand{\va}{68}
    }
  }
}

\question[1] $AB$ and $CD$ are two \textbf{equal} chords of a circle whose center is $O$ (see figure).
$OM\perp AB$ and $ON\perp CD$. If $\angle OMN = \ang\va$, then what is $\angle ONM = ?$

\watchout

\figinit{pt}
  \figpt 100: $O$(50,50)
  \figpt 200:(95,50) % reference pt that is rotated
  \figptrot 101:$A$= 200 /100,80/
  \figptrot 102:$C$= 200 /100,100/
  \figptrot 103:$B$= 200 /100,190/
  \figptrot 104:$D$= 200 /100,350/
  \figptorthoprojline 300:$M$= 100 /101,103/
  \figptorthoprojline 301:$N$= 100 /102,104/
\figdrawbegin{}
  \figdrawcirc 100(45)
  \figdrawline [101,103] 
  \figdrawline [102,104] 
  \figdrawaltitude 5 [100,101,103]
  \figdrawaltitude 5 [100,102,104]
  \figdrawline [300,301]
\figdrawend
\figvisu{\figBoxA}{}{%
  \figset write(mark=$\bullet$)
  \figwrites 100:(2)
  \figwriten 101:(3)
  \figwriten 102:(3)
  \figwritew 103:(3)
  \figwritee 104:(3)
  \figwriten 300:(3)
  \figwriten 301:(3)
}

\vspace{1cm}
\centerline{\box\figBoxA}


\begin{solution}[\mcq]
  Chords of the same length are equidistant from the center. Which means, 
   \[ OM = ON \implies \bigtriangleup OMN \text{ is an isoceles triangle} \] 
  Hence
    \[ \angle OMN = \angle ONM = \ang\va \]
\end{solution}

\ifprintanswers\begin{codex}$\ang\va$\end{codex}\fi
