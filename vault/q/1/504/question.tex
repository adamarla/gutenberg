

\question Suppose we count the number of people arriving every 30 minutes at a Starbucks or a 
Coffee Day Express outlet. Let this be our $y$ variable

\begin{parts}
  \part[2] What are some $x$ variables which could help explain variation in $y$? Explain/justify why 
  each identified $x$ makes sense and should be considered

\begin{solution}[\halfpage]
    We could include  \textbf{``seasonality-like'' variables} to distinguish early morning, from lunch time, 
    late afternoon etc. Could also have \textbf{indicators for weekends or holidays, weekdays}. 
    Since arrivals in adjacent time periods could be correlated, we could also include \textbf{lagged dependent 
    variables}, i.e the number of arrivals in the previous time period.
  \end{solution}

  \part[3] Suppose we run two regressions, one linear and one Poisson. What is the difference in the interpretation of the coefficients?

\begin{solution}[\halfpage]
    In the \textbf{linear model}, the coefficients have interpretation in terms of changes in the numerical 
    value of \(y\), that is the number of arrivals.  So if \(\beta_{1}=2\), then in the linear model, 
    a unit increase in \(x_{1}\) leads to an increase of 2 additional arrivals in the 30 minute period.  

    In the \textbf{Poisson version} the interpretation is in terms of percentage change in the \(y\). If 
    \(\beta_{1}=.04\), then the Poisson interpretation is a \(.04\times 100\) = 4\% increase in the number of arrivals.
  \end{solution}

\end{parts}

