
\renewcommand{\va}{6}
\ifnumequal{\value{rolldice}}{0}{
  % variables 
  \renewcommand{\vb}{19}
}{
  \ifnumequal{\value{rolldice}}{1}{
    % variables 
    \renewcommand{\vb}{43}
  }{
    \ifnumequal{\value{rolldice}}{2}{
      % variables 
      \renewcommand{\vb}{21}
    }{
      % variables 
      \renewcommand{\vb}{26}
    }
  }
}

\SQUARE\va\vp
\MULTIPLY\vp\va\vc

\SUBTRACT\vb\va\vd
\SQUARE\vd\ve

\MULTIPLY\vp{4}\vz
\SUBTRACT\ve\vz\vr

\SQRT\vr\vt
%\ROUND[0]\vu\vt

\MULTIPLY\va{2}\vw
\ADD\vd\vt\vx
\SUBTRACT\vd\vt\vy

\FRACDIV\va{1}\vx\vw\p\q
\FRACMULT\va{1}\vx\vw\r\s

\question[3] If the product of three numbers in geometric progression is $\vc$ and 
their sum is $\vb$, then what are the three numbers? 

\watchout

\begin{solution}[\halfpage]
	Let the three numbers be $\dfrac{a}{r}$, $a$ and $ar$. And so, 
	\begin{align}
		\dfrac{a}{r}\cdot a \cdot ar &= \vc \implies a^3 = \vc \implies a = \va
	\end{align}

	The three numbers, therefore, are $\dfrac{\va}{r}$, $\va$ and $\va r$. Moreover, 

	\begin{align}
		\dfrac{\va}{r} + \va + \va r &= \vb \implies \va r^2-\vd r + \va = 0 \\
		\implies r &= \dfrac{\vd\pm\sqrt{\vd^2-4\cdot \va\cdot \va}}{2\times \va} = 
    \WRITEFRAC\vx\vw\text{ or }\WRITEFRAC\vy\vw
	\end{align}

  And hence, the numbers are $\p,\,\va,\,\r ( r=\WRITEFRAC\vx\vw )$ or 
  $\r,\,\va,\,\p (r=\WRITEFRAC\vy\vw)$
\end{solution}

\ifprintanswers\begin{codex}$\p,\,\va\text{ and } \r$\end{codex}\fi
