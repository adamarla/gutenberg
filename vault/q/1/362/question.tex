

\ifnumequal{\value{rolldice}}{0}{
  % variables 
  \renewcommand{\va}{7}
  \renewcommand{\vb}{5}
}{
  \ifnumequal{\value{rolldice}}{1}{
    % variables 
    \renewcommand{\va}{6}
    \renewcommand{\vb}{3}
  }{
    \ifnumequal{\value{rolldice}}{2}{
      % variables 
      \renewcommand{\va}{5}
      \renewcommand{\vb}{7}
    }{
      % variables 
      \renewcommand{\va}{7}
      \renewcommand{\vb}{3}
    }
  }
}

\renewcommand{\vc}{\va + \sqrt{\vb}}
\renewcommand{\vd}{\va - \sqrt{\vb}}
\SQUARE\va\a
\SUBTRACT\a\vb\b
\MULTIPLY\va{4}\c
\FRACTIONSIMPLIFY\c\b\p\q

\question[3] What are the values of $a$ and $b$ if 
\[ \dfrac{\vc}{\vd} - \dfrac{\vd}{\vc} = a + b\cdot\sqrt{\vb} \]


\watchout

\begin{solution}[\halfpage]
	\begin{align}
		\dfrac{\vc}{\vd} &- \dfrac{\vd}{\vc} = \dfrac{(\vc)^2 - (\vd)^2}{\va^{2}-\vb} \\
		&= \dfrac{(\va^2 + 2\cdot\va\cdot\sqrt{\vb} + \vb) - 
		(\va^2 - 2\cdot\va\cdot\sqrt{\vb} + \vb)}{\b} \\
		&= \dfrac{\p}{\q}\cdot\sqrt{\vb}
	\end{align}
	On comparing this with the form $a + b\cdot\sqrt{\vb}$, we can see that 
  \[ a=0\text{ and } b = \frac{\p}{\q} \]
\end{solution}

\ifprintanswers\begin{codex}
$a=0\text{ and } b=\dfrac\p\q$ 
\end{codex}\fi
