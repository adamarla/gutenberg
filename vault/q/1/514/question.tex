
\ifnumequal{\value{rolldice}}{0}{
  \renewcommand\va{2}
  \renewcommand\vb{3}
  \renewcommand\vc{(1,5),(3,7)}
  \renewcommand\vd{1,3}
  \renewcommand\vr{5,7}
}{
  \ifnumequal{\value{rolldice}}{1}{
    \renewcommand\va{3}
    \renewcommand\vb{4}
    \renewcommand\vc{(1,7),(2,8),(4,13)}
    \renewcommand\vd{1,2,4}
    \renewcommand\vr{7,8,13}
  }{
    \ifnumequal{\value{rolldice}}{2}{
      \renewcommand\va{4}
      \renewcommand\vb{3}
      \renewcommand\vc{(1,7),(3,13)}
      \renewcommand\vd{1,3}
      \renewcommand\vr{7,13}
    }{
      \renewcommand\va{3}
      \renewcommand\vb{5}
      \renewcommand\vc{(1,8),(5,16)}
      \renewcommand\vd{1,5}
      \renewcommand\vr{8,16}
    }
  }
}

\providecommand\tqvqfn[1]{
  \MULTIPLY\va{#1}\vx
  \FRACADD\vx{1}\vb{#1}\vy\vz
  \left(#1,\WRITEFRAC\vy\vz \right),
}

\question Given the following relation \textbf{R}
\[ R = \lbrace (x,y) : y=\va x + \frac\vb{x} ; x,y\in\mathbb{N}\text{ and } x < 6 \rbrace \]


\watchout

\begin{parts}
  \part[2] Write the relation in roster form - that is - as a set of ordered-pairs $R=\lbrace (a,b),(c,d)\ldots\rbrace$

\begin{solution}[\halfpage]
    Had there been no constraint on $y$, then $R$ would have been 
    \[
      \left\{ \forcsvlist\tqvqfn{1,2,3,4,5,6} \right\}
    \]
    But there is a constraint on $y$ - namely, $y\in\mathbb{N}$. And hence, 
    \[
      R = \left\{ \vc \right\}
    \]
  \end{solution}

  \part[1] What will be the domain of the relation? 

\begin{solution}[\mcq]
    The domain of $R$ is $=\lbrace \vd \rbrace$
  \end{solution}

  \part[1] $\ldots$ and what will be its range? 

\begin{solution}[\mcq]
    And the range is $= \lbrace \vr \rbrace$
  \end{solution}
\end{parts}

\ifprintanswers
  \begin{codex}
    $(a)\,\lbrace\vc\rbrace\quad (b)\,\lbrace\vd\rbrace\quad (c)\,\lbrace\vr\rbrace$
  \end{codex}
\fi
