

\ifnumequal{\value{rolldice}}{0}{
  \renewcommand{\va}{2}
  \renewcommand{\vb}{5}
  \renewcommand{\vc}{7}
}{
  \ifnumequal{\value{rolldice}}{1}{
    \renewcommand{\va}{3}
    \renewcommand{\vb}{3}
    \renewcommand{\vc}{5}
  }{
    \ifnumequal{\value{rolldice}}{2}{
      \renewcommand{\va}{2}
      \renewcommand{\vb}{2}
      \renewcommand{\vc}{9}
    }{
      \renewcommand{\va}{3}
      \renewcommand{\vb}{2}
      \renewcommand{\vc}{6}
    }
  }
}

\POWER\va\vb\a
\POWER\va\vc\b
\SUBTRACT\vc\vb\c
\POWER\va\c\d
\MULTIPLY\vc\d\e

\question[4] Evaluate the following limit 
\[ \lim_{x\to \va}\left(\dfrac{x^\vb-\a}{x^\vc-\b}\right)\]

\watchout[-45pt]

\begin{solution}[\halfpage]
  The easiest way to find this limit is to use the fact 
  \[\lim_{x\to a}\left( \dfrac{x^n-a^n}{x-a}\right) = na^{n-1}\]
  Note also that, 
  \[ \a = \va^\vb\text{ and } \b = \va^\vc \]
  Using both of the above facts, we get  
  \begin{align}
    \lim_{x\to \va}\left(\dfrac{x^\vb-\a}{x^\vc-\b}\right) &= 
    \lim_{x\to \va}\left[ \dfrac{\left(\dfrac{x^\vb-\va^\vb}{x-\va}\right)}
    {\left(\dfrac{x^\vc-\va^\vc}{x-\va}\right)}\right] \\
    &= \dfrac{\lim_{x\to\va}\left(\dfrac{x^\vb-\va^\vb}{x-\va}\right)}
    {\lim_{x\to\va}\left(\dfrac{x^\vc-\va^\vc}{x-\va}\right)} \\ 
    &= \dfrac{\vb\cdot\va^{\vb - 1}}{\vc\cdot\va^{\vc - 1}} = \WRITEFRAC\vb\e
  \end{align}
\end{solution}

\ifprintanswers\begin{codex}$\WRITEFRAC[false]\vb\e$\end{codex}\fi
