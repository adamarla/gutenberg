
\ifnumequal{\value{rolldice}}{0}{
  \renewcommand\vm{37}
  \renewcommand\vp{24}
  \renewcommand\vc{43}
  \renewcommand\vx{19} % maths and physics 
  \renewcommand\vy{29} % maths and chemistry 
  \renewcommand\vz{20} % physics and chemistry
  \renewcommand\vt{50}
}{
  \ifnumequal{\value{rolldice}}{1}{
    \renewcommand\vm{24}
    \renewcommand\vp{30}
    \renewcommand\vc{15}
    \renewcommand\vx{19} % maths and physics 
    \renewcommand\vy{11} % maths and chemistry 
    \renewcommand\vz{8} % physics and chemistry
    \renewcommand\vt{70}
  }{
    \ifnumequal{\value{rolldice}}{2}{
      \renewcommand\vm{42}
      \renewcommand\vp{29}
      \renewcommand\vc{32}
      \renewcommand\vx{23} % maths and physics 
      \renewcommand\vy{19} % maths and chemistry 
      \renewcommand\vz{17} % physics and chemistry
      \renewcommand\vt{85}
    }{
      \renewcommand\vm{41}
      \renewcommand\vp{37}
      \renewcommand\vc{23}
      \renewcommand\vx{32} % maths and physics 
      \renewcommand\vy{16} % maths and chemistry 
      \renewcommand\vz{16} % physics and chemistry
      \renewcommand\vt{95}
    }
  }
}

\EXPR[0]\va{\vt-\vm-\vp-\vc + (\vx+\vy+\vz)}

\question[4] Of the $\vt$ students taking examinations in Mathematics, Physics amd Chemistry 
- each passed in \textbf{atleast} one of the subjects. Moreover, $\vm$ passed in 
Mathematics, $\vp$ in Physics and $\vc$ in Chemistry. \textbf{At most}, $\vx$ passed in 
Maths and Physics, (at most) $\vy$ in Maths and Chemistry and (at most) $\vz$ in Physics 
and Chemistry.

What then is the maximum possible number of students who could've passed in all three subjects?

\watchout[-100pt]

\begin{solution}[\fullpage]
  If $M$, $P$ and $C$ be the sets of students passing in Maths, Physics and 
  Chemistry respectively, then here is what we know 
  \begin{align}
    n(M) = \vm, n(P)&=\vp, n(C)=\vc \\
    n(M\cap P) = \vx, n(M\cap C) &= \vy, n(P\cap C) = \vz \\
    \underbrace{n(M\cup P\cup C)'}_{\text{failed in all}} = 0 &\implies
    \underbrace{n(M\cup P\cup C)}_{\text{passed in one or more}} = \vt
  \end{align}

  The question we have to answer is \[ n(M\cap P\cap C)\leq ?\]
  
  Now, we know
  \begin{align}
    n(M\cup P\cup C) &= n(M) + n(P) + n(C) - n(M\cap P) \nonumber \\
                     &- n(M\cap C) - n(P\cap C) + n(M\cap P\cap C) \\
    \implies n(M\cap P\cap C) &= n(M\cup P\cup C) - n(M) - n(P) - n(C) \nonumber \\
                              &+ n(M\cap P) + n(M\cap C) + n(P\cap C)
  \end{align}
  To find the maximum possible value for $n(M\cap P\cap C)$, we must use the maximum possible 
  values for $n(M\cap P)$, $n(M\cap C)$ and $n(P\cap C)$. And therefore, 
  \begin{align}
    n(M\cap P\cap C) &\leq \vt-\vm-\vp-\vc + (\vx + \vy + \vz) \\
                     &\leq \va
  \end{align}
  Hence, \textbf{at most} $\va$ students could have passed in all three exams
\end{solution}

\ifprintanswers
  \begin{codex}
    $\va$
  \end{codex}
\fi 

