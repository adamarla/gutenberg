


\ifnumequal{\value{rolldice}}{0}{
  % variables 
  \renewcommand{\va}{4}
  \renewcommand{\vb}{2}
  \renewcommand{\vc}{9}
  \renewcommand{\vd}{-\dfrac{21}{2}}
}{
  \ifnumequal{\value{rolldice}}{1}{
    % variables 
    \renewcommand{\va}{4}
    \renewcommand{\vb}{3}
    \renewcommand{\vc}{8}
    \renewcommand{\vd}{-\dfrac{56}{9}}
  }{
    \ifnumequal{\value{rolldice}}{2}{
      % variables 
      \renewcommand{\va}{5}
      \renewcommand{\vb}{2}
      \renewcommand{\vc}{8}
      \renewcommand{\vd}{70}
    }{
      % variables 
      \renewcommand{\va}{4}
      \renewcommand{\vb}{3}
      \renewcommand{\vc}{7}
      \renewcommand{\vd}{-\dfrac{35}{3}}
    }
  }
}

\SUBTRACT\va{1}\p
\SUBTRACT\vc\p\q
\MULTIPLY\q{3}\a
\MULTIPLY\p{2}\b

\SUBTRACT\a\b\x
\SUBTRACT\p\q\y

\question[3] What will be the $\va-$th term in the expansion of 
 \[ \left( \dfrac{x^3}\vb - \dfrac\vb{x^2} \right)^{\vc} \] 
 if the higher powers of $x$ are written before the lower powers in the expansion.

\watchout

\begin{solution}[\halfpage]
  Every term in the required expansion will be of the form 
  \[ T_{m+1} = \binom\vc{m}\cdot\left( \dfrac{x^3}\vb\right)^{\vc-m}\cdot\left( -\dfrac\vb{x^2} \right)^{m}, m\text{ goes from }0\to\vc\]
  \textbf{Alternately,} you could say that the general term is
  \[ T_{m+1} = \binom\vc{m}\cdot\left( \dfrac{x^3}\vb\right)^{m}\cdot\left( -\dfrac\vb{x^2} \right)^{\vc-m} \]
  \textbf{But then,} $m$ would have to go from $\vc\to 0$. Either is fine as long as you know \textbf{which values of }$m$ 
  will give higher powers of $x$. 
  
  However, we will use the first form. And the $\va-$th term will be
  \begin{align}
    T_{\p+1} &= \binom\vc\p\cdot
    \left( \dfrac{x^3}\vb \right)^{\q} \cdot
    \left( -\dfrac\vb{x^2}\right)^{\p} \\
    &= \binom\vc\p\cdot\dfrac{x^{\a}}{x^{\b}}\times\dfrac{\vb^{\p}}{\vb^{\q}}
    \times (-1)^{\p} \\
    &= \binom\vc\p\cdot x^{\x}\cdot\vb^{\y} = \vd\cdot x^{\x}
  \end{align}

\end{solution}

\ifprintanswers\begin{codex}$\vd\cdot x^{\x}$\end{codex}\fi
