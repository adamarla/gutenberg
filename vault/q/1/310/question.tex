


\ifnumequal{\value{rolldice}}{0}{
  % variables 
  \renewcommand{\va}{1} % p
  \renewcommand{\vb}{3} % q 
  \renewcommand{\vc}{5} % r 
}{
  \ifnumequal{\value{rolldice}}{1}{
    % variables 
    \renewcommand{\va}{6}
    \renewcommand{\vb}{14}
    \renewcommand{\vc}{21}
  }{
    \ifnumequal{\value{rolldice}}{2}{
      % variables 
      \renewcommand{\va}{20}
      \renewcommand{\vb}{35}
      \renewcommand{\vc}{42}
    }{
      % variables 
      \renewcommand{\va}{5}
      \renewcommand{\vb}{10}
      \renewcommand{\vc}{14}
    }
  }
}

\ADD\va\vb\tp
\ADD\vb\vc\tq
\EXPR[2]\tr{\tq / \tp}
\EXPR[0]\ts{(\tr * \va + \vc) / (\vb - \tr * \va)}

\question[4] If some \textbf{consecutive} coefficients in the expansion of $(1+x)^n$ are in the 
ratio $\va:\vb:\vc$, then find $n$.

\watchout

\begin{solution}[\halfpage]
	Let the $(m-1)^{\text{th}}$, $m^{\text{th}}$ and $(m+1)^{\text{th}}$ terms be in the given ratio
	\begin{align}
		\dfrac{\encr{n}{m-1}}{\encr{n}{m}} &= \dfrac{\va}{\vb} \\
		\implies \dfrac{\fncr{n}{m-1}}{\fncr{n}{m}} &= \dfrac{\va}{\vb} \\
		\implies \dfrac{m}{n-m+1} &= \dfrac{\va}{\vb} \implies \tp\cdot m = \va\cdot(n+1)
	\end{align}
	Similarly,
	\begin{align}
		\dfrac{\encr{n}{m}}{\encr{n}{m+1}} &= \dfrac{\vb}{\vc} \implies \dfrac{m+1}{n-m} = \dfrac{\vb}{\vc} \\
		\implies \tq\cdot m &= \vb\cdot n - \vc \\
		\text{And therefore, } \dfrac{\tq\cdot m}{\tp\cdot m} &= \dfrac{\vb\cdot n - \vc}{\va\cdot(n+1)}\implies n = \ts
	\end{align}
\end{solution}

\ifprintanswers\begin{codex}$\ts$\end{codex}\fi
