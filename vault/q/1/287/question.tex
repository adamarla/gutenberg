
\ifnumequal{\value{rolldice}}{0}{
  % variables 
  \renewcommand{\va}{5}
  \renewcommand{\vb}{3}
  \renewcommand{\vc}{7}
  \renewcommand{\vd}{1}
  \renewcommand\ve{6}
  \renewcommand\vg{8400}
}{
  \ifnumequal{\value{rolldice}}{1}{
    % variables 
    \renewcommand{\va}{4}
    \renewcommand{\vb}{1}
    \renewcommand{\vc}{2}
    \renewcommand{\vd}{6}
    \renewcommand\ve{7}
    \renewcommand\vg{33,600}
  }{
    \ifnumequal{\value{rolldice}}{2}{
      % variables 
      \renewcommand{\va}{6}
      \renewcommand{\vb}{7}
      \renewcommand{\vc}{3}
      \renewcommand{\vd}{4}
      \renewcommand\ve{8}
      \renewcommand\vg{100,800}
    }{
      % variables 
      \renewcommand{\va}{9}
      \renewcommand{\vb}{7}
      \renewcommand{\vc}{2}
      \renewcommand{\vd}{5}
      \renewcommand\ve{9}
      \renewcommand\vg{201,600}
    }
  }
}

\SUBTRACT\ve{2}\vf

\question[2] In a certain city, all telephone numbers have $\ve$ digits. The first two digits
are always either $\va\vb$ or $\va\vc$ or $\va\vd$ or $\vd\vc$ or 
$\vd\va$. Given this, how many telephone numbers have all $\ve$ digits distinct?


\watchout[-20pt]

\ifprintanswers
\fi 

\begin{solution}[\mcq]
  For each combination of the first two digits, we need another $\ve-2=\vf$ digits (from the remaining eight) 
  for the resulting phone number to have all digits distinct. 

  And doing so will give us 
    \[ 5\times\binom{8}\vf\times\vf! = \vg\text{ possible numbers}\] 
\end{solution}

\ifprintanswers
  \begin{codex}
    $\vg$
  \end{codex}
\fi

