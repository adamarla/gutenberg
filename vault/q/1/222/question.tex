
\ifnumequal{\value{rolldice}}{0}{
  % variables 
  \renewcommand\va{3}
  \renewcommand\vb{4}
  \renewcommand\vz{36.9}
}{
  \ifnumequal{\value{rolldice}}{1}{
    % variables 
    \renewcommand\va{4}
    \renewcommand\vb{3}
    \renewcommand\vz{53.1}
  }{
    \ifnumequal{\value{rolldice}}{2}{
      % variables 
      \renewcommand\va{12}
      \renewcommand\vb{5}
      \renewcommand\vz{67.4}
    }{
      % variables 
      \renewcommand\va{5}
      \renewcommand\vb{12}
      \renewcommand\vz{22.6}
    }
  }
}

\DIVIDE\vz{2}\vy
\MULTIPLY\vb{2}\vc
\SQUARE\va\vm
\SQUARE\vb\vn
\ADD\vm\vn\vo
\SQRT\vo\vp

\ADD{-\vb}\vp\vq
\SUBTRACT{-\vb}\vp\vr
\FRACTIONSIMPLIFY\vq\va\vs\vt

\question[3] Without using a calculator, find the value of $\tan\ang\vy$ given that $\tan^{-1}\frac\va\vb=\ang\vz$

\watchout

\begin{solution}[\halfpage]
  Half the battle is won if you realized that 
  \[ \ang\vz = 2\times \ang\vy\]

  Hence, if we let $\tan\ang\vy=x$, then 
  \begin{align}
    \tan\ang\vz &= \dfrac\va\vb = \underbrace{\dfrac{2\tan\ang\vy}{1-\tan^2\ang\vy} = \dfrac{2x}{1-x^2}}_{\tan 2\theta=\dfrac{2\tan\theta}{1-\tan^2\theta}} \\
    \implies \va x^2 +\vc x -\va &= 0\text{ or } x = \WRITEFRAC\vq\va,\WRITEFRAC\vr\va
  \end{align}

  But $\tan\ang\vy > 0$ as the angle lies in the first quadrant. 

  Hence, the only acceptable answer is $\tan\ang\vy=\dfrac\vs\vt$
\end{solution}

\ifprintanswers\begin{codex}$\dfrac\vs\vt$\end{codex}\fi
