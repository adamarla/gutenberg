

\question[4]   Commander Spock - of Star-Trek fame - has some \textit{Vulcan} dollars that he
would like to invest for a period of one Vulcan year. He has a choice of 2 banks - an Earth-based
bank and a bank on his home planet Vulcan. The Earth bank offers compounding at the
rate of 3\% every 6 \textit{Earth} months. The Vulcan bank - on the other hand - offers 6\% compounding
every 6 \textit{Vulcan} months. Which bank should Commander Spock put his money in given that
1 Vulcan year = 2 Earth years


\ifprintanswers
  % stuff to be shown only in the answer key - like explanatory margin figures
  \marginnote[-0.5cm]{Note that the \textit{nominal} interest rate offered by banks 
  is the same - 12\% per Vulcan year}
  \marginnote[0.5cm] {But in order to do a comparison, one should consider only how 
  often the compounding happens and at what rate. In other words, one must consider
  only the \textit{effective} interest rate}
\fi 

\begin{solution}[\fullpage]
	The number of times the Earth-based bank would compound the amount in one \textit{Vulcan}
	year is 
	\begin{align}
		&= \SI{1}{vulcan-year}\times\SI{2}{earth-years\per vulcan-year}\times
		\dfrac{\SI{1}{time}}{\frac{1}{2}\text{earth-year}} \\
		&= 4\text{ times}
	\end{align}
	
	Similarly, the bank on Vulcan will compound the amount
	\begin{align}
		&= \SI{1}{vulcan-year}\times\dfrac{\SI{1}{time}}{\frac{1}{2}\text{vulcan-year}} \\
		&= 2\text{ times}
	\end{align}
	And therefore, if $P_0$ be the amount Commander Spock puts in initially, 
	$P_{\texttt{earth}}$ be the amount he realizes at the end of his investment period with the bank
	on Earth and $P_{\texttt{vulcan}}$ the amount he realizes on Vulcan, then 
	
	\begin{align}
		\dfrac{P_\texttt{earth}}{P_0} &= \left( 1 + \dfrac{3}{100}\right)^4 \\
		                       &= 1.1255 \\
		\dfrac{P_\texttt{vulcan}}{P_0} &= \left( 1 + \dfrac{6}{100}\right)^2 \\
		                        &= 1.1236
	\end{align}
	Clearly, Commander Spock is better off investing his
	money with the bank on Earth
\end{solution}
\ifprintanswers\begin{codex}\end{codex}\fi
