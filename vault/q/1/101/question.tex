

\ifnumequal{\value{rolldice}}{0}{
  \renewcommand{\va}{8}
  \renewcommand{\vb}{4}
  \renewcommand{\vc}{16}
  \renewcommand{\vd}{64}
  \renewcommand{\ve}{48}
}{
  \ifnumequal{\value{rolldice}}{1}{
    \renewcommand{\va}{10}
    \renewcommand{\vb}{5}
    \renewcommand{\vc}{20}
    \renewcommand{\vd}{100}
    \renewcommand{\ve}{60}
  }{
    \ifnumequal{\value{rolldice}}{2}{
      \renewcommand{\va}{6}
      \renewcommand{\vb}{3}
      \renewcommand{\vc}{12}
      \renewcommand{\vd}{36}
      \renewcommand{\ve}{36}
    }{
      \renewcommand{\va}{12}
      \renewcommand{\vb}{6}
      \renewcommand{\vc}{24}
      \renewcommand{\vd}{144}
      \renewcommand{\ve}{72}
    }
  }
}

\question[2] Break up the number $\va$ into two summands such that the sum of their cubes
is the least possible.

\watchout 

\begin{solution}[\halfpage]
   Basically, we have to minimize $y = x^3 + (\va-x)^3$
   \begin{align}
      y &= x^3 + (\va -x)^3, \text{ then } \\
      \dydx &= 3x^2 - 3(\va - x)^2 \text{ and } \\
      \dnydxn{2} &= 6x + 6(\va-x)
   \end{align}
   
   Now, 
   \begin{align}
      \dydx &= 0 \text { when } 3x^2 - 3(\va - x)^2 = 0 \\
      \implies x^2 &= \vd + x^2 - \vc x \text{ or } x = \vb \\
      \text{Also, }\left[\dnydxn{2}\right]_{x=\vb} &= \ve > 0 \implies \text{ minima }
   \end{align}
   
   Hence, the required split is $\vb$ and $\vb$
\end{solution}
\ifprintanswers\begin{codex}Each summand = $\dfrac{N}{2}$\end{codex}\fi
