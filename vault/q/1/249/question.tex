
\ifnumequal{\value{rolldice}}{0}{
  % variables 
  \renewcommand\va{3}
}{
  \ifnumequal{\value{rolldice}}{1}{
    % variables 
    \renewcommand\va{4}
  }{
    \ifnumequal{\value{rolldice}}{2}{
      % variables 
      \renewcommand\va{5}
    }{
      % variables 
      \renewcommand\va{6}
    }
  }
}

\SUBTRACT\va{1}\vb

\question[3] There are an \textbf{even} number of terms in a geometric progression. 
If the sum of \textbf{all} terms $(S_a)$ is $\va$ times as large as the sum of terms in odd positions $(S_o)$
then what is the common ratio of the progression?

\watchout

\begin{solution}[\halfpage]
  If the total number of terms be $n$, then 
  \[ n = 2k \text{ for some } k\in\mathbb{N} \]
  Of these, the terms in odd positions are 
  \[ \underbrace{a_1, a_3, a_5\ldots a_{2k-1}}_{k=\dfrac{n}{2}\text{ terms }} \] 
  \textbf{Insight \#1: Common ratio of odd-positioned terms}

  The common ratio of this latter sequence is simply
  \[ \dfrac{a_3}{a_1} = \dfrac{a_5}{a_3} \ldots = \dfrac{ar^2}{a} = r^2 \]
  And hence, the sum of the $k$ terms that make the sequence up is 
  \[ S_o = \dfrac{a\cdot((r^2)^k - 1)}{r^2-1} = \dfrac{a\cdot ((r^2)^{\frac{n}{2}}-1)}{r^2-1} = \dfrac{a\cdot (r^n-1)}{r^2-1} \]
  whereas the sum of \textbf{all} terms is 
  \[ S_a = \dfrac{a\cdot (r^n-1)}{r-1} \]
  Which means 
  \begin{align}
    \dfrac{S_a}{S_o} &= \dfrac{\dfrac{a\cdot(r^n-1)}{r-1}}{\dfrac{a\cdot(r^n-1)}{r^2-1}} = \va \\
    \implies \dfrac{r^2-1}{r-1} &= r + 1 = \va\text{ or } r = \vb
  \end{align}
\end{solution}

\ifprintanswers\begin{codex}$\vb$\end{codex}\fi
