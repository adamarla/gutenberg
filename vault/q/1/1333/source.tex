\documentclass[14pt,fleqn]{extarticle}
\RequirePackage{prepwell}

\previewoff
\begin{document}

Three bags contain balls as show in the table below 

$\\$ \begin{tabular}{c|c|c|c}
\hline
	Bag & \# White & \# Black  & \# Red \\
\hline
	$B_1$ & 1 & 2 & 3\\
\hline
	$B_2$ & 2 & 1 & 1\\
\hline
	$B_3$ & 4 & 3 & 2\\
\hline
\end{tabular} 
\newline

A bag is chosen at random and two balls are drawn from it. 
They happen to be red and white. What is the probability 
that the balls came from $B_3$ ? 

\newcard 

If events be defined as follows $\ldots$, then \newline

\begin{tabular}{|c|c|}
\hline
	Event & Meaning\\
\hline
	$A$ & 1 red \& 1 white drawn\\
\hline
	$B_1$ & Bag $B_1$ picked\\
	\hline
	$B_2$ & Bag $B_2$ picked\\
	\hline
	$B_3$ & Bag $B_3$ picked\\
\hline
\end{tabular}\newline\newline

$\ldots$ then we need to find $P \left(B_3\vert A \right)$
\newcard 

If events be defined as follows $\ldots$
\newline

\begin{tabular}{|c|c|}
\hline
	Event & Meaning\\
\hline
	$A$ & 1 red \& 1 white drawn\\
\hline
	$B_1$ & Bag $B_1$ picked\\
	\hline
	$B_2$ & Bag $B_2$ picked\\
	\hline
	$B_3$ & Bag $B_3$ picked\\
\hline
\end{tabular}

$\ldots$ then we need to find $P \left(B_3\cap A \right)$
\newcard 

\underline{If neither $A$ nor $B_3$} have happened, then yes, it would 
make sense to evaluate $P \left(A\cap B_3 \right)$. 

But $A$ has already happened. Hence, the probability we are looking 
for is \[\qquad\qquad P \left(B_3\vert A \right)\]
\newcard 
\begin{align}
	P \left(B_3\vert A \right) &= \dfrac{P \left(B_3\cap A \right)}{P(A)} \\
	&= \dfrac{P \left(A\vert B_3 \right)\cdot P(B_3)}{\sum_{N=1}^3 P \left(A\vert B_N \right)\cdot P \left(B_N \right)}
\end{align}
\newline 
where $P(B_N)$ is the probability of drawing the $N-$th bag.
\newcard 
\underline{This is simply Bayes' Theorem}
\begin{align}
P \left(B_3\vert A \right) &= \dfrac{P \left(A\vert B_3 \right)\cdot P(B_3)}{P(A)} \\
\text{where } P \left(A \right) &= \sum_{N=1}^3 P \left(A\vert B_N \right)\cdot P(B_N) 
\end{align}

\underline{$P(A)$ is the probability} of drawing a red and a white ball -- from any bag. 
\newcard 

Moreover\newline 

\begin{tabular}{|c|c|c|c|}
\hline
	& \# Balls & $P \left(A\vert B_N \right)$ & $P \left(B_N \right)$\\
\hline
	$B_1$ & 6 & $\frac{1}{5}$ & $\frac{1}{3}$ \\
\hline 
	$B_2$ & 4 & $\frac{1}{3}$ & $\frac{1}{3}$\\
\hline
	$B_3$ & 9 & $\frac{2}{9}$ & $\frac{1}{3}$\\
\hline 
\end{tabular}\newline 

And therefore 
\[ \qquad\quad  \underbrace{P \left(B_3\vert A \right) = \frac{5}{17}}_{\text{Required answer}}\]

\newcard 

Moreover \newline 

\begin{tabular}{|c|c|c|c|}
\hline
	& \# Balls & $P \left(A\vert B_N \right)$ & $P \left(B_N \right)$\\
\hline
	$B_1$ & 6 & $\frac{1}{6}\cdot\frac{3}{5}=\frac{1}{10}$ & $\frac{1}{3}$ \\
\hline 
	$B_2$ & 4 & $\frac{2}{4}\cdot\frac{1}{3}=\frac{1}{6}$ & $\frac{1}{3}$\\
\hline
	$B_3$ & 9 & $\frac{4}{9}\cdot\frac{2}{8} = \frac{1}{9}$ & $\frac{1}{3}$\\
\hline 
\end{tabular}

And therefore 
\[ \qquad \underbrace{P \left(B_3\vert A \right) = \frac{17}{45}}_{\text{Required answer}}\]
\newcard 

$\underbrace{P \left(B_1 \right) = P \left(B_2 \right) = P \left(B_3 \right) = \frac{1}{3}}_{\text{All three bags equally likely to be picked}}$\newline\newline

The (slightly) tricky bit is finding $P \left(A\vert B_N \right)$ for the three bags.

Now, if $B_N$ has $R$ red balls, $W$ white balls and $T$ total balls, then 
\[ \qquad P \left(A\vert B_N \right) = \dfrac{^RC_1\times ^WC_1}{^TC_2} \] 

And hence\newline 

\begin{tabular}{|c|c|c|}
\hline
	& \# Balls & $P \left(A\vert B_N \right)$ \\
\hline
	$B_1$ & 6 & $\dfrac{^3C_1\times ^1C_1}{^6C_2} = \frac{1}{5}$ \\
\hline 
	$B_2$ & 4 & $\dfrac{^2C_1\times ^1C_1}{^4C_2} = \frac{1}{3}$ \\
\hline
	$B_3$ & 9 & $\dfrac{^4C_1\times ^2C_1}{^9C_2} = \frac{2}{9}$ \\
\hline 
\end{tabular}\newline 

Now we know everything we need to find $P \left(B_3\vert A \right)$

\begin{align}
P \left(B_3\vert A \right) &= \dfrac{P \left(A\vert B_3 \right)\cdot P(B_3)}{\sum_{N=1}^3 P \left(A\vert B_N \right)\cdot P \left(B_N \right)} \\
&= \dfrac{2/9}{1/3 + 1/5+2/9} = \frac{5}{17} 
\end{align}
\end{document}
