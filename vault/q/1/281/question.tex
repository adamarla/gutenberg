


\ifnumequal{\value{rolldice}}{0}{
  % variables 
  \renewcommand{\va}{8}
  \renewcommand{\vb}{2}
  \renewcommand{\vc}{3}
  \renewcommand{\vd}{1}
  \renewcommand{\ve}{4}
  \renewcommand{\vh}{1,440}
}{
  \ifnumequal{\value{rolldice}}{1}{
    % variables 
    \renewcommand{\va}{7}
    \renewcommand{\vb}{3}
    \renewcommand{\vc}{2}
    \renewcommand{\vd}{2}
    \renewcommand{\ve}{6}
    \renewcommand{\vh}{720}
  }{
    \ifnumequal{\value{rolldice}}{2}{
      % variables 
      \renewcommand{\va}{9}
      \renewcommand{\vb}{4}
      \renewcommand{\vc}{3}
      \renewcommand{\vd}{3}
      \renewcommand{\ve}{8}
      \renewcommand{\vh}{21,600}
    }{
      % variables 
      \renewcommand{\va}{8}
      \renewcommand{\vb}{4}
      \renewcommand{\vc}{2}
      \renewcommand{\vd}{3}
      \renewcommand{\ve}{8}
      \renewcommand{\vh}{4,320}
    }
  }
}

\gcalcexpr[0]{\vf}{\ve - \vd + 1}
\SUBTRACT\va\vb\vg

\question[2] There are $\va$ chairs in a room - numbered $1$ to $\va$ - for the $\vb$ women and $\vc$ men present.
If the women get to choose first from amongst chairs numbered $\vd$ to $\ve$, and then the men 
from the remaining, then find the total number of ways in which they can be seated


\watchout[-45pt]

\begin{solution}[\mcq]
	The total number of chairs available to the ladies to choose from is $\ve - \vd + 1 = \vf$.
	And once they have picked, the men can pick from amongst $\va - \vb = \vg$ chairs. Hence,
	the total number of arrangements possible is
	\begin{align}
		N_{\texttt{total}} &= \enpr{\vf}{\vb} \times \enpr{\vg}{\vc} \\
		&= \fnpr{\vf}{\vb} \cdot \fnpr\vg\vc = \vh
	\end{align}
\end{solution}

\ifprintanswers
  \begin{codex}
    $\vh$
  \end{codex}
\fi 

