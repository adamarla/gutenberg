
\ifnumequal{\value{rolldice}}{0}{
	\renewcommand\va{63}
	\renewcommand\vb{2}
	\renewcommand\vc{3}
	\renewcommand\vd{7}
}{
	\ifnumequal{\value{rolldice}}{1}{
		\renewcommand\va{54}
		\renewcommand\vb{3}
		\renewcommand\vc{6}
		\renewcommand\vd{11}
	}{
		\ifnumequal{\value{rolldice}}{2}{
			\renewcommand\va{7}
			\renewcommand\vb{3}
			\renewcommand\vc{37}
			\renewcommand\vd{2}
		}{
			\renewcommand\va{13}
			\renewcommand\vb{6}
			\renewcommand\vc{-11}
			\renewcommand\vd{7}
		}
	}
}

\ADD\va\vb\vw
\ADD\vw\vb\vx
\ADD\vc\vd\vp
\ADD\vp\vd\vq

\SUBTRACT\va\vc\vm
\SUBTRACT\vd\vb\vn
\DIVIDE\vm\vn\p
\ADD\p{1}\vz

\question[2]  For what value of $n$ are the $n^{th}$ terms of the following two arithmetic progressions equal? 
\[ A = \lbrace \va,\vw,\vx,\ldots\rbrace\text{ and } B = \lbrace \vc,\vp,\vq,\ldots \rbrace \] 

\watchout

\begin{solution}[\halfpage]
	The two progressions can be written as 
	\begin{align}
		a_n &= \va + (n-1)\cdot\vb \\
		b_n &= \vc + (n-1)\cdot\vd 
	\end{align}
	For two terms of the two sequences to be equal
	\begin{align}
		\va + (n-1)\cdot\vb &= \vc+ (n-1)\cdot\vd \\
		\implies n &= \dfrac{\va - \vc}{\vd-\vb} + 1 = \vz
	\end{align}
\end{solution}
\ifprintanswers\begin{codex}$\vz$\end{codex}\fi
