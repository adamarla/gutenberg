\documentclass[14pt,fleqn]{extarticle}
\RequirePackage{prepwell}
\previewoff

\newcommand\intgf{e^{\tan x}}
\newcommand\expa{\tan x\cdot\sec^2 x}

\begin{document}
\begin{question}
\statement
	
%text
Solve the differential equation 
\[ \qquad \cos^2 x\cdot \frac{dy}{dx} + y = \tan x \]
%

\begin{step}
  \begin{options} 
     \correct 
     \[ \qquad \cos^2 x\cdot \dydx + y = \tan  x\]
     is a first-order linear differential equation        
     \incorrect
        
        \begin{align}
	\cos^2 x\cdot\dydx + y &= \tan x \\ 
	\implies \dydx &= \dfrac{\tan x - y}{\cos^2 x} 
\end{align}

Hence, the given differential equation is homogenous
    \end{options} 
     \reason 
     
     \begin{align}
	\cos^2 x\cdot\dydx + y &= \tan x \\
	\implies \dydx + \sec^2 x\cdot y &= \expa 
\end{align}  

This equation is of the form 
\[ \qquad \dydx + P(x)\cdot y = Q(x) \]
which we know is a \underline{first-order linear} differential equation
\end{step}


\begin{step}
  \begin{options} 
     \correct 
       \begin{align}
	\dydx + \sec^2x\cdot y &= \tan x\cdot\sec^2 x \\
	\implies \ddx \left(\intgf\cdot y \right) &= \intgf \left(\expa \right)
\end{align}
     \incorrect
        
        \begin{align}
	\dydx + \sec^2x\cdot y &= \tan x\cdot\sec^2 x \\
	\implies \ddx \left(e^{\sec x}\cdot y \right) &= e^{\sec x} \left(\expa \right)
\end{align}
    \end{options} 
     \reason 
     
     As this is a first-order linear differential equation, we first need to find the integrating factor $(I)$\newline 
     
     \begin{align}
     I &= e^{\int P(x)\cdot dx} = e^{\int \sec^2\cdot dx} = \intgf 
\end{align}

On multiplying throughout with $I$, we get 
\smallmath\begin{align}
\intgf \dydx + \intgf \left(\sec^2x \cdot y \right) &= \intgf \left(\expa \right) \\[-20pt]
\overbrace{\ddx \left(\intgf\cdot y \right)}^{\uparrow} &= \intgf \left(\expa \right)
\end{align}

This is the point of the integrating factor. It takes an equation of the form 
\[ \qquad \dydx + P(x)\cdot y = Q(x) \]
and converts it into 
\[ \qquad \ddx \left[I(x)\cdot y \right] = G(x) \]
which makes it easier for us to find $y$ 
\end{step}

\begin{step}
  \begin{options} 
     \correct 
       
       \begin{align}
	\ddx \left(\intgf \cdot y \right) &= \intgf \left(\expa \right) \\
	\therefore \intgf\cdot y &= \int \intgf \left(\expa \right)\cdot dx 
\end{align}

\underline{To evaluate the integral}, we must do the following substitution 
\[ \qquad \qquad z = \tan x \]

     \incorrect
     
     \begin{align}
	\ddx \left(\intgf \cdot y \right) &= \intgf \left(\expa \right) \\
	\therefore \intgf\cdot y &= \int \intgf \left(\expa \right)\cdot dx 
\end{align}

\underline{To evaluate the integral}, we must do the following substitution 
\[ \qquad \qquad z = \sec x \]
        
    \end{options} 
     \reason 
     
     \begin{align}
	\ddx \left(\intgf \cdot y \right) &= \intgf \left(\expa \right) \\
	\therefore \intgf\cdot y &= \int \intgf \left(\expa \right)\cdot dx 
\end{align}

So far, so good. But notice also that 
\[ \qquad \ddx \tan x = \sec^2 x\]

And hence, if we let $z = \tan x$, then $dz = \sec^2x\cdot dx$ and 
\begin{align}
\int \intgf \left(\expa \right)\cdot dx &= \int e^z\cdot z \cdot dz 
\end{align}
       
\end{step}

\begin{step}
  \begin{options} 
     \correct 
     \begin{align}
	y\cdot \intgf &= \int e^z z\cdot dz = \left(z-1 \right)\cdot e^z + C \\[-20pt]
	\therefore y &= \left(\tan x -1 \right) + \frac{C}{\intgf} 
\end{align}
        
    \end{options} 
     \reason 
     
     \begin{align}
     A &= y\cdot\intgf = \int e^z z\cdot dz \\
     &= \underbrace{z\int e^z\cdot dz - \int \left[ \frac{d}{dz}z\int e^z\cdot dz\right] \cdot dz}_{\text{Integration by parts}} \\
     &= ze^z - \int \left[\int e^z\cdot dz \right]\cdot dz \\
     &= ze^z - \int e^z\cdot dz = ze^z - e^z + C \\
     &= \left(z-1 \right)\cdot e^z + C \\
     &= \left(\tan x - 1 \right)\cdot \intgf + C 
\end{align}

And hence, 
\begin{align}
y\cdot\intgf &= \left(\tan x -1 \right)\cdot \intgf + C \\
\text{or } y &= \left(\tan x - 1 \right) + \frac{C}{\intgf} 
\end{align}
\end{step}

\end{question}
\end{document}
