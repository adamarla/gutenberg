\documentclass[14pt,fleqn]{extarticle}
\RequirePackage{prepwell-eng}
\previewoff

\begin{document}

\begin{problem}
	\statement 

An insurance company insured $2000$
scooter drivers, $4000$ car drivers and 
$6000$ truck drivers. The probability of 
an accident involving a scooter, a car 
and a truck are $0.01, 0.03$ and $0.15$ 
respectively.  One of the insured persons
meets with an accident. What is the probability that he is 
a scooter driver?
    
\begin{step}
  \begin{options} 
     \correct 
       
       Given the following events 
       \begin{center}
  \begin{tabular}{cccc}
   \toprule
        $S$ & $C$ & $T$ & $A$  \\
   \midrule 
   Scooter & Car Driver & Truck Driver  & Accident \\
   Rider & & & occurs \\
    \bottomrule
  \end{tabular}
\end{center}

We have been told the following 
\begin{center}
  \begin{tabular}{NN}
   \toprule
        \prob{S} = \frac{1}{6} & \condp{A}{S} = 0.01 \\
   \midrule 
   \prob{C} = \frac{1}{3} & \condp{A}{C} = 0.03 \\
   \midrule 
   \prob{T} = \frac{1}{2} & \condp{A}{T} = 0.15 \\
    \bottomrule
  \end{tabular}
\end{center}
       
     \incorrect

Given the following events 
       \begin{center}
  \begin{tabular}{cccc}
   \toprule
        $S$ & $C$ & $T$ & $A$  \\
   \midrule 
   Scooter & Car Driver & Truck Driver  & Accident \\
   Rider & & & occurs \\
    \bottomrule
  \end{tabular}
\end{center}

We have been told the following 
\begin{center}
  \begin{tabular}{NN}
   \toprule
        \prob{S} = \frac{1}{6} & \prob{A\cap S} = 0.01 \\
   \midrule 
   \prob{C} = \frac{1}{3} & \prob{A\cap C} = 0.03 \\
   \midrule 
   \prob{T} = \frac{1}{2} & \prob{A\cap T} = 0.15 \\
    \bottomrule
  \end{tabular}
\end{center}
        
    \end{options} 
     \reason 
       
     The insurance company has ensured $12,000$ vehicles in all 
     $\left(2000 + 4000 + 6000 \right)$ 
     
     \begin{center}
  \begin{tabular}{NcN}
   \toprule
        &  Insured vehicle is a$\ldots$ & \text{Value} \\
   \midrule 
   \prob{S} & Scooter & \frac{2000}{12,000} = \frac{1}{6} \\
    \midrule 
    \prob{C} & Car & \frac{4000}{12,000} = \frac{1}{3} \\
    \midrule
    \prob{T} & Truck & \frac{6000}{12,000} = \frac{1}{2} \\
    \bottomrule
  \end{tabular}
\end{center}

We have also been told the probability of an accident \underline{given} the vehicle 
type 

\begin{center}
  \begin{tabular}{NNN}
   \toprule
        & \text{Probability} & \text{Not} \\
   \midrule 
   \condp{A}{S} & 0.01 & \prob{A\cap S} \\
    \midrule 
    \condp{A}{C} & 0.03 & \prob{A\cap C} \\
    \midrule 
    \condp{A}{T} & 0.15 & \prob{A\cap T} \\
    \bottomrule
  \end{tabular}
\end{center}
\end{step}

\begin{step}
  \begin{options} 
     \correct 
       
       And we need to find $\condp{S}{A}$ 
     \incorrect
        
        And we need to find $\prob{S\cap A}$ 
        
    \end{options} 
     \reason 
     
     The accident has \underline{already happened}. We need to 
     find the probability of the vehicle being a scooter\newline 
     
     Hence, $\condp{S}{A}$ 
       
\end{step}

\begin{step}
  \begin{options} 
     \correct 
     
       The probability that there is an accident is 
       \begin{align}
       \prob{A} &= \frac{13}{150} \\
       \therefore \condp{S}{A} &= \left[ \dfrac{\frac{1}{100}\cdot\frac{1}{6}}{\frac{13}{150}}\right] =\frac{1}{52} 
\end{align}
     \incorrect
     
     The probability that there is an accident is 
       \begin{align}
       \prob{A} &= \frac{9}{200} \\
       \therefore \condp{S}{A} &= \left[ \dfrac{\frac{1}{100}\cdot\frac{1}{6}}{\frac{13}{150}}\right] =\frac{1}{27} 
\end{align}
        
    \end{options} 
     \reason 
       
     As per Bayes' Theorem 
     \[ \qquad \condp{S}{A} = \fcondp{S}{A}  \]
     
     We know all the terms in the numerator. But we will need to find $\prob{A}$
     \begin{align}
	\prob{A} = &\overbrace{\condp{A}{S}\cdot\prob{S} + \condp{A}{C}\cdot\prob{C}}^{\text{Theorem of Total Probability}} \\[-10pt]
	&+ \condp{A}{T}\cdot\prob{T} \\
	&= \left[ \frac{1}{100}\cdot\frac{1}{6} + \frac{3}{100}\cdot\frac{1}{3} + \frac{15}{100}\cdot\frac{1}{2} \right] \\
	&= \frac{13}{150} 
\end{align}

And therefore, 
\begin{align}
\condp{S}{A} &= \fcondp{S}{A} \\
&= \dfrac{ \left( \frac{1}{100}\cdot\frac{1}{6}\right)}{\frac{13}{150}} = \frac{1}{52}
\end{align}

\end{step}
          
\end{problem} 

\end{document}
