

\ifnumequal{\value{rolldice}}{0}{
  % variables 
  \renewcommand{\va}{9}
  \renewcommand{\vb}{27}
}{
  \ifnumequal{\value{rolldice}}{1}{
    % variables 
    \renewcommand{\va}{10}
    \renewcommand{\vb}{50}
  }{
    \ifnumequal{\value{rolldice}}{2}{
      % variables 
      \renewcommand{\va}{12}
      \renewcommand{\vb}{36}
    }{
      % variables 
      \renewcommand{\va}{8}
      \renewcommand{\vb}{32}
    }
  }
}

\FRACTIONSIMPLIFY\va\vb\a\b %\tan\alpha
\FRACMULT{2}{1}\a\b\c\d % 2\tan\alpha
\FRACMULT\a\b\a\b\e\f %\tan^{2}\alpha
\FRACMINUS{1}{1}\e\f\x\y % 1- \tan^2\alpha
\FRACDIV\c\d\x\y\m\n

\FRACMULT\n\m\va{1}\j\k
\FRACMINUS\vb{1}\j\k\ansx\ansy

\question[4] An eagle is perched on a tree $\va$ meters high. A snake emerges 
from a hole $\vb$ meters from the base of the tree and starts moving 
towards the tree. Seeing the snake, the eagle pounces on it. If the eagle can descend 
only in a straight line and only as fast as the snake is crawling, then how far from 
the hole is the snake caught?

\watchout[-70pt]

\ifprintanswers
  % stuff to be shown only in the answer key - like explanatory margin figures
		\vspace{0.75cm}
    \figinit{cm}
    \figpt 1: (0,0)
    \figpt 2: (5,0)
    \figpt 3: (5,3)
    \figpt 4: (3,0)
    \figpt 5: (1.5,0)
    \figpt 6: (4.75,0)
    \figpt 7: (1,0.35)
    \figpt 8: (3.6,0.35)
  	\figdrawbegin{}
    \figdrawline [1,2,3,1]
    \figdrawline [3,4]
    \figdrawend
    \figvisu{\figBoxA}{Figure 1}{%
			\large
      \figwritew 1:A(0.1)
      \figwritee 3:B(0.1)
      \figwrites 4:C(0.1)
      \figwritee 2:D(0.1)
      \figwritese 5:$x$(0.2)
      \figwritesw 6:$\vb-x$(0.2)
      \figwritee 7:$\alpha$(0.2)
      \figwritee 8:$2\alpha$(0.2)
      \psarc{<->}{1}{0}{30}
      \psarc{<->}(3,0){0.65}{0}{60}
    }
    \centerline{\box\figBoxA}
\fi 


\begin{solution}[\fullpage]
	The situation is as shown in the figure. The snake starts at $A$. 
	And the eagle descends from $B$ and catches the snake at $C$. 
	
	Now, 
	\begin{align}
		\tan\alpha &= \dfrac\va\vb = \dfrac\a\b 
	\end{align}
	Morever, because both the snake and the eagle move at the \textbf{same speed} and 
	for the \textbf{same time},
	\begin{align}
		AC = BC&\implies \angle BAC=\angle ABC=\alpha \\
		\implies\angle BCD &= \underbrace{\angle ABC + \angle BAC = 2\alpha}_{\text{exterior angle = sum of interior angles}} \\
	   \therefore\tan\angle BCD &= \tan 2\alpha = \dfrac{2\tan\alpha}{1-\tan^2\alpha}
	                               = \dfrac\va{\vb - x}
	\end{align}
	Solving (4), we get 
	\begin{align}
		\tan\angle BCD &= \dfrac{\va}{\vb-x} = \dfrac{2\cdot\frac\a\b}{1-(\frac\a\b)^2} \\
		\implies \dfrac{\va}{\vb-x} &= \dfrac\m\n\implies x = \WRITEFRAC\ansx\ansy
	\end{align}
	
	So, the snake goes $\WRITEFRAC\ansx\ansy$ meters before it is caught!
\end{solution}


\ifprintanswers\begin{codex}$\WRITEFRAC\ansx\ansy$ meters\end{codex}\fi
