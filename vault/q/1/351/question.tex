

\ifnumequal{\value{rolldice}}{0}{
  % variables 
  \renewcommand{\va}{3}
  \renewcommand{\vb}{2}
  \renewcommand{\vc}{9}
  \renewcommand{\vd}{1}
}{
  \ifnumequal{\value{rolldice}}{1}{
    % variables 
    \renewcommand{\va}{7}
    \renewcommand{\vb}{4}
    \renewcommand{\vc}{7}
    \renewcommand{\vd}{3}
  }{
    \ifnumequal{\value{rolldice}}{2}{
      % variables 
      \renewcommand{\va}{4}
      \renewcommand{\vb}{5}
      \renewcommand{\vc}{6}
      \renewcommand{\vd}{3}
    }{
      % variables 
      \renewcommand{\va}{5}
      \renewcommand{\vb}{3}
      \renewcommand{\vc}{5}
      \renewcommand{\vd}{6}
    }
  }
}

\FRACMULT\vd\va\vc{1}\ve\a
\FRACMULT\vd\va\vb{1}\p\q

\question[1] For what value of $p$ would the vectors $\vec{a} = \WRITEVECTOR{\va}{\vb}{\vc}$ and 
$\vec{b} = \WRITEVECGENERAL{\vd}{p}{\ve}$ be parallel?


\watchout

\ifprintanswers
\fi 

\begin{solution}[\mcq]
	For two vectors $\vec{a} = \WRITEVECGENERAL{a_1}{a_2}{a_3}$ and $\vec{b} = \WRITEVECGENERAL{b_1}{b_2}{b_3}$ 
	to be parallel to each other, 
	\begin{align}
		\dfrac{a_1}{b_1} &= \dfrac{a_2}{b_2} = \dfrac{a_3}{b_3} \\
		\implies \dfrac{\va}{\vd} &= \dfrac{\vb}{p} = \dfrac{\vc}{\ve} \implies p = \dfrac{\p}{\q}
	\end{align}
\end{solution}

\ifprintanswers\begin{codex}$\dfrac\p\q$\end{codex}\fi
