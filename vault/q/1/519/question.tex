
\ifnumequal{\value{rolldice}}{0}{
  \renewcommand\va{7}
  \renewcommand\vb{2}
}{
  \ifnumequal{\value{rolldice}}{1}{
    \renewcommand\va{5}
    \renewcommand\vb{2}
  }{
    \ifnumequal{\value{rolldice}}{2}{
      \renewcommand\va{8}
      \renewcommand\vb{3}
    }{
      \renewcommand\va{9}
      \renewcommand\vb{7}
    }
  }
}

\SUBTRACT\va\vb\vc

\question[4] Using mathematical induction, prove the following proposition
\[ \va^n-\vb^n,\,n\in\mathbb{N} \text{ is a multiple of }\vc \]

\insertQR{}

\watchout

\begin{solution}[\halfpage]
  It is easy to easy to see that the proposition is true for $n=1$. 
  So, \textbf{let us assume} that it is true for some $n\implies\va^n-\vb^n=\vc m$.

  Now, 
  \begin{align}
    \va^{n+1} - \vb^{n+1} &= (\vb + \vc)\cdot\va^n - \vb\cdot\vb^n \\
    \implies\dfrac{\va^{n+1}-\vb^{n+1}}{\vc} &= \dfrac{\vc\cdot\va^n + \vb\cdot (\va^n - \vb^n)}{\vc} \\
    &= \va^n +\vb\cdot\underbrace{\dfrac{\va^n-\vb^n}{\vc}}_{= m}
  \end{align}

  What the last equation says is \textbf{ that if} $\va^n-\vb^n$ is a multiple of $\vc$, then 
  so is $\va^{n+1}-\vb^{n+1}$. 

  And we have already seen that the proposition is true for $n=1$. It must therefore also 
  be true for $n=2,3,\ldots,\infty$. Hence proved.
\end{solution}

