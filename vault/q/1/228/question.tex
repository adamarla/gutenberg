

\ifnumodd{\value{rolldice}}{
  % equations
  % variables 
  \renewcommand{\va}{10}
  \renewcommand{\vb}{50}
  \renewcommand{\vc}{30}
  \renewcommand{\vd}{70}
  \renewcommand{\ve}{\dfrac{1}{16}}
  \renewcommand{\vf}{\dfrac{1}{2}}
  \renewcommand{\vg}{110}
}{
    % equations
    % variables 
    \renewcommand{\va}{20}
    \renewcommand{\vb}{40}
    \renewcommand{\vc}{60}
    \renewcommand{\vd}{80}
    \renewcommand{\ve}{\dfrac{3}{16}}
    \renewcommand{\vf}{\dfrac{\sqrt{3}}{2}}
    \renewcommand{\vg}{100}
}

\question[4] Without using a calculator, find the value of 
\[ \sin\ang{\va}\times\sin\ang{\vb}\times\sin\ang{\vc}\times\sin\ang{\vd} \]

\watchout

\begin{solution}[\halfpage]
    \textbf{Method \#1: Using standard trigonometric results} 

    Start by using the fact 
    \[ \fProdOfSin{a}{b}\] 
    to re-write the above expression as

    \begin{align}
    	&\sin\ang{\vc}\cdot\underbrace{\dfrac{1}{2}\cdot 2\sin\ang{\va}\sin\ang{\vb}}_{\fProdOfSin{a}{b}}\cdot\sin\ang{\vd} \\
    	&= \dfrac{\sqrt{3}}{2}\cdot\dfrac{1}{2}\cdot(\cos\ang{\va}-\underbrace{\cos\ang{\vc}}_{= \frac{1}{2}})\cdot\sin\ang{\vd} \\
    	&= \dfrac{\sqrt{3}}{4}\cdot\left[\dfrac{1}{2}\cdot \underbrace{2\cos\ang{\va}\sin\ang{\vd}}
    	_{\eProdCosSin{a}{b}} - \dfrac{1}{2}\sin\ang{\vd}\right] \\
    	&= \dfrac{\sqrt{3}}{4}\cdot\left[ \dfrac{1}{2}\cdot(\sin\ang{\vg} - \sin(\ang{-\vc}))
    	-\dfrac{1}{2}\sin\ang{\vd}\right ] \\
      &= \dfrac{\sqrt{3}}{4}\cdot\left[ \dfrac{\sin\ang\vg}{2}+\dfrac{\sin\ang\vc}{2}-\dfrac{\sin\ang\vd}{2} \right]
    \end{align}

    Now, $ \ang{\vg} = \ang{180}-\ang{\vd}$ 
    \[ \implies\sin\ang{\vg} = \sin(\ang{180}-\ang{\vd}) = \sin\ang{\vd} \]
    And therefore, the last equation becomes 
    \begin{align}
    	 \dfrac{\sqrt{3}}{4}\cdot\dfrac{\sin\ang{\vc}}{2} &= \ve
    \end{align}

    \textbf{Method \#2: Using a little known identity}

    \begin{align}
       \sin\ang{\va}\cdot\sin\ang{\vb}&\cdot\sin\ang{\vc}\cdot\sin\ang{\vd} = 
       \dfrac{1}{4}\times\overbrace{(4\cdot\sin\ang{\va}\cdot\sin\ang{\vb}\cdot\sin\ang{\vd})}
       ^{\sin 3\theta = 4\sin\theta\cdot\sin\left( \frac{\pi}{3} - \theta\right)\cdot\sin\left(\frac{\pi}{3} +
        \theta\right)}\cdot\sin\ang{\vc} \\
       &= \dfrac{1}{4}\cdot\sin\ang{\vc}\cdot\sin\ang{\vc} \\
       &= \dfrac{1}{4}\cdot\left(\vf \right)^2 = \ve
    \end{align}
\end{solution}
\ifprintanswers\begin{codex}$\ve$\end{codex}\fi
