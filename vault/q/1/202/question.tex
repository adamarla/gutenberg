
\ifnumequal{\value{rolldice}}{0}{
  % variables 
  \renewcommand{\va}{6}
  \renewcommand{\vb}{16}
  \renewcommand{\vc}{4}
  \renewcommand{\vd}{30}
}{
  \ifnumequal{\value{rolldice}}{1}{
    % variables 
    \renewcommand{\va}{9}
    \renewcommand{\vb}{21}
    \renewcommand{\vc}{15}
    \renewcommand{\vd}{9}
  }{
    \ifnumequal{\value{rolldice}}{2}{
      % variables 
      \renewcommand{\va}{9}
      \renewcommand{\vb}{5}
      \renewcommand{\vc}{6}
      \renewcommand{\vd}{14}
    }{
      % variables 
      \renewcommand{\va}{14}
      \renewcommand{\vb}{8}
      \renewcommand{\vc}{6}
      \renewcommand{\vd}{7}
    }
  }
}

\FRACTIONSIMPLIFY\vb\va\vp\vq
\FRACTIONSIMPLIFY\vd\vc\vr\vs

\question[3] For what $x$ is $\dfrac{\va x-\vb}{\vc x+\vd} < 0$ ?

\watchout

\begin{solution}[\mcq]
  \begin{align}
    \dfrac{\va x - \vb}{\vc  x + \vd } &= 
    \dfrac\va\vc\cdot\overbrace{\left(\dfrac{x-\frac\vp\vq}{x+\frac\vr\vs}\right)}^{\text{focus on this}} < 0
    \implies\left(\dfrac{x-\frac\vp\vq}{x+\frac\vr\vs}\right) < 0
  \end{align}
  \textbf{Case \#1}
  \[ x-\dfrac\vp\vq > 0 \text{ and } x+\dfrac\vr\vs < 0\implies x>\dfrac\vp\vq\text{ and }x<-\dfrac\vr\vs \]
  \textbf{Case \#2}
  \[ x-\dfrac\vp\vq < 0 \text{ and } x+\dfrac\vr\vs > 0\implies x<\dfrac\vp\vq\text{ and }x>-\dfrac\vr\vs \]

  Only the second case makes sense. It is not possible for $x$ to be $>\dfrac\vp\vq$\textbf{ and simultaneously } be $<-\dfrac\vr\vs$.

  Hence, 
  $x\in\left( -\dfrac\vr\vs, \dfrac\vp\vq \right)$
\end{solution}

\ifprintanswers
  \begin{codex}
    $x\in\left( -\dfrac\vr\vs, \dfrac\vp\vq \right)$
  \end{codex}
\fi
