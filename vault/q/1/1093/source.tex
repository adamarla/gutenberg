\documentclass[14pt,fleqn]{extarticle}
\RequirePackage{prepwell}
\previewoff

\newcommand\fx{2x^3 - 9mx^2 + 12m^2 x + 1}

\begin{document}
\begin{question}
	\statement 
    
    If the function 
    \[ \qquad f(x) = \fx \]
    where $m > 0$ attains its maximum and minimum values at $p$ and $q$ respectively and if $p^2 = q$, then find the value of $m$ 
    
    \begin{step}
  \begin{options} 
     \correct 
       
       \begin{center}
  \begin{tabular}{NNN}
   \toprule
        \text{At} &  f'(x) & f''(x) \\
   \midrule 
   p & 0 & < 0 \\
    \midrule 
    q & 0 & > 0 \\
    \bottomrule
  \end{tabular}
\end{center}
     \incorrect
        
               \begin{center}
  \begin{tabular}{NNN}
   \toprule
        \text{At} &  f'(x) & f''(x) \\
   \midrule 
   p & 0 & > 0 \\
    \midrule 
    q & 0 & < 0 \\
    \bottomrule
  \end{tabular}
\end{center}
    \end{options} 
     \reason 
       
     $p$ and $q$ are extrema points. Which is why $f'(x) = 0$ at $x = p,q$ \newline 
     
     And as $p$ is a maxima and $q$ a minima, therefore 
            \begin{center}
  \begin{tabular}{NNNc}
   \toprule
        \text{At} &  f'(x) & f''(x) & Because \\
   \midrule 
   p & 0 & < 0 & Maxima \\
    \midrule 
    q & 0 & > 0 & Minima \\
    \bottomrule
  \end{tabular}
\end{center}  

Now that we have a strategy in place, we can start to find $m$ 
\end{step}  

\begin{step}
  \begin{options} 
     \correct 
       
       \begin{align}
	f'(x) &= 6\cdot \left(x^2 - 3mx + 2m^2 \right) \\
	\therefore f'(x) &= 0 \text{ when } x = m, 2m 
\end{align}
     \incorrect
        
        \begin{align}
	f'(x) &= 6\cdot \left(x^2 - 4mx + 3m^2 \right) \\
	\therefore f'(x) &= 0 \text{ when } x = m, 3m 
\end{align}
    \end{options} 
     \reason 
       
       \begin{align}
	f'(x) &= \ddx \left(\fx \right) \\
	&= 6x^2 - 18mx + 12m^2 \\ 
	&= 6\cdot \left(x^2 - 3mx + 2m^2 \right) \\
	&= 6\cdot \left(x-m \right)\cdot \left(x-2m \right)
\end{align}

Therefore, extremas are at $x = m, 2m$
\end{step}

\begin{step}
  \begin{options} 
     \correct 
       
       \[ p = m\text{ and } q = 2m \]
     \incorrect
        
        \[ p = 2m\text{ and } q = m \]
    \end{options} 
     \reason 
       
       Now $f''(x)$ comes into play 
       
       \begin{align}
		f''(x) &= \ddx f'(x) = 6 \ddx \left(x^2 - 3mx + 2m^2 \right) \\
		&= 6\cdot \left(2x - 3m \right)
\end{align}

Note also that 

\[ f''(x) = \begin{cases} 
6\cdot (2m-3m) < 0 \text{ when } x = m \\
6\cdot (4m - 3m) > 0\text{ when } x = 2m 
\end{cases} \]

Hence, $\underbrace{p = m}_{\text{Maxima}}$ and $\underbrace{q = 2m}_{\text{Minima}}$ 
\end{step}

\begin{step}
  \begin{options} 
     \correct 
       
       \[ p^2 = q \implies m = 2 \]
        
    \end{options} 
     \reason 
       
     Every fact in the question has been given for a reason\newline 
     
     So, let us use the fact that $p^2 = q$ 
     
     \begin{align}
	p &= m \text{ and } q = 2m \\
	\text{Also }p^2 &= q \implies m^2 = 2m \\
	\text{ or } m\cdot(m-2) &= 0\implies m = 0,2 
\end{align}

But $m > 0$ (not $\geq 0$). Therefore, $m=2$ is the only acceptable solution 
\end{step}
\end{question} 
\end{document}