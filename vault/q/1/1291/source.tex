\documentclass[14pt,fleqn]{extarticle}
\RequirePackage{prepwell}
\previewoff
\begin{document}


\newcommand\fx{\left(x-4 \right)\cdot \left(x-5 \right)}
\newcommand\fxe{ \left(x^2-9x +20 \right)}
\newcommand\intg{\int \frac{6x+7}{\sqrt{\fx}}\cdot dx}
%text
Evaluate the following \[ \qquad \int \frac{6x+7}{\sqrt{\fx}}\cdot dx \]
%

\newcard

\underline{If $f(x) = \fx$}, then the following will \textbf{not} work 
\[ \quad \frac{6x+7}{\sqrt{f(x)}} = \frac{A}{\sqrt{x-4}} + \frac{B}{\sqrt{x-5}}\]

But \textbf{this will}

\[ \quad \frac{6x+7}{\sqrt{f(x)}} = \frac{A\cdot f'(x) + B}{\sqrt{f(x)}}\]

\newcard 

$f(x) = \fx$ is a polynomial \newline 

But $g(x) = \sqrt{f(x)}$ is not. Hence, we \underline{cannot} do the following 
\[ \qquad \frac{6x+7}{\sqrt{f(x)}} = \frac{A}{\sqrt{x-4}} + \frac{B}{\sqrt{x-5}}\]

Note also that $f(x) = \fxe$. And therefore, $f'(x)$ will be of the form $Ax+B$ -- just as the numerator is \newline 

Hence, we must look to express the numerator in terms of $f'(x)$

\newcard 

\small\[I = \int \frac{6x+7}{\sqrt{f(x)}}\cdot dx = \int \frac{3\cdot f'(x)  + 34}{\sqrt{f(x)}}\cdot dx\]\normalsize

\newcard 
\small\[I = \int \frac{6x+7}{\sqrt{f(x)}}\cdot dx = \int \frac{3\cdot f'(x)  -26}{\sqrt{f(x)}}\cdot dx\]\normalsize 

\newcard 

\begin{align}
f(x) = \fxe &\implies f'(x) = 2x - 9 \\
\text{Hence, if }6x + 7 &= A\cdot f'(x) + B \\
\text{then }6x + 7 &= A\cdot (2x-9) + B \\
&= 2Ax + \left(B-9A \right)
\end{align}

On comparing coefficients, we get 
\begin{align}
2A &= 6 \implies A = 3 \\
\text{and } B-9A &= 7 \implies B = 7 + 27 = 34 
\end{align}

Therefore 
\[ I = \int \frac{6x+7}{\sqrt{f(x)}}\cdot dx = \int \frac{3f'(x) + 34}{\sqrt{f(x)}}\cdot dx\]

\newcard 

In the following statement 
\small\[ \int \frac{6x+7}{\sqrt{f(x)}}\cdot dx = \underbrace{3\int \frac{f'(x)}{\sqrt{f(x)}}\cdot dx}_A + \underbrace{34\int \frac{dx}{\sqrt{f(x)}}}_B \]\normalsize

\[ \qquad A = 6\sqrt{\fxe} + C_1\]

\newcard 

In the following statement 
\small\[ \int \frac{6x+7}{\sqrt{f(x)}}\cdot dx = \underbrace{3\int \frac{f'(x)}{\sqrt{f(x)}}\cdot dx}_A + \underbrace{34\int \frac{dx}{\sqrt{f(x)}}}_B \]\normalsize

\[ \qquad A = 3\cdot\log\vert \fxe\vert  + C_1\]

\newcard 

\begin{align}
A &= 3\int \frac{f'(x)}{\sqrt{f(x)}} \cdot dx \\
\text{Hence, if } z &= f(x)\text{ then } f'(x)\cdot dx = dz \\
\implies A  &= 3\int \frac{dz}{\sqrt{z}} = 3 \left[\frac{\sqrt{z}}{\frac{1}{2}} \right] = 6\sqrt{z} + C_1 \\
&= 6\sqrt{\fxe} + C_1 
\end{align}

\newcard 

And integral $B$ will be evaluated as follows 

\begin{align}
B &= 34 \int \frac{dx}{\sqrt{\fxe}}\cdot dx \\
&= 34\int\frac{dx}{\sqrt{\left(x-\frac{9}{2}\right)^2 -\frac{1}{2^2}}} \\
&= 34\cdot\log\left\vert \left( x-\frac{9}{2}\right) + \sqrt{x^2-9x+20} \right\vert + C_2 
\end{align} 

Which means 
\begin{align}
	I &= 6\sqrt{\fxe}  \\
	&+ 34\cdot\log\left\vert \left( x-\frac{9}{2}\right) + \sqrt{x^2-9x+20} \right\vert + C 
\end{align} 

\newcard 

%text
To evaluate integrals of the form 
\[\qquad \int\frac{dx}{\sqrt{cx^2+dx+e}} \] 

the best strategy is to \underline{complete the squares} 
-- as shown below. 
%
\begin{align}
x^2-9x+20 &= \left(x-\frac{9}{2} \right)^2 -\left(\frac{9}{2}\right)^2 + 20  \\
&= \left(x-\frac{9}{2}\right)^2 -\frac{1}{4}\\
&= \left(x-\frac{9}{2}\right)^2 -\left(\frac{1}{2}\right)^2
\end{align}\\
\begin{align}
\therefore B &= 34\int\frac{dx}{\sqrt{x^2-9x+20}} \\
&= 34\underbrace{\int\frac{dx}{\sqrt{\left(x-\frac{9}{2}\right)^2 -\frac{1}{2^2}}}}
_{\int\frac{dx}{\sqrt{x^2-a^2}}= \log\vert x + \sqrt{x^2-a^2}\vert} \\
&= 34\cdot\log\left\vert \left( x-\frac{9}{2}\right) + \sqrt{x^2-9x+20} \right\vert + C_2 
\end{align}

We now know both $A$ and $B$ can hence write down $I = A+B$
\end{document}