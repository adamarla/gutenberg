
\ifnumequal{\value{rolldice}}{0}{
  % variables 
  \renewcommand{\va}{8}
  \renewcommand{\vb}{4}
  \renewcommand{\ve}{240}
}{
  \ifnumequal{\value{rolldice}}{1}{
    % variables 
    \renewcommand{\va}{10}
    \renewcommand{\vb}{5}
    \renewcommand{\ve}{4200}
  }{
    \ifnumequal{\value{rolldice}}{2}{
      % variables 
      \renewcommand{\va}{11}
      \renewcommand{\vb}{6}
      \renewcommand{\ve}{40,320}
    }{
      % variables 
      \renewcommand{\va}{9}
      \renewcommand{\vb}{3}
      \renewcommand{\ve}{90}
    }
  }
}

\SUBTRACT\vb{1}\vc 
\SUBTRACT\va{3}\vd

\question[2] There are $\va$ persons $(P_1\to P_{\va})$ of whom $\vb$ must be arranged
in a line. However, $P_1$ must \textit{always} be present in the line whereas $P_4$ and $P_5$ must \textit{never} be. 
How many such arrangements are possible?

\watchout[-30pt]

\begin{solution}[\mcq]
	One out of $\vb$ places in the line is always taken by $P_1$. Hence, its now a question of picking
	$\vc$ other people from amongst the $\vd$ remaining ( after excluding $P_4$ and $P_5$ ). Moreover,
	the order in which individuals stand in the line is important. Hence, 
	\begin{align}
		N_{\texttt{total}} &= \encr\vd\vc\cdot \vb\,! \\
		&= \fncr\vd\vc\cdot \vb\,! = \ve
	\end{align}
\end{solution}

\ifprintanswers
  \begin{codex}
    $\ve$
  \end{codex}
\fi 

