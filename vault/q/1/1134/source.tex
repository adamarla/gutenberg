\documentclass[14pt,fleqn]{extarticle}
\RequirePackage{prepwell-eng}
\previewoff

\newcommand\cx{a\sin^3\theta}
\newcommand\cy{a\cos^3\theta} 

\newcommand\pta{ \left(\frac{a}{2\sqrt{2}}, \frac{a}{2\sqrt{2}} \right)}
\begin{document}

\begin{problem}
	\statement 
    
    Find the equation of the tangent and normal to the curve 
    \[ \qquad x = a\sin^3\theta \text{ and } y = a\cos^3\theta \] 
    at $\theta = \dfrac\pi{4}$
    
\begin{step}
  \begin{options} 
     \correct 
       
     The tangent and normal will pass through the point $A = \pta$
        
    \end{options} 
     \reason 
     
     What we have here is a parametric curve in which both $x$ and $y$ are defined in
     terms of a third parameter $\theta$ \newline 
     
     
     And for $\theta = \frac{\pi}{4}$, the point $A$ on the curve is given by 
     \[ A = \left(a\sin^3 \frac{\pi}{4}, a\cos^3 \frac{\pi}{4} \right) = \pta \]
       
\end{step}


\begin{step}
  \begin{options} 
     \correct 
       
     If $m_T$ be the slope of the tangent and $m_N$ the slope of the normal at $A$, then 
     
     \begin{center}
  \begin{tabular}{NNN}
   \toprule
        & \text{Expression} & \text{Value} \\
   \midrule 
   m_T & \frac{dy}{dx} & -1 \\
    \midrule 
    m_N & & 1 \\
    \bottomrule
  \end{tabular}
\end{center}

     \incorrect

     If $m_T$ be the slope of the tangent and $m_N$ the slope of the normal at $A$, then 
     
     \begin{center}
  \begin{tabular}{NNN}
   \toprule
        & \text{Expression} & \text{Value} \\
   \midrule 
   m_T & \frac{dy}{dx} & 1 \\
    \midrule 
    m_N & & -1 \\
    \bottomrule
  \end{tabular}
\end{center}        
    \end{options} 
    
     \reason 
    
    It might be a parametric curve. But it is still drawn in the $x-y$ plane \newline 
    
    Hence, the slope of the tangent $(m_T)$ at point $A$ is
    \begin{align}
	m_T &= \dydx = \left[ \frac{dy / dt}{dx / dt}\right]_{\frac\pi{4}} 
	= \left[\frac{\frac{d}{dt} \left(\cy \right)}{\frac{d}{dt} \left(\cx \right)} \right]\\
	&= \underbrace{\left[\frac{3\cos^2\theta\cdot (-\sin\theta)}{3\sin^2\theta\cdot (\cos\theta)} \right]}_{\text{Chain Rule}} = -\cot\theta \\
	&= -1\text{ when } \theta = \frac\pi{4}
\end{align}

Morever $m_T\cdot m_N = -1$ as the tangent and normal are perpendicular to each other \newline 

Hence, $m_N = 1$ 
\end{step}

\begin{step}
  \begin{options} 
     \correct 
       
     \begin{center}
  \begin{tabular}{lN}
   \toprule
        &  \text{Equation} \\
   \midrule 
   Tangent & x + y = \frac{a}{\sqrt{2}}\\
    \midrule 
    Normal & x = y \\ 
    \bottomrule
  \end{tabular}
\end{center}
     \incorrect
        
        \begin{center}
  \begin{tabular}{lN}
   \toprule
        &  \text{Equation} \\
   \midrule 
   Tangent & x - y = \frac{a}{2\sqrt{2}}\\
    \midrule 
    Normal & y = x + \frac{a}{\sqrt{2}} \\ 
    \bottomrule
  \end{tabular}
\end{center}

    \end{options} 
     \reason 
     
     We know that the one point the two lines pass through. And we know their slopes\newline 
     
     That is more than enough information to find the equations of the two lines \newline 
     
     \begin{center}
  \begin{tabular}{lNNN}
   \toprule
        & \text{Slope} & \text{Setup} & \text{Equation} \\
   \midrule 
   Tangent & -1 & \frac{y-\frac{a}{2\sqrt{2}}}{x-\frac{a}{2\sqrt{2}}} = -1 & x + y = \frac{a}{\sqrt{2}} \\
    \midrule 
    Normal & 1 & \frac{y- \frac{a}{2\sqrt{2}}}{x- \frac{a}{2\sqrt{2}}} = 1 & x = y \\
    \bottomrule
  \end{tabular}
\end{center}
       
\end{step}
\end{problem}     
\end{document}