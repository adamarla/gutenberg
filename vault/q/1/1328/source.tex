\documentclass[14pt,fleqn]{extarticle}
\RequirePackage{prepwell}
\previewoff
\begin{document}

\newcommand\prange{\left[-\frac{\pi}{2}, \frac{\pi}{2} \right]}
%text
What will be the principal value of the following? 
\[\qquad  \sin^{-1}\sin\left(\frac{3\pi}{5}\right) \]
%

\newcard

\[ \sin^{-1} x \in \prange \]

\newcard 
\[ \sin^{-1} x \in \left[0,\pi\right] \]

\newcard 

When we say $\sin^{-1} x = \theta$, then $\theta$ belongs to what is called 
the \underline{principal range} \newline 

It is the smallest range that gives all possible values of $\sin\theta$\newline 

And for $\sin^{-1} x$, this range is $\prange$ 

\newcard 

\begin{align}
	\sin \left(\frac{3\pi}{5} \right) &= \sin \left(\frac{2\pi}{5} \right) \\
	\frac{3\pi}{5} \notin \prange &\text{ but } \frac{2\pi}{5}\in \prange \\
	\therefore \sin^{-1} \left(\sin \frac{3\pi}{5} \right) &= \frac{2\pi}{5}
\end{align}

\newcard 

\begin{align}
	\sin^{-1} \left(\sin \frac{3\pi}{5} \right) &= \frac{3\pi}{5} 
\end{align}
\newcard 

We established that the principal range of $\sin^{-1} x$ is $\prange$\newline 

However, note that 
\[ \quad \frac{3\pi}{5}\notin \prange\, \left(\because \frac{3\pi}{5} > \frac{\pi}{2} \right) \] 

Hence, the principal value \underline{cannot} be $\frac{3\pi}{5}$. But given that $\sin \left(\pi - x \right) = \sin x$, we can infer that 
\[ \qquad \sin \frac{3\pi}{5} = \sin \left[\pi - \frac{3\pi}{5} \right] = \sin \frac{2\pi}{5} \]

And $\frac{2\pi}{5}\in\prange$. Therefore, 
\[ \qquad \sin^{-1} \left(\sin \frac{3\pi}{5} \right) = \frac{2\pi}{5}\]
\end{document}