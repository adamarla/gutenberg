


\ifnumequal{\value{rolldice}}{0}{
  % variables 
  \renewcommand{\va}{POSITRON } % word
  \renewcommand{\vb}{S } % pivot
  \renewcommand{\vc}{5} % # of characters in target word 
  \renewcommand{\vd}{8} % length of original 
  \renewcommand{\ve}{7} % # distinct characters in original
  \renewcommand{\vf}{O } % repeating character
  \renewcommand{\vg}{720} % part a 
  \renewcommand{\vh}{1800} % part b
  \renewcommand{\vi}{7776} % part c
  \renewcommand{\vj}{9031} % part d
}{
  \ifnumequal{\value{rolldice}}{1}{
    % variables 
    \renewcommand{\va}{NUCLEUS }
    \renewcommand{\vb}{E }
    \renewcommand{\vc}{4}
    \renewcommand{\vd}{7}
    \renewcommand{\ve}{6}
    \renewcommand{\vf}{U }
    \renewcommand{\vg}{120}
    \renewcommand{\vh}{240}
    \renewcommand{\vi}{625}
    \renewcommand{\vj}{671}
  }{
    \ifnumequal{\value{rolldice}}{2}{
      % variables 
      \renewcommand{\va}{QUASAR }
      \renewcommand{\vb}{R }
      \renewcommand{\vc}{4}
      \renewcommand{\vd}{6}
      \renewcommand{\ve}{5}
      \renewcommand{\vf}{A }
      \renewcommand{\vg}{24}
      \renewcommand{\vh}{96}
      \renewcommand{\vi}{256}
      \renewcommand{\vj}{369}
    }{
      % variables 
      \renewcommand{\va}{GALAXY }
      \renewcommand{\vb}{X }
      \renewcommand{\vc}{4}
      \renewcommand{\vd}{6}
      \renewcommand{\ve}{5}
      \renewcommand{\vf}{A }
      \renewcommand{\vg}{24}
      \renewcommand{\vh}{96}
      \renewcommand{\vi}{256}
      \renewcommand{\vj}{369}
    }
  }
}

\SUBTRACT\ve{1}\vy
\SUBTRACT\vc{1}\vz

\question How many $\vc$-letter words can be formed using characters in \va if

\watchout

\begin{parts}
  \part[2] \vb is \textit{never} included and no character repeats? You might want to 
  think about the number of \textit{distinct} characters in \va

\begin{solution}[\mcq]
    The word has $\vd$ characters - with \vf occuring twice. And hence, there are $\ve$ \textit{distinct}
    characters in the word
    
    If, however, \vb can never occur, then we have $\vy$ characters left to work with. And, as neither 
    can occur more than once, the required number of ways is 
    \begin{align}
    	N &= \binom\vy\vc\cdot\vc ! = \vg
    \end{align}
  \end{solution}

  \part[1] \vb is \textit{always} included \textit{but} none of the included characters repeat
\begin{solution}[\mcq]
  	If one slot is booked for \vb, then there are $\vz$ slots left to fill using the $\vy$ distinct characters
  	
  	The required number of ways then is
  	\begin{align}
  		N &= \vc \times \binom\vy\vz\cdot\vz ! = \vh
  	\end{align}
  \end{solution}

  \part[1] \vb is \textit{never} included \textit{but} the other characters - if included - can repeat.

  \begin{solution}[\mcq]
  	This is easy. From the $\ve$ distinct characters, we always have to ignore one - \vb. 
  	And given that any of the remaining $\vy$ can repeat, the required number of ways is 
  	\begin{align}
  		N &= \vy^\vc = \vi
  	\end{align}
  \end{solution}

  \part[3] \vb is \textit{always} included \textit{and} the any of the included characters can repeat

\begin{solution}[\mcq]
  	Think of it like this. If, from all possible $\vc$-letter words made using $\ve$ distinct 
  	characters - with repetition - one were to remove all the words in part (c), then the remaining 
  	words would always have the character \vb
  	\begin{align}
  		N &= \ve^\vc - \vy^\vc = \vj
  	\end{align}
  \end{solution}

\end{parts}

\ifprintanswers
  \begin{codex}
    $(a)\,\vg\quad (b)\,\vh\quad (c)\,\vi\quad (d)\,\vj$
  \end{codex}
\fi
