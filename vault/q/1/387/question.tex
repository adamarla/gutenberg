
\ifnumequal{\value{rolldice}}{0}{
  \renewcommand{\va}{3}
}{
  \ifnumequal{\value{rolldice}}{1}{
    \renewcommand{\va}{5}
  }{
    \ifnumequal{\value{rolldice}}{2}{
      \renewcommand{\va}{7}
    }{
      \renewcommand{\va}{9}
    }
  }
}

\MULTIPLY\va{2}\vb

\question[4] Evaluate the following limit \[ \lim_{x\to\frac\pi{2}}\left(\dfrac{\tan\vb x}{x-\frac{\pi}{2}}\right)\]

\watchout

\begin{solution}[\halfpage]
  If we let 
  \[ \theta = x-\dfrac\pi{2} \implies \theta\to 0\text{ as }x\to\frac\pi{2} \]
  then,
  \begin{align}
    \lim_{x\to\frac\pi{2}}\left( \dfrac{\tan\vb x}{x-\frac\pi{2}}\right) &= 
    \lim_{\theta\to 0}\left[ 
    \dfrac{\tan\left\lbrace\vb\cdot\left( \theta + \frac\pi{2} \right)\right\rbrace}{\theta}
    \right] \\
    &= \lim_{\theta\to 0}\left[ \dfrac{\tan (\va\pi + \vb\theta)}{\vb\theta} \times\vb\right]
  \end{align}
  Now, 
  \[ \tan(\va\pi + \vb\theta) = \tan\vb\theta \]
  And therefore, 
  \begin{align}
    \lim_{\theta\to 0}\left[ \dfrac{\tan (\va\pi + \vb\theta)}{\vb\theta} \times\vb\right]
    &= \vb\times\lim_{\theta\to 0}\left( 
    \dfrac{\tan\vb\theta}{\vb\theta}\right) \\
    &= \vb\times\underbrace{\lim_{\alpha\to 0}\left( \dfrac{\tan\alpha}{\alpha}\right)}_{\alpha=\vb\theta} = \vb
  \end{align}

\end{solution}
\ifprintanswers\begin{codex}$\vb$\end{codex}\fi
