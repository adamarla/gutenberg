

\question[3] After 4 seconds of motion, the speed of a moving particle is $1$ cm/s. 
Given that the speed of the particle is propotional to square of the time, what is
the distance travelled by the particle in the first $10$ seconds?


\ifprintanswers
  % stuff to be shown only in the answer key - like explanatory margin figures
\fi 

\begin{solution}[\halfpage]
  \begin{align}
     \text{Speed} &= \dfrac{\ud x}{\ud t} \propto t^2 \\
     \Rightarrow \dfrac{\ud x}{\ud t} &= k\cdot t^2
  \end{align}
  And so, if at $t=4$, $\dfrac{\ud x}{\ud t} = 1$, then 
  \begin{align}
     \dfrac{\ud x}{\ud t} = 1 &= k\cdot 4^2 \\
     \Rightarrow k &= \frac{1}{16} \\
     \therefore \dfrac{\ud x}{\ud t} &= \dfrac{t^2}{16}
  \end{align}
  
  Now that we have an equation for speed, finding the distance covered \textit{upto}
  a time $t$ is easy
  \begin{align}
     \text{Distance covered} &= \int_{0}^{t} \ud x = \int_0^{t}\dfrac{t^2}{16}\ud t \\
     &= \left[ \dfrac{t^3}{48}\right]_0^{t} = \dfrac{t^3}{48}
  \end{align}
  
  And hence, the distance covered in the first 10 seconds is $ = \dfrac{10^3}{48} = 20\frac{5}{6}cm$
\end{solution}
