
\ifnumequal{\value{rolldice}}{0}{
  % variables 
  \renewcommand{\va}{5}
  \renewcommand{\vb}{6}
  \renewcommand{\vc}{3}
  \renewcommand{\vd}{7}
  \renewcommand\ve{63}
}{
  \ifnumequal{\value{rolldice}}{1}{
    % variables 
    \renewcommand{\va}{9}
    \renewcommand{\vb}{8}
    \renewcommand{\vc}{5}
    \renewcommand{\vd}{11}
    \renewcommand\ve{72.8}
  }{
    \ifnumequal{\value{rolldice}}{2}{
      % variables 
      \renewcommand{\va}{4}
      \renewcommand{\vb}{7}
      \renewcommand{\vc}{17}
      \renewcommand{\vd}{10}
      \renewcommand\ve{89.3}
    }{
      % variables 
      \renewcommand{\va}{6}
      \renewcommand{\vb}{7}
      \renewcommand{\vc}{9}
      \renewcommand{\vd}{10}
      \renewcommand\ve{82.6}
    }
  }
}

\FRACADD\va\vb\vc\vd\vp\vq
\FRACMULT\va\vb\vc\vd\vr\vs
\FRACMINUS{1}{1}\vr\vs\vt\vu
\FRACDIV\vp\vq\vt\vu\vx\vy

\question[1] What is $\alpha+ \beta$ equal to (in degrees) if 
\[ \tan\alpha=\dfrac\va\vb\qquad\tan\beta=\dfrac\vc\vd \]

\watchout

\begin{calcaid}
  \begin{tabular}{c c c c}
    $\tan\ang{63}=\dfrac{53}{27}$ & $\tan\ang{89.3}=\dfrac{159}{2}$ & $\tan\ang{82.6}=\dfrac{123}{16}$ & $\tan\ang{72.8}=\dfrac{139}{43}$
  \end{tabular}
\end{calcaid}

\begin{solution}[\mcq]
  \begin{align}
    \tan(\alpha + \beta) &= \eTanOfSum\alpha\beta \\
    &= \dfrac{\frac\va\vb + \frac\vc\vd}{1-\frac\va\vb\cdot\frac\vc\vd} = \dfrac\vx\vy \\
    \implies \alpha+\beta &= \tan^{-1}\dfrac\vx\vy = \ang\ve
  \end{align}
\end{solution}

\ifprintanswers\begin{codex}$\ang\ve$\end{codex}\fi
