


\ifnumequal{\value{rolldice}}{0}{
  % variables 
  \renewcommand{\va}{even }
  \renewcommand{\vb}{1,7,6,8,4,3}
  \renewcommand{\vc}{3} % # odd 
  \renewcommand{\vd}{3} % # even
  \renewcommand{\ve}{360}
}{
  \ifnumequal{\value{rolldice}}{1}{
    % variables 
    \renewcommand{\va}{odd }
    \renewcommand{\vb}{5,8,3,2,4,1,9}
    \renewcommand{\vc}{4}
    \renewcommand{\vd}{3}
    \renewcommand{\ve}{2880}
  }{
    \ifnumequal{\value{rolldice}}{2}{
      % variables 
      \renewcommand{\va}{even }
      \renewcommand{\vb}{7,8,3,4,1,6,2,9}
      \renewcommand{\vc}{4}
      \renewcommand{\vd}{4}
      \renewcommand{\ve}{20,160}
    }{
      % variables 
      \renewcommand{\va}{odd }
      \renewcommand{\vb}{2,5,7,9,3,4}
      \renewcommand{\vc}{4}
      \renewcommand{\vd}{2}
      \renewcommand{\ve}{480}
    }
  }
}

\ADD\vc\vd\vt
\ifthenelse{\equal\va{odd }}{
  \ADD\vc{0}\vx
}{
  \ADD\vd{0}\vx
}

\question[1] How many \textit{\va} numbers can be formed using $\lbrace \vb \rbrace$. All digits 
must be used and no digit repeats.

\watchout

\begin{solution}[\mcq]
	There are a total of $\vt$ digits available - of which $\vc$ are odd and $\vd$ are even. 
	To form an \textit{\va} number, it must end with one of the $\vx$ \va digits. The other 
	digits can be in any order.
	
	And therefore
  \[ N = (\vt - 1)!\times\vx = \ve \]
\end{solution}

\ifprintanswers
  \begin{codex}
    $\ve$
  \end{codex}
\fi
