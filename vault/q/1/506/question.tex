

\question[2] Why and when do we appeal to large sample properties?

\begin{solution}[\halfpage]
  \textbf{In cross section analysis} we appeal to large sample properties when the normality assumption (MLR6) seems 
  inappropriate
  
  \textbf{In time series analysis}, we appeal to large sample properties after making additional 
  assumptions regarding the stochastic process that generates the data, namely weak dependence
  
  \textbf{In both cases}, we use the assumptions to justify appeals to the law of large numbers. This in turn leads to sampling 
  distributions which are approximately (asymptotically) normally distributed. With that in hand, we can proceed 
  with hypothesis testing as usual, although the test statistics now follow only approximate \(t\) or 
  approximate \(F\) distributions. The tests are said to be asymptotically valid.
\end{solution}

