\documentclass[14pt,fleqn]{extarticle}
\RequirePackage{prepwell-eng}
\previewoff

\newcommand\fx{\vert x-1\vert}

\begin{document}
\begin{problem}
\statement
	

      Find the domain and range of the
      following function 
      \[ \qquad f(x) = \fx \]
    
\begin{step}
	\begin{options}
		\correct
		
		$f(x) = \fx$ is defined for all $x \in\mathbb{R}$. Hence, its domain 
		is $\mathbb{R}$ 
		
		\incorrect
		
		$f(x) = \fx$ is defined \underline{only for} $x > 1$. Hence, its domain 
		is $(1,\infty)$
	\end{options}
	\reason
	
	Had $f(x) = \dfrac{1}{\fx}$, then yes, one would have to worry about 
	the deonominator turning $0$ (at $x=1$) \newline 
	
	But if $f(x) = \fx$, then there is no such problem. In fact, $f(x)$ is defined for any $x\in\mathbb{R}$. Therefore the domain is $\mathbb{R}$ 
      
\end{step}
\begin{step}
	\begin{options}
		\correct
		
		\begin{center}
  \begin{tabular}{NNN}
   \toprule
       f(x) & \text{At} & \text{Value} \\
   \midrule 
   \text{Minimum} & x=1 & 0 \\
    \midrule 
    \text{Maximum} & x = -\infty, \infty & \infty \\
    \bottomrule
  \end{tabular}
\end{center}

Hence, the \underline{range of $f(x)$} is $[0,\infty)$  
     

		\incorrect
		
		\begin{center}
  \begin{tabular}{NNN}
   \toprule
        f(x) & \text{At} & \text{Value} \\
   \midrule 
   \text{Minimum} & x=0 & 1 \\
    \midrule 
    \text{Maximum} & x = -\infty, \infty & \infty \\
    \bottomrule
  \end{tabular}
\end{center}

Hence, the range of $f(x)$ is $[1,\infty)$ 


	\end{options}
	\reason

$f(x) = \fx > 0$ for all $x$. But more specifically, 
	\begin{align}
	f(x) = \vert x-1\vert &= \begin{cases}
       -(x-1), x< 1 \\
       (x-1), x\geq 1
     \end{cases} \\
     &= \begin{cases}
       1-x, x< 1 \\
       x-1, x\geq 1
     \end{cases} 
\end{align}
     
     And as one can see, $f(x)$ is minimum when $x=1$ with $f(1) = 0$\newline
     
     Hence, the range of $f(x)$ is $[0,\infty)$
     
\end{step}
\end{problem}
\end{document}
