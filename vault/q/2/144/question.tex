\ifnumequal{\value{rolldice}}{0}{
  \renewcommand{\va}{12}
  \renewcommand{\vb}{25}
  \renewcommand{\vc}{20}
  \renewcommand{\vd}{30}
}{
  \ifnumequal{\value{rolldice}}{1}{
    \renewcommand{\va}{6}
    \renewcommand{\vb}{16}
    \renewcommand{\vc}{8}
    \renewcommand{\vd}{24}
  }{
    \ifnumequal{\value{rolldice}}{2}{
      \renewcommand{\va}{8}
      \renewcommand{\vb}{18}
      \renewcommand{\vc}{12}
      \renewcommand{\vd}{24}
    }{
      \renewcommand{\va}{13}
      \renewcommand{\vb}{27}
      \renewcommand{\vc}{18}
      \renewcommand{\vd}{36}
    }
  }
}

\MULTIPLY\vb{2}\m
\MULTIPLY\va\m\n

\question[4] Two workers working together can complete a job in $\va$ days. If the first worker completes half the work and then the second worker takes over, the job will be completed in $\vb$ days. How many days would it take each worker to do the job if they worked alone?


\ifprintanswers
  % stuff to be shown only in the answer key - like explanatory margin figures
\fi 

\begin{solution}[\fullpage]
  Let the time it takes the two worker to do the entire job by themselves be $t_1$ days and $t_2$ days respectively. If you think of a job as a unit of work, then you can describe the speed of doing work as follows,
  \begin{align}
    \text{Speed of first worker}  &= \frac{1}{t_1}(jobs/day) \\
    \text{Speed of second worker} &= \frac{1}{t_2}(jobs/day) \\
    \text{Effective speed when both work together}     
    				  &= \left(\frac{1}{t_1}+\frac{1}{t_2}\right)(jobs/day)
  \end{align}
  
  Now, working together they complete a job as a whole in $\va$ days, therefore,
  \begin{align}
	\dfrac{1(job)}
	      {\left(\dfrac{1}{t_1}+\dfrac{1}{t_2}\right)(jobs/day)} &= \va(days) \\
	\implies \dfrac{t_1t_2}{t_1+t_2}                             &=\va 
  \end{align}
  
  Dividing the job into two halves, where each worker completes one half and the second one starts only after the first one finishes, they take $\vb$ days, therefore,
  \begin{align}
    \dfrac{\frac{1}{2}(job)}{\dfrac{1}{t_1}(jobs/day)} +
      \dfrac{\frac{1}{2}(job)}{\dfrac{1}{t_2}(jobs/day)} &= \vb(days) \\
    \implies t_1+t_2                                     &= \m 
  \end{align}
  
  Using results (5) and (7), we get,
  \begin{align}
    \dfrac{t_1(\m-t_1)}{t_1+(\m-t_1)} &= \va \\
    \m t_1-t_1^2                       &= \n \\
    t_1^2-\m t_1+\n                   &= 0 \\     
    t_1                               &= \vc \\
    t_2                               &= \vd 
  \end{align}
  Therefore, working by themselves, the workers would take $\vc$ and $\vd$ days respectively to complete the job.

\end{solution}

\ifprintanswers\begin{codex}
  $\vc$ and $\vd$ days
\end{codex}\fi
