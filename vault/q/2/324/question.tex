


\ifnumequal{\value{rolldice}}{0}{
  % variables 
}{
  \ifnumequal{\value{rolldice}}{1}{
    % variables 
  }{
    \ifnumequal{\value{rolldice}}{2}{
      % variables 
    }{
      % variables 
    }
  }
}

\question A zoo sponsored a special one day contest to name a new baby elephant.
Zoo visitors deposited entries in a special box between noon $(t=0)$ and $8$ P.M.
$(t=8)$. The number of entries in the box $t$ hours after noon is modeled a
differentiable function $E(t)$ for $0 \leq t \leq 8$. Value of $E(t)$, in hundreds
of entries at various times are shown in the table below.\\
\textit{Use of calculator is permitted for this questions}
  \begin{tabular}{|c|c|c|c|c|c|}
    \hline
    $t$(hours) &0 &2 &5 &7 &8 \\ \hline
    $E(t)$(hundreds of entries) &0 &4 &13 &21 &23 \\
    \hline
  \end{tabular}



\ifprintanswers
  % stuff to be shown only in the answer key - like explanatory margin figures
\fi 

\begin{parts}
  \part[1] Use the data in the table to approximate the rate, in hundreds of entries 
  per hour, at which entries were being deposited at $t=6$. Show the computation
  that lead to your answer.  
\begin{solution}[\mcq]
    We can approximate the rate $R$ at $t=6$ using the data provided as follows:
    \begin{align}
      R = \dfrac{E(7)-E(5)}{7-5} = \dfrac{8}{2} = 4 \nonumber
    \end{align}
  \end{solution}

  \part[3] Use a trapezoidal sum with the four subintervals given by the table to
  approximate the value of $\dfrac{1}{8}\int_0^8 E(t)\ud t$. Using the correct
  units explain the meaning $\dfrac{1}{8}\int_0^8 E(t)\ud t$ in terms of the 
  number of entries.
\begin{solution}[\mcq]
    \begin{align}
      \dfrac{1}{8}\int_0^8 E(t)\ud t &= \dfrac{1}{8}\left( 
        2\dfrac{E(0)+E(2)}{2}+3\dfrac{E(2)+E(5)}{2}+
        2\dfrac{E(5)+E(7)}{2}+\dfrac{E(7)+E(8)}{2}\right) \nonumber \\
        &= 10.687 \nonumber
    \end{align}
    $\dfrac{1}{8}\int_0^8 E(t)\ud t$ gives the average number of hundreds of entries 
    in the box between noon and $8$ P.M.
  \end{solution}

  \part[2] At $8$ P.M., volunteers began to process the entries. They processed the 
  entries at a rate modeled by the function $P$. Where $P(t) = t^3 - 30t^2 + 298t - 976$
  hundreds of entries per hour for $8 \leq t \leq 12$. According to the model, how many 
  entries had not yet been processed by midnight $(t=4)$?
  \begin{solution}[\mcq]
    Let $T$ be the total number of entries processed. Then,
    \begin{align}
      T &= \int_8^{12} P(t) \ud t \nonumber \\
        &= \int_8^{12} t^3 - 30t^2 + 298t - 976 \nonumber \\
        &= \left[\dfrac{t^4}{4} - 30\dfrac{t^3}{3} + 298\dfrac{t^2}{2} - 976t \right]_8^{12} \nonumber \\
        &= 16 \nonumber
    \end{align}
    The number of hundreds of entries yet to be processed is $23-16=7$
  \end{solution}

  \part[3] According to the model from part(c), at what time were the entries being processed
  the most quickly. Justify your answer.
\begin{solution}[\mcq]
    To calculate this, we first need to find the extrema (maxima/minima) points of the functon
    and then calculate its value at each of them and the domain extremes $0, 8$.
    \begin{align}
      P'(t) = 3t^2 - 60t + 298 \nonumber
    \end{align}
    $P'(t) = 0$ at $t=9.183503$ and $t=10.816497$. Therefore candidate values for $P_{max}$ are
    {$0$, $5.088662$, $2.911338$ and $8$}. Entries are being processed most quickly at time $t=12$
    at the rate $8$ hundred entries per hour.
  \end{solution}  
\end{parts}

\ifprintanswers\begin{codex}\end{codex}\fi
