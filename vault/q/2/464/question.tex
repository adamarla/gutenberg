

\ifnumequal{\value{rolldice}}{0}{
  % variables 
  \renewcommand{\va}{4}
  \renewcommand{\vb}{5}
  \renewcommand{\vc}{20}
}{
  \ifnumequal{\value{rolldice}}{1}{
    % variables 
    \renewcommand{\va}{6}
    \renewcommand{\vb}{3}
    \renewcommand{\vc}{11}
  }{
    \ifnumequal{\value{rolldice}}{2}{
      % variables 
      \renewcommand{\va}{5}
      \renewcommand{\vb}{7}
      \renewcommand{\vc}{8}
    }{
      % variables 
      \renewcommand{\va}{8}
      \renewcommand{\vb}{3}
      \renewcommand{\vc}{9}
    }
  }
}

\SQUARE\va\aa
\SQUARE\vb\bb
\SQUARE\vc\cc

\FRACMAX\vc\va\vc\vb\vm\vn
\FRACMIN\vc\va\vc\vb\vp\vq

\MIN\va\vb\da
\MAX\va\vb\db

\SQUARE\da\daa
\SQUARE\db\dbb
\FRACMINUS{1}{1}\daa\dbb\j\k
\SQRT\j\p
\SQRT\k\q
\DIVIDE\p\q\e
\MULTIPLY\e\vc\nf
\DIVIDE\nf\da\tempf
\ROUND[2]\tempf\f
\ROUND[2]\e\vz

\renewcommand\vd{\dfrac{x^2}{a^2} + \dfrac{y^2}{b^2}}
\renewcommand\ve{\dfrac{x^2}{b^2} + \dfrac{y^2}{a^2}}

\ifnumodd{\value{rolldice}}{ % vertical
  \renewcommand\vf\ve
  \renewcommand\vg{A}
  \renewcommand\vh{\left(0, \pm a\right)}
  \renewcommand\vi{\left(0, \pm\WRITEFRAC\vc\da\right)}
  \renewcommand\vj{\left(0, \pm ae\right)}
  \providecommand\vbeleven{\left(0,\pm\f \right)}
}{
  \renewcommand\vf\vd
  \renewcommand\vg{B}
  \renewcommand\vh{\left(\pm a, 0\right)}
  \renewcommand\vi{\left(\pm\WRITEFRAC\vc\da, 0\right)}
  \renewcommand\vj{\left(\pm ae, 0\right)}
  \providecommand\vbeleven{\left(\pm\f,0\right)}
}

\question The figure alongside shows two ellipses - $A$ and $B$. One of them has the  
equation \[\aa x^2 + \bb y^2 = \cc \]

\watchout

\figinit{pt}
  \def\Xori{45}
  \def\Yori{45}
  \def\rxa{30} % A:x-radius
  \def\rya{40} % A:y-radius
  \def\rxb{40} % B:x-radius
  \def\ryb{30} % B:y-radius
  \def\Xmin{0}
  \def\Xmax{60}
  \def\Ymin{0}
  \def\Ymax{60}
  \figpt 100:$C$(\Xori,\Yori) % center of ellipse
  \ifnumodd{\value{rolldice}}{
    \figptell 101:$V_2$: 100;\rxa,\rya (90,0)
    \figptell 102:$V_1$: 100;\rxa,\rya (270,0)
  }{
    \figptell 101:$V_2$: 100;\rxb,\ryb (0,0)
    \figptell 102:$V_1$: 100;\rxb,\ryb (180,0)
  }
  \figptell 500:$A$: 100;\rxa,\rya (60,0)
  \figptell 501:$B$: 100;\rxb,\ryb (23,0)
  \figvectP 200 [100,101]
  \figvectP 201 [100,102]
  \figpttra 103:$F_2$= 100 /0.6,200/
  \figpttra 104:$F_1$= 100 /0.6,201/
\figdrawbegin{}
  %\figset arrowhead(fillmode=yes,angle=15,length=6)
  %\figset general(color=0.7)
  %\figdrawaxes 100(-\Xmax,\Xmax,-\Ymax,\Ymax)
  \drawAxes{100}{-\Xmax}\Xmax{-\Ymax}\Ymax
  %\figreset{general}
  \figdrawarcell 100;\rxa,\rya (0,360,0)
  \figdrawarcell 100;\rxb,\ryb (0,360,0)
\figdrawend
\figvisu{\figBoxA}{}{%
  \Large
  \figsetmark{$\bullet$}
  \figwritese 100:(3)
  \ifprintanswers
    \ifnumodd{\value{rolldice}}{
      \figwritene 101:(3)
      \figwritese 102:(3)
      \figwritew 103:(3)
      \figwritew 104:(3)
    }{
      \figwritese 101:(3)
      \figwritesw 102:(3)
      \figwritenw 103:(3)
      \figwritene 104:(3)
    }
  \fi
  \figsetmark{}
  \figwritene 500:(3)
  \figwritene 501:(3)
}

\vspace{0.7cm}
\centerline{\box\figBoxA}

\begin{parts}
  \part[2] Which of the two ellipses - $A$ or $B$ - is the equation for? Justify your answer 

\begin{solution}[\mcq]
  The equation for an ellipse can be in one of two forms 
  \begin{align}
    \dfrac{x^2}{b^2} + \dfrac{y^2}{a^2} &= 1\implies\text{ an ellipse like A} \\
    \dfrac{x^2}{a^2} + \dfrac{y^2}{b^2} &= 1\implies\text{ an ellipse like B} 
  \end{align} 

  \textbf{In both the forms, $a\geq b$}

  You can know what $a$ and $b$ are \textbf{only after} writing the equation in the standard form. 

  And for the given ellipse 
  \begin{align}
    \aa x^2 + \bb y^2 &= \cc \\
    \implies \dfrac{x^2}{\left(\WRITEFRAC\vc\va\right)^2} + \dfrac{y^2}{\left(\WRITEFRAC\vc\vb \right)^2} &= 1 \\
    \implies a = \WRITEFRAC\vm\vn\text{ and } b = \WRITEFRAC\vp\vq
  \end{align}
  Which means, that the given equation is of the form 
  \[ \vf \implies \text{ an ellipse like }\vg \] 
  \end{solution}

  \part[1] Mark the vertices of the ellipse and find their coordinates. Call the vertices $V_1$ and $V_2$.
\begin{solution}[\mcq]
    The vertices of ellipse $\vg$ are simply $\vh = \vi$. 
    
    These are the points where the ellipse makes the sharpest turns.
  \end{solution}

  \part[1] Find the eccentricity ($e$) of the ellipse 

\begin{solution}[\mcq]
    \begin{align}
      b^2 &= a^2\cdot (1-e^2) \implies e = \sqrt{1-\frac{b^2}{a^2}} \\
      &= \sqrt{\dfrac{\j}{\k}} = \vz
    \end{align}
  \end{solution}


  \part[1] Mark the focii and find their coordinates. Call the focii $F_1$ and $F_2$

\begin{solution}[\mcq]
    The focii of the ellipse would be $\vj = \vbeleven$
  \end{solution}

\end{parts}

\ifprintanswers
  \begin{codex}
    \begin{tabular}{l l l l}
      (a) Ellipse $\vg$ & (b) $\vi$ & (c) $e=\vz$ & (d) $\vbeleven$
    \end{tabular}
  \end{codex}
\fi
