


\ifnumequal{\value{rolldice}}{0}{
  % variables 
  \renewcommand{\va}{heptagon}
  \renewcommand{\vb}{7}
  \renewcommand{\vc}{14}
}{
  \ifnumequal{\value{rolldice}}{1}{
    % variables 
    \renewcommand{\va}{decagon}
    \renewcommand{\vb}{10}
    \renewcommand{\vc}{35}
  }{
    \ifnumequal{\value{rolldice}}{2}{
      % variables 
      \renewcommand{\va}{octagon}
      \renewcommand{\vb}{8}
      \renewcommand{\vc}{20}
    }{
      % variables 
      \renewcommand{\va}{nonagon}
      \renewcommand{\vb}{9}
      \renewcommand{\vc}{27}
    }
  }
}

\question[2] How many diagonals does a \textbf{\va } have? A \text{\va } has $\vb$ vertices. 

\watchout

\begin{solution}[\mcq]
  \textbf{Insight \#1: If its not a side then its a diagonal}

  Which means, that if $N_{\text{all}} = $ number of ways of connecting any two of the $\vb$ points, then 
  \begin{align}
    N_{\text{all}} &= \binom\vb{2} \\
    \text{ and }N_{\text{diagonals}} &= N_{\text{all}} - N_{\text{sides}} = \binom\vb{2} - \vb = \vc
  \end{align}
\end{solution}
\ifprintanswers\begin{codex}$\vc$\end{codex}\fi
