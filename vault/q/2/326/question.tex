


\ifnumequal{\value{rolldice}}{0}{
  % variables 
}{
  \ifnumequal{\value{rolldice}}{1}{
    % variables 
  }{
    \ifnumequal{\value{rolldice}}{2}{
      % variables 
    }{
      % variables 
    }
  }
}

\question Let $f$ be a continuous function defined on $[-4,3]$ whose graph,
consisting of three line segments and a semi-circle centered at the origin,
is given in the figure shown. Let $g$ be the function given by $g(x) = 
\int_1^x f(t)\ud t$.


\watchout

  \begin{marginfigure}
    \figinit{pt}
      \figpt 10:(0,0)
      \figpt 20:(-80,20)
      \figpt 30:(-40,60)
      \figpt 40:(-20,0)
      \figpt 50:(0,-20)
      \figpt 60:(20,0)
      \figpt 70:(60,-20)
      % extremeties
      \def\Xmax{100}
      \def\Ymax{80}
      \def\Xmin{-100}
      \def\Ymin{-40}
      % pts for numbering the axes (5, 10, 15...)
      \figpt 200:$\tiny\text{-4}$(-80,0)
      \figpt 201:$\tiny\text{-3}$(-60,0)
      \figpt 202:$\tiny\text{-2}$(-40,0)
      \figpt 203:$\tiny\text{-1}$(-20,0)
      \figpt 204:$\tiny\text{O}$(0,0)
      \figpt 205:$\tiny\text{1}$(20,0)
      \figpt 206:$\tiny\text{2}$(40,0)
      \figpt 207:$\tiny\text{3}$(60,0)
      \figpt 208:$\tiny\text{4}$(80,0)
      \figpt 209:$\tiny\text{5}$(100,0)
      \figpt 210:$\tiny\text{-2}$(0,-40)
      \figpt 211:$\tiny\text{-1}$(0,-20)
      \figpt 212:$\tiny\text{1}$(0,20)
      \figpt 213:$\tiny\text{2}$(0,40)
      \figpt 214:$\tiny\text{3}$(0,60)
      % label graph
      \figpt 100:(\Xmax,0)
      \figpt 101:(0,\Ymax)
      \figpt 102:(0,\Ymin)
    \figdrawbegin{}
      \figset arrowhead(length=4, fillmode=yes)
      \figdrawaxes 10(\Xmin, \Xmax, \Ymin, \Ymax)
      \figdrawline [20,30]
      \figdrawarccirc 10 ; 20 (180, 360)
      \figdrawline [30,40]
      \figdrawline [60,70]
    \figdrawend
    \figvisu{\figBoxA}{}{%
      \figwritee 100: $x$(2 pt)
      \figsetmark{$\tiny\text{|}$}
      \figwrites 200,201,202,203,205,206,207,208,209 :(2 pt)
      \figwritesw 204: (2 pt)
      \figsetmark{$\tiny\text{-}$}
      \figwrites 210,211,212,213,214 :(2 pt)
      \figsetmark{}
      \figwritew 20:$\text{(-4,1)}$(2 pt)
      \figwritee 30:$\text{(-2,3)}$(2 pt)
      \figwriten 60:$\text{(1,0)}$(2 pt)
      \figwrites 70:$\text{(3,-1)}$(2 pt)
      \figwrites 102: $\text{Graph of f}$(10 pt)
    }
    \centerline{\box\figBoxA}
  \end{marginfigure}

\begin{parts}
  \part[3] Find the values of $g(2)$ and $g(-2)$.
\begin{solution}[\mcq]
    $g(x)$ is the area under the curve $f$ in the
    said interval. Therefore,
    \begin{align}
      g(2) &= \int_1^2 f(t)\ud t = -\dfrac{1}{2}(1)(\dfrac{1}{2}) = -\dfrac{1}{4}
    \end{align}
    \begin{align}
      g(-2) &= \int_1^{-2} f(t)\ud t = -\int_{-2}^1f(t)\ud t \nonumber \\
            &= -\left(\dfrac{3}{2} - \dfrac{\pi}{2}\right) = \dfrac{\pi}{2} - \dfrac{3}{2}
    \end{align}
  \end{solution}

  \part[2] For $g'(-3)$ and $g"(-3)$, find the value if it exists, 
  or state that it does not exist.
\begin{solution}[\mcq]
    $g'(x)$ = $f(x)$ and $g"(x) = f'(x)$. Therefore, from a simple
    examination of the graph shown in the figure,
    \begin{align}
      g'(-3) = f(-3) = 2 \nonumber
    \end{align}
    \begin{align}
      g"(-3) = f'(-3) = 1 \nonumber
    \end{align}
  \end{solution}

  \part[3] Find the $x-coordinate$ of each point at which the graph of $g$ has
  a horizontal tangent line. For each of these points, determine whether $g$ 
  has a relative maximum, a relative minimum, or neither a maximum nor a 
  minimum at that point. Justify your answers.
  
\begin{solution}[\mcq]
    The graph of $g$ would have a tangent where $g'(x) = f(x) = 0$. This 
    occurs at $x=-1$ and $x=1$. \\
    $g'(x)$ changes sign from positive to negative at $x = -1$. Therefore
    $g$ has a relative maximum at $x=-1$. \\  
    $g'(x)$ does not change sign at $x=1$. Therefore, $g$ has neither a 
    relative maximum nor a relative minimum at $x=1$.   
  \end{solution}

  \part[2] For $-4 \leq 3$, find all values of $x$ for which the graph of $g$ 
  has a point of inflection. Explain your reasoning.
\begin{solution}[\mcq]
    A point of inflection occurs when the second derivative of the function 
    changes sign. For the graph of $g$, $g"(x) = f'(x)$ which changes sign
    at $x=-2$, $x=0$ and $x=1$.  
  \end{solution}

\end{parts}

\ifprintanswers\begin{codex}\end{codex}\fi
