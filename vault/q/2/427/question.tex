


\ifnumequal{\value{rolldice}}{0}{
  \renewcommand{\va}{a}
  \renewcommand{\vb}{b}
  \renewcommand{\vc}{c}
  \renewcommand{\vd}{d}
  \renewcommand{\ve}{e}
  \renewcommand{\vf}{f}
}{
  \ifnumequal{\value{rolldice}}{1}{
    \renewcommand{\va}{p}
    \renewcommand{\vb}{q}
    \renewcommand{\vc}{r}
    \renewcommand{\vd}{s}
    \renewcommand{\ve}{t}
    \renewcommand{\vf}{u}
  }{
    \ifnumequal{\value{rolldice}}{2}{
      \renewcommand{\va}{m}
      \renewcommand{\vb}{n}
      \renewcommand{\vc}{o}
      \renewcommand{\vd}{p}
      \renewcommand{\ve}{q}
      \renewcommand{\vf}{r}
    }{
      \renewcommand{\va}{h}
      \renewcommand{\vb}{i}
      \renewcommand{\vc}{j}
      \renewcommand{\vd}{k}
      \renewcommand{\ve}{l}
      \renewcommand{\vf}{m}
    }
  }
}

\question[2] Assume the \texttt{Apriori} algorithm identified the following
seven \texttt{4-itemsets} that satisfy a user given support threshold.
\begin{align}
    L_4 = \{\va\vb\vc\vd, \va\vb\vc\ve, \va\vb\vc\vf, \va\vc\vd\ve, 
          \va\vd\ve\vf, \vb\vc\vd\ve, \vb\vc\ve\vf\} \nonumber
\end{align}
List the candidate \texttt{5-itemsets} created by the \texttt{Apriori} 
algorithm (justify your answer).

\watchout

\begin{solution}[\fullpage]
  The following candidate \texttt{5-itemsets} are formed by performing a
  self-join on $L$.
  \begin{align}
    C_5 = \{\overbrace{\va\vb\vc\vd\ve}^{A}, \overbrace{\va\vb\vc\vd\vf}^{B}, 
          \overbrace{\va\vb\vc\ve\vf}^{C}\} \nonumber  
  \end{align}
  Now let us apply the \texttt{Apriori} property that \textbf{subsets of a frequent
  itemset must also be frequent}. The \texttt{4-itemsets} subsets of $A$, $B$ and $C$
  are,
  \begin{align}
    A_4 &= \{\va\vb\vc\vd, \va\vb\vc\ve, \textbf{\va\vb\vd\ve}, 
             \va\vc\vd\ve, \vb\vc\vd\ve \} \nonumber \\
    B_4 &= \{\va\vb\vc\vd, \va\vb\vc\vf, \textbf{\va\vb\vd\vf}, 
             \textbf{\va\vc\vd\vf}, \textbf{\vb\vc\vd\vf\}} \nonumber \\
    C_4 &= \{\va\vb\vc\ve, \va\vb\vc\vf, \textbf{\va\vb\ve\vf}, 
             \textbf{\va\vc\ve\vf}, \vb\vc\ve\vf \} \nonumber
  \end{align}
  Since each of the candidate itemsets have \textbf{subsets that are NOT frequent},
  they are are all pruned. Therefore there are no valid \texttt{5-itemsets} for this
  dataset for the specified support threshold.
\end{solution}
