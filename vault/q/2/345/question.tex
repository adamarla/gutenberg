

\ifnumequal{\value{rolldice}}{0}{
  \renewcommand{\va}{3}
  \renewcommand{\vb}{14}
}{
  \ifnumequal{\value{rolldice}}{1}{
    \renewcommand{\va}{3}
    \renewcommand{\vb}{7}
  }{
    \ifnumequal{\value{rolldice}}{2}{
      \renewcommand{\va}{10}
      \renewcommand{\vb}{14}
    }{
      \renewcommand{\va}{10}
      \renewcommand{\vb}{7}
    }
  }
}

\question[5] Suppose that we are using \textit{extendable hashing} 
on a file that contains records with the following search-key values:
\begin{align}
  \va, 8, 11, \vb, 15, 16, 17, 19, 20, 33, 43, 48 \nonumber
\end{align}
Show the final extendable hash structure for this file, if the hash 
function is $H(x) = x\,\text{mod}(7)$, e.g. $(H(8) = 1 = 00001)$. A bucket
can hold a maximum of $3$ records. \\
You need to use the suffix (for example, $01$ is the $2$ bit suffix 
of $00001$) for inserting records in the buckets. The records are 
inserted in the order in which they appear.

\watchout

\begin{solution}[\fullpage]
  Let us begin by computing and writing the hash for the search 
  key-values in binary:
  \begin{align}
    H(\mathbf{\va}) &= 3 = 0011 \quad H(08) = 1 = 0001 \quad H(11) = 4 = 0100 \nonumber \\
    H(\mathbf{\vb}) &= 0 = 0000 \quad H(15) = 1 = 0001 \quad H(16) = 2 = 0010 \nonumber \\
    H(17) &= 3 = 0011 \quad H(19) = 5 = 0101 \quad H(20) = 6 = 0110 \nonumber \\
    H(33) &= 5 = 0101 \quad H(43) = 1 = 0001 \quad H(48) = 6 = 0110 \nonumber        
  \end{align}  
  Start inserting key-values based on a $1$-bit suffix in the order given
  above (left to right, top to bottom). The order does not really affect
  the final outcome, it is only being set to aid explanation. 
  \begin{align}
    \begin{tabular}{|c|c|c|}
      \hline
      \textbf{Hash Suffix$(1)$} & $\mathbf{0}$ & $\mathbf{1}$ \\
      \hline
      Buckets & $11,14,16$ & $3,8,15$ \\
      \hline
    \end{tabular}\nonumber
  \end{align}
  $17$ forces us into $2$-bit suffixes since $\mathbf{1}$ will overflow. 
  Note that $17$ only requires us to split $\mathbf{1}$ into $\mathbf{01}$ and
  $\mathbf{11}$. But eventually $20$ will require us to split $\mathbf{0}$ as
  well.
  \begin{align}
    \begin{tabular}{|c|c|c|c|c|}
      \hline
      \textbf{Hash Suffix$(2)$} & $\mathbf{00}$ & $\mathbf{10}$ & $\mathbf{01}$ & $\mathbf{11}$ \\
      \hline 
      Buckets & $11,\vb$ & $16,20$ & $8,15,19$ & $\va,33$ \\
      \hline
    \end{tabular}\nonumber  
  \end{align}
  $33$ causes $\mathbf{01}$ to overflow and forces us into $3$-bit suffixes
  and the following final structure.
  \begin{align}
    \begin{tabular}{|c|c|c|c|c|c|}
      \hline
      \textbf{Hash Suffix$(3)$} & $\mathbf{000,100}(2)$ & $\mathbf{010,110}(2)$ & 
          $\mathbf{001}(3)$ & $\mathbf{101}(3)$ & $\mathbf{011,111}(2)$ \\
      \hline 
      Buckets & $11,\vb$ & $16,20,48$ & $8,15,43$ & $19,33$ & $\va,33$ \\
      \hline
    \end{tabular}\nonumber
  \end{align}

\end{solution}
