
\ifnumequal{\value{rolldice}}{0}{
  % variables 
  \renewcommand{\va}{4}
  \renewcommand{\vb}{6}
  \renewcommand{\vc}{5}
  \renewcommand{\vd}{3\sqrt{2}}
}{
  \ifnumequal{\value{rolldice}}{1}{
    % variables 
    \renewcommand{\va}{6}
    \renewcommand{\vb}{4}
    \renewcommand{\vc}{3}
    \renewcommand{\vd}{4}
  }{
    \ifnumequal{\value{rolldice}}{2}{
      % variables 
      \renewcommand{\va}{4}
      \renewcommand{\vb}{4}
      \renewcommand{\vc}{4}
      \renewcommand{\vd}{2\sqrt{3}}
    }{
      % variables 
      \renewcommand{\va}{6}
      \renewcommand{\vb}{6}
      \renewcommand{\vc}{2}
      \renewcommand{\vd}{2\sqrt{5}}
    }
  }
}

\DIVIDE\va{2}\a
\DIVIDE\vb{-2}\b
\SQUARE\a\aa
\SQUARE\b\bb
\EXPR[0]\rr{\aa + \bb + \vc}

\question[2] Find the center and radius of the circle represented 
by the following equation
  \[ x^2 + y^2 - \va x + \vb y - \vc = 0 \]

\watchout

\begin{solution}[\halfpage]
  The equation of a circle centered at $(a,b)$ with radius $R$ is given by 
  \begin{align}
    (x-a)^2 &+ (y-b)^2 = R^2 \\
    \implies (x^2 - 2ax + a^2) &+ (y^2-2by + b^2) -R^2 = 0 \\
    \implies x^2 + y^2 - 2ax - 2by &+ (a^2 + b^2 - R^2) = 0
  \end{align}
  Compare this with the equation we have been given and you will see 
  \[ \underbrace{x^2+y^2}_{\text{same}}\overbrace{-2ax}^{-\va x}\underbrace{-2by}_{\vb y}\overbrace{+ (a^2+b^2-R^2)}^{-\vc} = 0\]
  In other words, 
  \begin{align}
    -2a &= -\va \implies a = \a \\
    -2b &= \vb \implies b = \b \\
    a^2 + b^2 - R^2 &= -\vc \implies R^2 = \aa + \bb + \vc = \rr \nonumber\\
    \implies R &= \vd\text{ units}
  \end{align}
  Hence, the center of the circle $C$ and its radius $R$ are 
  \[ C = (a,b) = (\a,\b)\text{ and } R = \vd \]
\end{solution}

\ifprintanswers\begin{codex}
  Center = $(\a,\b)$, Radius = $\vd$ units
\end{codex}\fi

