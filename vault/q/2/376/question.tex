


\ifnumequal{\value{rolldice}}{0}{
  \renewcommand{\vb}{755}  % variables 
  \renewcommand{\vc}{18}  % variables 
  \renewcommand{\vd}{5}  % variables 
  \renewcommand{\va}{2.5}
  \renewcommand{\ve}{302}
}{
  \ifnumequal{\value{rolldice}}{1}{
    \renewcommand{\vb}{855}  % variables 
    \renewcommand{\vc}{15}  % variables 
    \renewcommand{\vd}{4}  % variables 
    \renewcommand{\va}{1.8}
    \renewcommand{\ve}{475}
  }{
    \ifnumequal{\value{rolldice}}{2}{
      \renewcommand{\vb}{690}  % variables 
      \renewcommand{\vc}{16}  % variables 
      \renewcommand{\vd}{5}  % variables 
      \renewcommand{\va}{2.3}
      \renewcommand{\ve}{300}
    }{
      \renewcommand{\vb}{800}  % variables 
      \renewcommand{\vc}{14}  % variables 
      \renewcommand{\vd}{5}  % variables 
      \renewcommand{\va}{2.0}
      \renewcommand{\ve}{400}
    }
  }
}

\MULTIPLY\vc\vd\m
\gcalcexpr[2]\final{\ve * \vc / 100.00}

\question[2] A freezer vcreciates at the rate of $\vc$\% per year. It
was purchased new for \$$\vb$. How fast is it vcreciating in dollar
terms (\textit{in \$\$/year}) when it is exactly $\vd$ years old.\\
\textit{Calculation Aid: $e^{0.\m}=\va$}

\watchout

\begin{solution}[\halfpage]
  Let the price of the refrigerator as a function of time be $p(t)$.
  Now, we know that,
  \begin{align}
    \dfrac{dp}{dt}    &= -0.\vc p \\
    \dfrac{dp}{p}     &= -0.\vc dt \\
    \int\dfrac{dp}{p} &= \int -0.\vc dt \\
    \ln p(t)          &= -0.\vc t + C
  \end{align}
  Using the fact that at $t=0$, the original price of the refrigerator 
  was $\vb$, we can use the equation in step (4) to get the value of 
  the constant of integration $C$ to be $\ln\vb$. \\
  Substituting this value of $C$ back in the equation we get,
  \begin{align}
    \ln p(t)                &= -0.\vc t + \ln \vb \\
    \ln \dfrac{p(t)}{\vb} &= -0.\vc t \\
    \dfrac{p(t)}{\vb}     &= e^{-0.\vc t} \\
    p(t)                    &= \dfrac{\vb}{e^{0.\vc t}}
  \end{align}
  At $t=\vd$ years, the price of the refrigerator $p(\vd)$ can be
  calculated as shown,
  \begin{align}
    p(\vd) &= \dfrac{\vb}{e^{0.\vc\times\vd}}
             = \dfrac{\vb}{e^{0.\m}} = \dfrac{\vb}{\va}
             = \ve
  \end{align}
  Therefore vcreciation rate at the end of the $\vd_{th}$ year is
  \begin{align}
    0.\vc\times\ve = \$\final\,\textit{per year}
  \end{align}

\end{solution}

\ifprintanswers
  \begin{codex}
    $\$\final\,\textit{per year}$
  \end{codex}
\fi

