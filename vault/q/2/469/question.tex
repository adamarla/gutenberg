\ifnumequal{\value{rolldice}}{0}{
  % variables 
  \renewcommand{\vj}{240}
  \renewcommand{\vk}{720}
  \renewcommand{\vl}{1080}
  \renewcommand\vx{2}
  \renewcommand\vy{3}
}{
  \ifnumequal{\value{rolldice}}{1}{
    % variables 
    \renewcommand{\vj}{15}
    \renewcommand{\vk}{90}
    \renewcommand{\vl}{270}
    \renewcommand\vx{1}
    \renewcommand\vy{3}
  }{
    \ifnumequal{\value{rolldice}}{2}{
      % variables 
      \renewcommand{\vj}{576}
      \renewcommand{\vk}{2160}
      \renewcommand{\vl}{4320}
      \renewcommand\vx{2}
      \renewcommand\vy{3}
    }{
      % variables 
      \renewcommand{\vj}{18}
      \renewcommand{\vk}{135}
      \renewcommand{\vl}{540}
      \renewcommand\vx{1}
      \renewcommand\vy{3}
    }
  }
}

\DIVIDE\vk\vj\va
\DIVIDE\vl\vk\vb
\MULTIPLY\va{2}\vc
\MULTIPLY\vb{3}\vd 
\EXPR[0]\vn{(\vd - 2*\vc)/(\vd -\vc)}
\SUBTRACT\vn{1}\vm

\question[4] If the $2^{nd}$, $3^{rd}$ and $4^{th}$ terms in the expansion of 
$(a+b)^n$ are $\vj$, $\vk$ and $\vl$ respectively, find the values
of $a$, $b$ and $n$.

\watchout

\ifprintanswers
  % stuff to be shown only in the answer key - like explanatory margin figures
\fi 

\begin{solution}[\fullpage]
  If the $(r+1)^{th}$ term in the exapansion of $(a+b)^n$ is given by 
  \[ T_{r+1} = \binom{n}{r} a^{n-r}\cdot b^r \]
  then the second $(r=1)$, third $(r=2)$ and fourth $(r=3)$ terms would be 
  \begin{align}
    T_1 &= \binom{n}{1}a^{n-1} b = \vj\\
    T_2 &= \binom{n}{2}a^{n-2} b^2 = \vk \\
    T_3 &= \binom{n}{3}a^{n-3}\cdot b^3 = \vl \\
    \implies\dfrac{T_2}{T_1} &= \dfrac{\frac{n\cdot(n-1)}{2} a^{n-2}\cdot b^2}{n\cdot a^{n-1}\cdot b} \nonumber\\
          &= \left( \dfrac{n-1}{2} \right)\cdot\dfrac{b}{a} = \va \nonumber \\
          &\implies \dfrac{b}{a} = \dfrac\vc{n-1} \\
    \implies\dfrac{T_3}{T_2} &= \dfrac{\frac{n\cdot (n-1) \cdot(n-2)}{6} a^{n-3}\cdot b^3}{\frac{n\cdot(n-1)}{2} a^{n-2}\cdot b^2}\nonumber \\
                  &=\left( \dfrac{n-2}{3} \right)\cdot\dfrac{b}{a} = \vb\nonumber \\
                  &\implies \dfrac{b}{a} =  \dfrac\vd{n-2} 
  \end{align}
  What we have from (4) and (5) is that 
  \begin{align}
    \dfrac{b}{a} &= \dfrac\vc{n-1} = \dfrac\vd{n-2} \implies n = \vn \\
    \text{ And therefore, }\dfrac{b}{a} &= \dfrac\vc{\vn-1} \\ 
    \implies T_1 &= \vj = \binom\vn{1}\cdot a^{\vn-1}\cdot\left(\dfrac\vc\vm a\right) \\
    \implies a = \vx\text{ and } b = \vy
  \end{align}
\end{solution}

\ifprintanswers
  \begin{codex}
    $a=\vx, b=\vy, n=\vn$
  \end{codex}
\fi
