\ifnumequal{\value{rolldice}}{0}{
  \renewcommand{\va}{7}
}{
  \ifnumequal{\value{rolldice}}{1}{
    \renewcommand{\va}{5}
  }{
    \ifnumequal{\value{rolldice}}{2}{
      \renewcommand{\va}{9}
    }{
      \renewcommand{\va}{3}
    }
  }
}

\question Find the domain and range of the following 
function assuming $x, f(x) \in \mathbf{R}$.
\begin{align}
  f(x) = \va\sqrt{x-x^2}
\end{align}

\begin{solution}
  To find domain of $f(x)$, we must find all values of $x$ for
  which $(x-x^2)$ is positive, since $f(x) \in \mathbf{R}$.
  \begin{align}
    x-x^2 \geq 0 \\
    \Rightarrow \underbrace{(x)}_{T_1}\cdot\underbrace{(1-x)}_{T_2} \geq 0    
  \end{align}
  For the relation in $(2)$ to hold, both $T_1$ and $T_2$ must either
  be simultaneously \textit{positive} or \textit{negative}.\\
  Assuming $T_1$ and $T_2$ are both \textit{positive},
  \begin{align}
                &x \geq 0 \text{ and } 1 - x \geq 0 \\
    \Rightarrow &x \geq 0 \text{ and } 1 \geq x \\
    \Rightarrow &\mathbf{0 \leq x \leq 1}
  \end{align}
  Assuming $T_1$ and $T_2$ are both \textit{negative},
  \begin{align}
                &x \leq 0 \text{ and } 1 - x \leq 0 \\
    \Rightarrow &x \leq 0 \text{ and } 1 \leq x \text{not possible}
  \end{align}
  
  From the above analysis we can see that only valid condition is 
  from statement $(5)$. Therefore domain of $f(x)$ is $\mathbf{[0,1]}$.\\
  
  The range of $f(x)$ is dependent on the product of $T_1$ and $T_2$. 
  A few calculations show how the $T_1\cdot T_2$ changes with $x$,
  \begin{align}
    x=0          &\Rightarrow T_1\cdot T_2=\mathbf{0} \\
    x=\frac{1}{4}&\Rightarrow T_1\cdot T_2=\dfrac{3}{16} \\
    x=\frac{1}{2}&\Rightarrow T_1\cdot T_2=\mathbf{\dfrac{1}{4}} \\
    x=\frac{3}{4}&\Rightarrow T_1\cdot T_2=\dfrac{3}{16} \\
    x=1          &\Rightarrow T_1\cdot T_2=\mathbf{0}
  \end{align}
  For all values of $x$ in the domain, the product of $x$ and $(1-x)$ 
  is highest at $x=\dfrac{1}{2}$ and lowest at $x=0,1$. Therefore, the
  range of $f(x)$ is $\mathbf{\left[0, \dfrac{\va}{2}\right]}$.

\end{solution}

