
\ifnumequal{\value{rolldice}}{0}{
  % variables 
  \renewcommand{\va}{1} % x-avg
  \renewcommand{\vb}{7} % y-avg
  \renewcommand{\vc}{2} % m-1
  \renewcommand{\vd}{3} % m-2
  \renewcommand{\ve}{4} % c-1
  \renewcommand{\vf}{2} % c-2
}{
  \ifnumequal{\value{rolldice}}{1}{
    % variables 
    \renewcommand{\va}{2}
    \renewcommand{\vb}{5}
    \renewcommand{\vc}{3}
    \renewcommand{\vd}{4}
    \renewcommand{\ve}{5}
    \renewcommand{\vf}{3}
  }{
    \ifnumequal{\value{rolldice}}{2}{
      % variables 
      \renewcommand{\va}{4}
      \renewcommand{\vb}{7}
      \renewcommand{\vc}{1}
      \renewcommand{\vd}{2}
      \renewcommand{\ve}{5}
      \renewcommand{\vf}{3}
    }{
      % variables 
      \renewcommand{\va}{3}
      \renewcommand{\vb}{-5}
      \renewcommand{\vc}{2}
      \renewcommand{\vd}{3}
      \renewcommand{\ve}{7}
      \renewcommand{\vf}{9}
    }
  }
}


\MULTIPLY\va{2}\j
\MULTIPLY\vb{2}\k
\ADD\ve\vf\a
\SUBTRACT\a\k\b
\MULTIPLY\b{-1}\c
\SOLVELINEARSYSTEM(\vc,\vd;1,1)(\c,\j)(\g,\h)
\EXPR[0]\v{(\vc * \g + \ve)}
\EXPR[0]\w{(\vd * \h + \vf)}

\SUBTRACT\v\vb\pp
\SUBTRACT\g\va\qq
\FRACTIONSIMPLIFY\pp\qq\p\q
\EXPR[0]{\acn}{(\q * \vb - \p * \va)}

\question[4] The segment of a line between two other lines $L_1: \vc x - y + \ve = 0$ and $L_2: \vd x -y + \vf = 0$ 
is bisected at the point $M = (\va, \vb)$. Find the equation of the line.

\watchout[-10pt]

\figinit{pt}
  \figpt 1:(10,90)
  \figpt 2:$L_2$(80,30)
  \figvectP 20 [1,2]
  \figpt 3:(10,30)
  \figpt 4:$L_1$(80,50)
  \figvectP 21 [3,4]
  \figpt 5:(25,25)
  \figpt 6:(25,70)
  \figvectP 22 [5,6]
  \figptinterlines 30:$B$[1,20;5,22]
  \figptinterlines 31:$A$[3,21;5,22]
  \figvectP 23 [30,31]
  \figpttra 7:$M$= 30 /0.5,23/
\figdrawbegin{}
  \figdrawline [1,2]
  \figdrawline [3,4]
  \figdrawline [30,31]
\figdrawend
\figvisu{\figBoxA}{}{%
  \Large
  \figwritee 2:(3)
  \figwritee 4:(3)
  \figset write(mark=$\bullet$)
  \figwritene 30:(3)
  \figwritese 31:(3)
  \figwritew 7:(3)
}

\ifprintanswers
  \vspace{0.7cm}
  \centerline{\box\figBoxA}
\fi 

\begin{solution}[\halfpage]
	If the required line intersects $L_1$ at $A = (a,b)$ and $L_2$ at $B = (m,n)$, then 
	\begin{align}
		\dfrac{a + m}{2} &= \va \implies a+m = \j \\
		\dfrac{b + n}{2} &= \vb \implies b+n = \k
	\end{align}
	Moreover, as $(a,b)$ lies on $L_1$ and $(m,n)$ lies on $L_2$ 
	\begin{align}
    (L_1) : \vc a + \ve &= b  \\
    (L_2) : \vd m + \vf &= n  \\
    \implies b+n &= \k = \vc a + \vd m + (\ve + \vf) \\
    \implies \vc a + \vd m &= \c
	\end{align}

  Solving equations (1) and (6), we get 
  \begin{align}
    a = \g &\text{ and } m = \h \\
    \text{And }\therefore, b = \v &\text{ and } n = \w 
  \end{align}

  We now have more than enough information to find the equation of the required line. Take \textbf{any two} 
  of the three points - $(\va, \vb), (\g,\v)$ or $(\h,\w)$ and find the line
  \begin{align}
    \dfrac{y-\vb}{x - \va} &= \dfrac{\v-\vb}{\g-\va} = \WRITEFRAC\p\q \\
    \implies \q y &= \p x + \acn
  \end{align}
\end{solution}

\ifprintanswers\begin{codex}$\q y =\p x+\acn$\end{codex}\fi

