\documentclass[14pt,fleqn]{extarticle}
\RequirePackage{prepwell}

\newcommand\Q{\mathbb{Q}}

\previewoff 

\begin{document} 
\begin{question}
	\statement 
    
    Let $A = \Q\times\Q$, where $\Q$ is the set of rational numbers, and $*$ be a binary operation defined on $A$ as 
    \[ \quad \left(a,b \right) * \left(c,d \right) = \left(a c,b + a d \right)\] where $\left(a,b \right), \left(c,d \right)\in A$. Given this, find 
    \begin{itemize}
    \item{ the identity element in $A$} 
    \item{ the inverse of some $(a,b)\in A$}
    \end{itemize}
    
    \begin{step}
  \begin{options} 
     \correct 
       
     As per the question, 
     \begin{align}
	   \left(a,b \right), \left(c,d \right)&\in A = \Q\times\Q \\
	   \implies a,b,c,d &\in \Q 
\end{align}
       
     \incorrect
        
      As per the question 
     \begin{align}
	   \left(a,b \right), \left(c,d \right)&\in A = \Q\times\Q \\
	   \implies a,b,c,d &\in A 
\end{align}
    \end{options} 
     \reason
     
     Nothing in the problem statement is written without a reason. 
     Which is why it is important to understand every bit of it \newline 
     
     $A = \Q\times\Q \implies A$ is the \underline{Cartesian product} of 
     two $\Q$ sets \newline 
     
     Hence, $A$ will be made up of elements like $\left(x,y \right)$ where both 
     $x,y\in\Q$ \newline 
     
     Had $A = \Q\times\Q\times\Q$, then $A$ would have been composed of 
     elements like $\left(x,y,z \right)$ where $x,y,z\in\Q$. See the pattern? \newline 
     
     
     And therefore, in this question 
     \[ \quad \left(a,b \right), \left(c,d \right)\in A\text{ but } a,b,c,d\in\Q \]
      
\end{step}

\begin{step}
  \begin{options} 
     \correct 
       
     If $I = (i,j)$ be the identity element, then 
     \begin{align}
	I * (a,b) &= (a,b)*I = (a,b) \\
	\therefore I &= (1,0) 
\end{align}

     \incorrect
        
             If $I = (i,j)$ be the identity element, then 
     \begin{align}
	I * (a,b) &= (a,b)*I = (a,b) \\
	\therefore I &= (0,1) 
\end{align}
    \end{options} 
     \reason 
     
     If $I$ be the identity element, then all of the following conditions are true for $I$
     \begin{center}
  \begin{tabular}{N}
   \toprule
        \text{Condition}  \\
   \midrule 
   I\text{ is an element like }\left(i,j \right)\text{ where }i,j\in\Q \\
    \midrule 
    I \in A \\
    \midrule 
    \left(a,b \right)*\underbrace{\left(i,j \right)}_I = \left(i,j \right)* \left(a,b \right) = \left(a,b \right) \\
    \bottomrule
  \end{tabular}
\end{center}  

Using the last condition, 
\begin{align}
	(a,b) * (i,j) &= \left(ai, b + aj \right) = (a,b) \\ 
	\implies ai &= a \text{ and } b + aj = b \\
	\therefore i &= 1 \text{ and } j = 0 \\
	\text{or } I &= (i,j) = (1,0) \in A 
\end{align}

Notice also that 
\begin{align}
I * (a,b) &= (1,0)*(a,b) \\
&= (a, 0 + b) = (a,b) 
\end{align}
       
\end{step}

\begin{step}
  \begin{options} 
     \correct 
       
     If $X = (x,y) = (a,b)^{-1}$, then 
     \[ \qquad X = \left(\frac{1}{a}, -\frac{b}{a} \right),\, a \neq 0\]
     
    Therefore, the inverse of $(a,b)$ is $\left(\dfrac{1}{a}, -\dfrac{b}{a} \right)$ as long as $a\neq 0$ 
    \end{options} 
     \reason 
      
      If $X = (x,y) = (a,b)^{-1}$, then 
      \begin{align}
      (a,b) * (x,y) &= I = (1,0) \\
	\implies \left(ax, b + ay \right) &= (1,0) \\
	\text{or } ax &= 1 \text{ and } b + ay = 0 \\[-10pt]
	\therefore x &= \frac{1}{a} \text{ and } y = \frac{-b}{a}
\end{align} 

Hence, the \underline{inverse element} of $(a,b)$ is  
\[\qquad X = \left(\dfrac{1}{a},-\dfrac{b}{a} \right), a \neq 0\]

Notice that any element with $a=0$, like $(0,3)$, has \underline{no inverse element}
\end{step}
\end{question} 
\end{document} 