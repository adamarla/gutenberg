


\ifnumequal{\value{rolldice}}{0}{
  % variables 
  \renewcommand{\va}{-4} % open left  
  \renewcommand{\vc}{y^2 = \vb x}
  \renewcommand\vi{y^2\geq 0\implies \vb x\geq 0\implies x\leq 0}
}{
  \ifnumequal{\value{rolldice}}{1}{
    % variables 
    \renewcommand{\va}{3} % open up
    \renewcommand{\vc}{y = \vb x^2}
    \renewcommand\vi{\vb x^2\geq 0\implies y\geq 0\text{ for all }x}
  }{
    \ifnumequal{\value{rolldice}}{2}{
      % variables 
      \renewcommand{\va}{-5} %open left
      \renewcommand{\vc}{y^2 = \vb x}
      \renewcommand\vi{y^2\geq 0\implies \vb x\geq 0\implies x\leq 0}
    }{
      % variables 
      \renewcommand{\va}{-6} %open down
      \renewcommand{\vc}{y = \vb x^2}
      \renewcommand\vi{\vb x^2 \leq 0\text{ for all }x\implies y\leq 0\text{ for all }x}
    }
  }
}

\ifnumodd{\value{rolldice}}{
  \renewcommand\vd{x^2 = 4py}
  \renewcommand\ve{ F = (0,\va) }
}{
  \renewcommand\vd{y^2 = 4px}
  \renewcommand\ve{ F = (\va, 0) }
}

\MULTIPLY\va{4}\vb
\ABSVALUE\vb\vf

\question Given the equation $\vc$ of a parabola

\watchout

\figinit{pt}
  \def\Xori{49}
  \def\Yori{45}
  \figpt 1:$O$(\Xori,\Yori)
  \figpt 2:(55,55)
  \figpt 3:(55,35)
  \figpt 4:(100,75)
  \figpt 5:$A$(100,15)
  \figptscontrol 10[4,2,3,5]
  \ifprintanswers
    \ifnumodd{\value{rolldice}}{ % x = 4y^2
      \ifnumequal{\value{rolldice}}{1}{ % y = 4x^2
        \figpt 20:(40,49)
        \figpt 21:(58,49)
        \figpt 22:(15,90)
        \figpt 23:(75,90)
      }{ % y = -4x^2
        \figpt 20:(40,40)
        \figpt 21:(58,40)
        \figpt 22:(20,5)
        \figpt 23:(85,5)
      }
    }{
      \figpt 20:(43,35)
      \figpt 21:(43,55)
      \figpt 22:(0,15)
      \figpt 23:(0,75)
    }
    \figptscontrol 30[22,20,21,23]
  \fi
\figdrawbegin{}
  \figdrawBezier 1[4,10,11,5]
  \ifprintanswers
    \figdrawBezier 1[22,30,31,23]
  \fi
  \drawAxes{1}{-5}{60}{-5}{60}
\figdrawend
\figvisu{\figBoxA}{}{%
  \figwritesw 1:(3)
  \figwritese 5:(3)
  \ifprintanswers
    \figwritew 22:$B$(2) 
  \fi
}

\vspace{0.7cm}
\centerline{\box\figBoxA}

\begin{parts}
  \part[1] Can curve $A$ be the plot of the given parabola? If not, then draw the correct plot.

\begin{solution}[\mcq]
  From the given equation, we can infer that 
  \[ \vi \]
  
  Hence, if anything, the parabola should look like curve $B$.
  \end{solution}

  \part[1] Find coordinates of the focus($F$) and the vertex ($V$). Mark both in the corrected plot.

\begin{solution}[\mcq]
    The vertex of the parabola is simply at $(0,0)$.

    And given the form of the equation 
    \[ \vd \implies p = \va \] 
    the focus is simply $\ve$
  \end{solution}

  \part[1] Find the length of the latus rectum of the parabola.

\begin{solution}[\mcq]
    The length of the latus rectum is
    \[ \vert 4\times p\vert = \vert 4\times \va\vert = \vf \]
  \end{solution}

\end{parts}

\ifprintanswers\begin{codex}
  \begin{tabular}{l l l}
    (a) No & (b) Vertex = $(0,0)$, Focus = $\ve$ & (c) $\vf$
  \end{tabular}
\end{codex}\fi
