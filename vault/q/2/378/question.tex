

\question We perform a decomposition to normalize a single schema to be of
certain normal forms. For such a decomposition to be \textit{equivalent} 
to the original schema, it is desirable that it be lossless. To examine this 
let us consider the original schema $R(A, B, C)$, which we then decompose into 
$R_1(A, B)$ and $R_2(A, C)$.

\begin{parts}
  \part[4] Will this decomposition always be lossless? Answer \textit{yes} or 
  \textit{no} and briefly explain why (marks will be awarded only if both the
  answer and the justification are correct).

\begin{solution}[\fullpage]
    No. \\
    Unless $A\rightarrow B$ or $A\rightarrow C$ is satisfied in the original
    relation, $R_1$ \textit{join} $R_2$ may result in entries that were not 
    originally present in $R$.
  \end{solution}

  \part[4] Give an example instance for $R$ and demonstrate its decomposition as
  evidence to support your answer to part (a) of this question.

\begin{solution}[\fullpage]
    Consider that $R$ looks like this,
    \begin{align}
      \begin{tabular}{ccc}      
        A &B &C \\
        \hline
        1 &2 &3 \\
        1 &4 &5 \\
      \end{tabular} \nonumber
    \end{align}
    Decomposition of $R$ into $R_1$ and $R_2$ would look like this,
    \begin{align}
      \begin{tabular}{cc|cc}
        A &B &A &C \\
        \hline
        1 &2 &1 &3 \\
        1 &4 &1 &5 \\ 
      \end{tabular} \nonumber
    \end{align}
    Performing $R_1 \bowtie R_2$ would result in this,
    \begin{align}
      \begin{tabular}{ccc}
        A &B &C \\
        \hline
        1 &2 &3 \\
        1 &4 &5 \\
        1 &2 &5 \\
        1 &4 &3 \\
      \end{tabular}\nonumber    
    \end{align}
    The \textit{join} generates extra records that did not exist in 
    the original $R$. Therefore this was not a lossless operation.
  \end{solution}

\end{parts}

