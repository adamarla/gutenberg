

\question[3] For what value of $M$ would the following be true? 
\[ \dfrac{\tan 2x\cdot\tan x}{\tan 2x - \tan x} = \sin Mx \] 

\ifprintanswers
  % stuff to be shown only in the answer key - like explanatory margin figures
\fi 

\begin{solution}[\halfpage]
  \textbf{Method \# 1}

  \begin{align}
    \dfrac{\tan 2x\cdot \tan x}{\tan 2x-\tan x} &= 
    \dfrac{\dfrac{\sin 2x}{\cos 2x}\cdot\dfrac{\sin x}{\cos x}}{\dfrac{\sin 2x}{\cos 2x}-\dfrac{\sin x}{\cos x}} \\
    &= \dfrac{\sin 2x\cdot \sin x}{\underbrace{\sin 2x\cdot\cos x - \cos 2x\cdot\sin x}_{\sin(a-b)=\eSinOfDiff{a}{b}}} \\
    &= \dfrac{\sin 2x\cdot\sin x}{\sin (2x-x)} = \sin 2x \\
    \implies M &= 2
  \end{align}

  \textbf{Method \# 2}

	\begin{align}
    \dfrac{\tan 2x\cdot \tan x}{\tan 2x-\tan x} &= 
    \dfrac{\left(\dfrac{2\tan x}{1-\tan^2 x} \right)\cdot \tan x}{\left(\dfrac{2\tan x}{1-\tan^2 x}\right)-\tan x} \\
    &= \dfrac{2\tan^2 x}{2\tan x - (\tan x -\tan^3 x)} \\
    &= \dfrac{2\tan^2 x}{\tan x + \tan^3 x} = \dfrac{2\tan x}{1 + \tan^2 x} \\
    &= \sin 2x \implies M = 2
	\end{align}
\end{solution}
\ifprintanswers\begin{codex}$2$\end{codex}\fi
