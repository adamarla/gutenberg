\ifnumequal{\value{rolldice}}{0}{
  \renewcommand{\va}{20}
  \renewcommand{\vb}{\dfrac{\pi}{9}}
}{
  \ifnumequal{\value{rolldice}}{1}{
    \renewcommand{\va}{30}
    \renewcommand{\vb}{\dfrac{\pi}{6}}
  }{
    \ifnumequal{\value{rolldice}}{2}{
      \renewcommand{\va}{10}
      \renewcommand{\vb}{\dfrac{\pi}{18}}
    }{
      \renewcommand{\va}{15}
      \renewcommand{\vb}{\dfrac{\pi}{12}}
    }
  }
}

\ADD{270}\va\p
\SUBTRACT{270}\va\q

\question[5] Without resorting to a calculator or tables, verify
\begin{align}
  \dfrac{1}{\cos \p^\circ}+\dfrac{1}{\sqrt{3}\sin \q^\circ} 
    = \dfrac{4}{\sqrt{3}} \nonumber
\end{align}

\begin{solution}[\fullpage]
  \begin{align}
    \p^\circ &= \left(\dfrac{3\pi}{2}+\vb\right) 
      \text{radians} \\
    \q^\circ &= \left(\dfrac{3\pi}{2}-\vb\right) 
      \text{radians}
  \end{align}
  Therefore Left Hand Side (L.H.S) may be rewritten as,
  \begin{align}
    \text{L.H.S} &= \dfrac{1}{\cos(\dfrac{3\pi}{2}+\vb)}+
      				  \dfrac{1}{\sqrt{3}
      				    \sin(\dfrac{3\pi}{2}-\vb)} \\
      		     &= \dfrac{1}{\sin\vb}-
      				  \dfrac{1}{\sqrt{3}\cos\vb} \\
      		     &= \dfrac{\sqrt{3}\cos\vb-\sin\vb}
      		              {\sqrt{3}\cos\vb\sin\vb} \\
      		     &= \dfrac{2\left(\dfrac{\sqrt{3}}{2}\cos\vb-
      		                  \dfrac{1}{2}\sin\vb\right)}
            		        {\dfrac{\sqrt{3}}{2}\left(2\cos\vb
            		            \sin\vb\right)} \\
      		     &= \dfrac{4}{\sqrt{3}}\left(
      		          \dfrac{\sin \dfrac{\pi}{3}\cos\vb-
      		                 \cos \dfrac{\pi}{3}\sin\vb}
            		          {2\cos\vb\sin\vb}\right) \\
      		     &= \dfrac{4}{\sqrt{3}}\left(
      		          \dfrac{\sin(\dfrac{\pi}{3}-\vb)}
      		                {\sin 2(\vb)}\right) \\
      		     &= \dfrac{4}{\sqrt{3}}
  \end{align}
\end{solution}
