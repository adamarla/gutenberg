


\ifnumequal{\value{rolldice}}{0}{
  % variables 
  \renewcommand{\vbone}{odd} % type 
  \renewcommand{\vbtwo}{0,1,4,7} % available 
  \renewcommand{\vbthree}{350} % less than 
  \renewcommand{\vbfour}{1,7} % units 
  \renewcommand{\vbfive}{0,1,4,7} % tens 
  \renewcommand{\vbsix}{1} % hundreds 
}{
  \ifnumequal{\value{rolldice}}{1}{
    % variables 
    \renewcommand{\vbone}{even}
    \renewcommand{\vbtwo}{3,6,8,9}
    \renewcommand{\vbthree}{500}
    \renewcommand{\vbfour}{6,8}
    \renewcommand{\vbfive}{3,6,8,9}
    \renewcommand{\vbsix}{3}
  }{
    \ifnumequal{\value{rolldice}}{2}{
      % variables 
      \renewcommand{\vbone}{odd}
      \renewcommand{\vbtwo}{2,5,7,8}
      \renewcommand{\vbthree}{495}
      \renewcommand{\vbfour}{5,7}
      \renewcommand{\vbfive}{2,5,7,8}
      \renewcommand{\vbsix}{2}
    }{
      % variables 
      \renewcommand{\vbone}{even}
      \renewcommand{\vbtwo}{6,7,8,9}
      \renewcommand{\vbthree}{700}
      \renewcommand{\vbfour}{6,8}
      \renewcommand{\vbfive}{6,7,8,9}
      \renewcommand{\vbsix}{6}
    }
  }
}

\question[2] How many 3-digit $\vbone$ numbers less than $\vbthree$ can be 
 formed using the digits $\vbtwo$ - if repetition of digits is allowed


\watchout

\ifprintanswers
    \begin{tabular}{ccc}
      \toprule
      Units & Tens & Hundreds \\
      \midrule
      $\vbfour$ & $\vbfive$ & $\vbsix$ \\
      \bottomrule
    \end{tabular}
\fi 

\begin{solution}[\mcq]
  The table above shows the digits that can appear in the units, tens and 
  hundreds places so that all conditions are satisified (number < $\vbthree$ and 
  number is \vbone)

  Therefore, the number of ways of creating a valid number are
  \begin{align}
    \text{\# of ways} &= \underbrace{2}_{\texttt{units}}\times\underbrace{4}_{\texttt{tens}}\times\underbrace{1}_{\texttt{hundreds}} \\
    &= 8
  \end{align} 
\end{solution}
