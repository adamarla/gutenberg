\documentclass[14pt,fleqn]{extarticle}
\RequirePackage{prepwell}

\newcommand\R{\mathbb{R}}
\newcommand\fgx{\left(f\circ g \right)\left(x \right)}
\newcommand\ifgy{\left(f\circ g \right)^{-1}\left(y \right)}
\newcommand\ifgx{\left(f\circ g \right)^{-1} \left(x \right)}

\previewoff

\begin{document}
\begin{question}
\statement

If the function $f:\R\to\R$ be defined as $f(x) = 2x-3$ and $g:\R\to\R$ be defined as $g(x) = x^3 + 5$, then find the value of 
\[ \qquad\qquad \ifgx \]	

\begin{step}
  \begin{options} 
     \correct 
       
       \begin{align}
	\text{If } f(x) = 2x - 3 &\text{ and } g(x) = x^3 + 5 \\
	\text{then } \fgx &= 2x^3 + 7 
\end{align}

     \incorrect

       \begin{align}
	\text{If } f(x) = 2x - 3 &\text{ and } g(x) = x^3 + 5 \\
	\text{then } \fgx &= 2x^4 - 3x^3 + 10x + 15  
\end{align}
        
    \end{options} 
     \reason 
     
     $\fgx$ is \underline{simply shorthand} for $f \left(g(x) \right)$
     \begin{align}
     \therefore \fgx &= f \left( g(x) \right) = 2g(x) - 3 \\
     &= 2 \left(x^3 + 5\right) - 3 \\
     &= 2x^3 + 7 
\end{align}
       
\end{step}

\begin{step}
  \begin{options} 
     \correct 
       
       \[ \therefore \ifgx = \sqrt[3]{\dfrac{x-7}{2}} \]
     \incorrect
     
     \[ \therefore \ifgx = \dfrac{1}{2x^3 + 7} \]
        
    \end{options} 
     \reason 

In $ y= \fgx$, we know $x$ (independent variable) and use $\fgx$ to find $y$ (dependent variable)\newline 

But to \underline{find $x$ given $y$}, we need the inverse function $\ifgy$ which we can find as follows 
\begin{align}
y = 2x^3 + 7 &\implies x = \sqrt[3]{\dfrac{y-7}{2}} \\
\therefore \left(f\circ g \right)^{-1}(y) &= x = \sqrt[3]{\dfrac{y-7}{2}}
\end{align}

But \underline{by convention}, the \underline{independent} variable is 
written as $x$ and the \underline{dependent} variable as $y$. Hence, 
\[ \qquad \ifgx = \sqrt[3]{\dfrac{x-7}{2}} \]
       
\end{step}
\end{question}
\end{document}
