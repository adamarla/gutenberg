


\ifnumequal{\value{rolldice}}{0}{
  % variables 
}{
  \ifnumequal{\value{rolldice}}{1}{
    % variables 
  }{
    \ifnumequal{\value{rolldice}}{2}{
      % variables 
    }{
      % variables 
    }
  }
}

\question [TODO]: This slot can be reused - question is a repeat



  \begin{marginfigure}
    \figinit{pt}
      \figpt 1:(-10,-10)
      \figpt 2:(-10,0)
      \figpt 10:(0,0)
      \figpt 20:(20,5)
      \figpt 30:(30,20)
      \figpt 40:(40,60)
      \figpt 50:(20,32)
      \figpt 60:(40,48)
      \figpt 70:(50,38)
      \figpt 80:(60,20)
      \figpt 90:(50,90)
      % extremeties
      \def\Xmax{100}
      \def\Ymax{80}
      \def\Xmin{-100}
      \def\Ymin{-40}
      % pts for numbering the axes (5, 10, 15...)
      \figpt 204:$\tiny\text{O}$(0,0)
      \figpt 205:$\tiny\text{1}$(0,48)
      \figpt 206:$\tiny\text{R}$(15,20)
      % label graph
      \figpt 100:(\Xmax,0)
      \figpt 101:(0,\Ymax)
      \figpt 102:(0,\Ymin)
    \figdrawbegin{}
      \figset arrowhead(length=4, fillmode=yes)
      \figdrawaxes 10(\Xmin, \Xmax, \Ymin, \Ymax)
      \figdrawcurve [1,10,50,60,70,80]
      \figdrawcurve [2,10,20,30,40,90]
    \figdrawend
    \figvisu{\figBoxA}{}{%
      \figwritene 60: $\text{(1/2,1)}$(2 pt)
      \figwritee 100: $x$(2 pt)
      \figwritesw 204: (2 pt)
      \figwritee 206: $R$(1 pt)
      \figsetmark {$-$}
      \figwritew 205: $1$(2 pt)
    }
    \centerline{\box\figBoxA}
  \end{marginfigure}

\begin{parts}
  \part[2] Write an equation for the line tangent to the graph of $f$ at $x=\dfrac{1}{2}$.

\begin{solution}[\mcq]
    \begin{align}
      f(x) = 8x^3 \Rightarrow f(\dfrac{1}{2}) = 1 
    \end{align}    
    To find the tangent at a we must first find the derivative of $f(x)$ to get slope.
    \begin{align}
      f'(x) = 24x^2 \Rightarrow f'(\dfrac{1}{2}) = 6 
    \end{align}    
    Therefore equation of the tangent at $x=\dfrac{1}{2}$ is
    \begin{align}
      y=1+6(x-\dfrac{1}{2}) 
    \end{align}
  \end{solution}

  \part[4] Find the area of $R$.
\begin{solution}[\mcq]
    Let the area of $R$ be $A_R$. Since $R$ extends from $x=0$ to $x=\dfrac{1}{2}$ area 
    can be calculated as follows,
    \begin{align}
      A_R &= \int_0^{1/2} (g(x)-f(x))\ud x \\ \nonumber
          &= \int_0^{1/2} (\sin(\pi x)-8x^3)\ud x \\ \nonumber
          &= \left[-\dfrac{1}{\pi}\cos(\pi x)-2x^4\right]_0^{1/2} \\ \nonumber
          &= -\dfrac{1}{8}+\dfrac{1}{\pi} \nonumber
    \end{align}
  \end{solution}

  \part[3] Write, but do not, an integral expression for the volume of the solid generated
  when $R$ is rotated about the horizontal line $y=1$.
\begin{solution}[\mcq]
    Let Volume of generated solid be $V$. To calculate the volume $V$ we must sum up the
    areas of all the discs of thickness $\ud x$ from $x=0$ to $x=1/2$. The area $dA$ of each
    disc being the $\pi((1-f(x))^2-(1-g(x))^2)\ud x$. Therefore expression for Volume is,
    \begin{align}
      V = \pi((1-8x^3)^2-(1-\sin(\pi x))^2)\ud x \\ \nonumber 
    \end{align}
  \end{solution}

\end{parts}

