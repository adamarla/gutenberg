


\ifnumequal{\value{rolldice}}{0}{
  % variables 
  \renewcommand{\vbone}{\sqrt{2}}
  \renewcommand{\vbtwo}{22}
}{
  \ifnumequal{\value{rolldice}}{1}{
    % variables 
    \renewcommand{\vbone}{\sqrt{3}}
    \renewcommand{\vbtwo}{19}
  }{
    \ifnumequal{\value{rolldice}}{2}{
      % variables 
    }{
      % variables 
    }
  }
}
\SUBTRACT{51}\vbtwo\l
\SUBTRACT\vbtwo{1}\k 
\question Find the term of the expansion $(a+b)^{50}$ which is greatest in absolute value if $\dfrac{a}{b} = \vbone$ where $\left[ r \in \mathbb{N} \right]$.


\watchout

\ifprintanswers
  % stuff to be shown only in the answer key - like explanatory margin figures
  \begin{marginfigure}
    \figinit{pt}
      \figpt 100:(0,0)
      \figpt 101:(0,0)
    \figdrawbegin{}
      \figdrawline [100,101]
    \figdrawend
    \figvisu{\figBoxA}{}{%
    }
    \centerline{\box\figBoxA}
  \end{marginfigure}
\fi 

\begin{solution}
Suppose the general term $T_{r}$ is the greatest term then it can said that $|\dfrac{T_{r}}{T_{r+1}}| \geq 1$ \\ 
and also $|\dfrac{T_{r}}{T_{r-1}}| \geq 1$.\\
Expanding these terms : 
\begin{align}
{T}_{r} &= \encr{50}{r-1} \cdot a^{51-r} \cdot b^{r-1} \\ 
{T}_{r+1} &= \encr{50}{r} \cdot a^{50-r} \cdot b^{r} \\
{T}_{r-1} &= \encr{50}{r-2} \cdot a^{52-r} \cdot b^{r-2} \\
\nonumber \\
\dfrac{T_{r}}{T_{r+1}} &\geq 1 \\
\dfrac{\encr{50}{r-1} \cdot a^{51-r} \cdot b^{r-1}}{\encr{50}{r} \cdot a^{50-r} \cdot b^{r}} &\geq 1 \\
\dfrac{r}{51-r} \cdot \dfrac{a}{b} &\geq 1 \\
\dfrac{r}{51-r} \cdot \vbone &\geq 1 \\
\Rightarrow r \geq \dfrac{51}{1+\vbone} \\
\nonumber \\
\dfrac{T_{r}}{T_{r-1}} &\geq 1 \\
\dfrac{\encr{50}{r-1} \cdot a^{51-r} \cdot b^{r-1}}{\encr{50}{r-2} \cdot a^{52-r} \cdot b^{r-2}} &\geq 1 \\
\dfrac{52-r}{r-1} \cdot \dfrac{b}{a} &\geq 1 \\
\dfrac{52-r}{r-1} \cdot &\geq \vbone \\
\Rightarrow r \leq \dfrac{52+\vbone}{1+\vbone} \\
\end{align}  
Now as r is a natural number by solving the two inequalities we get r = $\vbtwo$. \\
Thus, the greatest term by absolute value is 
\begin{align}
T_{\vbtwo} &=\encr{50}{\vbtwo- 1} \cdot a^{51-\vbtwo} \cdot b^{\vbtwo- 1}  \\
&=\encr{50}{\vbtwo- 1} \cdot \left({\dfrac{a}{b}}\right)^{51-\vbtwo} \cdot b^{50} \\
&=\encr{50}{\k} \cdot \left({\vbone}\right)^{\l} \cdot b^{50}  
\end{align}
\end{solution}


