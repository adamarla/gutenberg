
 \setcounter{rolldice}{1}

\ifnumequal{\value{rolldice}}{0}{
  % variables 
  \renewcommand{\vbone}{\dfrac{1}{1+\textit{i}}}
}{
  \ifnumequal{\value{rolldice}}{1}{
    % variables 
    \renewcommand{\vbone}{\dfrac{1}{1-\textit{i}}}
  }{
    \ifnumequal{\value{rolldice}}{2}{
      % variables 
    }{
      % variables 
    }
  }
}

\question Find the modulus and argument of the complex number $\vbone$


\watchout

\ifprintanswers
  % stuff to be shown only in the answer key - like explanatory margin figures
  \begin{marginfigure}
    \figinit{pt}
      \figpt 100:(0,0)
      \figpt 101:(0,0)
    \figdrawbegin{}
      \figdrawline [100,101]
    \figdrawend
    \figvisu{\figBoxA}{}{%
    }
    \centerline{\box\figBoxA}
  \end{marginfigure}
\fi 

\begin{solution} 
To find modulus and argument we convert the function in polar form i.e $r\left( \cos \theta + \textit{i} \sin \theta \right)$ where r is the modulus and $\theta$ is the argument.
\ifnumequal{\value{rolldice}}{0}{
$\dfrac{1}{1+\textit{i}}$ \\
By rationalizing this expression, 
\begin{align}
\dfrac{1}{1+\textit{i}} \times \dfrac{1 - \textit{i}}{1-\textit{i}}\\
&= \dfrac{1 - \textit{i}}{2} \\
&= \dfrac{1}{2} - \dfrac{\textit{i}}{2}
\end{align}
Now, this is in the form of $a + b \textit{i}$ \\
To convert this in polar form, we put  
\begin{align}
r \cos \theta = \dfrac{1}{2} \\
r \sin \theta = -\dfrac{1}{2} \\
\Rightarrow r = \sqrt{\dfrac{1}{2}^{2} + {-\dfrac{1}{2}}^{2}} = \dfrac{1}{\sqrt{2}} 
\end{align}
Plugging this value of r in equations 4 and 5, we get $\theta = -\dfrac{\pi}{4}$
Thus, modulus $(r) = \dfrac{1}{\sqrt{2}}$ and Argument$(\theta) = -\dfrac{\pi}{4}$
}


\ifnumequal{\value{rolldice}}{1}{
$\dfrac{1}{1-\textit{i}}$ \\
By rationalizing this expression, 
\begin{align}
\dfrac{1}{1-\textit{i}} \times \dfrac{1 + \textit{i}}{1+\textit{i}} \\
&= \dfrac{1 + \textit{i}}{2}  \\
&= \dfrac{1}{2} + \dfrac{\textit{i}}{2}
\end{align}
Now, this is in the form of $a + b \textit{i}$ \\
To convert this in polar form, we put  
\begin{align}
r \cos \theta = \dfrac{1}{2}\\
r \sin \theta = \dfrac{1}{2} \\
\Rightarrow r = \sqrt{\dfrac{1}{2}^{2} + {\dfrac{1}{2}}^{2}} = \dfrac{1}{\sqrt{2}} 
\end{align}
Plugging this value of r in equations 4 and 5, we get $\theta = \dfrac{\pi}{4}$ \\
Thus, modulus $(r) = \dfrac{1}{\sqrt{2}}$ and Argument$(\theta) = \dfrac{\pi}{4}$
}

\end{solution}


