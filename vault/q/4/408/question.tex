


\ifnumequal{\value{rolldice}}{0}{
  % variables 
  \renewcommand{\vbone}{1}
  \renewcommand{\vbtwo}{\sqrt{3}}
  \renewcommand{\vbthree}{2}
  \renewcommand{\vbfour}{\frac{\pi}{3}}
}{
  \ifnumequal{\value{rolldice}}{1}{
    % variables 
    \renewcommand{\vbone}{\sqrt{3}}
    \renewcommand{\vbtwo}{1}
    \renewcommand{\vbthree}{2}
    \renewcommand{\vbfour}{\frac{\pi}{6}}
  }{
    \ifnumequal{\value{rolldice}}{2}{
      % variables 
      \renewcommand{\vbone}{1}
      \renewcommand{\vbtwo}{1}
      \renewcommand{\vbthree}{\sqrt{2}}
      \renewcommand{\vbfour}{\frac{\pi}{4}}
    }{
      % variables 
    }
  }
}

\question Convert the complex number $z =\vbone+\vbtwo i$ to polar form \\ $\left[ r(\cos \theta + i \sin \theta) \right] $ 


\watchout

\ifprintanswers
  % stuff to be shown only in the answer key - like explanatory margin figures
  \begin{marginfigure}
    \figinit{pt}
      \figpt 100:(0,0)
      \figpt 101:(0,0)
    \figdrawbegin{}
      \figdrawline [100,101]
    \figdrawend
    \figvisu{\figBoxA}{}{%
    }
    \centerline{\box\figBoxA}
  \end{marginfigure}
\fi 

\begin{solution}
\begin{align}
z = \vbone + \vbtwo i \\
\vbone = r \cos \theta  \quad \vbtwo = r \sin \theta 
\end{align}
Squaring and adding
\begin{align}
r^{2} &= (\vbone)^{2} + (\vbtwo)^{2}  \\
\Rightarrow r &= \vbthree \\
\cos \theta = \dfrac{\vbone}{\vbthree} \\ 
\sin \theta = \dfrac{\vbtwo}{\vbthree} \\
\Rightarrow \theta = \vbfour 
\end{align} 
Thus, \\
In polar form $z = \vbthree \left( \cos \vbfour + i \sin \vbfour \right)$ 
\end{solution}


