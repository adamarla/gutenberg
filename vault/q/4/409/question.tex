


\ifnumequal{\value{rolldice}}{0}{
  % variables 
}{
  \ifnumequal{\value{rolldice}}{1}{
    % variables 
  }{
    \ifnumequal{\value{rolldice}}{2}{
      % variables 
    }{
      % variables 
    }
  }
}

\question How many 3-digit numbers can be formed from the digits 1- 5 assuming that


\watchout

\ifprintanswers
  % stuff to be shown only in the answer key - like explanatory margin figures
  \begin{marginfigure}
    \figinit{pt}
      \figpt 100:(0,0)
      \figpt 101:(0,0)
    \figdrawbegin{}
      \figdrawline [100,101]
    \figdrawend
    \figvisu{\figBoxA}{}{%
    }
    \centerline{\box\figBoxA}
  \end{marginfigure}
\fi 

\begin{parts}
  \part[3] Repetition of digit is allowed

\begin{solution}[\mcq]
  The units place can be filled by any of these 5 digits in 5 ways. \\
  Since, the repetition of digits is allowed the tens place can again be filled by any of these 5 digits and so is the case with the hundredth place. \\
  Thus by using multiplication method the total number of ways are $5\times 5 \times 5 = 125$\\  
  \end{solution}

  \part[3] Repetition is not allowed

\begin{solution}[\mcq]
  The units place is filled first, by anyone of the 5 digits in 5 ways.\\
  Since repetition is not allowed, the number that is at the units place can not be used at other places.\\
  Hence, the tens place can be filled by the remaining 4 digits only. Similarly, the hundredth place can be filled by remaining 3 digits only in 3 ways.\\
  By multiplication method the total number of ways are\\ $5\times 4 \times 3 = 60$\\
   \end{solution}
\end{parts}

