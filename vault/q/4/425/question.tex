


\ifnumequal{\value{rolldice}}{0}{
  % variables 
  \renewcommand{\va}{4}
  \renewcommand{\vb}{5}
}{
  \ifnumequal{\value{rolldice}}{1}{
    % variables 
    \renewcommand{\va}{4}
    \renewcommand{\vb}{6}
  }{
    \ifnumequal{\value{rolldice}}{2}{
      % variables 
      \renewcommand{\va}{3}
      \renewcommand{\vb}{7}
    }{
      % variables 
      \renewcommand{\va}{6}
      \renewcommand{\vb}{8}
    }
  }
}

\MULTIPLY\va\vb\vc

\question[5] A bag contains $\va$ mangoes and and $\vb$ oranges. 
In how many ways can a person take out \textbf{atleast} one mango and one orange from the bag? 
There is no difference between any two mangoes nor is there any difference between any two oranges.

\watchout

\begin{solution}[\halfpage]
  \textbf{Insight \#1: The importance of being identical}

  Had each mango (or orange) been of a different variety, then it would have mattered 
  which two or three were picked from the bag. 

  But as the mangoes (and oranges) are all identical to each other, the only difference 
  between any two outcomes is \textbf{how many} of each were picked - not which ones.

  Which means, there can be \textbf{only one way to pick} $N$ mangoes (or oranges) from the bag.
  Think about it. \textbf{You cannot have two ways of picking three mangoes from the bag.}

  The above argument is summarized in the table below. 

  \begin{tabular}{c c c c c c c}
    \toprule
    \# picked & 1 & 2 & 3 & $\ldots$ & N & Total \\
    \midrule 
    \# ways  & 1 & 1 & 1 & $\ldots$ & 1 & $\sum_{k=1}^N 1 = N$ \\
    \bottomrule
  \end{tabular}

  Given this, the \textbf{total number of ways of picking $\geq 1$} 
  \[ \text{Mangoes} = \sum_{k=1}^{\va} = \va\text{ and Oranges} = \sum_{k=1}^\vb = \vb \] 
  And hence, the \textbf{the total number of ways} is 
  \[ N_{\text{total}} = \va\times\vb = \vc \] 
\end{solution}

\ifprintanswers\begin{codex}$\vc$\end{codex}\fi
