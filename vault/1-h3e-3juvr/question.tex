% This is an empty shell file placed for you by the 'examiner' script.
% You can now fill in the TeX for your question here.

% Now, down to brasstacks. ** Writing good solutions is an Art **. 
% Eventually, you will find your own style. But here are some thoughts 
% to get you started: 
%
%   1. Write the solution as if you are writing it for your favorite
%      14-17 year old to help him/her understand. Could be your nephew, 
%      your niece, a cousin perhaps or probably even you when you 
%      were that age. Just write for them.
%
%   2. Use margin-notes to "talk" to students about the critical insights
%      in the question. The tone can be - in fact, should be - informal
%
%   3. Don't shy away from creating margin-figures you think will help
%      students understand. Yes, it is a little more work per question. 
%      But the question & solution will be written only once. Make that
%      attempt at writing a solution count.
%
%   4. At the same time, do not be too verbose. A long solution can
%      - at first sight - make the student think, "God, that is a lot to know".
%      Our aim is not to scare students. Rather, our aim should be to 
%      create many "Aha!" moments everyday in classrooms around the world
% 
%   5. Ensure that there are *no spelling mistakes anywhere*. We are an 
%      education company. Bad spellings suggest that we ourselves 
%      don't have any education. Also, use American spellings by default
% 
%   6. If a question has multiple parts, then first delete lines 40-41
%   7. If a question does not have parts, then first delete lines 43-69

\question \begin{fullwidth} A couple has two children. Find the probability that both 
the children are girls if \end{fullwidth}

\insertQR{}

\ifprintanswers
  % stuff to be shown only in the answer key - like explanatory margin figures
  \begin{table}
  	\begin{tabular}{ccc}
  	   \toprule
  	   Combination & First child & Second child \\
  	   \midrule
  	   1 & Girl & Girl \\
  	   2 & Girl & Boy \\
  	   3 & Boy & Girl \\
  	   4 & Boy & Boy \\
  	   \bottomrule
  	\end{tabular}
  \end{table}
\fi 

\begin{parts}
  \part Atleast one child is a girl

  \insertQR{}
  \begin{solution}
     Which means that combinations 1,2 and 3 are the only possible outcomes. And of these, 
     only the first combination is a favorable outcome. Therefore, the required probability
     is $\dfrac{1}{3}$
  \end{solution}

  \part The older child is a girl 

  \insertQR{}
  \begin{solution}
     If the first child was a girl, then our set of possible outcomes is down to just
     the first two. And therefore, the required probability is $\dfrac{1}{2}$
  \end{solution}

\end{parts}
