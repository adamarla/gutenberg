% This is an empty shell file placed for you by the 'examiner' script.
% You can now fill in the TeX for your question here.

% Now, down to brasstacks. ** Writing good solutions is an Art **. 
% Eventually, you will find your own style. But here are some thoughts 
% to get you started: 
%
%   1. Write the solution as if you are writing it for your favorite
%      14-17 year old to help him/her understand. Could be your nephew, 
%      your niece, a cousin perhaps or probably even you when you 
%      were that age. Just write for them.
%
%   2. Use margin-notes to "talk" to students about the critical insights
%      in the question. The tone can be - in fact, should be - informal
%
%   3. Don't shy away from creating margin-figures you think will help
%      students understand. Yes, it is a little more work per question. 
%      But the question & solution will be written only once. Make that
%      attempt at writing a solution count.
%
%   4. At the same time, do not be too verbose. A long solution can
%      - at first sight - make the student think, "God, that is a lot to know".
%      Our aim is not to scare students. Rather, our aim should be to 
%      create many "Aha!" moments everyday in classrooms around the world
% 
%   5. Ensure that there are *no spelling mistakes anywhere*. We are an 
%      education company. Bad spellings suggest that we ourselves 
%      don't have any education. And, use American spellings

\question[5]  The mid-points of the sides of a triangle are $(3,4)$, $(4,6)$ and $(5,7)$. Find the coordinates of the vertices of the triangle
\insertQR{QRC}

\ifprintanswers
	\begin{marginfigure}
		\figinit{pt}
		\figpt 1:(10,0)
		\figpt 2:(90,0)
		\figpt 3:(30,60)
		\figvectP 10 [1,2]
		\figvectP 20 [2,3]
		\figvectP 30 [3,1]
		\figpttra 11:$A(3,4)$= 1/0.5,10/
		\figpttra 21:$B(4,6)$= 2/0.5,20/
		\figpttra 31:$C(5,7)$= 3/0.5,30/
		\figdrawbegin{}
			\figdrawline[1,2,3,1]
			\figdrawline[11,21,31,11]
		\figdrawend
		\figvisu{\figBoxA}{}{
			\figsetmark{$\bullet$}
			\figwritew 1:$X$(5)
			\figwritee 2:$Y$(5)
			\figwriten 3:$Z$(5)
			\figwrites 11:(5)
			\figwritee 21:(5)
			\figwritew 31:(5)
		}
		\centerline{\box\figBoxA}
	\end{marginfigure}

  % stuff to be shown only in the answer key - like explanatory margin figures
	\marginnote[0.5cm] {This question uses 3 facts}
	\marginnote[0.2cm] {The line joining mid-points of two sides is parallel to the third side}
	\marginnote[0.2cm] {Lines that are parallel to each other have the same slope}
	\marginnote[0.2cm] {The third mid-point lies on the third side (obviously)}
\fi 

\begin{solution}[\fullpage]
	The equation of the 3 lines joining the 3 mid-points are given by
	\begin{align}
		AB &: \dfrac{y-4}{x-3} = \dfrac{6-4}{4-3} \Rightarrow y = 2x - 2 \\
		BC &: \dfrac{y-6}{x-4} = \dfrac{7-6}{5-4} \Rightarrow y = x + 2 \\
		CA &: \dfrac{y-4}{x-3} = \dfrac{7-4}{5-3} \Rightarrow 2y = 3x - 1
	\end{align}
	
	The equation for the three sides of the triangle are therefore given by
	
	\begin{align}
		XZ &: \dfrac{y-7}{x-5} = 2 \Rightarrow y = 2x - 3 \\
		XY &: \dfrac{y-4}{x-3} = 1 \Rightarrow y = x + 1 \\
		YZ &: \dfrac{y-6}{x-4} = \dfrac{3}{2} \Rightarrow y = \dfrac{3}{2}x
	\end{align}
	
	And the vertices are therefore given by
	\begin{align}
		X &: 2x-3 = x + 1 \Rightarrow x = 4, y = 5 \\
		Y &: \dfrac{3}{2}x = x + 1 \Rightarrow x = 2, y = 3 \\
		Z &: 2x-3 = \dfrac{3}{2}x \Rightarrow x = 6, y = 9
	\end{align}
	The three vertices are therefore $(4,5)$, $(2,3)$ and $(6,9)$
	
	
	
	
	
\end{solution}
