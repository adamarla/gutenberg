

\question[4] Two workers working together can complete a job in $12$ days. If the first worker completes half the work and then the second worker takes over, the job will be completed in $25$ days. How many days would it take each worker to do the job if they worked alone?


\ifprintanswers
  % stuff to be shown only in the answer key - like explanatory margin figures
\fi 

\begin{solution}[\fullpage]
  Let the time it takes the two worker to do the entire job by themselves be $t_1$ days and $t_2$ days respectively. If you think of a job as a unit of work, then you can describe the speed of doing work as follows,
  \begin{align}
    \text{Speed of first worker}  &= \frac{1}{t_1}(jobs/day) \\
    \text{Speed of second worker} &= \frac{1}{t_2}(jobs/day) \\
    \text{Effective speed when both work together}     
    				  &= \left(\frac{1}{t_1}+\frac{1}{t_2}\right)(jobs/day)
  \end{align}
  
  Now, working together they complete a job as a whole in $12$ days, therefore,
  \begin{align}
	\dfrac{1(job)}
	      {\left(\dfrac{1}{t_1}+\dfrac{1}{t_2}\right)(jobs/day)} &= 12(days) \\
	\Rightarrow \dfrac{t_1t_2}{t_1+t_2}                          &= 12
  \end{align}
  
  Dividing the job into two halves, where each worker completes one half and the second one starts only after the first one finishes, they take $25$ days, therefore,
  \begin{align}
    \dfrac{\frac{1}{2}(job)}{\dfrac{1}{t_1}(jobs/day)}+
    	\dfrac{\frac{1}{2}(job)}{\dfrac{1}{t_1}(jobs/day)} &= 25(days) \\
    \Rightarrow t_1+t_2                                    &= 50
  \end{align}
  
  Using results (5) and (7), we get,
  \begin{align}
    \dfrac{t_1(50-t_1)}{t_1+(50-t_1)} &= 12 \\
    50t_1-t_1^2                       &= 600 \\
    t_1^2-50t_1+600                   &= 0 \\     
    t_1                               &= 20 \\
    t_2                               &= 30
  \end{align}
  Therefore, working by themselves, the workers would take $20$ and $30$ days respectively to complete the job.

\end{solution}

