\question Consider the universal relation $R=\{A,B,C,D,E,F,G,H,I\}$ and
the set of functional dependencies,
\begin{align}
  \{\{A,B\}\rightarrow \{C\},
  \{A\}\rightarrow\{D,E\},
  \{B\}\rightarrow\{F\},
  \{F\}\rightarrow\{G,H\},
  \{D\}\rightarrow\{I,J\}\} \nonumber
\end{align}
Compute the following and show your steps and justification as you do so.

\begin{parts}
  \part[2] $\{A,B\}^+$

\begin{solution}[\mcq]
    \begin{align}
      \text{Trivially, }\Rightarrow
        &\{A,B\}^+\rightarrow\{A,B\} \nonumber\\
      \{A\}\rightarrow\{D,E\},\{B\}\rightarrow\{F\} \Rightarrow
        &\{A,B\}^+\rightarrow\{A,B,D,E,F\} \nonumber\\
      \{A,B\}\rightarrow\{C\}\Rightarrow   
        &\{A,B\}^+\rightarrow\{A,B,C,D,E,F\} \nonumber\\
      \{F\}\rightarrow\{G,H\},\{D\}\rightarrow\{I,J\}
        &\{A,B\}^+\rightarrow\{A,B,C,D,E,F,G,H,I,J\} \nonumber
    \end{align}
  \end{solution}

  \part[1] Is $\{A,B\}^+$ a candidate key?

\begin{solution}[\mcq]
    Yes. Since $\{A,B\}^+$ covers the entire relation $R$, therefore 
    $AB$ is a candidate key.
  \end{solution}

  \part[2] Convert the relation $R$ into $2$NF.
\begin{solution}[\mcq]
    To normalize, we separate the attributes which are dependent on part of
    the key i.e. either $A$ or $B$, and place them into relations $R_1$ and
    $R_2$.
    \begin{align}
      \{A\}^+ &= \{\overline{A},D,E,I,J\} \\
      \{B\}^+ &= \{\overline{B},F,G,H\}
    \end{align}
    The remaining part forms relation $R_3$ for which the candidate key is 
    $A$ and $B$ taken together forms a separate relation $R_3$. Therefore the
    $2$NF form would contain,
    \begin{align}
      R_1 &= \{\overline{A},D,E,I,J\} \\
      R_2 &= \{\overline{B},F,G,H\} \\
      R_3 &= \{\overline{A,B},C\}
    \end{align}
    
  \end{solution}

\end{parts}

