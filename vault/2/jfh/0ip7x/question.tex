


\ifnumequal{\value{rolldice}}{0}{
  \renewcommand{\va}{A}
  \renewcommand{\vb}{B}
  \renewcommand{\vc}{2}
  \renewcommand{\vd}{3}
}{
  \ifnumequal{\value{rolldice}}{1}{
    \renewcommand{\va}{P}
    \renewcommand{\vb}{Q}
    \renewcommand{\vc}{3}
    \renewcommand{\vd}{4}
  }{
    \ifnumequal{\value{rolldice}}{2}{
      \renewcommand{\va}{M}
      \renewcommand{\vb}{N}
      \renewcommand{\vc}{2}
      \renewcommand{\vd}{4}
    }{
      \renewcommand{\va}{J}
      \renewcommand{\vb}{K}
      \renewcommand{\vc}{3}
      \renewcommand{\vd}{5}
    }
  }
}

\ADD\vc\vd\s
\MULTIPLY\vc\vd\m

\question Consider the following transactions:
\begin{align}
  T1:\;&\text{read}(\va,t);\;t:=t+\vc;\;\text{write}(\va,t);\;
    {read}(\vb,t);\;t:=t*\vd;\;\text{write}(\vb,t) \nonumber \\
  T2:\;&\text{read}(\vb,s);\;s:=s*\vc;\;\text{write}(\vb,s);\;
    \text{read}(\va,s);\;s:=s+\vd;\;\text{write}(\va,s) \nonumber
\end{align}
where, read$(X,y)\Rightarrow$ assign $X$ to $y$ and 
write$(X,y)\Rightarrow$ assign $y$ to $X$.

\watchout

\begin{parts}
  \part[4] Give an example of a schedule (interleaved transactions) 
  which gives the same outcome as a serial schedule but is not 
  \textit{conflict serializable}. Justify your answer.

  \begin{solution}[\fullpage]
    To begin with, let us compute the output of the following schedules.
    Regardless of whether they were executed as $T_1\rightarrow T_2$ or 
    $T_2\rightarrow T_1$, the resulting values would be,
    \begin{align}
      \va &= \va + \s \\
      \vb &= \vb \times \m
    \end{align}
    Consider the schedule below:
    \begin{align}
      \begin{tabular}{|c|c|c|c|c|}
        \hline
        t & T1 & T2 & \va & \vb \\
        \hline
        1 & read$(\va,t)$      & - & $\va$ & $\vb$ \\
        2 & - & read$(\vb,s)$  & $\va$ & $\vb$ \\
        3 & $t=t+\vc$ & -      & $\va$ & $\vb$ \\
        4 & - & $s=s\times\vc$ & $\va$ & $\vb$ \\
        5 & - & write$(\vb,s)$ & $\va$ & $\vb\times\vc$ \\
        6 & write$(\va,t)$ & - & $\va+\vc$ & $\vb\times\vc$ \\
        7 & - & read$(\va,s)$  & $\va+\vc$ & $\vb\times\vc$ \\
        8 & read$(\vb,t)$ & -  & $\va+\vc$ & $\vb\times\vc$ \\
        9 & $t=t\times\vd$ & - & $\va+\vc$ & $\vb\times\vc$ \\
        10& write$(\vb,t)$ & - & $\va+\vc$ & $\vb\times\m$ \\
        11& - & $s=s+\vd$      & $\va+\vc$ & $\vb\times\m$ \\
        12& - & write$(\va,s)$ & $\va+\s$ & $\vb\times\m$ \\
        \hline
      \end{tabular} \nonumber
    \end{align}
    In the above schedule, resource $\vb$ is written to by $T_2(t=5)$
    before it is read by $T_1(t=8)$ which implies $T_2\rightarrow T_1$.
    However, resource $\va$ is written to by $T_1(t=6)$ before it is 
    read by $T_2(t=7)$ which implies $T_1\rightarrow T_2$.
    Therefore the precedence graph for the above schedule contains a
    cycle and the schedule is not \textit{conflict serializable}.
  \end{solution}

  \part[4] Give an example of a schedule (interleaved transactions) 
  which results in database inconsistency (non-serializable). 
  Justify your answer.

  \begin{solution}[\fullpage]
    Consider the following schedule:
    \begin{align}
      \begin{tabular}{|c|c|c|c|c|}
        \hline
        t & T1 & T2 & \va & \vb \\
        \hline
        1 & read$(\va,t)$      & - & $\va$ & $\vb$ \\
        2 & - & read$(\vb,s)$  & $\va$ & $\vb$ \\
        3 & $t=t+\vc$ & -      & $\va$ & $\vb$ \\
        4 & - & $s=s\times\vc$ & $\va$ & $\vb$ \\
        5 & - & write$(\vb,s)$ & $\va$ & $\vb\times\vc$ \\
        6 & - & read$(\va,s)$  & $\va$ & $\vb\times\vc$ \\
        7 & - & $s=s+\vd$      & $\va$ & $\vb\times\vc$ \\
        8 & - & write$(\va,s)$ & $\va+\vd$ & $\vb\times\vc$ \\
        9 & write$(\va,t)$ & - & $\va+\vc$ & $\vb\times\vc$ \\
        10& read$(\vb,t)$ & -  & $\va+\vc$ & $\vb\times\vc$ \\
        11& $t=t\times\vd$ & - & $\va+\vc$ & $\vb\times\vc$ \\
        12& write$(\vb,t)$ & - & $\va+\vc$ & $\vb\times\m$ \\
        \hline
      \end{tabular} \nonumber
    \end{align}
    As can be seen, update by $T_2$ on $\va(t=8)$ is lost due
    to $T_1$ overwriting it with it's own update on $\va(t=9)$. 
    Therefore the resulting state of the database is inconsistent
    with the original schedule.
  \end{solution}

\end{parts}

