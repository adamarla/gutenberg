
\ifnumequal{\value{rolldice}}{0}{
  % variables 
  \renewcommand{\vbone}{4}
  \renewcommand{\vbtwo}{6}
  \renewcommand{\vbthree}{5}
  \renewcommand{\vbfour}{3\sqrt{2}}
}{
  \ifnumequal{\value{rolldice}}{1}{
    % variables 
    \renewcommand{\vbone}{6}
    \renewcommand{\vbtwo}{4}
    \renewcommand{\vbthree}{3}
    \renewcommand{\vbfour}{4}
  }{
    \ifnumequal{\value{rolldice}}{2}{
      % variables 
      \renewcommand{\vbone}{4}
      \renewcommand{\vbtwo}{4}
      \renewcommand{\vbthree}{4}
      \renewcommand{\vbfour}{2\sqrt{3}}
    }{
      % variables 
      \renewcommand{\vbone}{6}
      \renewcommand{\vbtwo}{6}
      \renewcommand{\vbthree}{2}
      \renewcommand{\vbfour}{2\sqrt{5}}
    }
  }
}

\DIVIDE\vbone{2}\a
\DIVIDE\vbtwo{2}\b

\question[2] Find the center and radius of the circle represented 
by the following equation
\begin{align}
  x^2 + y^2 - \vbone x + \vbtwo y - \vbthree = 0 \nonumber
\end{align}


\watchout

\ifprintanswers
  % stuff to be shown only in the answer key - like explanatory margin figures
  \begin{marginfigure}
    \figinit{pt}
      \figpt 100:(0,0)
      \figpt 101:(0,0)
    \figdrawbegin{}
      \figdrawline [100,101]
    \figdrawend
    \figvisu{\figBoxA}{}{%
    }
    \centerline{\box\figBoxA}
  \end{marginfigure}
\fi 

\begin{solution}[\halfpage]
  Let us begin by expanding the general equation of a circle with 
  its center at $(a, b)$ and a radius $r$,
  \begin{align}
                &(x-a)^2 + (y-b)^2 = r^2 \\
    \Rightarrow &x^2 + y^2 - 2a -2b + (a^2 + b^2 -r^2) = 0
  \end{align}
  On comparing coefficients with the equation given in the question 
  we get,
  \begin{align}
      2a = \vbone &\Rightarrow a=\a \\
      2b = -\vbtwo  &\Rightarrow b=-\b \\
      a^2+b^2-c^2 = -\vbthree &\Rightarrow c=\vbfour
  \end{align}
  Therefore, the given circle has $(\a, \b)$ as center and a radius
  of $\vbfour$ units.
\end{solution}
