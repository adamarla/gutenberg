


\ifnumequal{\value{rolldice}}{0}{
  % variables 
  \renewcommand{\vbone}{3}
  \renewcommand{\vbtwo}{4}
  \renewcommand{\vbthree}{64}
}{
  \ifnumequal{\value{rolldice}}{1}{
    % variables 
    \renewcommand{\vbone}{3}
    \renewcommand{\vbtwo}{5}
    \renewcommand{\vbthree}{125}
  }{
    \ifnumequal{\value{rolldice}}{2}{
      % variables 
      \renewcommand{\vbone}{4}
      \renewcommand{\vbtwo}{5}
      \renewcommand{\vbthree}{625}
    }{
      % variables 
      \renewcommand{\vbone}{2}
      \renewcommand{\vbtwo}{3}
      \renewcommand{\vbthree}{9}
    }
  }
}

\question[4] In how many ways can $\vbone$ letters addressed to different people be posted in $\vbtwo$ postboxes?


\watchout

\ifprintanswers
  % stuff to be shown only in the answer key - like explanatory margin figures
  \begin{marginfigure}
    \figinit{pt}
      \figpt 100:(0,0)
      \figpt 101:(0,0)
    \figdrawbegin{}
      \figdrawline [100,101]
    \figdrawend
    \figvisu{\figBoxA}{}{%
    }
    \centerline{\box\figBoxA}
  \end{marginfigure}
\fi 

\begin{solution}[\mcq]
Each letter can be posted in $\vbtwo$ post boxes. First into $\vbtwo$, second into $\vbtwo$ and so on...\\
Thus the total ways of posting these letters are $\underbrace{\vbtwo \times \vbtwo \cdots}_{\vbone \text{times}}$\\
\begin{align}
\Rightarrow {\vbtwo}^{\vbone} = {\vbthree}
\end{align}  
\end{solution}

