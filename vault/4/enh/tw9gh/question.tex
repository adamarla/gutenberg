


\ifnumequal{\value{rolldice}}{0}{
  % variables 
  \renewcommand{\vbone}{4}
  \renewcommand{\vbtwo}{6}
  \renewcommand{\vbthree}{2}
}{
  \ifnumequal{\value{rolldice}}{1}{
    % variables 
    \renewcommand{\vbone}{2}
    \renewcommand{\vbtwo}{4}
    \renewcommand{\vbthree}{5}
  }{
    \ifnumequal{\value{rolldice}}{2}{
      % variables 
      \renewcommand{\vbone}{3}
      \renewcommand{\vbtwo}{4}
      \renewcommand{\vbthree}{3}
    }{
      % variables 
      \renewcommand{\vbone}{6}
      \renewcommand{\vbtwo}{4}
      \renewcommand{\vbthree}{4}
    }
  }
}

\question[4] $\vbone$ different math books, $\vbtwo$ different physics books, $\vbthree$ different chemistry books are arranged on a shelf. How many different arrangements are possible is books of each particular subject stand together?   


\watchout

\ifprintanswers
  % stuff to be shown only in the answer key - like explanatory margin figures
  \begin{marginfigure}
    \figinit{pt}
      \figpt 100:(0,0)
      \figpt 101:(0,0)
    \figdrawbegin{}
      \figdrawline [100,101]
    \figdrawend
    \figvisu{\figBoxA}{}{%
    }
    \centerline{\box\figBoxA}
  \end{marginfigure}
\fi 

\begin{solution}[\halfpage]
Since all the books of the same  subject are to be kept together. We can assume that the math books are kept in one tray, all physics books in the second tray and all chemistry books in the third tray.\\
\\ 
Now, $\vbone$ math books can be arranged in the first tray in $\vbone!$ ways. $\vbtwo$ math books can be arranged in the second tray in $\vbtwo!$ ways and $\vbthree$ math books can be arranged in the third tray in $\vbthree!$ ways.\\
\\
And the three trays can be rearranged among themselves in $3!$ ways. Thus,\\
total number of arrangements $= 3! \times \vbone! \times \vbtwo! \times \vbthree! $
\end{solution}


