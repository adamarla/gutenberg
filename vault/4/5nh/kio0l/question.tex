


\ifnumequal{\value{rolldice}}{0}{
  % variables 
  \renewcommand\va{15}
  \renewcommand\vb{16}
}{
  \ifnumequal{\value{rolldice}}{1}{
    % variables 
    \renewcommand\va{4}
    \renewcommand\vb{7}
  }{
    \ifnumequal{\value{rolldice}}{2}{
      % variables 
      \renewcommand\va{3}
      \renewcommand\vb{5}
    }{
      % variables 
      \renewcommand\va{7}
      \renewcommand\vb{11}
    }
  }
}

\FRACMINUS{2}{1}\va\vb\vc\vd
\FRACMINUS{1}{1}\va\vb\ve\vf
\FRACDIV{-1}{7}\ve\vf\vg\vh

\question[5] Evaluate the following integral 
    \[ \int\dfrac{dx}{(x+3)^{\frac\va\vb}\cdot (x-4)^{\frac\vc\vd}}\]


\watchout

\begin{solution}[\fullpage]
  Notice that $\frac\va\vb + \frac\vc\vd = 2$ (an integer). The sum of two fractions can be anything. 
  But when it is an integer, perhaps something interesting is happening. So, let us work with this new fact. 

  Rewriting the above integral as follows
  \begin{align}
    \int\dfrac{dx}{(x+3)^{\frac\va\vb}\cdot (x-4)^{\frac\vc\vd}} &= 
    \int\dfrac{dx}{\left( \dfrac{x+3}{x-4} \right)^{\frac\va\vb}\cdot (x-4)^2}
  \end{align}

  Now, if we let $z=\dfrac{x+3}{x-4}$, then 
  $\dfrac{d}{dx} z = \underbrace{\dfrac{(x-4)-(x+3)}{(x-4)^2}}_{\texttt{Quotient Rule}} = \dfrac{-7}{(x-4)^2}$

  And therefore, our original integral can be re-written as 
  \begin{align}
    \int\dfrac{dx}{\left( \dfrac{x+3}{x-4} \right)^{\frac\va\vb}\cdot (x-4)^2} &= 
    -\dfrac{1}{7}\int\dfrac{dz}{z^{\frac\va\vb}} = -\dfrac{1}{7}\int z^{-\frac\va\vb} dz \\
    &= -\dfrac{1}{7}\cdot\left( \dfrac{z^{1-\frac\va\vb}}{1-\frac\va\vb}\right) + C \\
    &= -\dfrac\vg\vh z^\frac\ve\vf + C  \\
    &= -\dfrac\vg\vh\left( \dfrac{x+3}{x-4} \right)^\frac\ve\vf + C 
  \end{align}
\end{solution}

