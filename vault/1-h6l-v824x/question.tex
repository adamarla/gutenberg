% This is an empty shell file placed for you by the 'examiner' script.
% You can now fill in the TeX for your question here.

% Now, down to brasstacks. ** Writing good solutions is an Art **. 
% Eventually, you will find your own style. But here are some thoughts 
% to get you started: 
%
%   1. Write the solution as if you are writing it for your favorite
%      14-17 year old to help him/her understand. Could be your nephew, 
%      your niece, a cousin perhaps or probably even you when you 
%      were that age. Just write for them.
%
%   2. Use margin-notes to "talk" to students about the critical insights
%      in the question. The tone can be - in fact, should be - informal
%
%   3. Don't shy away from creating margin-figures you think will help
%      students understand. Yes, it is a little more work per question. 
%      But the question & solution will be written only once. Make that
%      attempt at writing a solution count.
%
%   4. At the same time, do not be too verbose. A long solution can
%      - at first sight - make the student think, "God, that is a lot to know".
%      Our aim is not to scare students. Rather, our aim should be to 
%      create many "Aha!" moments everyday in classrooms around the world
% 
%   5. Ensure that there are *no spelling mistakes anywhere*. We are an 
%      education company. Bad spellings suggest that we ourselves 
%      don't have any education. Also, use American spellings by default
% 
%   6. If a question has multiple parts, then first delete lines 40-41
%   7. If a question does not have parts, then first delete lines 43-69

\question The sum of the first four terms of a geometric progression is 30 and the 
sum of the \textit{next} four terms is 480. Find the sum of the first twelve terms

\calculator{2^{10} = 1024}

\insertQR{}

\ifprintanswers
\fi 

\begin{solution}
	\begin{align}
		\eSumOfGP{r}{4} &= 30 \\
		\underbrace{\overbrace{\eSumOfGP{r}{8}}^{\text{first 8 terms}} 
		- \eSumOfGP{r}{4}}_{\text{next 4 terms}} &= 480 \\
		\Rightarrow \eSumOfGP{r}{8} &= 480 + 30 = 510 \\
		\Rightarrow \dfrac{\eSumOfGP{r}{8}}{\eSumOfGP{r}{4}} &= 
		\dfrac{r^8-1}{r^4-1} = \dfrac{510}{30} = 17
	\end{align}
	Let $r^4 =x$ so that
	\begin{align}
		\dfrac{r^8-1}{r^4-1} &= \dfrac{x^2-1}{x-1} = 17 \\
		\Rightarrow x &= r^4 = 16 \Rightarrow r = 2 \\
		\text{which means } a &= \dfrac{30\cdot(r-1)}{(r^4-1)} = 2
	\end{align}
	Now we are in a position to calculate the sum of the first twelve terms
	\begin{align}
		S_{12} &= \eSumOfGP[2]{2}{12} = \dfrac{2\cdot(1-4096)}{-1} = 8190
	\end{align}
\end{solution}
