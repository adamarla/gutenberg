% This is an empty shell file placed for you by the 'examiner' script.
% You can now fill in the TeX for your question here.

% Now, down to brasstacks. ** Writing good solutions is an Art **. 
% Eventually, you will find your own style. But here are some thoughts 
% to get you started: 
%
%   1. Write to be understood - but be crisp. Your own solution should not take 
%      more space than you will give to the student. Hence, if you take more than 
%      a half-page to write a solution, then give the student a full-page and so on...
%
%   2. Use margin-notes to "talk" to students about the critical insights
%      in the question. The tone can be - in fact, should be - informal
%
%   3. Don't shy away from creating margin-figures you think will help
%      students understand. Yes, it is a little more work per question. 
%      But the question & solution will be written only once. Make that
%      attempt at writing a solution count.
%      
%      3b. Use bc_to_fig.tex. Its an easier way to generate plots & graphs 
% 
%   4. Ensure that there are *no spelling mistakes anywhere*. We are an 
%      education company. Bad spellings suggest that we ourselves 
%      don't have any education. Also, use American spellings by default
% 
%   5. If a question has multiple parts, then first delete lines 40-41
%   6. If a question does not have parts, then first delete lines 43-69
%   
%   7. Create versions of the question when possible. Use commands defined in 
%      tufte-tweaks.sty to do so. Its easier than you think

% \noprintanswers
% \setcounter{rolldice}{0}
% \printrubric

\ifnumequal{\value{rolldice}}{0}{
  % variables 
  \renewcommand{\vbone}{}
  \renewcommand{\vbtwo}{}
  \renewcommand{\vbthree}{}
  \renewcommand{\vbfour}{}
  \renewcommand{\vbfive}{}
  \renewcommand{\vbsix}{}
  \renewcommand{\vbseven}{}
  \renewcommand{\vbeight}{}
  \renewcommand{\vbnine}{}
  \renewcommand{\vbten}{}
}{
  \ifnumequal{\value{rolldice}}{1}{
    % variables 
    \renewcommand{\vbone}{}
    \renewcommand{\vbtwo}{}
    \renewcommand{\vbthree}{}
    \renewcommand{\vbfour}{}
    \renewcommand{\vbfive}{}
    \renewcommand{\vbsix}{}
    \renewcommand{\vbseven}{}
    \renewcommand{\vbeight}{}
    \renewcommand{\vbnine}{}
    \renewcommand{\vbten}{}
  }{
    \ifnumequal{\value{rolldice}}{2}{
      % variables 
      \renewcommand{\vbone}{}
      \renewcommand{\vbtwo}{}
      \renewcommand{\vbthree}{}
      \renewcommand{\vbfour}{}
      \renewcommand{\vbfive}{}
      \renewcommand{\vbsix}{}
      \renewcommand{\vbseven}{}
      \renewcommand{\vbeight}{}
      \renewcommand{\vbnine}{}
      \renewcommand{\vbten}{}
    }{
      % variables 
      \renewcommand{\vbone}{}
      \renewcommand{\vbtwo}{}
      \renewcommand{\vbthree}{}
      \renewcommand{\vbfour}{}
      \renewcommand{\vbfive}{}
      \renewcommand{\vbsix}{}
      \renewcommand{\vbseven}{}
      \renewcommand{\vbeight}{}
      \renewcommand{\vbnine}{}
      \renewcommand{\vbten}{}
    }
  }
}

\question The radius r of a sphere is increasing at a constant rate of 
 $0.04$ centimetres per second. \textit{note: The Volume $V$ of a sphere with
 radius $r$ is given by $\dfrac{4}{3}\pi r^3$.}

\insertQR{}

\ifprintanswers
\fi 

\begin{parts}
  \part[3] At the time when the radius of the sphere is 10 centimetres, what is 
  the rate of  increase of its volume?

\begin{solution}[\mcq]
    Rate of increase of volume $\dfrac{\ud V}{\ud t}$ is given by,
    \begin{align}
      &\dfrac{\ud V}{\ud t} = \dfrac{4\pi}{3}\cdot3r^2\cdot\dfrac{\ud r}{\ud t} 
                            = 4\pi r^2\dfrac{\ud r}{\ud t} \\
      &\dfrac{\ud V}{\ud t} \vert_{r=10} = 4\pi (100)(0.04) = 16\pi
    \end{align}
    $V$ is changing at the rate of $16\pi (cm^3/sec)$ when $r=10(cm)$.
  \end{solution}

  \part[3] At the time when the volume of the sphere is $36\pi$ cubic centimetres,
  what is the rate of increase of the area of a cross section through the center 
  of the sphere?

\begin{solution}[\mcq]
    When $Volume$ is $36\pi (cm^3)$,
    \begin{align}
      \dfrac{4}{3}\pi r^3 = 36\pi \Rightarrow r^3 = 27 \Rightarrow r=3      
    \end{align}
    Rate of increase of area $\dfrac{\ud A}{\ud t}$ is given by,
    \begin{align}
      &\dfrac{\ud A}{\ud t} = \pi\cdot\dfrac{\ud r^2}{\ud t} 
                            = 2\pi r\dfrac{\ud r}{\ud t} \\
      &\dfrac{\ud A}{\ud t} \vert_{r=3} = 2\pi(3)(0.04) = 0.24\pi
    \end{align}
    $A$ is changing at the rate of $0.24\pi (cm^2/sec)$ when $r=3(cm)$.
  \end{solution}

  \part[3] At the time when the volume and the radius of the sphere are increasing 
  at the same \textit{numerical} rate, what is the radius?

\begin{solution}[\mcq]
    To find $r$ when $\dfrac{\ud V}{\ud t}$ and $\dfrac{\ud r}{\ud t}$ are 
    numerically equal
    \begin{align}
      &4\pi r^2\cdot\dfrac{\ud r}{\ud t} = \dfrac{\ud r}{\ud t} \\
      &4\pi r^2 = 1 \Rightarrow r=\sqrt{\dfrac{1}{4\pi}}
    \end{align}
    $r$ equals $\sqrt{\dfrac{1}{4\pi}}(cm)$ when $\dfrac{\ud V}{\ud t}$ and
    $\dfrac{\ud r}{\ud t}$ are numerically equal.
  \end{solution}

\end{parts}

\ifprintrubric
  \begin{table}
  	\begin{tabular}{ p{5cm}p{5cm} }
  		\toprule % in brief (4-6 words), what should a grader be looking for for insights & formulations
  		  \sc{\textcolor{blue}{Insight}} & \sc{\textcolor{blue}{Formulation}} \\ 
  		\midrule % ***** Insights & formulations ******
  		\toprule % final numerical answers for the various versions
        \sc{\textcolor{blue}{If question has $\ldots$}} & \sc{\textcolor{blue}{Final answer}} \\
  		\midrule % ***** Numerical answers (below) **********
  		\bottomrule
  	\end{tabular}
  \end{table}
\fi
